\documentclass[a4paper,12pt]{book}


\usepackage{../../../utilidades/guia_linux}

% Comando para incluir porciones de texto
\newcommand\incluir[1]{\input{../../../contenido/#1}}
\setcounter{chapter}{2}

\begin{document}
\renewcommand{\chaptername}{M�dulo}

\chapter{M�dulo 3: El Entorno gr�fico}

%%%%%%%%%%%
% Clase 1 %
%%%%%%%%%%%
% * Modelo de capas del protoco X11 (X/Window) - detallado
\incluir{ServidorX/ModeloDeCapas}

% * Introducci�n a bibliotecas gr�ficas y diferentes entornos
\incluir{ServidorX/BibliotecasGraficas}

%%%%%%%%%%%
% Clase 2 %
%%%%%%%%%%%
% * KDE: Historia e introducci�n
% * Ejemplo de la filosof�a del software libre en Santa Fe
% * Descripci�n del entorno
% * Administrador de ventanas
% * Panel

%%%%%%%%%%%
% Clase 3 %
%%%%%%%%%%%
\section{Personalizaci�n del entorno}
\sonsubsecciones
 % * Personalizaci�n del entorno
 \incluir{KDE2/Configuracion/Introduccion}
 %   * Idioma
 \incluir{KDE2/Configuracion/PersonalizacionDelIdioma}
 %   * Configuraci�n del teclado 
 \incluir{KDE2/Configuracion/ConfiguracionDelTipoDeTeclado}
 %   * Escritorio - Entorno
 \incluir{KDE2/Configuracion/PersonalizacionDelEscritorio/ConfiguracionDelEscritorio}
 \incluir{KDE2/Configuracion/PersonalizacionDelEscritorio/FondoDePantalla}
 \incluir{KDE2/Configuracion/PersonalizacionDelEscritorio/GestorDeTemas}
 \incluir{KDE2/Configuracion/PersonalizacionDelEscritorio/Colores}
 \incluir{KDE2/Configuracion/PersonalizacionDelEscritorio/Estilo}
 \incluir{KDE2/Configuracion/PersonalizacionDelEscritorio/Fuentes}
 \incluir{KDE2/Configuracion/PersonalizacionDelEscritorio/IconosDelEscritorio}

 \incluir{KDE2/Configuracion/PersonalizacionDeLasVentanas}


 %   * Comportamiento de las ventanas
 %\incluir{KDE2/Configuracion/Ventanas/Ventanas}
 %\incluir{KDE2/Configuracion/Ventanas/BarraDeTitulo}
 %\incluir{KDE2/Configuracion/Ventanas/Propiedades}
 %\incluir{KDE2/Configuracion/Ventanas/Botones}
 %\incluir{KDE2/Configuracion/Ventanas/Avanzado}
 %   * Configuraci�n del Panel
\sonsecciones
%%%%%%%%%%%
% Clase 4 %
%%%%%%%%%%%
% * Konqueror
%   * Uso como navegador de disco
%   * Tipos MIME
%   * Navegador de  web
%   * configuraci�n de plug-ins

%%%%%%%%%%%
% Clase 5 %
%%%%%%%%%%%
% * Aplicaciones preinstaladas
% * Introduccion a la administraci�n del sistema
\end{document}
