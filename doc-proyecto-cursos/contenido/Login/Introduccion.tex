\porcion{Introducci�n}
\autor{\LDP}
\colaborador{Pedro Pablo Fabrega}
\revisor{\LLC}
\traductor{}

Los tipos de trabajos que se pueden realizar sobre un Unix cualquiera,
pero particularmente sobre GNU/Linux, difieren en la forma de
interactuar con el usuario y el formato de la interfaz de usuario. A
simple vista, se puede decir que hay dos tipos de acceso interactivo
en lo que respecta al formato de la interfaz: usando interfaz gr�fica
o usando interfaz de texto. En este curso se le dar� especial
importancia a la interfaz de texto, ya que es lo m�s normal que se
encuentra en los equipos que funcionan como servidores, y las
herramientas basadas en interfaz de texto tienen mayor tiempo de
desarrollo que las otras, lo que las hace m�s convenientes para la
tarea de administrar un sistema GNU/Linux.

