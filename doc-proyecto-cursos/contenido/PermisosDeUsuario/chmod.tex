\porcion{Cambio de permisos: el comando \comando{chmod}}
\autor{\LDP}
\colaborador{\NC}
\revisor{\LLC}
\traductor{}

Para cambiar los permisos de los archivos se usa el comando
\comando{chmod}. Su sintaxis es la siguiente:

\begin{vscreen}
chmod [-R] modo archivo...
\end{vscreen}

La opci�n \comando{-R} permite cambiar recursivamente los permisos de
todos los archivos dentro de un directorio.

El argumento \comando{modo} est� compuesto por alguna combinaci�n de
las letras \emph{u} (usuario due�o), \emph{g} (grupo due�o), y
\emph{o} (otros), seguido de un s�mbolo + o - dependiendo si se quiere
activar o desactivar un permiso, siguiendo por �ltimo una combinaci�n
de las letras correspondientes a los distintos permisos: \emph{r},
\emph{w} y \emph{x}. 

\begin{ejemplo}

Veremos algunos ejemplos comunes.

Si se necesita dar permisos de ejecuci�n al usuario y al grupo de un
archivo, el comando deber� ejecutarse de la siguiente manera:

\begin{vscreen}
chmod ug+x nombre-de-archivo
\end{vscreen}

Reci�n creado un archivo puede tener permisos no deseados, por ejemplo
lectura y escritura para el grupo y de lectura para el resto. Para
modificar este estado se utiliza 'go-rw'. 'g' es \emph{grupo}, 'o' es
\emph{otros}, '-' significa \emph{eliminar atriburos} y 'rw' es lectura y escritura
respectivamente.

\begin{vscreen}
$ touch archivo
$ ls -l archivo 
-rw-rw-r--    1 usuario     usuario        0 oct 21 14:09 archivo
$ chmod go-rw archivo 
$ ls -l archivo 
-rw-------    1 usuario     usuario        0 oct 21 14:09 archivo
$
\end{vscreen}
%$


O si se necesita sacar el permiso de lectura y ejecuci�n de todos los
archivos y subdirectorios del directorio
\archivo{/home/usuario/prueba} para el \emph{grupo} y los
\emph{otros}, se debe ejecutar:

\begin{vscreen}
chmod -R go-rx /home/usuario/prueba
\end{vscreen}


\end{ejemplo}
