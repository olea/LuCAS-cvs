\porcion{Representaci�n octal}
\autor{\LDP}
\colaborador{}
\revisor{\LLC}
\traductor{}

Existe una manera m�s �gil de representar los permisos de archivo.
Teniendo en cuenta que cada grupo de 3 bits es un n�mero binario, la
representaci�n en octal consiste en traducir cada grupo a un n�mero
octal, de tal manera que quede como resultado un n�mero de 3 d�gitos,
cada d�gito representando a un grupo de 3 bits.

Mejor aclarar esto con un ejemplo:

\begin{vscreen}
rwx rw- r--  representaci�n escrita
111 110 100  representaci�n binaria
 7   6   4   representaci�n octal
 |   |   |
 |   |  otros
 |  grupo
usuario
\end{vscreen}

La tabla \ref{tab:permisos} da una gu�a de la traducci�n de n�meros
binarios a octales.

\begin{table}[htbp]
  \begin{center}
    \begin{tabular}{|c|c|} \hline
      \emph{Binario} & \emph{Octal} \\ \hline \hline
      000 & 0 \\
      001 & 1 \\
      010 & 2 \\
      011 & 3 \\
      100 & 4 \\
      101 & 5 \\
      110 & 6 \\
      111 & 7 \\ \hline
    \end{tabular}
    \caption{Traducci�n de binario a octal}
    \label{tab:permisos}
  \end{center}
\end{table}

Entonces se puede concluir que los siguientes comandos son
equivalentes:

\begin{vscreen}
chmod u+rwx go-rwx nombre-de-archivo.txt

chmod 700 nombre-de-archivo.txt
\end{vscreen}
