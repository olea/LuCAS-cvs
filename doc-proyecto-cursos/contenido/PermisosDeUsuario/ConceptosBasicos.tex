\porcion{Conceptos b�sicos}
\autor{\LDP}
\colaborador{}
\revisor{\LLC}
\traductor{}
\label{sec:permisos}

Los permisos de un archivo cualquiera (inclusive los directorios) se
agrupan en 3 grupos de 3 bits cada uno, como se muestra m�s abajo:

\begin{vscreen}
rwx   rwx   rwx
 |     |     |
 |     |    otros
 |    grupo
usuario
\end{vscreen}

Como se ha dicho, cada grupo posee 3 bits:

\begin{description}
\item[Bit r:] Lectura
\item[Bit w:] Escritura
\item[Bit x:] Ejecuci�n
\end{description}

Con las diferentes combinaciones, se puede configurar un archivo para
que pueda ser le�do y modificado por su due�o, y s�lo le�do por el
grupo y los dem�s, por ejemplo el archivo \archivo{/etc/passwd}:

\begin{vscreen}
-rw-r--r--    1 root    root    1509 Apr  4 12:44 /etc/passwd
\end{vscreen}

Este archivo es del usuario \textbf{root}, y del grupo del mismo
nombre, solamente se puede modificar (bit <<w>> de escritura) por su
usuario due�o, y leer por el grupo y los dem�s.

Los grupos son un tema m�s que nada administrativo y no lo tocaremos en
esta secci�n. S�lo hay que tener en cuenta que generalmente en un
sistema GNU/Linux, un usuario cualquiera pertenece a su grupo (grupo
del mismo nombre que su nombre de usuario) y al grupo \emph{users}.

A diferencia de sistemas operativos como \emph{DOS} y \emph{Windows},
el hecho de que un archivo tenga una extensi�n \comando{.com} o
\comando{.exe} no significa que ser� un programa ejecutable. Al
necesitar restringir los derechos de ejecuci�n de cualquier
archivo\footnote{Siempre teniendo en cuenta a los archivos
  ejecutables, es decir programas.}, la acci�n de ejecutar cualquier
programa estar� supeditada al permiso correspondiente (bit <<x>> de
ejecuci�n). Esto es importante tenerlo en cuenta a la hora de
escribir programas que ser�n interpretados, ya que a fin de cuentas
los archivos ser�n de texto, y para que se ejecuten se les deber�
activar el permiso de ejecuci�n.

