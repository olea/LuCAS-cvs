\porcion{Un caso especial, los directorios}
\autor{\LDP}
\colaborador{}
\revisor{\LLC}
\traductor{}

Quiz�s a m�s de un lector le ha asaltado la siguiente duda: ?`para qu�
servir� el bit de ejecuci�n en los directorios?. Obviamente, los
directorios no se ejecutan, y evidentemente, el bit <<x>> en los
directorios existe. Como se ha aclarado anteriormente, en estos casos,
dicho bit tiene un significado especial.

El bit de ejecuci�n en los directorios permite  poder ver la
informaci�n acerca de los archivos que contienen.

El bit de lectura permite listar los contenidos de un directorio.

El bit de escritura permite crear y borrar archivos dentro de un
directorio.

Generalmente es conveniente manejar los permisos de lectura y
ejecuci�n de los directorios en forma conjunta, para evitar
confusiones.

