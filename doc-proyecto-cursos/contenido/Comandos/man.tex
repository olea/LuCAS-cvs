\porcion{El comando \comando{man}}
\autor{\LDP}
\colaborador{}
\revisor{\LLC}
\traductor{}

Quiz�s uno de los comandos m�s importantes para cualquier aprendiz (y
a veces no tan aprendiz); el comando \comando{man} sirve para
desplegar en pantalla las \emph{p�ginas de manual}, que proporcionan
ayuda en l�nea acerca de cualquier comando, funci�n de programaci�n,
archivo de configuraci�n, etc.

Hay diferentes tipos de p�ginas de manual, cada tipo se diferencia por
un n�mero, que se detallan a continuaci�n:
\begin{enumerate}
\item Programas ejecutables y guiones\footnote{En ingl�s, scripts, son
    programas creados en el lenguaje del int�rprete de comandos}del
  int�rprete de comandos.
\item Llamadas del sistema (funciones servidas por el n�cleo).
\item Llamadas de la biblioteca (funciones contenidas en las
  bibliotecas del sistema).
\item Archivos especiales (se encuentran generalmente en \archivo{/dev}).
\item Formato de archivos y convenios, por ejemplo \archivo{/etc/passwd}.
\item Juegos.
\item Paquetes de macros y convenios, por ejemplo \comando{man(7)},
  \comando{groff(7)}
\item Comandos de administraci�n del sistema (generalmente s�lo son para
  root).
\item Rutinas del n�cleo.
\end{enumerate}
