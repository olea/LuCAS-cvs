\porcion{El comando \comando{cal}}
\autor{\LDP}
\colaborador{}
\revisor{\LLC}
\traductor{}

Este es un comando bastante �til, que aunque no tenga mucha relaci�n
con los anteriormente dados, sirve para demostrar que las herramientas
basadas en texto no son in�tiles para tareas dom�sticas. \comando{cal}
es una herramienta que sirve para mostrar en pantalla el calendario.
Su sintaxis es la siguiente:

\begin{vscreen}
cal [-jy] [[mes] a�o]
\end{vscreen}

Si \comando{cal} se ejecuta sin argumentos, mostrar� en pantalla el
calendario del mes y a�o actuales, por ejemplo:

\begin{vscreen}
usuario@maquina:~/$ cal
     April 2001
 S  M Tu  W Th  F  S
 1  2  3  4  5  6  7
 8  9 10 11 12 13 14
15 16 17 18 19 20 21
22 23 24 25 26 27 28
29 30
\end{vscreen}
%$

Las opciones de este comando son:

\begin{description}
\item[-j] Muestra la fecha en formato Juliano.
\item[-y] Muestra el calendario completo del a�o actual.
\end{description}
