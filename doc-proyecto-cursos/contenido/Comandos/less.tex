\porcion{El comando \comando{less}}
\autor{\LDP}
\colaborador{\NC}
\revisor{\LLC}
\traductor{}

Este comando es de mucha utilidad; su funci�n es paginar texto en
pantalla. Muchas veces ocurre que cuando se ejecuta alg�n comando, la
salida del mismo aporta demasiada informaci�n como para que se pueda leer
en la pantalla del monitor. Entonces se puede redireccionar esta
salida a \comando{less} para que permita al usuario leer sin mayores
problemas, pudiendo avanzar o retroceder en el texto con las flechas
de cursor del teclado. Tambi�n se utiliza para visualizar archivos de
texto almacenados en disco.

La idea de \comando{less} proviene de un paginador llamado
\comando{more}, un cl�sico en los UNIX. El \comando{more} no era lo
suficientemente amigable,  por eso  hicieron \comando{less}. Su
sintaxis es la siguiente:

\begin{vscreen}
less [archivo...]
\end{vscreen}

El comando \comando{less} es un programa interactivo, es por lo que no
se hablar� de argumentos sino de comandos:

\begin{description}
\item[\boton{ESPACIO}] Si se oprime la barra espaciadora, less avanzar� un
  n�mero de l�neas igual al n�mero de l�neas por pantalla que posea la
  terminal que se est� usando.
\item[\boton{ENTER}] Pulsando la tecla \boton{ENTER} se va avanzando
  de l�nea en l�nea.
\item[\boton{G}] Ir al final del texto.
\item[\boton{g}] Ir al inicio del texto.
\item[\boton{/}] Ingresar una palabra a ser buscada avanzando dentro
  del texto.
\item[\boton{?}] Ingresar una palabra a ser buscada retrocediendo
  dentro del texto.
\item[\boton{n}] Buscar la siguiente ocurrencia de la b�squeda.
\item[\boton{AvP�g}] Avanzar una pantalla de texto.
\item[\boton{ReP�g}] Retroceder una pantalla de texto.
\item[\boton{v}] Cargar el editor de texto en el lugar donde se
  encuentre el usuario dentro del archivo. El editor que normalmente
  se utiliza es el \comando{vi}, el cual se explicar� en la secci�n
  \ref{sec:vi}.
\item[\boton{q}] Salir del programa.
\item[\boton{R}] Repintar la pantalla. �til cuando se est�
  visualizando un archivo que ha sido modificado por otro programa.
\end{description}

\begin{ejemplo}

Para visualizar un archivo de texto llamado \archivo{arch1.txt}, 
se puede utilizar less,
\begin{vscreen}
$ less arch1.txt
\end{vscreen}
%$

Tambi�n se puede utilizar tuber�as y \comando{cat} para realizar lo mismo.
\begin{vscreen}
$ cat arch1.txt | less
\end{vscreen}
%$

\end{ejemplo}
