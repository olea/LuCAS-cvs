\porcion{El comando \comando{touch}}
\autor{\LDP}
\colaborador{\NC}
\revisor{\LLC}
\traductor{}

Este comando se utiliza para cambiar la fecha de acceso y/o
modificaci�n a un archivo. Su sintaxis es la que sigue:

\begin{vscreen}
touch [opci�n...] archivo...
\end{vscreen}

Si el argumento \comando{archivo} corresponde al nombre de un archivo
que no existe, a menos que se le diga, \comando{touch} crear� el
archivo con dicho nombre y sin ning�n contenido. Sus opciones de mayor
importancia son:

\begin{description}
\item[-a] Cambia solamente el tiempo de acceso.
\item[-c] No crear archivos que no exist�an antes.
\item[-d fecha] Usar \comando{fecha} en lugar de la fecha actual. El
  formato de \comando{fecha} es el siguiente: \comando{MMDDHHMMAAAA},
  por ejemplo para representar el 7 de abril de 2002 a la 1:00 a.m.,
  se escribir�: 040701002002. Si el a�o a usar es el a�o actual, se
  puede obviar, entonces el ejemplo anterior quedar�a as�: 04070100.
\end{description}

Este comando es muy �til cuando se necesita recompilar cierta parte de
un programa evitando compilar todo el programa completo, s�lo aquellos
sectores modificados\footnote{De hecho, en la compilaci�n del n�cleo
  se utiliza}.


\begin{ejemplo}

En el caso que no exista en el directorio actual el archivo
\archivo{arch1.txt},
\begin{vscreen}
$ touch arch1.txt
\end{vscreen}
%$
crear� un archivo llamado \archivo{arch1.txt} vac�o (tama\~{n}o 0).
es de suponer que la fecha de creaci�n y modificaci�n ser�n el momento
actual.


\end{ejemplo}
