\porcion{El comando \comando{locate}}
\autor{\LDP}
\colaborador{}
\revisor{\LLC}
\traductor{}

\comando{locate} es un comando de b�squeda de archivos, bastante
parecido al comando \comando{find}. La diferencia de \comando{locate}
es que la b�squeda la hace en una base de datos indexada para aumentar
significativamente la velocidad de respuesta. Esto quiere decir, que
\comando{locate} realmente no busca en el disco del sistema, sino que
en un archivo con la lista de todos los archivos que existen en el
GNU/Linux. Generalmente todas las distribuciones de GNU/Linux ejecutan
a una hora determinada (generalmente cerca de las 4:00am, ya que tarda
alg�n tiempo realizar esta tarea) un comando para actualizar la base
de datos que utiliza
\comando{locate}, dicho comando se llama \comando{updatedb}. Su
sintaxis es:

\begin{vscreen}
locate PATR�N
\end{vscreen}

Donde \comando{PATR�N} corresponde al mismo tipo de patr�n que en el
comando \comando{find}. Ejemplo de ejecuci�n:

\begin{ejemplo}

\begin{vscreen}
usuario@maquina:~/$ locate locate
/usr/bin/locate
/usr/lib/locate
/usr/lib/locate/bigram
/usr/lib/locate/code
/usr/lib/locate/frcode
/usr/share/doc/kde/HTML/en/kcontrol/kcmlocate.docbook.gz
/usr/share/doc/xlibs-dev/XdbeAllocateBackBufferName.3.html
/usr/share/doc/xlibs-dev/XdbeDeallocateBackBufferName.3.html
/usr/share/doc/xlibs-dev/XtAllocateGC.3.html
/usr/share/emacs/20.7/lisp/locate.elc
/usr/share/gnome/help/gsearchtool/C/locate.png
/usr/share/man/man1/locate.1.gz
/usr/share/man/man5/locatedb.5.gz
/usr/X11R6/man/man3/XdbeAllocateBackBufferName.3x.gz
/usr/X11R6/man/man3/XdbeDeallocateBackBufferName.3x.gz
/usr/X11R6/man/man3/XtAllocateGC.3x.gz
/var/lib/locate
/var/lib/locate/locatedb
/var/lib/locate/locatedb.n
\end{vscreen}
%$

Como se puede observar, \comando{locate} ha listado todos aquellos
archivos que posean la palabra <<locate>> en su nombre (los
directorios est�n incluidos).

\end{ejemplo}
