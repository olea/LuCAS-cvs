\porcion{Introducci�n}
\autor{\LDP}
\colaborador{}
\revisor{\LLC}
\traductor{}

Existe un conjunto de comandos que todo usuario debe conocer para
poder manejarse en un sistema GNU/Linux. La mayor�a de estos comandos
est�n relacionados con el manejo de archivos y
directorios\footnote{Que en realidad son un tipo especial de
  archivos}. 

Como se ha dicho anteriormente, estas herramientas tienen
un tiempo de desarrollo y prueba mucho mayores que sus s�miles en
interfaz gr�fica, y aunque no lo parezca, una vez familiarizado con el
entorno de texto, resulta m�s �gil y c�modo para las tareas diarias.

Este grupo de comandos, llamados muchas veces <<Caja de herramientas
UNIX>>, posee detr�s de cada <<herramienta>>, una filosof�a de
desarrollo. Cada uno de los comandos, fue creado con dos ideas en
mente:

\begin{itemize}
\item Debe realizar una sola funci�n.
\item Dicha funci�n debe realizarse correctamente.
\end{itemize}

Con estos objetivos, la simpleza de las herramientas permite que puedan
combinarse para solucionar diferentes problemas que individualmente no
podr�an. La forma de combinar estas herramientas se ver� m�s adelante
en la secci�n \ref{sec:redireccion}.

Es importante hacer notar, que muchas de las tareas que se pueden
hacer con los comandos que se explican en esta secci�n pueden
realizarse con el administrador de archivos \comando{mc}, un clon del
\emph{Norton Commander}, muy bueno por cierto, pero igualmente no se
puede obtener toda la flexibilidad que se tiene con los comandos, como
ya se ver� m�s adelante.
