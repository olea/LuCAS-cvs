\porcion{El comando \comando{cd}}
\autor{\LDP}
\colaborador{\NC}
\revisor{\LLC}
\traductor{}

Este comando se usa para cambiar de directorio. Generalmente cuando el
usuario inicia una sesi�n en GNU/Linux, el directorio donde comienza
es su directorio personal. Desde ah� uno puede moverse a los diferentes
directorios donde se tenga acceso usando este comando. Su sintaxis es
la siguiente:

\begin{vscreen}
cd directorio
\end{vscreen}

�ste es un comando interno del int�rprete (por ejemplo,
\comando{bash}), y no lleva opciones que sean de relevancia como para
nombrarlas.

Con la utilizaci�n del comando \comando{cd} es bueno explicar
conceptos como \textbf{path relativos} y \textbf{path absolutos}.
 
\begin{description}
\item[path reltativo] si el path \emph{no} comienza con \archivo{/} entonces
se considera relativo al path actual. Ej. \archivo{dir} har�
referencia a \archivo{/\emph{directorio actual}/dir}

\item[path absoluto] si el path comiemnza con \archivo{/} entonces
har� referencia a un path en el directorio ra�z. Ej. \archivo{/bin}.
\end{description}

\begin{ejemplo}
Suponiendo que el directorio actual es \archivo{/home/usuario}
\begin{vscreen}
$ cd subdirectorio
\end{vscreen}
%$ emacs fontifica mal
cambiar� a \archivo{/home/usuario/subdirectorio}.

En cambio,
\begin{vscreen}
$ cd /subdirectorio
\end{vscreen}
%$ emacs fontifica mal
cambiar� a \archivo{/subdirectorio}, debido a que es un path
absoluto.

En el caso que \comando{cd} se ejecute sin par�metros, cambiar� al 
directorio personal o \emph{home} del usuario.
\begin{vscreen}
$ cd
\end{vscreen}
%$ emacs fontifica mal
Cambiar� a \archivo{/home/usuario}. Es equivalente a \comando{cd \~{}} o
\comando{cd \$HOME} en \comando{bash}.

Si se quiere cambiar al directorio personal de otro usuario se puede 
ejecutar
\begin{vscreen}
$ cd ~otro
\end{vscreen}
%$ emacs fontifica mal
cambiar� a \archivo{/home/otro}. En cambio \comando{cd ~/otro} cambiar� a
\archivo{/home/usuario/otro} pues \archivo{\~{}} se reemplaza por el directorio
personal del usuario.

El �ltimo directorio se guarda en una variable de entorno y 
se puede f�cilmente intercambiar con el directorio actual con \comando{cd -}
\begin{vscreen}
$ cd /bin
$ cd ~/prueba
$ cd -
\end{vscreen}
%$ emacs fontifica mal

volver� a \archivo{/bin} en el caso de ejecutar otra vez \comando{cd -}
ir� a \archivo{/home/usuario/prueba}

\end{ejemplo}
