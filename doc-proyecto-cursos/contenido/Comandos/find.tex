\porcion{El comando \comando{find}}
\autor{\LDP}
\colaborador{}
\revisor{\LLC}
\traductor{}

Se utiliza este comando para buscar archivos dentro de una jerarqu�a
de directorios. La b�squeda, como veremos m�s adelante, se puede
realizar mediante varios criterios. La sintaxis de este comando es:

\begin{vscreen}
find [camino...] [expresi�n]
\end{vscreen}

La \comando{expresi�n} se conforma de opciones, pruebas y acciones. En
este manual no enumeraremos todas las opciones, pruebas y acciones de
este comando, sino las expresiones que son m�s cotidianas. Dejamos al
alumno que investigue todo el potencial de este comando mediante
la lectura de la p�gina de manual por medio de la ejecuci�n del
siguiente comando:

\begin{vscreen}
man find
\end{vscreen}

Algunos de los criterios de b�squeda que se pueden utilizar son:
\begin{vscreen}
find CAMINO -name ARCHIVO
find CAMINO -name ARCHIVO -perm MODO
\end{vscreen}

\comando{ARCHIVO} corresponde al nombre entero o en parte del archivo
que se est� buscando, \comando{MODO} son los permisos del archivo a
buscar representados en octal. C�mo manejarse con permisos de usuario
se ver� en la secci�n \ref{sec:permisos}.

\begin{ejemplo}

Carlos recuerda haber almacenado en su directorio personal una foto de
su familia cuando estaban de vacaciones, y lo \'unico que recuerda es
que estaba en formato PNG, para intentar localizar dicha foto, usa el
comando \comando{find} de la siguiente forma:

\begin{vscreen}
$ find /home/carlos -name "*.png"
\end{vscreen}

\end{ejemplo}

\begin{ejemplo}

El administrador de un servidor de Internet necesita realizar una
auditor\'ia de seguridad, para ello una de las pruebas que necesita
realizar es identificar aquellos archivos o directorios que poseen
permisos de escritura para cualquier usuario, esto lo puede hacer como
sigue:

\begin{vscreen}
# find / -perm 777
\end{vscreen}

Este comando listar\'a tambi\'en los enlaces simb\'olicos, que aunque
en el listado aparecen con todos los permisos activados, no significa
que cualquier usuario los pueda modificar. Para evitar entonces este
inconveniente, se puede ejecutar el comando de esta manera:

\begin{vscreen}
# find / -perm 777 -follow
\end{vscreen}

La opci\'on \comando{-follow} instruye a \comando{find} para que en
lugar de hacer la prueba con los enlaces simb\'olicos, la haga con los
archivos apuntados por estos enlaces.

\end{ejemplo}

