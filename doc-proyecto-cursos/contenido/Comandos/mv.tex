\porcion{El comando \comando{mv}}
\autor{\LDP}
\colaborador{\NC}
\revisor{\LLC}
\traductor{}

Este comando se usa tanto para mover archivos, como para renombrarlos
(que, al fin de cuentas, es una manera de mover archivos); su sintaxis
es la siguiente:

\begin{vscreen}
mv [opci�n...] origen destino
mv [opci�n...] origen... destino
\end{vscreen}

Si el �ltimo argumento, \comando{destino}, es un directorio existente,
\comando{mv} mueve cada uno de los otros archivos a \comando{destino}.
Algunas opciones de este comando son:

\begin{description}
\item[-f] Borrar los archivos de destino existentes sin preguntar al
  usuario.
\item[-i] Lo contrario de \comando{-f}; pregunta por cada archivo a
  sobreescribir antes de hacerlo.
\item[-v] Muestra el nombre de cada archivo a ser movido.
\end{description}

\begin{ejemplo}

Si en el directorio actual existe 1 archivo llamado
\archivo{arch1.txt}.

\begin{vscreen}
$ mv arch1.txt /usr/doc/
\end{vscreen}
%$
mover� \archivo{arch1.txt} al directorio \archivo{/usr/doc/}
manteniendo el nombre de archivo. En cambio,

\begin{vscreen}
$ mv arch1.txt /usr/doc/archivoNuevo.txt
\end{vscreen}
%$
mover� \archivo{arch1.txt} al directorio \archivo{/usr/doc/} con
el nombre \archivo{archivoNuevo.txt}. 

\end{ejemplo}
