\porcion{El comando \comando{ln}}
\autor{\LDP}
\colaborador{}
\revisor{\LLC}
\traductor{}

Este comando sirve para establecer enlaces entre archivos. Un enlace
puede ser r�gido o simb�lico. El primer tipo es simplemente una forma
de dar otro nombre a un archivo. Por ejemplo teniendo el archivo
\archivo{/etc/passwd}, se puede hacer un enlace y tener el nuevo
nombre en \archivo{/home/usuario/claves}, y ambos nombres de archivos
refiri�ndose al mismo archivo. El segundo tipo es parecido al primero,
pero se pueden enlazar directorios, y adem�s de diferentes sistemas de
archivos. Este tipo de enlace es el que m�s se utiliza. La sintaxis
del comando \comando{ln} es:

\begin{vscreen}
ln [opciones] origen [destino]
ln [opciones] origen... directorio
\end{vscreen}

Sus opciones m�s importantes son las siguientes:

\begin{description}
\item[-d] Permite al \emph{super-usuario} hacer enlaces r�gidos a
  directorios.
\item[-s] Crear enlace simb�lico.
\item[-f] Borrar los archivos de destino que ya existen.
\end{description}

\begin{ejemplo}
Para el caso del ejemplo anterior, se deber�a ejecutar:

\begin{vscreen}
ln -s /etc/passwd /home/usuario/claves
\end{vscreen}

Cuando se ejecuta \comando{ls -l} en un directorio donde hay un enlace
simb�lico, �ste se muestra de la siguiente manera:

\begin{vscreen}
usuario@maquina:~/$ ls -l claves
lrwxrwxrwx    1 usuario usuario 11 Apr  8 13:33 claves -> /etc/passwd
\end{vscreen}
%$

La <<l>> al comienzo de la l�nea especifica el tipo de archivo
listado, en este caso, un \emph{link}.

\end{ejemplo}

\begin{ejemplo}

Sup\'ongase que el administrador de un sistema GNU/Linux necesita
instalar todos los directorios personales de sus usuarios en un disco
aparte al del sistema, y que este disco se encuentra \emph{montado} en
el directorio \comando{/mnt/disco2}. Luego de crear el directorio
\comando{/mnt/disco2/home}, se lo deber\'a \emph{enlazar} a
\comando{/home} de la siguiente manera:

\begin{vscreen}
# ln -s /mnt/disco2/home /home
\end{vscreen}

Tener en cuenta que el directorio \comando{/home} no debe existir
antes de ejecutar el comando.

\end{ejemplo}
