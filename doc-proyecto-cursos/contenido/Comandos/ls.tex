\porcion{El comando \comando{ls}}
\autor{\LDP}
\colaborador{}
\revisor{\LLC}
\traductor{}

Quiz�s uno de los comandos de mayor utilizaci�n, sirve para listar archivos.
Su sintaxis es:

\begin{vscreen}
ls [opciones] [archivo...]
\end{vscreen}

Si se ejecuta \comando{ls} sin argumentos, dar� como resultado un
listado de todos los archivos (incluyendo directorios) del directorio
donde el usuario est� posicionado. Sus opciones son:

\begin{description}
\item[-a] Lista todos los archivos, incluyendo aquellos que comienzan
  con un <<.>>\footnote{Como convenci\'on, los archivos cuyos nombres
  comienzan con un punto se les denomina <<ocultos>>.}.
\item[-d] Lista el nombre del directorio en vez de los archivos
  contenidos en �l.
\item[-l] Lista los archivos con mucho m�s detalle, especificando para
  cada archivo sus permisos, el n�mero de enlaces r�gidos, el nombre
  del propietario, el grupo al que pertenece, el tama�o en bytes, y la
  fecha de la �ltima modificaci�n.
\item[-r] Invierte el orden de listado de los archivos.
\item[-s] Muestra el tama�o de cada archivo en bloques de 1024 bytes a
  la izquierda del nombre.
\item[-h] Muestra los tama\~{n}os de archivo en t\'erminos de kilobytes,
  megabytes, etc.
\item[-t] Lista los archivos ordenados por el tiempo de modificaci�n
  en vez de ordenarlos alfab�ticamente.
\item[-A] Lista todos los archivos excepto el <<.>> y el <<..>>.
\item[-R] Lista los contenidos de todos los directorios
  recursivamente.
\item[-S] Ordena el listado por el tama�o de los archivos.
\item[--color=[cu�ndo] Especifica si emplear color para distinguir
  los diferentes tipos de archivos. El argumento \comando{cu�ndo}
  puede tener varios valores:
  \begin{description}
  \item[none] No usar colores. Esta opci�n es la predeterminada.
  \item[auto] Usar colores solamente cuando la salida est�ndar es una
    terminal.
  \item[always] Usar siempre colores. Si \comando{ls} se usa con la
    opci�n \comando{--color} sin especificar la opci�n de color, el
    resultado es el mismo que cuando se usa \comando{--color=always}.
  \end{description}
\end{description}

\begin{ejemplo}
Sup\'onganse que se tiene un directorio \comando{/usr/local/papers} donde
se alojan los documentos de un grupo de redactores de una revista, y se
quiere saber cu�les fueron los \'ultimos documentos modificados y su
tama\~{n}o para su inclusi\'on en el pr\'oximo n\'umero. Para esto se puede
ejecutar el comando \comando{ls} de la siguiente forma:

\begin{vscreen}
$ ls -lhtr /usr/local/papers
\end{vscreen}

Como se puede observar, se le pasan 4 opciones al comando \comando{ls}, de
tal manera que muestre un listado extendido, ordenado por tiempos de 
modificaci\'on de forma ascendente y que adem\'as muestre los tama\~{n}os 
de archivo en forma m�s legible. Otra manera de ejecutar el mismo comando
es la siguiente:

\begin{vscreen}
$ ls -l -h -t -r /usr/local/papers
\end{vscreen}
\end{ejemplo}
