\porcion{El comando \comando{du}}
\autor{\LDP}
\colaborador{}
\revisor{\LLC}
\traductor{}

Este comando informa al usuario de la cantidad de almacenamiento
utilizado por los archivos especificados, posee varias opciones, su
sintaxis es la siguiente:

\begin{vscreen}
du [opciones] [archivo...]
\end{vscreen}

Sus opciones m\'as significativas son:

\begin{description}
\item[-s] Muestra \'unicamente los tama\~{n}os de los archivos 
  especificados en la l\'inea de comandos. 
\item[-h] Muestra los tama\~{n}os de archivo en un formato m\'as legible.
\item[-c] Muestra en pantalla el espacio total ocupado por los archivos
  especificados.
\item[-x] Omite en el conteo aquellos directorios que pertenezcan a 
  otro sistema de archivos.
\end{description}

\begin{ejemplo}

El administrador de un servidor necesita saber el espacio en disco
ocupado por los distintos directorios del sistema, para hacer
limpieza. Para esto, se tiene en cuenta que no se deber\'an contar
aquellos directorios que est\'en en su propio sistema de archivos,
entonces se puede ejecutar de esta forma:

\begin{vscreen}
# du -sxh /*
\end{vscreen}

Con esto se obtiene una lista similar a esta:

\begin{vscreen}
6.8M	/bin
6.7M	/boot
351k	/dev
34M	/etc
2.8G	/home
43M	/lib
3.0k	/mnt
0	/proc
15M	/root
7.3M	/sbin
512	/swap
281k	/tmp
5.3G	/usr
758M	/var
\end{vscreen}


\end{ejemplo}
