%%%%%%%%%%%%%%%%%%%%%%%%%
% Porci�n: Introducci�n %
%%%%%%%%%%%%%%%%%%%%%%%%%
\porcion{Introducci�n}

\autor{\NC}
\colaborador{}
\revisor{}
\traductor{}

GNU/Linux no necesita un entorno de ventanas para
funcionar. Ciertamente, cuando comenz� no exist�a dicho entorno. La
pantalla era un int�rprete de comandos de aspecto similar a los Unix o
una ventana de MS-DOS.  Las aplicaciones que funcionan bajo terminales
o consolas las llamaremos \emph{aplicaciones de texto}.

Tiempo despu�s se port� un sistema de ventanas llamado X/Window, muy
popular en el mundo Unix.

Es un sistema de control de mouse y pantalla, pero no maneja las
ventanas y operaciones con ventanas (como mover, minimizar, cerrar,
etc.). Por lo tanto hay que utilizar alg�n programa administrador de
ventanas. Se eligi� para el curso es el KDE. Existen muchos
otros entre los cuales est� el GNOME, CDE, WindowMaker y AfterStep.

Las aplicaciones que funcionan bajo X/Window las llamaremos
\emph{aplicaciones gr�ficas}. Necesitan X/Window para funcionar pero no
necesitan un administrador de ventanas. Sin embargo el administrador
de ventanas facilita el uso de los programas.
