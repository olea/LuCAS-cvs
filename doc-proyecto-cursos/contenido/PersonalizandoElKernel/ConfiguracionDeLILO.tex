\porcion{Configurando LILO}
\autor{\LDP}
\colaborador{\NC}
\revisor{}
\traductor{}

Toda la configuraci�n de LILO se encuentra en \archivo{/etc/lilo.conf}
el contenido es similar a algo as�: 

\begin{vscreen} 
boot=/dev/hda
install=/boot/boot.b
default=linux
prompt
timeout=5
message=/boot/message
image=/boot/vmlinuz
        label=linux
        root=/dev/hda6
        append=" hdc=ide-scsi ide1=autotune ide0=autotune"
other=/dev/hda2
        label=windows
        table=/dev/hda
\end{vscreen}

Por ahora esto puede parecer inentendible pero vamos a analizar linea a linea:
\begin{vscreen}
boot=/dev/hda
\end{vscreen}

Significa que el dispositivo de arranque es
\archivo{/dev/hda}\footnote{\archivo{/dev/hda} el disco maestro de la
  controladora IDE primaria}. El \emph{sector de arranque} o
\emph{boot sector} de ese dispositivo contendr� a LILO cuando inicie
el equipo.

\begin{vscreen}
install=/boot/boot.b
\end{vscreen}

\archivo{/boot/boot.b} es un archivo usado como nuevo sector de
arranque.

\begin{vscreen}
default=linux
prompt
timeout=5
\end{vscreen}

Con estas tres opciones se especifica que:
\begin{description}
\item[prompt] Pregunte que n�cleo hay que utilizar (el caso contrario
  puede ser que haya uno s�lo y no se quiera elegir).
\item[default] En caso de no poner nada se elija ``linux''.
\item[timeout] tiempo en segundos a esperar si no se elije nada.
\end{description}

\begin{vscreen}
message=/boot/message
\end{vscreen}

Se muestra un mensaje que es el archivo \archivo{/boot/message} que puede
contener algo como:

\begin{vscreen}
Bienvenido a LILO, el selector de SO de arranque!

Elija un sistema operativo de la lista.
O espere 5 segundos para que arranque el sistema predeterminado.
\end{vscreen}

Luego vienen las configuraciones de los n�cleos en si. En el ejemplo
existen 2 n�cleos, uno de linux y el otro es un \emph{Windows}.

Las dos configuraciones son distintas pero tienen una linea en com�n.
Esta es {\tt label}. {\tt label} es el identificador de n�cleo para
LILO, es de suponer que tiene que ser �nico. Puedo tener varios
n�cleos de linux pero no con {\tt label=linux} en mas de uno de ellos.
Simplemente habr� que asignarlos de distinta manera como por ejemplo
{\tt label=linux-2.2.19} y {\tt label=linux-2.4.3}.

Si por un momento repasamos este concepto, nos vamos a dar cuenta que
{\tt default=linux} hace referencia al n�cleo que posee {\tt label=linux}.
Cuando cambiemos de configuraci�n a {\tt label=linux-nuevo} 
recordemos cambiar {\tt default} tambi�n.

Para correr un sistema no s�lo necesitamos el n�cleo, sino tambi�n
archivos, que componen el �rbol de directorios que surge de la
\emph{ra�z} o \emph{root}.

Por eso,
\begin{vscreen}
root=/dev/hda6
\end{vscreen}

especifica que se va a usar la sexta partici�n del disco como
\emph{directorio ra�z} o simplemente \emph{ra�z}, es decir que, todo lo
que est� en esa partici�n va a pasar a ser el directorio \archivo{/}
donde estar�n \archivo{/bin}, \archivo{/etc}, \archivo{/home},
\archivo{/usr}, \archivo{/lib}, etc.

En este concepto independizamos el n�cleo de los archivos que maneja.
Una vez que est� el n�cleo corriendo, los archivos se pueden obtener
de diferentes lugares. Por ejemplo particiones, otros discos, discos
flexibles, hasta un dispositivo que se encuentra a trav�s de una
red\footnote{un hipot�tico \archivo{/dev/red} o com�nmente denominado
  \archivo{/dev/nfsroot}}.  Casi cualquier archivo/dispositivo puede
ser {\tt root} siempre que este \emph{formateado} correctamente.

Por ultimo la linea:

\begin{vscreen}
append=" hdc=ide-scsi ide1=autotune ide0=autotune"
\end{vscreen}

pasa par�metros al n�cleo para ajustar configuraciones, las cuales
depender�n de cada sistema.
