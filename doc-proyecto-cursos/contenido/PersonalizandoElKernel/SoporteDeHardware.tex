\porcion{Soporte de hardware}
\autor{\LDP}
\colaborador{\NC}
\revisor{}
\traductor{}

En un principio, el hardware que el n�cleo Linux soportaba se deb�a
exclusivamente al arduo trabajo de unos cuantos hackers del n�cleo,
que pasaban incontables horas de trabajo intentando descubrir como una
pieza de hardware funcionaba (realizando lo que se llama
\emph{ingenier�a inversa}), para luego escribir un manejador para que
Linux pudiera utilizar ese dispositivo. La mayor�a de las empresas
fabricantes de dispositivos no entregaba la informaci�n necesaria a
los programadores del n�cleo, y el soporte para los nuevos
dispositivos muchas veces se tardaba un tiempo.

Hoy en d�a, con el aumento de colaboradores en el desarrollo del
n�cleo, y con la creciente cantidad de empresas que se han dado cuenta
que GNU/Linux vale la pena, el soporte para nuevos dispositivos no se
hace esperar demasiado, y es por eso que el n�cleo Linux soporta:

\begin{itemize}
\item Tarjetas de v�deo VGA, SVGA, Monocromo, etc.
\item Controladores de discos IDE, EIDE, MFM, RLL, RAID, SCSI.
\item Controladores de puertos seriales.
\item Tarjetas multipuerto.
\item Adaptadores de red Ethernet, ISDN, Frame Relay, Inal�mbricas,
  X25, SLIP, PPP, ARCnet, TokenRing, FDDI, AX.25, ATM.
\item Tarjetas de sonido.
\item Unidades de cinta.
\item Unidades de CD-ROM.
\item Unidades grabadoras de CD-R.
\item Unidades removibles, como por ejemplo Zip, Jaz, Bernoulli y
  tantas otras.
\item Mouse serie, PS/2 y otros.
\item M�dems normales, y tambi�n algunos \emph{winm�dems}.
\item Impresoras matriciales, de inyecci�n de tinta, y l�ser.
\item Plotters.
\item C�maras digitales.
\item Capturadoras de v�deo.
\item Unidades DVD-ROM.
\item Puertos y dispositivos USB.
\end{itemize}

La lista es demasiado grande como para detallarla en este curso, pero
para tener una idea mucho mas detallada, basta con leer el
<<Hardware-HOWTO>>, normalmente localizado en
\comando{/usr/doc/HOWTO}.

Tambi�n debemos tener en cuenta que no todo el soporte del hardware
listado es trabajo exclusivo del n�cleo Linux. Como ejemplo tenemos el
caso de las impresoras, en el n�cleo no se necesita definir
expl�citamente que clase de impresora se tiene conectada al equipo,
solamente se necesita activar el soporte para el puerto paralelo
(siempre y cuando se utilice una impresora de puerto paralelo
obviamente), y el resto del trabajo lo har� un programa a trav�s del
n�cleo; en el caso del ejemplo, el programa que se encarga de enviar
datos a la impresora se llama \comando{lpr}.

