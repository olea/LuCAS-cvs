\porcion{Mostrando los m�dulos cargados}
\autor{\LDP}
\colaborador{\NC}
\revisor{}
\traductor{}

Una vez que tenemos varios m�dulos en el directorio
\archivo{/lib/modules/(versi�n del N�cleo)/}\footnote{Este curso 
  muestra como generarlos, pero, la mayor�a de las distribuciones ya
  vienen casi todos los m�dulos compilados}
podemos listar aquellos que est�n siendo usados.

El comando \comando{lsmod} muestra los m�dulos usados. Una salida podr�a ser:

\begin{vscreen}
root@maquina:/root# lsmod
Module                  Size  Used by
loop                    9600   2  (autoclean)
lockd                  32208   1  (autoclean)
sunrpc                 54640   1  (autoclean) [lockd]
autofs                  9456   2  (autoclean)
8139too                12064   1  (autoclean)
via82cxxx_audio         9024   0 
soundcore               2800   2  [via82cxxx_audio]
ac97_codec              7088   0  [via82cxxx_audio]
ip_masq_vdolive         1440   0  (unused)
ip_masq_cuseeme         1184   0  (unused)
ip_masq_quake           1456   0  (unused)
ip_masq_irc             1664   0  (unused)
ip_masq_raudio          3072   0  (unused)
ip_masq_ftp             4032   0  (unused)
nls_cp437               3952   5  (autoclean)
vfat                    9408   2  (autoclean)
fat                    30432   2  (autoclean) [vfat]
supermount             14224   3  (autoclean)
ide-scsi                7664   0 
reiserfs              128592   2 
root@maquina:/root# 
\end{vscreen}

Como tambi�n podr�a estar vac�a. Si es que ning�n m�dulo se carg� o si
el n�cleo es monol�tico.

Tomemos el caso del m�dulo {\tt soundcore}:
\begin{vscreen}
soundcore               2800   2  [via82cxxx_audio]
\end{vscreen}

El tama�o en memoria del m�dulo es de 2800 bytes. Y el m�dulo {\tt
  via82cxxx\_audio} lo est� usando. Esto quiere decir que para sacar de
memoria a {\tt soundcore} primero hay que sacar a {\tt
  via82cxxx\_audio}. Y viceversa, si necesitamos agregar {\tt
  via82cxxx\_audio} primero tendremos que agregar {\tt soundcore}.

Podemos darnos cuenta de que existe un �rbol de dependencias entre
m�dulos. Y en alg�n lugar debe estar. Bueno as� es, es el archivo
\archivo{/lib/modules/(versi�n de n�cleo)/modules.dep} y es generado
en la compilaci�n.

