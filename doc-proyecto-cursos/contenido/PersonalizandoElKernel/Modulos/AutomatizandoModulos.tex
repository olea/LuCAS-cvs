\porcion{Automatizando un poco mas los m�dulos}
\autor{\LDP}
\colaborador{\NC}
\revisor{}
\traductor{}

\emph{Configuraciones en \archivo{/etc/modules.conf}}
Como estamos operando frente a una m�quina trataremos de automatizar
lo m�s posible las tareas rutinarias. Se supone que los cambios de
configuraci�n en el hardware se hacen infrecuentemente. Entonces los 
par�metros en la carga de m�dulos es siempre la misma.

La gran mayor�a de los m�dulos auto detectan su configuraci�n, pero en
ciertas ocasiones hay que parametrizarla.  Una alternativa poco
elegante ser�a crear un script que cargue al m�dulo con los par�metros
correspondientes. Pero se vuelve engorroso tener un script por m�dulo.

En reemplazo a eso, los comandos
\comando{insmod} y \comando{rmmod} utilizan una archivo de configuraci�n:
\archivo{/etc/modules.conf}.

Este archivo puede contener una l�nea del estilo
\begin{vscreen}
option nombre-modulo opt-1 [opt-2 .. opt-n] 
\end{vscreen}

Donde {\tt opt-1}, {\tt opt-2}, etc. son las opciones o par�metros del
m�dulo. 

Una configuraci�n interesante es la creaci�n de sobrenombre o
\emph{alias} a los m�dulos. Sirve para no tener que recordar nombres
como {\tt via82cxxx\_audio} y en reemplazo usar {\tt placa\_sonido} por
ejemplo. Su sintaxis es:

\begin{vscreen}
alias sobre-nombre nombre-modulo
\end{vscreen}

Algunos scripts de configuraci�n del sistema tienen palabras definidas
para cargar los m�dulos correspondientes y que el usuario edite
\archivo{/etc/modules.conf} y asigne el modulo.

Un ejemplo cl�sico ser�a:

\begin{vscreen}
alias sound sb
alias eth0 ne2k-pci
\end{vscreen}

que asigna el seud�nimo {\tt sound} a un m�dulo de \emph{Sound Blaster} y
{\tt eth0}\footnote{Corresponde a la primera placa de red Ethernet} al
m�dulo {\tt ne2k-pci} para la placa de red.

\emph{Auto cargado de m�dulos}
A medida que se utilizaban los m�dulos, era conveniente tener una
utilidad que cargue el o los m�dulos necesarios para hacer una tarea.
Esta utilidad est� contenida en el n�cleo, se llama {\tt kmod} y se
configura en

\emph{Loadable Module Suport} $\rightarrow$ 
\emph{Enable Loadable Module Support} $\rightarrow$ 
\emph{Kernel Module Loader}.

Con s�lo incluir eso, pr�cticamente no hay que cargar <<a mano>> ni
ning�n m�dulo.
