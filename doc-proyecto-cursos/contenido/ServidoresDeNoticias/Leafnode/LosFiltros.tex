\porcion{Los filtros}
\autor{Carlos P�rez P�rez}
\colaborador{\SGG}
\revisor{}
\traductor{}

Aqu� es d�nde apunta la directiva \comando{filterfile} del archivo
de configuraci�n. En este archivo se colocan todas aquellas
reglas destinadas al filtrado de mensajes que no queramos
recibir. Dentro de Usenet hay una pr�ctica que se va
extendiendo y que es perjudicial, el denominado Spam,
con el filtrado de las direcciones desde las que se
remiten los mensajes no solicitados se consigue que los
mensajes que leamos se ajusten a la materia del grupo
de noticias al que estemos apuntados. Otra manera de
controlar estos mensajes es con la directiva \comando{maxcrosspost}
del archivo de configuraci�n que evita leer mensajes que vayan a
m�s de una determinada cantidad de grupos a la vez, el denominado
crosspoting, y que es muy utilizado por los spammers.\\

Las expresiones para el fichero de filtros siguen la misma estructura
que las expresiones regulares de Perl.\\
Con la siguiente l�nea filtramos cualquier noticia que tenga como
remitente de correo \comando{todosexo@sex.com}:

\begin{vscreen}
^(i?:from):.*[< ]todosexo@sexo.com(>|$| )
\end{vscreen}

Ejemplo de filtro para el apartado Asunto (Subject en ingl�s):
\begin{vscreen}
^Subject.*\[Cursos-linux\]*.
\end{vscreen}
Con esta regla filtramos todas las noticias que tengan en
el asunto [Cursos-linux].

De esta forma podremos evitar bajar noticias que contengan unas
determinadas palabras o unos remitentes molestos.
En \sitio{http://www.escomposlinux.org/spam/} se pueden encontrar
consejos para filtrar el spam y filtros constantemente actualizados
que filtran los spammers dentro de la jerarqu�a es.comp.os.linux.*.