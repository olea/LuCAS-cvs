\porcion{Obtenci�n e Instalaci�n}
\autor{Carlos P�rez P�rez}
\colaborador{\SGG}
\revisor{}
\traductor{}

Se puede obtener Leafnode desde la p�gina
oficial del programa \sitio{http://www.leafnode.org}.
Si se quieren bajar paquetes en formato rpm se
puede acudir a \sitio{http://rpmfind.net} para el
formato Debian se deber� acudir a \sitio{http://www.debian.org}.
Aunque normalmente el paquete de Leafnode viene con
la mayor�a de las distribuciones.\\

\figura{P�gina de Leafnode}{leafnode.org}

Su instalaci�n es sencilla:\\
\begin{itemize}
\item Paquetes RPM: desde la l�nea de comandos ejecutaremos
la orden \comando{rpm -i leafnode-1.9.17-1.rpm}, teniendo
en cuenta que la versi�n puede ser distinta, en este caso, 1.9.17-1.
\item Paquetes Deb: desde la l�nea de comandos ejecutamos
\comando{apt-get install leafnode} y con esto nos aseguramos
que se instalan todas las dependencias necesarias.
\item C�digo fuente: procederemos a desempaquetar el archivo
con las fuentes con la orden \comando{tar xvzf archivo} y
buscaremos el fichero README para ver la informaci�n de
compilaci�n.
\end{itemize}
Una vez realizada la instalaci�n, habr� que verificar la existencia del usuario news.
Dentro del fichero /etc/inetd.conf debe existir:\\
\comando{nntp stream tcp nowait news /usr/sbin/tcpd /usr/local/sbin/leafnode}

Estas comprobaciones no son necesarias porque el sistema de instalaci�n las
trae automatizadas, pero servir�n para descartar problemas si el servicio
no funciona correctamente.


Si estamos utilizando xinet en lugar de inet, deberemos crear un fichero llamado
\comando{leafnode} dentreo del directorio \comando{/etc/xinetd.d}, con el siguiente
contenido:

\begin{vscreen}
service nntp
{
	disable	= no
	socket_type		= stream
	wait			= no
	user			= root
	server			= /usr/sbin/leafnode
	log_on_success		+= USERID
	log_on_failure		+= USERID
}
\end{vscreen}


El siguiente paso l�gico es editar el fichero de configuraci�n con el
objeto de adpatar el servicio a nustras necesidades e indicarle a
Leafnode una informaci�n vital para su correcto funcionamiento.

El �ltimo paso ser�a automatizar los comandos para bajar los grupos
y las noticias. Para esto se puede poner el \comando{fetchnews} dentro
del script ip-up para bajarlos cuando hagamos la conexi�n a Internet
o bien a�adiendo una l�nea dentro del cron. Aunque siempre nos queda
el m�todo manual, esto es, ejecutarlo desde la l�nea de comandos.

Reiniciamos el demonio inetd o xinetd:\\
\comando{kill -HUP `cat /var/run/inetd.pid`} \\
\comando{kill -HUP `cat /var/run/xinetd.pid`} \\
o bien:\\
\comando{/etc/init.d/xinetd reload}

En el fichero /etc/hosts.deny pondremos:\\
\comando{leafnode: ALL EXCEPT LOCAL}\\
Ejecutamos \comando{fetchnews} despu�s de haber hecho los
cambios en el fichero de configuraci�n, con lo que se
bajar�n la lista de grupos disponibles. Luego nos conectaremos
desde nuestro cliente favorito, teniendo en cuenta que el
servidor ser� local y no se necesitar� usuario ni contrase�a.
Ejecutamos \comando{fetchnews} por segunda vez y comenzar�n
a bajar las noticias de los grupos suscritos hasta un m�ximo
que ha sido definido en el ficher de configuraci�n.