\porcion{Competencia de GNU/Linux vs Windows}
\autor{}
\colaborador{}
\revisor{\LLC}
\traductor{}

Largas discusiones se han realizado sobre el tema <<GNU/Linux vs
Windows>>, y tiempo atr�s quiz�s no ten�a sentido compararlos por la
escasa cantidad de aplicaciones \emph{aptas para el usuario}
disponible en GNU/Linux; pero con el correr de los a\~nos el sistema
fue evolucionando cada vez m�s para dar soporte al usuario y brindarle
una plataforma estable y funcional para el trabajo de todos los d�as.

Originalmente GNU/Linux (como todo sistema de tipo Unix), hab�a
sido pensado para funcionar como servidor, y es por eso que las
herramientas que prove�a eran algo avanzadas y de uso espec�fico para
la administraci�n del sistema; pero en la actualidad existe una gran
variedad de aplicaciones que permiten al usuario realizar todas sus
tareas cotidianas: procesadores de texto, planillas de c�lculo,
programas de correo electr�nico, juegos, reproductores de CDs,
generadores de presentaciones, retocadores de im�genes, etc., lo que
hace al sistema GNU/Linux especial para la oficina o el hogar.

Un punto m�s a favor de GNU/Linux es que, dado que fue pensado para
funcionar como servidor, impl�citamente posee una robustez y
estabilidad importantes, lo que combinado con la amigabilidad y
funcionalidad del sistema de escritorio, genera un sistema sumamente
f�cil y seguro de utilizar.
