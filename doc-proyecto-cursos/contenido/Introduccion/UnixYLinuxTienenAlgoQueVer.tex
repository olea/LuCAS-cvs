\porcion{UNIX y GNU/Linux... ?'tienen algo que ver?}
\autor{}
\colaborador{}
\revisor{\LLC}
\traductor{}

GNU/Linux es una reimplementaci�n de la especifiaci�n <<POSIX>> con
extensiones de SysV y BSD, lo que significa que parece Unix pero no
proviene del mismo c�digo fuente base.

GNU/Linux es un sistema operativo gratuito y de libre distribuci�n
bajo las condiciones que establece la licencia GPL (\emph{GNU Public
License}). Tiene todas las caracter�sticas que uno puede esperar de un
sistema Unix moderno: multitarea real, memoria virtual, bibliotecas
compartidas, carga por demanda, soporte de redes TCP/IP, entre muchas
otras funcionalidades.

GNU/Linux funciona mayormente en computadoras PC, pero se ha portado a
otras plataformas tambi�n, como son Alpha, Macintosh, Sun y Silicon
Graphics.
