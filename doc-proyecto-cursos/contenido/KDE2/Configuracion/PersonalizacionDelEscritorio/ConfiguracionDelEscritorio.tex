\porcion{Configuraci�n del escritorio}
\label{sec:escritorio}

\autor{\JAB}
\colaborador{\LDP}
\revisor{}
\traductor{}


\figura{Seccion de configuracion del escritorio}{fig:CentroDeControlEscritorio-Configuracion}
{width=10cm}
{KDE2/Configuracion/PersonalizacionDelEscritorio/CentroDeControlEscritorio-Configuracion}


En la secci�n \emph{LookNFeel}...\emph{Escritorio} del \emph{Centro de
Centro KDE} ( figura \ref{fig:CentroDeControlEscritorio-Configuracion} )
podemos configurar espec�ficamente la funcionalidad y apariencia de
nuestro escritorio. Primeramente se encuentran las opciones de los
iconos que est�n en el escritorio, cabe aclarar que estos no son mas
que archivos que est�n en un directorio prefijado representando,
enlaces a aplicaciones, enlaces a carpetas o un archivo com�n ( un
documento, una imagen, una aplicaci�n, etc. ) y que poseen una
representaci�n gr�fica en nuestro escritorio. La opci�n de selecci�n
\emph{Mostrar archivos ocultos en el escritorio} nos permite
visualizar los iconos que representan archivos ocultos. Tambi�n
podemos habilitar un men� que aparece en la parte superior de la
pantalla con opciones y enlaces comunes del escritorio. En el cuadro
\emph{Mostrar previsualizaciones} le decimos a KDE que para
determinados tipos de archivos, en vez de <<dibujar>> el icono
predeterminado, dibuje ( en miniatura ) el contenido del archivo.

Tambi�n podemos configurar que men� se debe mostrar cuando se presiona
alguno de los botones del mouse en el escritorio, estos men�s pueden
ser:
\begin{description}
	\item[Men� de aplicaciones]: men� principal del panel KDE 
	\item[Men� de lista de ventanas]: ventanas abiertas y opciones de organizaci�n.
	\item[Men� del escritorio]: opciones del escritorio.
	\item[Ninguna acci�n].
\end{description}

Adem�s podemos configurar en que directorio est�n los archivos que se
mostraran como iconos en el escritorio, la ubicaci�n de la papelera,
las aplicaciones que se ejecutaran cuando inicie KDE, y un directorio
com�n para documentos.

La pagina \emph{Apariencia} nos permite personalizar las opciones de
texto de nuestro escritorio, pudiende configurar:
\begin{itemize}
	\item{Tama�o de la fuente}
	\item{Tipografia a utilizar}
	\item{Color de la fuente}
	\item{Color de fondo del texto}
	\item{Determinar el subrayado del texto}
\end{itemize}


Antes de seguir adelante, es conveniente aclarar el concepto de
<<m�ltiples escritorios>>, esta es una caracter�stica que proveen la
mayor�a de los entornos de escritorio, en una misma sesi�n de X tener
varios escritorios, esto posibilita trabajar con m�s comodidad. KDE
por defecto tiene configurados 4 escritorios, en la barra de
herramientas se pueden observar los botones llamados \boton{Uno}
\boton{Dos} \boton{Tres} y \boton{Cuatro}, los que se utilizan para
acceder a dichos escritorios. Tambi�n existe una forma de cambiar de
escritorios de una manera m�s directa: usando la combinaci�n de teclas
\comando{Ctrl-F1} para el primero, \comando{Ctrl-F2} para el segundo,
etc. Estos escritorios m�ltiples le proveen al usuario m�s espacio
para colocar sus aplicaciones, adem�s de una mayor organizaci�n de las
ventanas debido a que no es necesario tenerlas a todas en un mismo
escritorio.  En la pagina \emph{Numeros de escritorios} ademas de
determinar la cantidad de escritorios, hasta 16, podemos definir un
nombre para cada uno de estos.
