\porcion{Configuraci�n del tipo de teclado}

\autor{\LDP}
\colaborador{\JAB}
\revisor{}
\traductor{}

Otra de las funciones m�s importantes es la personalizaci�n del tipo
de teclado que poseemos en nuestras computadoras, si ovbservamos que
nuestro teclado no funciona correctamente, o sea no se corresponde lo
que uno teclea con lo que sale en pantalla, entonces es necesario
configurar el teclado, para esto cargamos el \emph{Centro de Control
KDE}, entrar en la categor�a \emph{Personalizaci�n} y seleccionar la
opci�n \emph{Distribuci�n del teclado} como en la figura
\ref{fig:ConfiguracionDelTeclado}

\figura{Selecci�n del mapa de caracteres del teclado}{fig:ConfiguracionDelTeclado}
{width=6cm}{KDE2/Configuracion/ConfiguracionDelTeclado}

Como se podr� ver, en la p�gina \emph{Disposici�n}, tenemos dos listas
desplegables donde podemos seleccionar el modelo de teclado, (si
nuestra maquina es un clon, seguramente elegiremos algun tipo
``generico'' de teclado), y el dise�o de este, o sea la distribuci�n
que tiene el teclado de acuerdo a un determinado idioma. Luego se
pueden elegir dise�os secundarios, los cuales pueden ser usados en
cualquier momento, utilizando las teclas rapidas configuradas para
realizar estos cambios (p�gina \emph{Tecla r�pida}).

Con \boton{Aplicar} los cambios tomar�n efecto.