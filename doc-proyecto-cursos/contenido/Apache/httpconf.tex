\porcion{El archivo \archivo{httpd.conf}}
\autor{\NC}
\colaborador{}

El archivo \archivo{httpd.conf} configura a el servidor Apache, e
incluye otros archivos de configuraci�n espec�ficos y m�s peque�os. El
archivo esta ampiamente documentado\footnote{Como muchos archivos de
configuraci�n cualquier l�nea que comience con ``#'' se ignora por lo
tanto es considerada comentario.}, cada directiva tiene una descripci�n
concreta de como usarlo, por lo tanto haremos una descripci�n breve de
las  directivas  m�s importantes.

\begin{description}
\item[User]         Define el usuario que ejecuta al proceso apache.

\item[Group]        Define el grupo que ejecuta al proceso apache.

\item[ServerName]   Nombre del servidor, si se omite, se asigna el nombre del equipo.

\item[ServerAdmin]  Email del administrador para el envio de errores.

\item[DocumentRoot] Directorio donde se encuentran los documentos a enviar,
                    debe tener permiso el \emph{User} y \emph{Group}
                    mencionados anteriormente.
\end{description}

Otro tipo de directivas son los \emph{bloques} que su formato es
similar a los tags HTML/XML, comienzan con {\tt <Directiva Opcion>} y
finalizan con {\tt </Directiva>}

\begin{description}
\item[Directory] Opcion: nombre de directorio. Define propiedades de 
		un directorio a utilizar en el servidor.

\item[DirectoryMatch] Opcion: expresi�n regular. Igual que \emph{Directory}, 
		pero acepta una expresi�n regular para generalizar 
		los directorios.

\item[Files] 	Opcion: nombre de archivo. Define propiedades para un archivo
		determinado.

\item[FilesMatch] Opcion: expresi�n regular. Igual que \emph{Files}, pero
		acepta una expresi�n regular para generalizar los archivos.

\item[Location] Opcion: URL. Especifica las propiedades de una URL del sitio.

\item[LocationMatch] Opcion: expresi�n regular. Como es de esperar, es igual
		que \emph{Location} pero se pueden especificar 
		varias URL's con expresiones regulares

\item[VirtualHost] Opcion: nombre de host. Especifica propiedades de un s�lo
		dominio. Veremos en mayor detalle esta directiva.

\end{description}

\begin{ejemplo}

Un archivo \archivo{httpd.conf} simple:

\begin{vscreen}
User apache
Group apache
ServerName www.dominio.org.ar
ServerAdmin webmaster@dominio.org.ar
DocumentRoot /var/www/

<Directory />
</Directory>

\end{vscreen}

\end{ejemplo}

