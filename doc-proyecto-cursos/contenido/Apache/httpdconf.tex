\porcion{El archivo \archivo{httpd.conf}}
\autor{\NC}
\colaborador{}
\revisor{\LLC}

El archivo \archivo{httpd.conf} configura el servidor Apache, e
incluye otros archivos de configuraci�n espec�ficos y m�s peque�os. El
archivo esta ampliamente documentado\footnote{Como muchos archivos de
configuraci�n cualquier l�nea que comience con  ``\#'' se ignora por lo
tanto es considerada comentario.}, cada directiva tiene una descripci�n
concreta de c�mo usarlo, por lo tanto haremos una descripci�n breve de
las directivas m�s importantes.

\begin{description}
\item[User]         Define el usuario que ejecuta al proceso apache.

\item[Group]        Define el grupo que ejecuta al proceso apache.

\item[ServerName]   Nombre del servidor, si se omite, se asigna el nombre del equipo.

\item[ServerAlias]  Uno o m�s alias del servidor.

\item[ServerAdmin]  Email del administrador para el envio de errores.

\item[DocumentRoot] Directorio donde se encuentran los documentos a enviar,
                    debe tener permiso el \emph{User} y \emph{Group}
                    mencionados anteriormente.

\item[Include]     Nombre  de archivo a incluir. Util para dividir la 
			configuraci�n en varios archivos peque�os.
\end{description}

Otro tipo de directivas son los \emph{bloques} que su formato es
similar a los tags HTML/XML, comienzan con {\tt \verb+<+Directiva
Opcion\verb+>+} y finalizan con {\tt \verb+<+/Directiva\verb+>+}

\begin{description}
\item[Directory] Opcion: nombre de directorio. Define propiedades de 
		un directorio a utilizar en el servidor.

\item[DirectoryMatch] Opcion: expresi�n regular. Igual que \emph{Directory}, 
		pero acepta una expresi�n regular para generalizar 
		los directorios.

\item[Files] 	Opcion: nombre de archivo. Define propiedades para un archivo
		determinado.

\item[FilesMatch] Opcion: expresi�n regular. Igual que \emph{Files}, pero
		acepta una expresi�n regular para generalizar los archivos.

\item[Location] Opcion: URL. Especifica las propiedades de una URL del sitio.

\item[LocationMatch] Opcion: expresi�n regular. Como es de esperar, es igual
		que \emph{Location} pero se pueden especificar 
		varias URL's con expresiones regulares

\item[VirtualHost] Opcion: nombre de host. Especifica propiedades de un s�lo
		dominio. Veremos en mayor detalle esta directiva.

\end{description}

\begin{ejemplo}

Un archivo \archivo{httpd.conf} simple:

\begin{vscreen}
User apache
Group apache
ServerName www.dominio.org.ar
ServerAdmin webmaster@dominio.org.ar
DocumentRoot /var/www/

<Directory /var/www>
Options Indexes
</Directory>
\end{vscreen}

En este caso, el usuario y grupo se llaman {\tt apache}, el servidor se
llama \emph{www.dominio.org.ar}, el email del administrador es
\emph{webmaster@dominio.org.ar}.

El directorio donde se encontrar�n los recursos es \archivo{/var/www}
y como opci�n se generan los �ndices de cada p�gina.
\end{ejemplo}

A nuestro ejemplo podemos agregarle un directorio llamado \archivo{images} que 
contenga todas las im�genes del sitio pero no queremos que nadie liste el 
contenido de ese directorio:

\begin{vscreen}
<Directory /var/www/images>
Options -Indexes
</Directory>
\end{vscreen}

Recordemos que se debe referenciar el \emph{path} completo. El `-'
sirve para deshabilitar la opci�n.

Por razones de seguridad, es �til agrupar todos los archivos tipo
\emph{CGI} en un directorio �nico, por ejemplo \archivo{cgi-bin}
el cual debe tener la opci�n {\tt ExecCGI}.

\begin{vscreen}
<Directory /var/www/cgi-bin>
Options -Indexes ExecCGI
</Directory>
\end{vscreen}

Y podemos mejorar la seguridad negando la ejecuci�n de CGI en el resto 
de los directorios:

\begin{vscreen}
<Directory /var/www>
Options Indexes -ExecCGI
</Directory>
\end{vscreen}

Queda nuestro archivo de la siguiente manera:

\begin{vscreen}
User apache
Group apache
ServerName www.dominio.org.ar
ServerAdmin webmaster@dominio.org.ar
DocumentRoot /var/www/

<Directory /var/www>
Options Indexes -ExecCGI
</Directory>

<Directory /var/www/images>
Options -Indexes
</Directory>

<Directory /var/www/cgi-bin>
Options -Indexes ExecCGI
</Directory>
\end{vscreen}
