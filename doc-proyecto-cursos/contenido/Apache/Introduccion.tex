\porcion{Introducci�n a Apache}
\autor{\NC}
\colaborador{}
\revisor{\LLC}

Apache es un servidor del protocolo \emph{http}, comunmente llamado \emph{servidor
web} pues es la mayor utilidad para dicho protocolo. Cuando pensamos en un servidor
web imaginamos un grupo de \emph{p�ginas web} que determinan un \emph{sitio web}.
Un servidor como Apache puede alojar varios sitios, y pueden coexistir varios
servidores Apache en un s�lo equipo.

B�sicamente Apache lee un directorio con todo el contenido posible a enviar y los
navegadores\footnote{Llamados \emph{browsers} en ingl�s} piden las p�ginas (o recursos)
para luego, por ejemplo, mostrarlos en pantalla. Es el funcionamiento m�s b�sico de
un servidor, sin embargo, los servidores actuales realizan muchas tareas complejas. 
Un ejemplo ser�a modificar el recurso para personalizarlo y luego enviarlo. O
ejecutar un programa y que la salida de este programa devuelva el recurso a enviar.
Comunmente estos programas se llaman \emph{scripts} y se tienden a escribir en lenguajes 
que fueron creados para ese prop�sito, como lo es PHP, Python o versiones actuales de
Perl.

El archivo de configuraci�n de Apache principal puede ser \archivo{/etc/httpd.conf} o 
\archivo{/etc/http.d/conf/httpd.conf} o \archivo{/etc/apache/httpd.conf} seg�n la 
distribuci�n y versi�n de Apache. En este archivo se determinan los directorios a 
utilizar, los m�dulos a cargar, permisos y much�simos detalles m�s. Veremos los 
m�s importantes.
