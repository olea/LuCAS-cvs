\porcion{Ping}
\autor{\NC}
\colaborador{}
\revisor{\LLC}
\traductor{}

?`C�mo sabemos si una computadora est� dentro de la red?

Es una pregunta relativamente simple de contestar sabiendo su \emph{
Direcci�n IP}. \comando{knu} se encuentra en \rama{K-Internet-Utilidades
de Red}. Se ingresa la direcci�n IP (por ejemplo {\tt 127.0.0.1}
\footnote{{\tt 127.0.0.1 es la m�quina local.}}) y a continuaci�n
\boton{!`Adelante!} como muestra la figura
\ref{fig:UtilidadesDeRed-Ping}.

\figura{Ejemplo del uso de \comando{ping} en \comando{knu}}{fig:UtilidadesDeRed-Ping}
{width=10cm}{Internet/UtilidadesDeRed-Ping}

Dar� informaci�n de la conexi�n. Normalmete en enlaces entre m�quinas
el dato que m�s importa es {\tt time}, que expresa la latencia de red.
Aqu� es irrelevante, puesto que es la m�quina local. De no funcionar esta
simple prueba, es posible que la configuraci�n de red no sea correcta.

Tambi�n se pueden poner nombres de m�quinas, como por ejemplo {\tt
www.google.com} siempre que est� clickeado el rect�ngulo que dice
\emph{Resolver el nombre}.

