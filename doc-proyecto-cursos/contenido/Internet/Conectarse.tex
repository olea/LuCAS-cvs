%%%%%%%%%%%%%%%%%%%%%%%
% Secci�n: Conectarse %
%%%%%%%%%%%%%%%%%%%%%%%
\porcion{Conectarse}
\autor{\NC}
\colaborador{}
\revisor{\LLC}
\traductor{}

Lo primero que hay que hacer es configurar una cuenta de Internet.
Los pasos son relativamente simples usando el \comando{kppp}.

El \comando{kppp} se puede ejecutar desde terminal o clickeando en 
\rama{K-Internet-Conexi�n a Internet}.

\figura{Vista inicial del \comando{kppp}}{fig:kppp-Inicial}
{width=5cm}{Internet/kppp-Inicial}
\figura{Configuracion del \comando{kppp}}{fig:kppp-Configuracion}
{width=5cm}{Internet/kppp-Configuracion}
\figura{Nueva conexi�n}{fig:kppp-NuevaConexion}
{width=5cm}{Internet/kppp-NuevaConexion}

Luego hay que ir a \boton{Configuraci�n}
(fig. \ref{fig:kppp-Configuracion}) para crear una conexi�n a Internet
con un click en \boton{Nueva} (fig. \ref{fig:kppp-NuevaConexion}).

Pudiendo as� rellenar los datos del provedor de Internet. Los datos
importantes son: el \emph{nombre} y el \emph{n�mero a marcar}. Luego
hay que ir a la leng�eta \emph{Servidor Nombres} y poner la direcci�n
IP del \emph{Servidor de Nombres} o \emph{Servidor DNS}, que es parte
de la informaci�n que nos brinda el proveedor de Internet.

\figura{Conexi�n a internet con todos los datos}{fig:kppp-completo}
{width=7cm}{Internet/kppp-completo}

Una vez configurado ya se puede elegir como parte de las posibles
configuraciones en \emph{Conectar con}. S�lo falta el nombre de
usuario y la clave como muestra la figura \ref{fig:kppp-completo}.

Con s�lo apretar \boton{Conectar} se deber�a conectar sin problemas
(fig \ref{fig:kppp-conectando}). 

\figura{Conect�ndose a Internet}{fig:kppp-conectando}
{width=8cm}{Internet/kppp-conectando}

De todas formas, muchas veces la realidad es muy distinta a la
teor�a. Esta es una lista de posibles problemas:

\begin{itemize}
\item Problemas con el modem
\item Problemas con pppd
\item Problemas relativos al proveedor
\end{itemize}

No pretende ser una lista exhaustiva, tan s�lo son los problemas 
m�s comunes.




