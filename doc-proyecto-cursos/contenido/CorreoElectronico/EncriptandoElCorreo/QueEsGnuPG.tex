\porcion{?`Qu� es GnuPG?}
\autor{}
\colaborador{\SGG}
\revisor{\LLC}
\traductor{}

\emph{GNU Privacy Guard} (com�nmente conocido como GPG) es una
herramienta de cifrado de informaci�n, que utiliza un sistema de
\emph{cifrado asim�trico}, ?`qu� quiere decir esto?, que cada usuario
tiene un par de claves:

\begin{description}
\item[Clave p�blica] La cual se puede (y debe) hacer conocer a todas
  aquellas personas que quieran enviarnos informaci�n cifrada.
\item[Clave privada] Esta clave se debe guardar y en ninguna
  circunstancia entregar a nadie. Se utiliza para descifrar los
  mensajes que nos env�an cifrados.
\end{description}

Este m�todo se utiliza porque es mucho m�s seguro que un sistema de
clave �nica. Por ejemplo: Si la persona A y la persona B quieren
comenzar a mandarse correo cifrado, deber�an establecer una clave con
la cual cifrar (y descifrar) la informaci�n (mensajes de correo en
este caso) que intercambian. La persona A establece una clave y se la
env�a por correo a B, pero por correo no cifrado (porque a�n B al no
tener la clave, no podr�a descifrar el mensaje de A). B al recibirla
comienza a cifrar los mensajes que env�a a A, y A usa la misma clave
para descifrarlos y cifrar los suyos hacia B. Esto parece seguro, pero
de hecho no lo es, ya que si alguna otra persona pudo interceptar el
mensaje de A hacia B (no cifrado) con la clave, luego usando dicha
clave podr�a descifrar todos los mensajes que A y B intercambien sin
ning�n problema.

El sistema de cifrado asim�trico soluciona este problema, ya que A y B
entonces tendr�an dos claves cada uno, entonces A le env�a su clave
p�blica a B y B env�a su clave p�blica a A. Luego A utiliza dicha
clave p�blica de B para cifrar los mensajes dirigidos a B. Lo mismo hace 
B con la clave p�blica de A. Aquellos mensajes cifrados con la
clave p�blica de B s�lo podr�n ser descifrados con la clave
\emph{privada} de B, al igual que los mensajes cifrados con la clave
p�blica de A podr�n ser descifrados con la clave privada de A. Como
dichas claves privadas est�n en poder de sus due�os y nadie m�s,
cualquier persona que intercepte los mensajes no cifrados con las
claves p�blicas de A y B, no podr� utilizarlas para descifrar ning�n
mensaje, ya que el �nico uso que tienen las claves p�blicas es cifrar
(y no descifrar) informaci�n que podr� ser descifrada s�lo por el
due�o de dicha clave p�blica.
