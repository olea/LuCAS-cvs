\porcion{Cifrando informaci�n}
\autor{}
\colaborador{\SGG}
\revisor{\LLC}
\traductor{}

Suponiendo que una persona quiere enviarle correo cifrado a uno de los
autores de este curso, Lucas Di Pentima, una vez que la clave p�blica
de este personaje est� en el anillo de claves p�blicas, se ver� algo
parecido a lo siguiente:

\begin{vscreen}
usuario@zeloran:~$ gpg --list-keys
/home/usuario/.gnupg/pubring.gpg
--------------------------------
pub  1024D/62B70584 2001-04-22 Usuario (233-8847) <usuario@maquina.dominio.com>
sub  2048g/7459EB6A 2001-04-22

pub  1024D/6AA54FC9 2001-03-22 Lucas Di Pentima (Tel: 54 342 4593122) <lucas@lunix.com.ar>
sub  1024g/9252D0E4 2001-03-22
\end{vscreen}
%$

Entonces para cifrar el archivo \archivo{/home/usuario/mensaje.txt} se
deber� ejecutar el siguiente comando:

\begin{vscreen}
usuario@maquina:~$ gpg --output mensaje.gpg --encrypt --recipient lucas@lunix.com.ar mensaje.txt
\end{vscreen}
%$

Una clave p�blica puede llegar a tener la certificaci�n de otras
personas de que es v�lida. No ahondaremos en este tema, pero es
oportuno hacer notar que si una clave p�blica que se utiliza para
cifrar un mensaje no est� certificada por otros, significa que no es
seguro que sea realmente de la persona que dice ser. Si esta clave
p�blica la recibimos de fuentes confiables (por ejemplo, directamente
del due�o), se puede utilizar de todas formas, aunque GPG avise de que
no es totalmente confiable.

El archivo con el mensaje cifrado ser�, obviamente, el archivo
\archivo{mensaje.gpg}.
