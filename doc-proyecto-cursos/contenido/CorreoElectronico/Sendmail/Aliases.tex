\porcion{Alias de correo en \comando{sendmail}}
\autor{\LDP}
\revisor{\LLC}
\colaborador{}
\traductor{}

Los \emph{alias de correo} generalmente se utilizan para redireccionar
los mensajes de las cuentas de correo de los usuarios administrativos
de un sistema a cuentas de correo de usuarios reales. Esto se hace
porque las cuentas administrativas s�lo se deben usar para realizar
tareas administrativas, y leer correo electr�nico no entra en esta
categor�a. Por ejemplo, si \emph{Juan P�rez} es el administrador de un
servidor, lo correcto es que redireccione los mensajes que van a la
cuenta {\tt root} a su propia cuenta {\tt jperez}. 

Esto es lo que normalmente se hace tambi�n con las cuentas de los
distintos subsistemas de un GNU/Linux normal: las cuentas de los
usuarios \emph{bin}, \emph{daemon}, \emph{games}, \emph{system}, etc
son redireccionadas a la cuenta del usuario \emph{root}.

Otro uso com�n de los \emph{alias de correo} es la generaci�n de
listas de distribuci�n: cuando un mensaje llega a una cuenta de correo
en especial, autom�ticamente se generan copias del mismo mensaje a
otras cuentas especificadas. No confundir este uso con las listas de
correo que masivamente se utilizan hoy en d�a. Para estos casos se
utiliza software especializado, ya que mantener una lista de correo
enorme con los aliases de correo de \comando{sendmail} ser�a demasiado
tedioso.

El archivo donde se definen los alias de correo se encuentra
generalmente en \archivo{/etc/mail/aliases}, y para explicar su
formato se da el siguiente ejemplo:

\begin{vscreen}
# Usuarios del sistema
bin:		root
daemon:		root
games:		root
nobody:		root
system:		root
postmaster	root

# Cuentas de los administradores
root:		jperez
admins:		jperez, rms, linus

# Ejemplo gen�rico
nombre:		valor
\end{vscreen}

Por cada entrada (l�nea) del archivo, el primer valor antes de los dos
puntos se refiere a una cuenta de correo local al equipo. En los
primeros 6 ejemplos, mensajes dirigidos a esos usuarios del sistema se
redireccionar�n a la cuenta del \emph{superusuario}, a su vez, luego
se define que los mensajes que lleguen a la cuenta del
\emph{superusuario} se redireccionen a la cuenta {\tt jperez}, del
 usuario Juan P�rez, el administrador del ejemplo anterior.

El pen�ltimo ejemplo define una lista de correo, todos los mensajes
dirigidos a la cuenta \emph{admins} se redireccionar�n a las cuentas
\emph{jperez}, \emph{rms} y \emph{linus}.

La �ltima entrada del ejemplo presentado tiene un car�cter un poco m�s
gen�rico, ya que {\tt valor} puede referirse a otro tipo de entradas:

\begin{description}
\item[direcci�n@de.correo] el mensaje se redireccionar� a una cuenta 
de correo externa al equipo.
\item[/path/a/un/archivo] el mensaje se a�adir� al final del archivo 
especificado.
\item[!comando] el mensaje se alimenta al comando especificado. Esta 
funcionalidad generalmente se utiliza cuando se necesita pasar los 
mensajes por filtros especiales.
\end{description}

Para que los cambios que se hagan en el archivo
\archivo{/etc/mail/aliases} sean efectivos, se debe ejecutar el
comando \comando{newaliases} (como {\tt root}), y al igual que en el
caso de los dominios virtuales, el \comando{sendmail} no debe
reiniciarse.

