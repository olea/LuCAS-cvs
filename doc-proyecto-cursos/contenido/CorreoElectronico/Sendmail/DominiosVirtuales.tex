\porcion{Dominios Virtuales con \comando{sendmail}}
\autor{\LDP}
\colaborador{}
\revisor{\LLC}
\traductor{}

Cuando es necesario administrar varios dominios en un mismo servidor
de correo, y estos dominios son utilizados por distintos grupos de
personas, es necesario utilizar \emph{dominios virtuales}. Un caso
t�pico ser�a el de un proveedor de Internet, que posee un servidor
para el manejo del correo electr�nico, el cual debe manejar los
dominios de sus clientes.

Los \emph{dominios virtuales} permiten:

\begin{itemize}
\item Redireccionar una cuenta de correo de un dominio espec�fico a 
una cuenta de usuario del servidor local.

\item Redireccionar una cuenta de correo de un dominio espec�fico a
una cuenta de correo de un servidor remoto.

\item Redireccionar todo el correo de un dominio espec�fico a una
cuenta de correo local o remota.
\end{itemize}

Quiz�s a simple vista, estas caracter�sticas no sean muy pr�cticas,
pero su practicidad se puede apreciar con el siguiente ejemplo:

Sup�ngase que se necesitan configurar dos dominios de correos
diferentes, para distintas empresas: \emph{Aberturas Guillermito
Puertas} con su dominio {\tt guillepuertas.com} y \emph{Ganader�a
Ricardo} con su dominio {\tt gnuslibres.com}, y ambos clientes
necesitan que exista en sus dominios la cuenta info y ventas. Esto sin
los dominios virtuales es un problema grave.

Si no se pudieran definir dominios virtuales y se creara una cuenta
de usuario llamada \emph{info} y otra \emph{ventas}, los mensajes de
correo dirigidos por ejemplo a \emph{ventas@gnuslibres.com} tambi�n
llegar�a a \emph{ventas@guillepuertas.com}.

El nombre que se le da a la caracter�stica de dominios virtuales en
\comando{sendmail} es \emph{virtusertable}, y el archivo donde se
definen las configuraciones de esta funcionalidad es normalmente el
\archivo{/etc/mail/virtusertable}. Su formato es el siguiente:

\begin{vscreen}
ventas@gnuslibres.com		ricardo
ventas@guillepuertas.com	bill@microsoft.com
@otrodominio.net		otrocliente@curso.org
\end{vscreen}

En el primer caso, los mensajes de correo dirigidos a {\tt
ventas@gnuslibres.com} se redireccionan a la cuenta de usuario local
\emph{ricardo}, en el segundo caso, los mensajes a {\tt
ventas@guillepuertas.com} se redireccionan a una cuenta de correo en
otro servidor. Finalmente, en el �ltimo caso se configura que
cualquier mensaje dirigido a cualquier cuenta del dominio
@otrodominio.net se redirecciona a una cuenta en otro servidor.

Agregar entradas al archivo \archivo{virtusertable} no es suficiente
para que se hagan efectivos los cambios en la configuraci�n. Dicho
archivo debe ser procesado por el comando \comando{makemap} para
generar un archivo binario que ser� usado por \comando{sendmail}. Una
vez que el archivo \emph{hash}\footnote{Un \emph{archivo hash} posee
un formato especial de tal forma que permite la b�squeda r�pida de
palabras claves.} se genera, \comando{sendmail} lo comienza a utilizar
sin necesidad de ser reiniciado.

Para generar entonces el archivo hash anteriormente mencionado, se
debe ejecutar el comando \comando{makemap} de la siguiente manera
(como superusuario):

\begin{vscreen}
# makemap hash /etc/mail/virtusertable < /etc/mail/virtusertable
\end{vscreen}

El resultado ser� la regeneraci�n del archivo
\archivo{/etc/mail/virtusertable.db} conteniendo los nuevos cambios
listos para que \comando{sendmail} los utilice.

Para activar la funcionalidad de dominios virtuales en
\comando{sendmail}, se debe definir lo siguiente en el archivo
\archivo{sendmail.mc}\footnote{El archivo \archivo{sendmail.mc} es el
archivo de \emph{macros m4} que se utiliza para generar el archivo de
configuraci�n de \comando{sendmail}, el archivo
\archivo{/etc/mail/sendmail.cf}}:

\begin{vscreen}
FEATURE(`virtusertable')
\end{vscreen}

