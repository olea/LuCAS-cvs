\porcion{?'Qu� es un MTA?}
\autor{\NC}
\colaborador{}
\revisor{\LLC}
\traductor{}

Las siglas MTA significan <<\emph{Mail Transfer Agent}>> que una
traducci�n aceptable puede ser \emph{Agente de Transferencia de
Correo}. Esta definici�n se contrasta con MUA, que significan
<<\emph{Mail User Agent}>> o podr�amos llamarlo \emph{Agente de
Correo del Usuario}.  Esta distinci�n se realiza para dividir la tarea
de env�o de correo, que est�n encargados los MTAs, de la tarea de
redacci�n y edici�n de correos, realizados mediante un MUA.

El protocolo\footnote{Los protocolos para las computadoras son como
los idiomas para los humanos.} m�s com�n utilizado entre los MUA y los
MTAs de Internet es SMTP o \emph{Simple Mail Transfer Protocol} en ingl�s
o \emph{Protocolo de Transferencia de Correo Simple}.
El servidor \comando{sendmail} es un MTA, o tambi�n denominado 
\emph{servidor SMTP}. 

Una transacci�n posible al enviar un email entre el MUA y el MTA
se describe a continuaci�n:

\begin{itemize}

\item El MUA se conecta con el MTA.

\item El MUA envia el remitente del correo.

\item El MTA acepta el remitente como un remitente v�lido.

\item El MUA envia el destinatario del correo.

\item El MTA acepta el destinatario como v�lido.

\item El MUA envia el ``subject'' o ``tema'' al MTA y el cuerpo del mensaje

\item El MTA lo acepta

\item El MUA termina la conexi�n

\end{itemize}


Los MTA pueden conectarse entre ellos para realizar la entrega de mail
utilizando tambi�n el protocolo SMTP. Inclusive pueden convertir correos
de varios formatos si es necesario.
