\porcion{Pine}
\autor{}
\colaborador{\SGG}
\revisor{\LLC}
\traductor{}

La mayor�a de las distribuciones (excepto Debian, virtualmente todas)
incluyen este programa de correo. Es un cliente de correo y grupos de
noticias muy avanzado, que posee una enorme cantidad de opciones.

La primera vez que el usuario ejecute el \comando{pine}, el programa
se iniciar� con una pantalla como se muestra en la figura
\ref{fig:ClientesDeCorreo-Pine-PantallaInicial}, donde se explica que
este programa se mantiene en la Universidad de Washington y el
proyecto sigue funcionando en la medida que aquellas personas sepan
que el programa se utiliza. Es por eso que se le pide al usuario
oprimir \boton{Enter} para ser contado entre los usuarios de
\comando{pine}. Si uno no quiere ser contado, puede salir de esa
pantalla pulsando la tecla \boton{E}. Esa pantalla no se volver� a
mostrar otra vez.

%\figura{Pantalla inicial del \comando{pine}}{fig:ClientesDeCorreo-Pine-PantallaInicial}{width=10cm}{ClientesDeCorreo-Pine-PantallaInicial}

El \comando{pine} dividide su �rea de trabajo en dos: la parte
inferior, que funciona a modo de barra de men�, y la parte superior, que 
toma varios formatos dependiendo de la secci�n donde se encuentre el
usuario. En la barra de men�, se ir�n mostrando las diferentes
funciones y sus teclas asociadas, como generalmente existen en cada
secci�n m�s opciones que lugar f�sico donde distribuirlas, pulsando la
tecla \boton{O} se mostrar�n m�s comandos. Aquella funci�n que
aparezca encerrada entre corchetes, es la que se ejecutar� por defecto
al pulsar \boton{Enter}.

Una vez que se ha salido de la pantalla de bienvenida, aparecer� la
pantalla principal, o \emph{Main Menu}, se puede acceder a la pantalla
principal desde las dem�s secciones del programa pulsando \boton{M}.
Como vemos en la figura
\ref{fig:ClientesDeCorreo-Pine-PantallaPrincipal}, la pantalla
principal se divide en varias secciones:

\begin{description}
\item[HELP] Es el sistema de ayuda del \comando{pine}, posee toda la
  documentaci�n en l�nea explicando cada detalle del programa. Es
  realmente recomendable utilizar esta ayuda para descubrir todas las
  posibilidades que provee este cliente de correo
\item[COMPOSE MESSAGE] Mediante el uso de esta opci�n desde la
  pantalla principal o pulsando \boton{C} desde otras secciones, el
  programa activa su modo de edici�n para enviar un mensaje nuevo.
  Esta funcionalidad la veremos m�s adelante.
\item[MESSAGE INDEX] Si seleccionamos esta opci�n, iremos al �ndice de
  mensajes de la carpeta que tengamos seleccionada. Apenas arranca el
  programa, la carpeta por defecto es \comando{INBOX}, pero se pueden
  agregar carpetas adicionales para organizar los mensajes por
  tem�tica. Esta opci�n puede accederse en otras secciones del
  \comando{pine} pulsando la tecla \boton{I}. El manejo de carpetas
  tambi�n se ver� con en detalle m�s adelante.
\item[FOLDER LIST] Esta opci�n lleva al usuario al �ndice, pero de las
  carpetas que existan en el programa, la tecla \boton{L} activa esta
  opci�n.
\item[ADDRESS BOOK] Para evitar tener que recordar grandes cantidades
  de cuentas de correo, o tener que escribirlas en papeles que siempre
  luego se pierden, el programa posee esta funcionalidad a modo de
  libreta de direcciones. En ella se pueden configurar adem�s listas
  de distribuci�n, ya se ver� m�s adelante c�mo funciona.
\item[SETUP] Seleccionando esta opci�n se ingresa en un submen� con
  varias posibilidades de configuraci�n y personalizaci�n del
  \comando{pine}. M�s adelante se ver�n algunas de las opciones de
  configuraci�n m�s importantes.
\item[QUIT] Esta opci�n es m�s que obvia. Se utiliza para salir del
  programa. En otras secciones del mismo, pulsando \boton{Q} se
  activa.
\end{description}

%\figura{Pantalla principal del \comando{pine}}{ClientesDeCorreo-Pine-PantallaPrincipal}


