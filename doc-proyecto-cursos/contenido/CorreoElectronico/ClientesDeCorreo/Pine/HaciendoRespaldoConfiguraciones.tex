\porcion{Haciendo respaldo de las configuraciones}
\autor{}
\colaborador{\SGG}
\revisor{\LLC}
\traductor{}

Como el alumno sabr�, el hacer copias de respaldo es una buena tarea
para evitar disgustos. Entonces es bueno enumerar aquellos archivos que
el \comando{pine} utiliza para guardar sus configuraciones. Estos
archivos son manipulados por el \comando{pine}, y aunque son archivos
de texto, no se recomienda editarlos directamente.

Las configuraciones del programa se guardan en un archivo localizado
en el directorio personal de cada usuario, con el nombre de
\comando{.pinerc}.

La libreta de direcciones con todos los contactos que se tengan, se
almacena en un archivo llamado \comando{.addressbook}.

Finalmente, la firma que se haya personalizado se almacena en un
archivo con nombre \comando{.signature}

Tambi�n es bueno recordar que las carpetas con los mensajes
almacenados (exceptuando \comando{INBOX}) se guardan en el directorio
personal, en un subdirectorio llamado \comando{mail/}.

A la hora de hacer respaldo de la correspondencia electr�nica, es
bueno conocer estos datos. Otro punto interesante a nombrar es que el
\comando{pine}, cada fin de mes preguntar� al usuario si quiere mover
las carpetas \comando{saved-messages} y \comando{sent-mail} a otro
lugar, para ir almacenando solamente los mensajes del mes actual. Es
una buena pr�ctica ir guardando los mensajes viejos en otro sitio, as�
el \comando{pine} no tarda tanto tiempo al intentar abrir una carpeta
que tiene miles de mensajes acumulados a trav�s de los meses o a�os.
