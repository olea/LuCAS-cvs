\porcion{Bibliotecas gr�ficas}

\autor{\LDP}
\colaborador{}
\revisor{}
\traductor{}

Para dibujar en pantalla cualquier cosa, el servidor X provee a los
programadores de la biblioteca gr�fica \biblioteca{Xlib}. Esta
biblioteca provee de funciones de dibujo demasiado primitivas, y el
realizar un entorno de ventanas con estas funciones se hace demasiado
complicado, es por �sto que se han desarrollado distintas bibliotecas
gr�ficas basadas en \biblioteca{Xlib}, pero que proveen un conjunto
mas rico y complejo de funciones que facilitan la tarea al programador
y mejoran la est�tica de los programas gr�ficos.

Como primeros ejemplos se tuvo a la biblioteca \biblioteca{Motif}, que
se utilizaba mucho en programas comerciales en los distintos
UNIX. \biblioteca{Motif} visualmente es muy simple y a la vez no
requiere de muchos recursos del equipo para funcionar con buena
velocidad.

Con el tiempo comenzaron a aparecer proyectos alternativos, primero
fue la versi�n libre de \biblioteca{Motif} llamada
\biblioteca{Lesstif}, luego bibliotecas mas pulidas como
\biblioteca{XForms} y finalmente \biblioteca{Qt} y
\biblioteca{GTK+}. En las figuras \ref{fig:BibliotecaQt} y
\ref{fig:BibliotecaGTK} se pueden apreciar las diferencias visuales de
\biblioteca{Qt} y \biblioteca{GTK+} respectivamente.

\figura{Entorno gr�fico basado en \biblioteca{Qt}}{fig:BibliotecaQt}{width=6cm}{ServidorX/BibliotecaQt}

\figura{Entorno gr�fico basado en \biblioteca{GTK+}}{fig:BibliotecaGTK}{width=6cm}{ServidorX/BibliotecaGTK}

Adem�s de la diferencia visual de las bibliotecas gr�ficas, estas
tambi�n difieren en lo que respecta a los lenguajes de programaci�n
soportados. Tomando como ejemplo las dos bibliotecas gr�ficas mas
populares en estos d�as, \biblioteca{Qt} (y las \biblioteca{kde-libs})
principalmente usan C++ para el desarrollo de aplicaciones, en cambio
las bibliotecas gr�ficas del proyecto \emph{GNOME} (\biblioteca{GDK} y
\biblioteca{GTK+}) tiene como lenguaje de programaci�n principal el
C. Sin embargo, esto no es una limitaci�n hoy en d�a, ya que estas
bibliotecas tambi�n soportan lenguajes alternativos como el Perl,
Python, Tcl/Tk, etc. de tal manera de poder dar cabida a m�s
programadores para cada entorno gr�fico.
