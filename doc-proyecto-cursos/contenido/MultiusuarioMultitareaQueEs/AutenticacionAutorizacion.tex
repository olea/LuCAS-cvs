%%%%%%%%%%%%%%%%%%%%%%%%%%%%%%%%%%%%%%%%%%%%%%%%%%%%%%
% Porci�n: Conceptos de Autenticaci�n y Autorizaci�n %
%%%%%%%%%%%%%%%%%%%%%%%%%%%%%%%%%%%%%%%%%%%%%%%%%%%%%%

\porcion{Conceptos de Autenticaci�n y Autorizaci�n}

\autor{\LDP}
\colaborador{}
\revisor{}
\traductor{}

En un entorno multiusuario como GNU/Linux, el sistema debe saber en
cada instancia qui�n lo est� operando, de manera tal de darle (o no)
los recursos que corresponde a cada usuario.

Esto se divide en dos acciones que el sistema debe realizar al tratar
con un usuario que intenta ingresar al sistema: primero la
\emph{autenticaci�n} y posteriormente la \emph{autorizaci�n}.

Cuando un usuario pretende ingresar al sistema de forma interactiva,
es decir, obtener acceso a una terminal de textos o a un entorno
gr�fico de ventanas, el sistema de alguna manera va a solicitar a
dicho usuario su identificaci�n y su contrase�a. La identificaci�n
normalmente se la conoce como \emph{<<nombre de usuario>>}, la cual le
dice al sistema qui�n es el usuario. La contrase�a es una palabra o
conjunto de caracteres que el usuario debe tener en secreto, y sirve
para probar al sistema que el usuario es quien dice ser. Esta etapa se
la conoce como \textbf{autenticaci�n}.

No s�lo en accesos interactivos la autenticaci�n tiene lugar. En otros
servicios como el correo electr�nico, acceso a bases de datos, etc. se
requiere de una autenticaci�n por obvias razones de privacidad.

Por otro lado, cuando un usuario ha probado ser quien dice ser, el
sistema debe establecer las actividades que dicho usuario tiene
permitido hacer en el sistema. Por ejemplo, no es lo mismo un usuario
de correo electr�nico que un usuario administrador del sistema. El
segundo obviamente tendr� m�s privilegios que el primero porque sus
funciones son diferentes. Entonces, cuando un usuario se
\emph{autentica} frente al sistema, el mismo debe establecer ciertos
niveles de acceso, entre los cuales pueden ser:

\begin{itemize}
\item Uso m�ximo de espacio en disco.
\item Cantidad m�xima de procesos simult�neos posible.
\item Cantidad m�xima de memoria ocupada por procesos en ejecuci�n.
\item Lista de servicios que el sistema provee y al cual el usuario
      tiene acceso.
\end{itemize}

Este procedimiento es lo que se denomina \textbf{autorizaci�n}. Como
se puede ver, la autorizaci�n generalmente tiene que ver con la
asignaci�n m�xima de recursos del sistema a un usuario dado, ya que
cuando un equipo es usado por muchas personas a la vez, el
administrador del sistema deber�a tener en cuenta la capacidad del
equipo y establecer las listas de acceso de tal manera que el servicio
que provee dicho equipo no disminuya en eficiencia a causa de la
saturaci�n.


