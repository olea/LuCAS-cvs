%%%%%%%%%%%%%%%%%%%%%%%%%%%%%%%%%%%
% Porci�n: GNU/Linux Multiusuario %
%%%%%%%%%%%%%%%%%%%%%%%%%%%%%%%%%%%

\porcion{GNU/Linux Multiusuario}

\autor{\LDP}
\colaborador{}
\revisor{\LLC}
\traductor{}

En los principios de la inform�tica, cuando las computadoras costaban
fortunas, ocupaban habitaciones enteras y s�lo algunas empresas u
organizaciones grandes pose�an estos ejemplares, la necesidad de
brindar servicios a varias personas al mismo tiempo hizo que se creara
un sistema que permitiera un solo equipo atender los requerimientos
de varios usuarios a la vez.

El hecho de que un sistema pueda proveer servicios a muchos usuarios
simult�neamente da por sentado que debe ser adem�s multitarea, ya que
cada usuario al trabajar en el sistema ejecuta al menos un proceso.

Esa �poca de grandes \emph{mainframes}\footnote{Servidores de gran
capacidad.} ve el nacimiento de los sistemas operativos UNIX, de los
cuales GNU/Linux deriva.

Con el pasar de los a�os, el abaratamiento del hardware y el
nacimiento de las computadoras personales (en la d�cada de los 80),
los sistemas operativos \emph{mono-usuarios} se hicieron
predominantes\footnote{Entre ellos, el famoso D.O.S.}. En la
actualidad, la computadora ha llegado a ser un electrodom�stico, y
como tal su uso es familiari. Esto quiere decir que muchas personas
utilizan un mismo equipo: el nombre de \emph{Computadora Personal} ha
dejado de tener sentido. Por esto, los sistemas operativos m�s usados,
entre ellos \emph{Microsoft Windows} han tratado de llegar a una
soluci�n mediante el uso de perfiles de usuario, en los cuales cada
perfil tiene su propia configuraci�n de correo, escritorio, etc. Sin
embargo, esta soluci�n no es la adecuada, dado que no se observa algo
muy importante en la persona: la privacidad.

Para asegurar la privacidad de las personas en un ambiente
inform�tico, no s�lo se debe mantener una vista personalizada del
ambiente de trabajo de cada usuario, sino que tambi�n se debe ofrecer
un sistema de protecci�n de los archivos y directorios personales, as�
como un mecanismo equivalente para los procesos. De esta manera, cada
archivo y cada proceso en el sistema tiene permisos que permiten la
manipulaci�n de los mismos solamente a las personas autorizadas.

GNU/Linux posee estas caracter�sticas, por lo tanto se adec�a a
diferentes usos, desde un servidor de Internet con muchos cientos de
usuarios, hasta el uso familiar, conservando siempre la privacidad de
sus usuarios.
