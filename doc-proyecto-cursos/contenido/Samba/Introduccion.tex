%****Introducci�n****
\porcion{Introducci�n}
\autor{\CPP}
\colaborador{}
\revisor{}
\traductor{}

Samba es un conjunto de programas que permiten que m�quinas que no trabajan
con Windows puedan utilizar el protocolo Server Message Block (SMB), que es
el protocolo de uso nativo de los sistemas Windows.

�De d�nde viene Samba?

Samba es la idea de Andrew Tridgell, quien actualmente lidera el equipo
de desarrollo de Samba development desde su casa de Canberra, Australia.
El proyecto naci� en 1991 cuando Andrew cre� un programa servidor de
ficheros para su red local, que soportaba un raro protocolo DEC de Digital
Pathworks. Aunque �l no lo supo en ese momento, aquel protocolo m�s tarde
se convertir�a en SMB. Unos cuantos a�os despu�s, �l lo expandi� como su
servidor SMB particular y comenz� a distribuirlo como producto por Internet
bajo el nombre de servidor SMB. Sin embargo, Andrew no pudo mantener ese
nombre -ya pertenec�a como nombre de producto de otra compa��a-, as� que
intent� lo siguiente para buscarle un nuevo nombre desde Unix:\\
\\
\comando{grep -i 's.*m.*b' /usr/dict/words}
\\
y la respuesta fue:salmonberry samba sawtimber scramble


De �sta manera naci� el nombre de Samba\footnote{Extraido de la traducci�n
del libro 'Using Samba' que puede encontrarse en \sitio{www.samtek.es/sobl}}.

