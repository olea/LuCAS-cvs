
%****Funcionamiento de Samba****

\porcion{Funcionamiento de Samba}
\autor{\CPP}
\colaborador{}
\revisor{}
\traductor{}

Samba utiliza NetBIOS sobre TCP/IP (NetBT) para entregar los cuatro servicios
SMB principales: servicios de archivo e impresi�n, autentificaci�n y autorizaci�n,
resoluci�n de nombre y visualizaci�n.

\begin{itemize}
\item Los servicios de archivo e impresi�n
permiten que Windows y Samba compartan ficheros e impresoras a trav�s de la red.

\item La autenticaci�n y autorizaci�n utilizan nombres de registro y contrase�as para
permitir a los usuarios autorizados acceder a la red.

\item La resoluci�n de nombres consiste en convertir nombre NetBIOS en direcciones IP.

\item La visualizacion permite que otros despositivos en la red conozcan los servicios
que se encuentran disponibles.

\end{itemize}

Samba puede hacer casi todas las funciones de un Windows NT/2000 Server. Soporta
muchas de las funciones de un PDC (controlador primario de dominio) pero no es
capaz de realizar las de BDC (controlador reserva de dominio).

Tampoco permite el intercambio de datos con un servidor WINS de Microsoft.

Lo m�s corriente es configurar Samba en un entorno mixto UNIX/Linux Windows puesto
que Samba requiere una m�quina relativamente m�s sencilla para realizar el trabajo
y es considerablemente m�s econ�mico.

