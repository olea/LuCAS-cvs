


\porcion{Obtenci�n e instalaci�n}
\autor{\CPP}
\colaborador{}
\revisor{}
\traductor{}

Samba se puede obtener desde internet aunque la mayor�a de las distribuciones la
traen. Para obtner la �ltima versi�n de Samba nos dirigimos a \sitio{http://www.samba.org}
y desde all� accedemos a uno de los m�ltiples servidores de r�plica que
existen en el mundo.\\
En este cap�tulo vamos a instalar y configurar Samba en una distribuci�n
Red Hat 7.1 aunque se tratar� de dar una orientaci�n sobre la instalaci�n en distribuciones
con otros sistemas de paquetes, as� como las especificaciones generales para su instalaci�n
desde el c�digo fuente.


Instalaci�n en diferentes formatos


Red Hat o paquetes RPM

Obtendremos la versi�n para nuestro sistema operativo desde los discos de instalaci�n
o desde \sitio{http://rpmfind.net} o sus distintos sitios de r�plica, as� como desde
las p�ginas de las distintas distribuciones. Los paquetes a obtener ser�n los correspondientes
a samba, samba-common y samba-client. Aunque este �ltimo no es del todo necesario.\\
La forma de instalaci�n es la com�n en los casos de paquetes RPM:\\

\begin{verbatim}
rpm -i paquete_a_instalar.rpm
\end{verbatim}

Debian

Para Debian habr� que tener los \comando{source-list} bien configurados y utilizar la
c�moda herramienta de configuraci�n provista por el sistema:
\comando{apt-get install samba}

Slackware

�Alguien se anima a explicar su instalaci�n?

Instalaci�n desde el c�digo fuente
