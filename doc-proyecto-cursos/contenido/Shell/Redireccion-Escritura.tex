\porcion{Escritura}
\autor{\NC}
\colaborador{}
\revisor{}
\traductor{}

Para escribir un archivo se utiliza (\verb+>+). Hay que tener mucho cuidado
de no borrar un archivo sobreescribi�ndolo. Cuando se utilizan
redirecciones, debido a su utilidad en los scripts, \textbf{no se
  realizan confirmaciones}. Si el archivo a escribir existe y posee
informaci�n valiosa, aplicar \comando{> archivo-importante} lo sobreescribe
con el contenido del flujo.

En cambio el operador (\verb+>>+) realiza un
\emph{agregado}\footnote{\emph{append} en ingl�s} de los datos en el flujo.

No hay nada mejor que un ejemplo clarificador:

\begin{vscreen}
$ escribe-en-salida-estandar > archivo.txt 
\end{vscreen}
%$

El (falso) comando \comando{escribe-en-salida-est�ndar} justamente
hace eso, escribe unas cuantas cosas en salida est�ndar. Puede ser un
\comando{ls}, un \comando{cal} o cualquier comando antes visto, as� como
tambi�n una combinaci�n de comandos por tuber�as.

En este punto, el contenido de \archivo{archivo.txt} es lo mismo que
saldr�a en pantalla.  Si ejecutamos otro comando redireccionado a
\archivo{archivo.txt}, �ste pierde su contenido y el resultado de la
operaci�n pasa  a estar en el archivo.

Cuando se necesita tener una lista de acontecimientos, no se quiere que 
un acontecimiento nuevo borre a todos los anteriores. Para lograr esto
\emph{agregamos} en vez de sobreescribir.

\begin{vscreen}
$ echo Este es el acontecimiento Nro. 1 > bitacora.log
$ echo Este es el segundo acontecimiento >> bitacora.log
\end{vscreen}

Va a escribir dos l�neas en el archivo \archivo{bitacora.log} 
sin eliminar nada.

\textbf{Ejemplo} Si queremos combinar el ejemplo de las tuber�as en
la secci�n \ref{subsection:tuberias} con lo aprendido recientemente
podr�amos escribir:

\begin{vscreen}
$ cat archivo.txt | sort | sacar-repetidas | diccionario >> glosario.txt
\end{vscreen}
%$

