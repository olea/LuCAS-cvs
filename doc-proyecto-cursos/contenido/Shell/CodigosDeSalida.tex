\porcion{C�digos de salida}
\autor{\NC}
\colaborador{}
\revisor{}
\traductor{}

Una vez que termina cada programa, puede brindar al entorno un 
\emph{C�digo de salida} para que otros programas o el int�rprete
sepan como concluy� la aplicaci�n.

Tomemos un ejemplo de la vida de un administrador. Es com�n que la
administraci�n sea remota, por lo que vamos a considerar que en
nuestra tarea no tenemos conocimiento de lo que est� pasando en una
m�quina alejada en la que se est� ejecutando
\comando{arreglar-base-de-datos}.

El script \comando{arreglar-base-de-datos} es un script que corrige
posibles errores en una hipot�tica gran base de datos. Y el resultado de
esa correcci�n interesa, especialmente, si no se pudo arreglar.

Vamos a suponer que hay 2 posibles alternativas:

\begin{description}
  
\item[Salida exitosa] La base de datos no tuvo ning�n error. En este
  caso s�lo hay que agregar al archivo \archivo{/var/log/BD.registro}
  una l�nea con la fecha de comprobaci�n y el responsable en ese momento.

%\item[Se efectuaron correcciones exitosamente] La base de datos tenia
%  errores, el script los detecto y los corrigi�. Este caso es un poco
%  m�s grave por lo tanto no s�lo hay que agregar al archivo
%  \archivo{/var/log/BD.registro} una l�nea con la fecha de chequeo y
%  el responsable en ese momento sino tambi�n un mail al responsable
%  con los errores encontrados y posteriormente corregidos para que no
%  vuelva a ocurrir
  
\item[Se detectaron errores pero no se repararon] Esta situaci�n es 
  peor. Hay que escribir informaci�n detallada
  en \archivo{/var/log/BD.registro} y adem�s enviar correo-e\footnote{Por 
rid�culo que parezca es la forma correcta de mencionar a los \emph{emails}}
 a una lista  de encargados y directivos de la empresa.
\end{description}

Para diferenciar cada caso se asigna un \emph{c�digo de salida} a cada uno.
Luego de ejecutar \comando{arreglar-base-de-datos} se verifica en base al
c�digo, los comandos a ejecutar.

El algoritmo ser�a algo similar a:

\begin{vscreen}
if arreglar-base-de-datos
then
  date >> /var/log/BD.registro
  echo $RESPONSABLE_BD >>  /var/log/BD.registro 
else
  informar-errores.sh >> /var/log/BD.registro
  enviar-mail "Errores en BD" lista-encargados lista-directivos
fi
\end{vscreen}
%$

?`Y d�nde est�n los c�digos de salida? Bueno, el \emph{comando
  interno}\footnote{\emph{Built-in command} en ingl�s.}  \comando{if}
analiza el c�digo de salida, y si es {\tt 0} (cero) ejecuta la lista
de comandos despu�s del \comando{then}, en caso contrario (y si
existe) la lista de comandos despu�s del \comando{else} hasta
encontrar un \comando{fi}.

Por lo tanto el script \comando{arreglar-base-de-datos} tiene que devolver
{\tt 0} en caso de �xito. \'{E}ste es  el comportamiento normal de la mayor�a de los 
comandos en Linux y otros Unix, y un valor para varios errores.

Las p�ginas \comando{man} suelen tener una secci�n llamada \textbf{Exit Status}
que contiene los c�digos que devuelve el programa.

