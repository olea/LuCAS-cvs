\porcion{Redirecci�n}
\autor{\NC}
\colaborador{}
\revisor{}
\traductor{}

Si bien nuestro ejemplo es ilustrativo, es bueno ver los resultados en
pantalla. En repetidas ocaciones en la vida de un sistema es mejor
tener todo en archivos, ya sea para guardar alg�n historial o para
automatizar ciertas funciones dentro de scripts.

Para almacenar o sacar informaci�n de archivos y 
vincularlas con entradas o salidas est�ndares se utilizan
\emph{Redirecciones}.

La redirecci�n se expresa con los s�mbolos <<Mayor>> (\verb+>+) y <<Menor>>
(\verb+<+). Se pueden utilizar en forma simple o doble.

\begin{ejemplo}

Utilizando el comando \comando{cat} se puede hacer una copia de 
\archivo{arch1.txt} a \archivo{arch2.txt} utilizando redirecci�n.

\begin{vscreen}
$ cat arch1.txt > arch2.txt
\end{vscreen}
%$

Se puede redireccionar una archivo para visualizarlo con \comando{less}.

\begin{vscreen}
$ less < arch1.txt
\end{vscreen}
%$

\end{ejemplo}
