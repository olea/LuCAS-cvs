\porcion{El comando \comando{for}}
\autor{\NC}
\colaborador{}
\revisor{}
\traductor{}

Para quienes programan en otros lenguajes el comando \comando{for} se
comporta distinto a la cl�sica sentencia \emph{for}. Este comando
asigna \emph{de} una lista de elementos, el valor \emph{a} una
variable y repite una lista de comandos con esa variable.

Si bien la explicaci�n puede ser un  poco confusa, el concepto es 
bastante f�cil de entender al ver un ejemplo.

\begin{vscreen}
for cantidad in dos tres cuatro cinco seis siete
do
  echo ${cantidad} elefantes se balancaban sobre la tela de una ara�a
  echo como veian que resist�a fueron a llamar a otro elefante...
done
\end{vscreen}
%$

\begin{description}
\item[dos (...) siete] son los elementos.
\item[cantidad] es la variable que iteraci�n a iteraci�n va tomando los valores
de la lista de elementos
\item[do; echo (...);done] es el bloque de comandos a iterar.
\end{description}

Esta es la forma m�s simple de utilizar el comando  \comando{for},
pero con pocas variaciones se puede realizar cosas muy �tiles, por ejemplo:

\begin{vscreen}
for archivo in `ls`
do
  touch ${archivo}
done
\end{vscreen}
%$

La lista de elementos se obtiene de el resultado del comando
\comando{ls}.  Es decir, primero se ejecuta \comando{ls}, el cual dar�
el listado de todos los archivos de un directorio, y a todos esos
archivos se les aplica un \comando{touch}\footnote{El comando
  \comando{touch} cambia la fecha de modificaci�n de un archivo
a la fecha actual}.

