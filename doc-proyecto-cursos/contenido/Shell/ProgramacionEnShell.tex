\porcion{Programaci�n en shell}
\autor{\NC}
\colaborador{}
\revisor{}
\traductor{}
\label{seccion:progbash}

Uno de las grandes ventajas que ofrece un int�rprete de comandos es la
programaci�n en un lenguaje r�stico pero poderoso para automatizar
infinidad de tareas.

Como todo lenguaje, posee reglas sint�cticas que establecen
la forma de escribir las sentencias a ejecutar.

Para quienes poseen conocimiento de otros lenguajes de programaci�n,
el signo <<punto y coma>> (;) es utilizado frecuentemente como
separador o terminador de sentencias. En \comando{bash} no es
necesario y puede ser reemplazado por \boton{Enter}. Es com�n encontrar
una l�nea de este tipo:

\begin{vscreen}
# comando1 ; comando2
(ejecuci�n de comando1 seguido de comando2)
\end{vscreen}

es equivalente a:

\begin{vscreen}
# comando1
(ejecuci�n de comando1)
# comando2
(ejecuci�n de comando2) 
\end{vscreen}

En el primer ejemplo con una sola l�nea se ejecutan ambos comandos.
Es muy buen ejemplo cuando se quiere encadenar tareas que consumen
mucho tiempo y tienen que ser seguidas.  

Hay que tener presente que no se ejecutan en paralelo. Cuando
termina de ejecutarse \comando{comando1} empieza a ejecutarse
\comando{comando2}.

