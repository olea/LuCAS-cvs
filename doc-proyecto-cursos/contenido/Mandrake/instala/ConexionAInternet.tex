\porcion{Configuraci�n de la conexi�n a Internet}
\autor{\jeg}
\colaborador{\NC}
\revisor{\LLC}
\traductor{}

La otra opci�n es la conexi�n usando m�dem, la cual es sin temor
a dudas, la m�s popular. La configuraci�n de la conexi�n pregunta
si intenta encontrar el m�dem y si no tiene �xito, preguntar�
el puerto serie al cual est� conectado. Tal como se explic� 
anteriormente, este dispositivo ser� {\it /dev/ttySx}, muy 
posiblemente el \verb+/dev/ttyS3+ correspondiente al COM4 en DOS/Windows.
Las caracter�sticas de la conexi�n se piden a continuaci�n en
una caja de di�logo (figura ~\ref{fig_configura_modem}).

\begin{itemize}
\item{{\bf Nombre de la conexi�n} para identificar esta conexi�n}
\item{{\bf N�mero de tel�fono} a cual se va a llamar}
\item{{\bf ID de conexi�n} Su identificaci�n de usuario }
\item{{\bf Contrase�a} Su palabra clave}
\item{{\bf Autenticaci�n} El tipo de autenticaci�n. Por defecto es PAP}
\item{{\bf Nombre del dominio} La extensi�n al nombre, ej: (skina.com.co).
		 No requerido}
\item{{\bf Primer servidor DNS} Servidor de Nombres primario.
		 No requerido pero recomendado}
\item{{\bf Segundo servidor DNS} Servidor de Nombres secundario.
		 No requerido pero recomendado}
\end{itemize}

\figura
{Configuraci�n de la conexi�n por m�dem}
{fig_configura_modem}
{width=11cm}{Mandrake/configura/install/pan_conexion_modem.png}



