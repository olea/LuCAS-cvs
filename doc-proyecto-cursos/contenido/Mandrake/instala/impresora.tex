%%% Parte de la Guia Completa a Linux ver 1.2.2
%%% de Jaime E. Gomez <jgomez@uniandes.edu.co>
%%% http://linuxcol.uniandes.edu.co/infolinux/docs/guia_linux/guia_linux.html
\porcion{Configuraci�n de la impresora \label{sec_impresora}}
\autor{\jeg}
\colaborador{\NC}
\revisor{\LLC}
\traductor{}

El soporte para impresi�n se incluye por defecto en la instalaci�n.
\distribucion Linux incluye dos sistemas de impresi�n: el
est�ndar Unix \verb+lpr+ ({\it Line printer}) usado por todas las 
distribuciones y
un nuevo sistema abierto promovido por la casa matriz \distribucion:
CUPS ({\it Sistema de Impresi�n com�n de Unix}), que es, por supuesto,
el sistema recomendado.

En la conexi�n de la impresora se escoge una impresora local.
GNU/Linux puede utilizar impresoras conectadas al computador que
se est� utilizando (paralelo, serie o USB) o impresoras 
remotas en otros computadores, utilizando diversos 
protocolos: LPR (Unix), SMB (MS-Windows), NCP (Novell Netware), 
CUPS o IPP (figura~\ref{fig_nombre_impresora}).

La primera pregunta a continuaci�n es el Nombre de la impresora, 
el cual tiene por defecto {\it lp}, legado del sistema {\it lpr},
para el cual deber�a dejarse de esta forma. Si desea 
otro nombre, a��dalo usando el car�cter {\it pipe} ``${|}$''. 
Por ejemplo {\sf lp${|}$laser}. La descripci�n y la localizaci�n de la 
impresora es �til para su anuncio en la red. 

\figura
{Selecci�n del nombre de la impresora}
{fig_nombre_impresora}
{width=11cm}{Mandrake/configura/install/pan_nombre_impresora.png}
                         
Como se escogi� una impresora local, a
continuaci�n se detecta el dispositivo donde est� conectada
y el modelo de la misma. 
Si no se tiene �xito, se pregunta al usuario por el 
dispositivo y el modelo. Si no se est� seguro de la respuesta 
se sugiere revisar la secci�n 
dedicada a la descripci�n del {\it hardware} del PC bajo el punto de 
vista de GNU/Linux. Normalmente la respuesta es {\it /dev/lp0}
como es sugerido. Tambi�n se presentar� ahora una lista 
de impresoras. Se selecciona la correcta o la que m�s se 
parezca al modelo a usar (figura~\ref{fig_op_drv_impresora}).

\figura
{Modelo de la impresora}
{fig_op_drv_impresora}
{width=11cm}{Mandrake/install/driver_impresora.png}

Una vez configurada la impresora se presentar� una pantalla de confirmaci�n
de la configuraci�n con la impresi�n de una p�gina de prueba. Si se est�
conforme con esta configuraci�n, se escoge [{\sf Hecho}] y se 
presiona [{\sf Aceptar}].
