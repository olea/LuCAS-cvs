%%% Parte de la Guia Completa a Linux ver 1.2.2
%%% de Jaime E. Gomez <jgomez@uniandes.edu.co>
%%% http://linuxcol.uniandes.edu.co/infolinux/docs/guia_linux/guia_linux.html
\porcion{Establecer la clave del {\it root} y otros usuarios}
\autor{\jeg}
\colaborador{\NC}
\revisor{\LLC}
\traductor{}

A continuaci�n hay que introducir una palabra clave o {\it password} para
el usuario {\it root}. Se introduce una clave dos veces, la segunda es de
confirmaci�n. No hay que preocuparse si no se ve lo que se escribe.
Se hace de esta forma para que nadie pueda ver en la pantalla la
clave (figura~\ref{fig_passwd_root}).

El {\it root} es el usuario con todos los privilegios en una
m�quina GNU/Linux. Es aquella persona que puede configurar el sistema y
a�adir otros usuarios menos ``privilegiados''.

\figura
{Password de {\it root}}
{fig_passwd_root}
{width=11cm}{Mandrake/configura/install/pan_cuenta_root.png}

A prop�sito, es tambi�n recomendado, tal como lo sugiere la 
siguiente pantalla, a�adir un usuario corriente, 
por decir {\it invitado} o su {\it usuario} favorito, para ser usado
cotidianamente 
en vez de {\it root}. El usuario {\it root} s�lo debe usarse para labores de
administraci�n y nunca debe usarse como una cuenta corriente, es
muy peligroso, ya que los errores de {\it root} tienen consecuencias para
todos los usuarios.

\distribucion tiene la opci�n de escoger un usuario para que el
sistema entre con �l siempre que se enciende sin necesidad de
ingresar la clave. En el 
modo experto esta opci�n no se ofrece, mientras que s� se 
hace en el recomendado, aunque siempre se puede configurar esta
caracter�stica 
despu�s de instalado. Esta elecci�n es recomendada para 
cuando se est� migrando o si la m�quina que se est� instalando 
puede ser usada por otras personas y no se desea tener una 
cuenta para cada una de ellas.

Es muy importante no olvidar estas claves. Si lo hace, en teor�a
no se podr�n recuperar (en la pr�ctica existen varios "trucos" \verb+:-)+ ).
M�s adelante se discutir� m�s detalladamente sobre {\it root},
las claves y los usuarios.

Una vez a�adidos los usuarios, presione [{\sf Hecho}] para continuar.
