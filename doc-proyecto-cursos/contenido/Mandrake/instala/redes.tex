%%% Parte de la Guia Completa a Linux ver 1.2.2
%%% de Jaime E. Gomez <jgomez@uniandes.edu.co>
%%% http://linuxcol.uniandes.edu.co/infolinux/docs/guia_linux/guia_linux.html
\porcion{Configuraci�n de la Red }\label{sec_inst_red}
\autor{\jeg}
\colaborador{\NC}
\revisor{\LLC}
\traductor{}

Una vez se ha culminado la creaci�n de usuarios, el
sistema proceder� a hacer la configuraci�n
de red: esto incluye conexi�n a red local y remota por
m�dem. 

La configuraci�n de redes se inicia con la autodetecci�n
de dispositivos. Se solicita aprobaci�n para realizar
este proceso ya que puede congelar la m�quina.
A continuaci�n se presenta el men� de 
elecci�n del tipo de conexi�n que se posee. Estos son:
\begin{itemize}
\item{Configurar una conexi�n por M�dem normal:
	Modulador/demodulador para conectarse v�a telef�nica con 
	el proveedor de servicio de acceso a Internet}
\item{Configurar una conexi�n por RDSI: Igual que el
	m�dem pero usando l�neas digitales }
\item{Configurar una conexi�n DSL o ADSL:
	 ({\it Asymmetric Digital Subscriber Loop/Line})
	 Nuevo tipo de conexi�n telef�nica hasta de 1.5 Mbps}
\item{Configurar una conexi�n por cable: Conexi�n por fibra �ptica,
	 la misma de la TV por cable}
\item{Configurar la red local: Conexi�n a red local}
\end{itemize}


Si se est� en una red local, y se tiene una tarjeta de 
red, el sistema intenta autodetectarla y confirma si es
la �nica tarjeta.
Como siempre, si no se tiene �xito en la autodetecci�n,
mostrar� una lista de los dispositivos
soportados para que el usuario escoja la correcta.

Mucha de la informaci�n necesaria para la red local debe
ser proveida por el administrador del sistema o el
departamento de soporte. Es posible que la red local 
en la que se encuentra use DHCP ({\it Dynamic Host 
Configuration Protocol}) de tal forma que un servidor remoto 
proveer� toda la informaci�n necesaria al computador. De lo contrario 
ser� necesario introducir los datos manualmente y pedir que active la
interfaz de red en el momento del boot (figura~\ref{fig_configura_red}).

%Todos los valores relevantes a la conexi�n son: 
%
%\input{../linux/valores_red}
%
%Es importante nuevamente resaltar que tanto los valores como el nombre
%no pueden ser asignados arbitrariamente, sino que son dados por el
%administrador de red, ya sea local o globalmente en Internet y muchos
%de estos son asignados por defecto por el sistema o tienen valores
%que siguen estandares universales.


\figura
{Configuraci�n de la red local}
{fig_configura_red}
{width=11cm}{Mandrake/configura/install/pan_red_dns.png}

