\porcion{Introducci�n}
\autor{\LDP}
\colaborador{\SGG}
\revisor{Leonardo Mart�nez}
\traductor{}

% Ideas, para la introducci�n, obtenidas de:
% http://grupos-locales.hispalinux.es/directorio-lugs/

Los grupos locales son agrupaciones de personas que f�sicamente se reunen y
realizan actividades en una zona determinada.

Estos grupos tienen unas particularidades que los diferencian de asociaciones
nacionales o grupos de trabajo virtuales: necesitan un punto de reuni�n
(un local,...), hacen actividades ``a nivel de suelo'' destinadas a la promoci�n del
Software Libre entre el ciudadano de la calle (conferencias, fiestas de
instalaci�n, colaborar con colegios, centros culturales, otras asociaciones,
etc.). Todo esto significa que sus cometidos son mucho m�s que un mero contacto
por Internet.

A continuaci�n se presenta una lista de los grupos locales existentes en el
mundo hispanohablante (si conoces alguno que no aparezca en esta lista,
contacta con nosotros, por favor):
