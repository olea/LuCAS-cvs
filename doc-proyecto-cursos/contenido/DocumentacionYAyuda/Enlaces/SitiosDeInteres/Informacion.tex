\porcion{Informaci�n}
\autor{\LDP}
\colaborador{\SGG}
\revisor{Leonardo Mart�nez}
\traductor{}

%++++++++++++++++++++++++++++ Informaci�n +++++++++++++++++++++++++++++

Colecci�n de enlaces con informaci�n relativa al sistema operativo GNU/Linux:

\begin{description}

   \item[es.comp.os.linux.*]
   	\sitio{http://www.escomposlinux.org}, P�gina de los grupos de noticias
	                                      de Linux de la jerarqu�a
					      es.comp.os.linux.*

   \item[El rinc�n de Linux]
   	\sitio{http://www.linux-es.com/}, Esta p�gina es un buen punto de partida
	                                  para aquellos que necesitan encontrar
					  informaci�n sobre Linux, principalmente
					  en castellano

   \item[GNU No Es Unix!]
   	\sitio{http://www.gnu.org/home.es.html}, Proyecto donde se origin�
	                                         la filosof�a del Software Libre.
						 Visita obligada para quien
						 quiera conocer que es el Software
						 Libre

   \item[La Gaceta de Linux]
   	\sitio{http://www.gacetadelinux.com/}, Traducci�n al castellano de la
	                                       conocida Gaceta inglesa \emph{Linux
					       Gazette}\footnote{La versi�n inglesa
					       puede encontrarse en:
					       \sitio{http://www.linuxgazette.com/}}.
					       El objetivo de esta publicaci�n es hacer
					       Linux un poco m�s divertido, compartir
					       ideas y descubrimientos

   \item[linux.com]
	\sitio{http://www.linux.com}, Sitio dedicado a dar informaci�n general
	                              sobre muchos aspectos del sistema

   \item[LinuxFocus]
   	\sitio{http://linuxfocus.org/}, Revista internacional y libre sobre linux. Est�
	                                traducida a muchos idiomas, entre los que se
					encuentra el espa�ol

   \item[Linux Today]
   	\sitio{http://linuxtoday.com/}, P�gina destinada a los profesionales interesados
	                                en mantener nivel de conocimiento alto sobre
					Linux y el C�digo Abierto

\end{description}
