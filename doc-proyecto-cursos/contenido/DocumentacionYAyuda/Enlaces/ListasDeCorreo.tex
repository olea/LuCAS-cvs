\porcion{Listas de correo}
\autor{\LDP}
\colaborador{\SGG}
\revisor{Leonardo Mart�nez}
\traductor{}

Las listas de correo son un recurso muy �til a la hora de solucionar
problemas, el hecho de que participe mucha gente en estas listas hace m�s
factible el recibir una respuesta a alg�n pedido o consulta que se
realice.

Las reglas de convivencia en dichas listas, aconsejan a los
participantes ayudar cada vez que se pueda, y antes de consultar algo,
leer toda la documentaci�n posible.
%(ver secci�n \ref{sec:rtfm})
De esta manera no se genera \emph{tr�fico} innecesario en la
lista.

HispaLinux\footnote{Asociaci�n de Usuarios Espa�oles de GNU/LiNUX} (\sitio{www.hispalinux.es})
dispone de un servicio de listas de correo para la comunidad de usuarios y desarrolladores
de Software Libre. Puedes obtener el listado de dichas listas de correo en
\sitio{http://listas.hispalinux.es/mailman/listinfo}, desde donde obtendr�s la informaci�n
sobre cada lista y la forma de subscribirte.

En \sitio{http://calvo.teleco.ulpgc.es/mailman/listinfo/l-linux} se encuentra disponible
la lista de correo \emph{L-Linux}. Desde la interfaz de Mailman puedes acceder al hist�rico
de la lista, obtener informaci�n sobre ella y subscribirte.

Normalmente los Grupos Locales disponen de listas de correo, si est�s interesando en alguna de ellas
visita sus p�ginas, donde obtendr�s toda la informaci�n necesaria.

% Hasta resolver el problema de los enlaces cruzados, no hago referencia al documento 'Grupos Locales'
