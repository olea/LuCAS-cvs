% Porci�n basada en la "Gu�a de instalaci�n oficial de Red Hat Linux"
% traducida al castellano. Esta gu�a est� disponible en:
% http://lucas.hispalinux.es/htmls/manuales.html

\porcion{Secciones de una P�gina del Manual}
\autor{Gu�a de instalaci�n oficial de Red Hat Linux}
\colaborador{\SGG}
\revisor{\LLC}
\traductor{}

Las p�ginas de manual proporcionan una gran cantidad de informaci�n en
muy poco espacio, por lo pueden ser dif�ciles de leer. Veamos brevemente
las secciones principales de la mayor�a de las p�ginas de manual:

\begin{description}
   \item [Nombre]
      el nombre del programa o programas documentados en la p�gina del
      manual. Puede haber m�s de un nombre, si los programas est�n
      estrechamente relacionados entre s�
   \item[Sinopsis]
      sintaxis de la �rden, mostrando brevemente todas sus opciones y
      argumentos
   \item[Descripci�n]
      una descripci�n corta de la funci�n del programa
   \item[Opciones]
      una lista de todas las opciones, con una descripci�n corta de
      cada una (a menudo combinada con la secci�n anterior)
   \item[Ver tambi�n]
      si existe, indica los nombres de otros programas que est�n
      relacionados de alguna manera con el actual
   \item[Ficheros]
      si existe, contiene una lista de ficheros que son usados y/o
      modificados por el programa
   \item[History]
      si existe, contiene hitos importantes del desarrollo del programa
   \item[Autores]
      los nombres de quienes han escrito el programa
      para abandonar info
\end{description}

Si es un reci�n llegado a GNU/Linux, no espere servirse de las p�ginas
del manual como gu�as paso a paso, ya que se entienden como material
conciso de referencia o consulta. Intentar aprender GNU/Linux utilizando
las p�ginas del manual es como pretender aprender a hablar ingl�s
leyendo un diccionario.
