\porcion{Avanzado}

\autor{\LDP}
\colaborador{\JAB}
\revisor{}
\traductor{}

\figura{Configuraciones avanzadas de ventana}{CentroDeControlVentanas-Avanzado}
De esta secci�n se ver�n solamente las opciones del cuadro superior,
(figura \ref{fig:CentroDeControlVentanas-Avanzado})
el cuadro inferior \emph{Filtros} no provee funcionalidades que valgan
la pena ver en este curso.

En el cuadro \emph{Teclado y rat�n} se podr� habilitar el uso de
\comando{Ctrl-TAB} para cambiar entre escritorios con la primer
opci�n. La segunda opci�n limita la lista de tareas a intercambiar con
\comando{Alt-TAB} del escritorio donde se est� trabajando; por
ejemplo, si en el escritorio 1 se tienen abiertos un editor de textos
y el navegador de web, y en el escritorio 2 est� abierto un
reproductor de archivos mp3, si se usa \comando{Alt-TAB} para hacer un
intercambio de tareas en el escritorio 1, la lista de tareas se
limitar� solamente al editor de textos y al navegador de web ya que el
reproductor de mp3 estar� en otro escritorio.

El \emph{Modo Alt-Tab} tiene dos opciones: KDE y CDE, si se elige la
opci�n KDE, el intercambio de tareas con \emph{Alt-TAB} se efectuar�
del mismo modo que se hace en Windows, es decir, al presionar
\emph{Alt-TAB} aparece un cuadro con la lista de tareas, y cuando se
suelta la tecla \boton{Alt} reci�n en ese momento se realiza el
intercambio. Si se selecciona el modo CDE, por cada presi�n a
\emph{Alt-TAB} se har� inmediatamente el intercambio de tareas.

La opci�n de \emph{Capturar el bot�n derecho del rat�n} provee total
control sobre el bot�n derecho del mouse al KDE, hay algunas
aplicaciones no espec�ficas de KDE que necesitan controlar
directamente el bot�n derecho del mouse, as� que habr� que tener en
cuenta esto cuando se use alguna aplicaci�n de este tipo.

