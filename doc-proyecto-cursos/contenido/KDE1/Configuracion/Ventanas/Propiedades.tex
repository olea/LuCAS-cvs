\porcion{Propiedades}

\autor{\LDP}
\colaborador{\JAB}
\revisor{}
\traductor{}

\figura{Movimiento y colocaci�n de ventanas en el escritorio}{CentroDeControlVentanas-Propiedades}

El cuadro superior \emph{Ventanas} (figura \ref{fig:CentroDeControlVentanas-Propiedades}) posee las siguientes opciones:

\begin{description}
\item[Maximizado vertical s�lo por defecto] Si esta opci�n es activada, 
las ventanas se maximizar�n solo verticalmente, manteniendo su ancho.
\item[Mostrar contenido en ventanas en movimiento] Esta opci�n, 
cuando est� activada, hace que al mover una ventana, se vea todo el
contenido de la ventana en movimiento, si se posee una m�quina no muy
potente, es aconsejable desactivar esta opci�n, as� al mover una
ventana solo veremos el recuadro en movimiento.
\item[Mostrar contenido en ventanas al redimensionar] Es la misma
 funcionalidad que la opci�n anterior, pero cuando se redimensiona una
 ventana.
\end{description}

El cuadro siguiente, \emph{Pol�tica de colocaci�n}, posee una lista
desplegable con las diferentes opciones de colocaci�n autom�tica o
manual de las ventanas que van abri�ndose a medida que se ejecutan los
programas.

El cuadro inferior, \emph{Tipo de focalizaci�n}, sirve para
seleccionar la forma en que las ventanas ``toman foco\footnote{Una
ventana tiene el foco cuando se puede interactuar con ella por teclado
y mouse, es decir cuando est� en primer plano}''

