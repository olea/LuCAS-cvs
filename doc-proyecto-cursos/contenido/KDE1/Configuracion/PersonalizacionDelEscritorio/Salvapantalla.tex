\porcion{Salvapantalla}

\autor{\LDP}
\colaborador{\JAB}
\revisor{}
\traductor{}

Esta funci�n se utiliza para configurar los distintos aspectos del
protector de pantalla del KDE. En la parte superior del cuadro (figura
\ref{fig:CentroDeControlEscritorio-Salvapantalla}) se puede observar
una representaci�n del monitor, donde aparecer� una muestra de la
animaci�n que produzca cada protector de pantalla que se selecciona en
el cuadro \emph{Salvapantalla}. Casi todos los protectores de pantalla
tienen algunos par�metros configurables, estos se modifican utilizando
el bot�n \boton{Configuraci�n...} debajo de la lista.

\figura{Protector de pantalla en KDE}{CentroDeControlEscritorio-Salvapantalla}

La barra de nivel que se encuentra en el cuadro \emph{Prioridad} se
usa para asignarle m�s o menos prioridad de procesador al protector de
pantalla, �sto es �til porque algunos protectores tienen animaciones
en tres dimensiones que necesitan de mucha prioridad para que se
visualicen correctamente.

En el cuadro \emph{Opciones} se puede establecer el tiempo de activaci�n
del protector, si al activarse bloquea la sesi�n y adem�s, que si la
bloquea, al ingresar la clave para desbloquearla, esta aparezca con
asteriscos para evitar que alguien m�s la vea al momento de tipearla.

Una caracter�stica interesante es la posibilidad de configurar las
esquinas de la pantalla para que produzcan alg�n efecto. Si se observa
con atenci�n en la parte superior, en el dibujo del monitor se puede
notar que hay unos sectores en color gris en cada esquina del
mismo. Estos sectores si se les hace un \emph{click} encima, se
despliega un men� con tres opciones:

\begin{description}
\item[Ignorar] La esquina no produce ning�n efecto
\item[Salvar Pantalla] La esquina activa el protector de pantalla
\item[Bloquear Pantalla] La esquina activa el protector y adem�s bloquea la sesi�n con contrase�a.
\end{description}

La idea es que por ejemplo si seleccionamos en la esquina superior
derecha, la opci�n \emph{Salvar Pantalla}, luego si se quiere activar
manualmente el protector, lo �nico que se debe hacer es llevar el
puntero del mouse a dicha esquina del monitor y el protector se
activar� autom�ticamente. Lo mismo pasa con el bloqueo de la pantalla.

En caso de que la sesi�n se bloquee, la contrase�a que es requerida es
la misma que se utiliza para entrar en la sesi�n.