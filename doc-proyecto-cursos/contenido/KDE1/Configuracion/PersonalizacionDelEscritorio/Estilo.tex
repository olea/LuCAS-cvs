\porcion{Estilo}

\autor{\LDP}
\colaborador{\JAB}
\revisor{}
\traductor{}

\figura{Estilo de widgets como Windows95}{CentroDeControlEscritorio-Estilo}

Otro aspecto en la personalizaci�n del escritorio, es el estilo de
dibujado de los \emph{widgets}. KDE permite elegir entre el estilo
t�pico de Windows95 y el estilo de los entornos de UNIX en
general. Podemos cambiar de estilo activando o desactivando la opci�n
\emph{Pintar widgets en el estilo Windows 95}. Se puede ver la diferencia de estilo de widgets observando las figuras \ref{fig:CentroDeControlEscritorio-Estilo} y \ref{fig:CentroDeControlEscritorio-Estilo-otro}

La opci�n que dice \emph{Barra de Men� arriba de la pantalla al estilo
de MacOS} simular�a de alguna manera el estilo de los men�es como se
utilizan en las computadoras Macintosh, es decir, en vez de que cada
ventana tenga su propia barra de men�es, existe una barra de men�es
general para todas las aplicaciones arriba de la pantalla, a medida
que se va intercambiando de ventana en ventana (de aplicaci�n en
aplicaci�n), esta barra de men�es va cambiando.

\figura{Estilo de widgets como UNIX}{CentroDeControlEscritorio-Estilo-otro}

La tercera opci�n, que dice \emph{Aplicar fuentes y colores a
aplicaciones no-KDE} sirve para darle a las aplicaciones que no son
espec�ficas de KDE (el Netscape por ejemplo), un ``look'' parecido a
las dem�s aplicaciones que si lo son, asign�ndole el color de ventanas
y botones bastante similares.

Debajo de �sto, hay un cuadro llamado \emph{Icon style} que se utiliza
para asignar el tama�o de los �conos de la barra de herramientas
(panel) y del escritorio.

