% Linux Installation and Getting Started    -*- TeX -*-
% gpl.tex
% Copyright (c) 1992-1994 by Matt Welsh <mdw@sunsite.unc.edu>
%
% This file is freely redistributable, but you must preserve this copyright 
% notice on all copies, and it must be distributed only as part of "Linux 
% Installation and Getting Started". This file's use is covered by the 
% copyright for the entire document, in the file "copyright.tex".
%
% Copyright (c) 1998 by Specialized Systems Consultants Inc. 
% <ligs@ssc.com>
%Revisado por Francisco javier fernandes <serrador@arrakis.es> el 16 de julio de 2002

\section{Acerca del Copyright}
\markboth{Introducci�n a {\linux}}{Acerca del Copyright}
\namedlabel{sec-intro-gpl}{Acerca del Copyright}

Linux est� regido por lo que se conoce como la {\em Licencia P�blica
General} de GNU, o {\em GPL, General Public License}. La GPL fue
desarrollada para el proyecto GNU por la {\em Free Software
Foundation}, que podemos traducir como ``Fundaci�n por el Software
Libre''. La licencia hace una serie de previsiones sobre la
distribuci�n y modificaci�n del ``software libre''. ``Free'' en
este sentido se refiere a libertad, y no necesariamente al coste. La GPL puede ser
interpretada de distintas formas, y esperamos que este resumen le
ayude a entenderla y c�mo afecta a Linux. Se incluye una copia
completa de la Licencia al final del libro, en el Ap�ndice~\ref{app-gpl-num}.
\index{free software}
\index{software libre}

Originalmente, Linus Torvalds lanz� Linux bajo una licencia m�s
restrictiva que la GPL, que permit�a que el software fuera libremente
distribuido y modificado, pero prohib�a su uso para ganar dinero. Sin
embargo, la GPL autoriza que la gente venda su software, aunque no le
permite restringir el derecho que su comprador tiene a copiarlo y
venderlo a su vez.

En primer lugar, hay que aclarar que el ``software libre'' de la GPL
{\em no es} software de dominio p�blico. El software de dominio
p�blico carece de {\it copyright} y pertenece literalmente al
p�blico. El software regido por la GPL s� tiene el copyright de su
autor o autores. Esto significa que est� protegido por las leyes
internacionales del copyright y que el autor del software est�
declarado legalmente. No solo porque un programa sea de libre
distribuci�n puede consider�rsele del dominio p�blico.

El software regido por la GPL tampoco es ``shareware''. Por lo
general, el ``shareware'' es propiedad del autor, y exige a los
usuarios que le paguen cierta cantidad por utilizarlo despu�s de la
distribuci�n. Sin embargo, el software que se rige por la GPL puede
ser distribuido y usado sin pagar a nadie.

La GPL permite a los usuarios modificar el software y
redistribuirlo. Sin embargo, cualquier trabajo derivado de un programa
GPL se regir� tambi�n por la GPL. En otras palabras, una compa��a
nunca puede tomar \linux, modificarlo y venderlo bajo una licencia
restringida. Si un software se deriva de \linux, �ste deber� regirse
por la GPL tambi�n.

La GPL permite distribuir y usar el software sin cargo alguno. Sin
embargo, tambi�n permite que una persona u organizaci�n gane dinero
distribuyendo el software. Sin embargo, cuando se venden programas
GPL, el distribuidor no puede poner ninguna restricci�n a la
redistribuci�n. Esto es, si usted compra un programa GPL, puede a su
vez redistribuirlo gratis o cobrando una cantidad.

Esto puede parecer contradictorio. ?`Por qu� vender software cuando la
GPL especifica que puede obtenerse gratis? Por ejemplo, supongamos que
una empresa decide reunir una gran cantidad de software GPL en un
CD-ROM y venderlo. La empresa necesitar� cobrar por el hecho de haber
producido el CD, y as�mismo querr� ganar dinero. Esto est� permitido
por la GPL.

Las organizaciones que vendan el software regido por la GPL deben
tener en cuenta algunas restricciones. En primer lugar, no pueden
restringir ning�n derecho al comprador del programa. Esto significa
que si usted compra un CD-ROM con software GPL, podr� copiar ese CD y
revenderlo sin ninguna restricci�n. En segundo lugar, los
distribuidores deben hacer saber que el software se rige por la
GPL. En tercer lugar, el vendedor debe proporcionar, sin coste
adicional, el c�digo fuente del software a distribuir. Esto permite a
cualquiera comprar el software y modificarlo a placer.

Permitir a una empresa distribuir y vender programas que son gratis es
bueno. No todo el mundo tiene acceso a Internet para conseguir los
programas, como \linux, gratis. La GPL permite a las empresas vender y
distribuir programas a esas personas que no pueden acceder al software
con un coste bajo. Por ejemplo, muchas empresas venden \linux en
disquetes o CD-ROM por correo, y hacen negocio de esas ventas. Los
desarrolladores de \linux pueden no tener constancia de estos
negocios. Por ejemplo, Linus sabe que ciertas compa��as venden \linux,
y �l no va a cobrar nada por esas ventas.

En el mundo del software libre, lo importante no es el dinero. El
objetivo es permitir desarrollar y distribuir software fant�stico
asequible a cualquiera. En la siguiente secci�n, hablaremos de c�mo
esto se aplica al desarrollo de \linux.



