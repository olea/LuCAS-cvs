% Linux Installation and Getting Started    -*- TeX -*-
% telecomm.tex
% Copyright (c) 1992-1994 by Matt Welsh <mdw@sunsite.unc.edu>
%
% This file is freely redistributable, but you must preserve this copyright 
% notice on all copies, and it must be distributed only as part of "Linux 
% Installation and Getting Started". This file's use is covered by the 
% copyright for the entire document, in the file "copyright.tex".
%
% Copyright (c) 1998 by Specialized Systems Consultants Inc. 
% <ligs@ssc.com>

\subsection{Telecomunicaciones y software para BBS}

Si dispone de un M�dem, podr� comunicarse con otras m�quinas gracias a los
paquetes de telecomunicaciones que proporciona Linux. Mucha gente usa
sus programas de telecomunicaciones para acceder a sistemas de BBS (Bulletin
Board System, Sistema de tabl�n de anuncios electr�nico), y a proveedores de
servicios en l�nea como Prodigy, Compuserve America Online. La gente utiliza
el m�dem para conectarse a los sistemas UNIX del trabajo o el centro educativo.
Con el m�dem se pueden enviar y recibir faxes.

Un conocido paquete de comunicaciones para Linux es {\tt seyon,} que 
nos proporciona un interfaz de usuario c�modo y configurable bajo X~Window, y 
que lleva incluido el soporte para los protocolos de transferencia de ficheros
Kermit y Z-Modem. Otros programas de telecomunicaciones son C-Kermit, {\tt pcomm}
y {\tt minicom}. Son parecidos a los programas de telecomunicaciones disponibles
para otros sistemas operativos, y resultan bastante f�ciles de utilizar.

Si no tiene acceso a un servidor SLIP o PPP (v�ase la secci�n anterior)
puede utilizar {\tt term} para multiplexar su l�nea serie. El programa
{\tt term} le hace posible abrir m�s de una sesi�n de login sobre una conexi�n 
por m�dem. Le permite redirigir conexiones de un cliente X a su servidor X local
a trav�s de una l�nea serie. Otro paquete de software, KA9Q, implementa un interfaz
parecido, estilo SLIP.

Ser un SySop de una BBS fue en tiempos una afici�n predilecta y una forma
de obtener ingresos para mucha gente. {\linux} soporta una amplia gama de 
software para BBS, que en general es mucho m�s potente que el disponible
para otros sistemas operativos. Con una l�nea telef�nica, un m�dem y
{\linux}, puedes transformar tu sistema en una BSS y proporcionar acceso
a los usuarios de todo el mundo. Entre los programas de BBS para linux  
est�n XBBS y UniBoard BBS.

La mayor�a de programas de BBS constri�en al usuario a un sistema de men�s
en el que s�lo est�n disponibles determinadas funciones y aplicaciones.
Una alternativa al acceso por BBS es el acceso UNIX completo, que permite
al usuario llamar a tu sistema y autentificarse normalmente. Esto �ltimo 
requiere de una nada despreciable tarea de administraci�n por parte del 
administrador, pero no es dif�cil proporcionar acceso p�blico a UNIX. 
Adem�s de la red TCP/IP, puede hacer posible el acceso al correo y las 
noticias en su sistema.

Si no dispone de acceso a una red TCP/IP o de una pasarela UUCP,
{\linux} le permite todav�a comunicarse con redes de BBS como Fidonet, que le
permiten intercambiar correo y noticias a trav�s de una l�nea telef�nica.
Para m�s informaci�n sobre telecomunicaciones y software de BBS bajo {\linux}, 
v�ase el Cap�tulo~\ref{chap-networking}.


