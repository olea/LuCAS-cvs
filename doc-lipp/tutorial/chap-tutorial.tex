% Linux Installation and Getting Started    -*- TeX -*-
% chap-tutorial.tex
% Copyright (c) 1992, 1993 by Matt Welsh, Larry Greenfield and Karl Fogel
%
% This file is freely redistributable, but you must preserve this copyright 
% notice on all copies, and it must be distributed only as part of "Linux 
% Installation and Getting Started". This file's use is covered by
% the copyright for the entire document, in the file "copyright.tex".

% A little bit of convention: Use the tscreen environment for all 
% examples. ALL of them. If you must nest verbatim environment inside,
% fine (see the listing of /etc below). However, inside the tscreen
% environment, default font is \tt. Use \em for commands which the user
% types. Use the \key macro for keypresses. Example:
% \begin{tscreen}
% /home/larry/foo\# {\em cp foo bar} \\
% /home/larry/foo\#
% \end{tscreen}
% You MUST use "\\" on the end of every line in the example, except the
% last (don't use it on the last line--it will add unwanted vertical space.
% Also the last line of every example must be the prompt, to indicate
% that output from the command is finished.
%Revisi�n por Francisco Javier Fern�ndez
%\chapter{Linux Tutorial}\label{chap-tutorial}
\chapter{Tutorial de \linux}
\label{chap-tutorial}
\label{chap-tutorial-num}
\markboth{Tutorial de Linux}{}

% \linux Installation and Getting Started    -*- TeX -*-
% tut-intro.tex
% Copyright (c) 1992, 1993 by Matt Welsh, Larry Greenfield and Karl Fogel
%
% This file is freely redistributable, but you must preserve this copyright 
% notice on all copies, and it must be distributed only as part of "\linux 
% Installation and Getting Started". This file's use is covered by
% the copyright for the entire document, in the file "copyright.tex".
%
% Copyright (c) 1998 by Specialized Systems Consultants Inc. 
% <ligs@ssc.com>
%Revisi�n 1 por Francisco Javier Fernandes <serrador@arrakis.es>
%gold
\section{Introducci�n}

Si es nuevo en UNIX y {\linux}, puede que est� un poco intimidado por el tama�o y
la aparente complejidad del sistema que tiene delante suya.
Este cap�tulo no profundiza en muchos detalles ni cubre
temas avanzados. Por contra, queremos que aterrice corriendo.

Aqu� se asume que posee pocos conocimientos, salvo quiz�s
algo de familiaridad con ordenadores personales, y MS-DOS. Sin embargo, incluso 
si no es un usuario de MS-DOS, deber�a ser capaz de entender todo esto. A primera 
vista, {\linux} se parece mucho a MS-DOS---despu�s de todo, hay partes de MS-DOS 
basadas en el sistema operativo CP/M, que a su vez se basaba en UNIX. Sin embargo,
s�lo las caracter�sticas m�s superficiales de {\linux} se parecen a MS-DOS. Incluso 
si es completamente nuevo en el mundo del PC, este tutorial deber�a serle de ayuda.

Y, antes de que comencemos: {\em No tenga miedo a experimentar \/} El sistema
no le morder�. No se puede destrozar todo s�lo por trabajar con el sistema.
{\linux} tiene incorporadas caracter�sticas de seguridad para evitar que usuarios 
''normales''da�en ficheros que sean imprescindibles para el sistema. Aun as�, 
lo peor que puede ocurrir es que borre todos o algunos de sus ficheros y tenga que
reinstalar el sistema. As� que, en este punto, no tiene nada
que perder.



% {\linux} Installation and Getting Started    -*- TeX -*-
% concepts.tex
% Copyright (c) 1992, 1993 by Matt Welsh, Larry Greenfield and Karl Fogel
%
% This file is freely redistributable, but you must preserve this copyright 
% notice on all copies, and it must be distributed only as part of "{\linux} 
% Installation and Getting Started". This file's use is covered by
% the copyright for the entire document, in the file "copyright.tex".
%
% Copyright (c) 1998 by Specialized Systems Consultants Inc. 
% <ligs@ssc.com>

\section{Conceptos b�sicos de {\linux}}
%\section{Basic {\linux} concepts.}
\markboth{Tutorial de {{\linux}}}{Conceptos B�sicos de {{\linux}}}
\index{{\linux}!conceptos b�sicos|(}
\index{{\linux}!multitarea!definici�n}
\index{multitarea!definici�n}
\index{multiusuario!definici�n}
\index{login!definici�n}
\index{contrase�a!definici�n}
{\linux} es un sistema operativo multitarea y multiusuario, lo que significa que 
mucha gente puede ejecutar diferentes aplicaciones en un ordenador al mismo tiempo.
En esto se diferencia de MS-DOS, donde s�lo una persona puede usar el sistema en un 
momento dado. Bajo {\linux}, para identificarse en el sistema, debe registrarse ``{\bf log in}'', lo 
que requiere que introduzca su nombre de usuario {\bf login name} (el nombre que el sistema usa para 
identificarle), y que introduzca su {\bf password}, que es su contrase�a personal 
para acceder a su cuenta. Como s�lo usted conoce su contrase�a, nadie m�s puede 
acceder al sistema con su nombre.

En los sistemas UNIX tradicionales, los administradores del sistema le asignan un 
nombre de usuario y una contrase�a inicial cuando se le da una cuenta en el sistema.
De todos modos, como en {\linux} {\tt usted} es el administrador del sistema, usted 
debe poner en marcha su propia cuenta antes de que pueda acceder a ella.
Para las pr�ximas discusiones, usaremos el nombre de usuario imaginario, ``{\tt 
larry}.''

\index{hostname!definici�n}
Adem�s, cada sistema tiene asignado un {\bf nombre de host }. Este nombre de  ''host'' le da
nombre a la m�quina adem�s de car�cter y encanto. El ''host name'' se usa para 
identificar m�quinas individuales en una red, pero incluso si su m�quina no est� 
conectada a una, deber�a tener un nombre ''host''. Para los ejemplos siguientes, el nombre
 ''host'' de la m�quina es ``{\tt mousehouse}''.

%\subsection{Creating an account.}
\subsection{Creaci�n de una cuenta}
\label{sec-create-account}
\index{cuenta de usuario!creaci�n}
\index{cuantas!creaci�n}
Antes de que pueda usar un sistema {\linux} reci�n instalado,  debe configurar 
una cuenta para s� mismo. No suele ser una buena idea usar la cuenta {\tt root} 
para un uso diario; deber�a reservar la cuenta {\tt root} para ejecutar 
�rdenes privilegiados y para el mantenimiento del sistema, como veremos m�s 
adelante. 

Para crear una cuenta para usted mismo, acceda al sistema como {\tt root} y use  el 
orden {\tt useradd} o {\tt adduser}. Vea Section~\ref{sec-add-user} para 
informaci�n acerca de este procedimiento.

\subsection{Registrarse en el sistema}
\index{logging in}
\index{login}
\index{entrada al sistema}
\index{registrarse en el sistema}
A la hora de entrar en el sistema, ver� algo como esto:

\begin{tscreen}
mousehouse login:
\end{tscreen}

Introduzca su nombre y pulse la tecla \key{Enter}. Nuestro h�roe, {\tt larry}, 
escribir�a:

\begin{tscreen}
mousehouse login: larry \\
Password: 
\end{tscreen}

Seguidamente, introduzca su contrase�a. Los car�cteres que introduzca no ser�n 
mostrados en la pantalla, as� que escriba con cuidado. Si se equivoca con la 
contrase�a, ver�:

\begin{tscreen}
Login incorrect
\end{tscreen}

y tendr� que probar de nuevo.

Una vez que haya introducido correctamente su nombre y contrase�a, usted ha entrado 
oficialmente en el sistema, y podr� comenzar a trabajar.

\subsection{Consolas Virtuales}
\index{consola!virtual}
\index{consola!definici�n}
La {\bf consola} del sistema es el monitor y el teclado conectados directamente al 
sistema (debido a que {\linux} es un sistema operativo multiusuario, puede tener otros
terminales conectados a los puertos serie de su sistema, pero �stos no constituir�n 
la consola). {\linux}, como otras versiones de UNIX, facilita el acceso a {\bf consolas 
virtuales} (o CVs), que le permiten tener m�s de una sesi�n en la consola a la vez.

Para comprobar esto, entre en el sistema. Entonces, pulse \key{Alt-F2}. Deber�a ver 
{\tt login:} de nuevo. Usted est� viendo la segunda consola virtual. Para cambiar a 
la primera CV, pulse \key{Alt-F1}. {\em Voila!\/}. Ha vuelto a su primera sesi�n.

Un sistema {\linux} reci�n instalado le permite acceder s�lo a las primeras seis (m�s 
o menos) CVs, presionando de \key{Alt-F1} hasta \key{Alt-F6}, o hasta cuantas CVs 
est�n configuradas en su sistema. Es posible habilitar hasta 12 CVs ---una para 
cada tecla de funci�n en su teclado. Como puede ver, el uso de CVs puede ser muy 
poderoso porque puede trabajar en diferentes sesiones a la vez.

Aunque el uso de CVs es algo limitado (despu�s de todo, s�lo puede ver una CV a la 
vez), deber�a permiterle hacerse una idea de las capacidades multiusuario de {\linux}. 
Mientras est� trabajando en la primera CV, puede cambiar a la segunda CV y trabajar 
en otra cosa diferente. 

\subsection{Int�rpretes de �rdenes y �rdenes}
\label{sec-shells-cmds}
\index{shells!definici�n}
\index{int�rprete de �rdenes!definici�n}
Para la mayor�a de sus exploraciones en el mundo de {\linux}, usted le hablar� al 
sistema a trav�s de un {\bf shell (int�rprete de �rdenes)}, un programa que recibe 
las �rdenes que escribe y los traduce en instrucciones al sistema operativo. Esto 
se puede comparar al programa {\tt COMMAND.COM} de MS-DOS, que hace esencialmente lo
mismo. Un int�rprete de �rdenes es �nicamente un interfaz para {\linux}. Hay muchos 
interfaces disponibles ---como el sistema X Window, que le permite ejecutar �rdenes 
usando el rat�n y el teclado.

A la vez que entra en el sistema, �ste inicia el int�rprete de �rdenes, y usted ya 
puede comenzar a introducir �rdenes. Aqu� tenemos un ejemplo r�pido. Larry entra 
en el sistema y el indicador `` {\bf indicador de �rdenes}'' del int�rprete de �rdenes queda a la espera de 
�rdenes.

\begin{tscreen}
mousehouse login: larry \\
Password: larry's password \\
Welcome to Mousehouse! \\
\\
/home/larry\# 
\end{tscreen}

\index{shells!indicador de �rdenes}
\index{int�rprete de �rdenes!indicador}
La �ltima l�nea de este texto es el indicador del int�rprete de �rdenes, comunicando 
que est� listo para recibir �rdenes. (M�s adelante veremos m�s cosas acerca del 
indicador).
Probemos a decirle al sistema que haga algo interesante:

\begin{tscreen}
/home/larry\# make love \\
make: *** No way to make target `love'. Stop. \\
/home/larry\#
\end{tscreen}

Bien, como podemos ver, {\tt make} es el nombre de un programa que hay en el 
sistema, y el int�rprete de �rdenes ejecut� este programa cuando le dimos la 
orden. (Desgraciadamente, el sistema se mostr� antip�tico.)

\index{orden!definici�n}
\index{orden!argumento!definici�n}
\index{argumento!orden!definici�n}
Esto nos lleva a la siguiente pregunta: �Qu� es una orden? �Qu� ocurre cuando
 escribe ``{\tt make love}''? La primera palabra en la l�nea de �rdenes, 
''{\tt make}'', es el nombre de la orden que debe ser ejecutada. Todo lo 
dem�s en la l�nea de �rdenes se toma como argumentos para esta orden. 
Ejemplo:

\begin{tscreen}
/home/larry\# cp foo bar 
\end{tscreen}

El nombre de esta orden es ``{\tt cp}'', y los argumentos son ``{\tt foo}'' y 
{\tt bar}''.

Cuando introduce una orden, el int�rprete de �rdenes hace varias cosas. Primero, 
comprueba la orden para ver si es interno al int�rprete de �rdenes. (Es decir, 
una orden que el int�rprete de �rdenes sabe ejecutar por s� mismo. Hay muchas de 
estas �rdenes, y las veremos m�s adelante). El int�rprete de �rdenes tambi�n 
comprueba si la orden es un alias, o nombre sustitutorio, para otra orden. Si 
no se cumple ninguna de estas dos condiciones, el int�rprete de �rdenes busca un 
programa, en el disco, que tenga el nombre especificado. 
Si tiene �xito, el int�rprete de �rdenes ejecuta el programa, mand�ndole los 
argumentos especificados en la l�nea de �rdenes.

En nuestro ejemplo, el int�rprete de �rdenes busca un programa llamado {\tt make}, 
y lo ejecuta con el argumento {\tt love}. La orden {\tt make} es un programa usado
a menudo para compilar grandes programas, y toma como argumentos el nombre de un 
''objetivo''para compilar. En el caso de ''{\tt make love}'', le dijimos a {\tt 
make} que compilara el objetivo {\tt love}. Como {\tt make} no puede encontrar un 
objetivo con ese nombre, falla con un divertido mensaje de error, y nos lleva de 
nuevo al indicador del int�rprete de �rdenes.

\index{mensaje de error orden no encontrada@mensaje de error {\tt command not found}}
\index{mensajes de error!orden no encontrada@mensajes de error!{\tt command not found}}
�Qu� ocurre si escribimos una orden en el int�rprete de �rdenes y �ste  
no puede encontrar un programa que tenga el nombre especificado? Bien, 
podemos probar lo siguiente:

\begin{tscreen}
/home/larry\# eat dirt \\
eat: command not found \\
/home/larry\#
\end{tscreen}

Bastante simple, si el sistema no puede encontrar un programa con el nombre dado en 
la l�nea de �rdenes (aqu�, ''{\tt eat}''), imprime un mensaje de error. A menudo se
 encontrar� con ese mensaje de error si se equivoca con la orden (por ejemplo, si 
hubiera escrito ``{\tt mkae love}'' en lugar de ``{\tt make love}'').

%\subsection{Logging out.}
\subsection{Salida del sistema}
\index{logging out!con orden exit@con orden {\tt exit}}
\index{exit@{\tt exit}}
\index{salir}
Antes de que sigamos, deber�amos decirle c�mo salir del sistema. En el indicador del 
int�rprete de �rdenes, use la orden

\begin{tscreen}
/home/larry\# exit
\end{tscreen}

para salir del sistema. Hay otras formas de salir, pero �sta es la m�s facilita.

\subsection{Cambiar la contrase�a}
\index{contrase�a!cambio de la contrase�a con passwd@cambiando con {\tt passwd}}
\index{passwd@{\tt passwd}}
Tambi�n deber�a saber c�mo cambiar su contrase�a. El orden {\tt passwd} le pide su
 antigua contrase�a y una nueva. Adem�s le pide que vuelva a introducir la nueva 
contrase�a para darla por v�lida. Tenga cuidado de no olvidar su contrase�a---si lo 
hace, tendr� que pedirle al administrador del sistema que la reinicie por usted. (Si
 usted es el administrador del sistema, vea la p�gina~\pageref{sec-manage-users}.)

\subsection{Ficheros y directorios}
\index{ficheros!definici�n}
Bajo la mayor�a de sistemas operativos (incluyendo {\linux}), existe el concepto de fichero, que es simplemente un conjunto de informaci�n con un nombre (llamado {\bf nombre de fichero}). 
Ejemplos de ficheros podr�an ser su examen de historia, un correo electr�nico (e-mail), o un programa que pueda ser ejecutado. B�sicamente, cualquier cosa almacenada en el disco es guardado en 
un fichero individual. 

\index{nombre de fichero!definici�n}
\index{fichero!nombre|(}
Los ficheros se identifican por sus nombres de fichero. Por ejemplo, el fichero que contiene su examen de historia podr�a estar almacenado con el nombre de fichero {\tt history-paper}. Estos nombres 
normalmente identifican el fichero y su contenido de una forma que tenga alg�n significado para usted. No existe ning�n formato est�ndar para los nombres de fichero, al contrario de lo que ocurre 
bajo MS-DOS y algunos otros sistemas operativos; en general, un nombre de fichero puede contener cualquier caracter (excepto el car�cter {\tt /}---vea la discusi�n acerca de los nombres de las rutas, 
m�s adelante) y est� limitado a 256 car�cteres de longitud.

\index{directorio!definici�n}
Junto con el concepto de ficheros tenemos el concepto de directorios. Un {\bf directorio} es una colecci�n de ficheros. Puede entenderse como una ''carpeta'' que contiene muchos ficheros diferentes. 
Los directorios tienen nombres, con los que se les identifica. Adem�s, los directorios se mantienen en una estructura de tipo �rbol; es decir, los directorios pueden contener otros directorios.

\index{ruta!definici�n}
\index{nombre de ruta!definici�n}

Por tanto, puede referirse a un fichero por su {\bf nombre de ruta}, que est� compuesto del nombre del fichero, precedido por el nombre del directorio que lo contiene. Por ejemplo, supongamos que 
Larry tiene un directorio llamado {\tt papers}, que contiene tres ficheros: {\tt history-final}, {\tt english-lit}, y {\tt masters-thesis}. Cada uno de estos tres ficheros contiene informaci�n para 
tres proyectos de Larry. Para referirnos al fichero {\tt english-lit}, Larry puede especificar el nombre de la ruta del fichero:

\begin{tscreen}
papers/english-lit
\end{tscreen}

\index{/@{\tt /}!en nombres de ruta}
Como puede ver, el directorio y el nombre del fichero est�n separados por una �nica barra ({\tt /}). Por esta raz�n, los nombres de los ficheros no pueden contener el car�cter {\tt /}. 
Los usuarios de MS-DOS encontrar�n familiar est� convenci�n, aunque en el mundo de MS-DOS se usa la barra invertida (\verb+\+) en su lugar.

\index{directorio!anidamiento}
Como ya mencionamos, los directorios pueden anidarse unos en otros. Por ejemplo, supongamos que hay otro directorio en {\tt papers}, llamado {\tt notes}. El directorio {\tt notes} contiene los ficheros 
{\tt math-notes} y {\tt cheat-sheet}. El nombre de la ruta del fichero {\tt cheat-sheet} ser�a 

\begin{tscreen}
papers/notes/cheat-sheet
\end{tscreen}

\index{directorio!padre}
\index{directorio padre}
Por tanto, el nombre de la ruta es realmente como la ruta hasta el fichero. El directorio que contiene un subdirectorio dado es conocido como {\bf directorio padre}. En nuestro caso, el directorio
 {\tt papers} es el padre del directorio {\tt notes}. 

\subsection{El �rbol de directorios}
\index{directorio!�rbol}
\index{directorio!estructura}
\index{directorio!ra�z!definici�n}
\index{/@{\tt /}!nombre del directorio ra�z}
\index{GNU/Linux!estructura de directorio}
La mayor�a de los sistemas {\linux} usa una distribuci�n de ficheros est�ndar para los ficheros de forma que los recursos del sistema y los programas puedan ser f�cilmente localizados. 
Esta distribuci�n forma el �rbol de directorios, que comienza en el directorio ``{\tt /}'', tambi�n conocido como ''directorio ra�z''. Directamente debajo de {\tt /} est�n importantes
subdirectorios: {\tt /bin}, {\tt /etc}, {\tt /dev}, y {\tt /usr}, entre otros. Estos directorios contienen otros directorios que contienen ficheros de configuraci�n del sistema, programas, etc�tera.

\index{directorio!inicial!definici�n}
En particular, cada usuario tiene un {\bf directorio de usuario}, que es el directorio preparado para que el usuario almacene sus propios ficheros. En los ejemplos de arriba, todos los ficheros de Larry 
(como {\tt cheat-sheet} y {\tt history-final}) est�n contenidos en el directorio de usuario de Larry. Normalmente, los directorios de usuario est�n contenidos bajo {\tt /home}, y se nombran con el 
nombre de usuario al que pertenecen. El directorio de usuario de Larry es {\tt /home/larry}.

\iflotex  % No dirtree if ASCII.
\else {

El diagrama en la P�gina~\pageref{dirtree} muestra un �rbol de directorios de ejemplo, que podr�a darle una idea de c�mo se organiza el �rbol de directorios de su sistema.

%\unitlength=1.0pt
%\begin{figure}[tb]
\begin{figure}
%\centerline{\epsffile{tutorial/dirtree.eps}}
\centerline{\includegraphics{tutorial/dirtree}}
\label{dirtree}
%\caption{A typical (abridged) {\linux} directory tree.}
\caption{Un t�pico �rbol de directorio {\linux} (resumido).}
\end{figure} } \fi % Kill dirtree if ASCII.

\subsection{Directorio de trabajo actual}
\index{directorio!de trabajo actual!definici�n}
\index{directorio de trabajo!definici�n}
En cualquier momento, se asume que los �rdenes que introduce se refieren a su {\bf directorio de trabajo actual}. Puede entender directorio de trabajo como el directorio en el que ''se encuentra'' 
en ese momento. Cuando accede por primera vez al sistema, su directorio de trabajo se configura como su directorio de usuario ---{\tt /home/larry}, en nuestro caso. Cuando haga referencia a un fichero,
puede referirse a �l en relaci�n a su directorio de trabajo actual, en vez de especificar el nombre de la ruta completa del fichero.

Aqu� tenemos un ejemplo. Larry tiene el directorio {\tt papers}, y {\tt papers} contiene el fichero {\tt history-final}. Si Larry quiere ver el contenido de este fichero, puede usar la orden

\begin{tscreen}
/home/larry\# more /home/larry/papers/history-final
\end{tscreen}

La orden {\tt more} simplemente muestra por pantalla un fichero, pantalla a pantalla. Como el directorio de trabajo actual de Larry es {\tt /home/larry}, se puede referir al fichero {\em en relaci�n\/} con su localizaci�n actual usando la orden

\begin{tscreen}
/home/larry\# more papers/history-final
\end{tscreen}

\index{ruta!relativa}
Si comienza el nombre del fichero (como {\tt papers/final}) con un car�cter diferente de {\tt /}, se est� refiriendo al fichero en t�rminos relativos a su directorio de trabajo actual. Esto se conoce como {\bf nombre de ruta relativo}.

\index{ruta!completa}
\index{ruta!absoluta}
Por otra parte, si comienza el nombre del fichero con una {\tt /}, el sistema lo interpreta como el nombre de la ruta absoluta ---es decir, un nombre de ruta que incluye la ruta completa hasta el fichero, comenzando en el directorio ra�z, {\tt /}. Esto se conoce como {\bf nombre de ruta absoluta}.

\subsection{Refiri�ndose al directorio inicial}
\index{directorio!inicial!\~ para referirse a@{\tt \~{}} para referise a}
\index{\~@{\tt \~{}}!para referirse al directorio inicial}
\index{home}
\index{directorio inicial}
\index{tilde}
\index{virgulilla}
Bajo {\tt tcsh} y {\tt bash}\footnote {{\tt tcsh} y {\tt bash} son dos {\em int�rprete de �rdeness} que funcionan bajo {\linux}. 
El int�rprete de �rdenes es un programa que lee los �rdenes del usuario y los ejecuta; la mayor�a de los sistema {\linux} habilitan 
{\tt tcsh} o {\tt bash} para las cuentas de los nuevos usuarios.}
puede especificar su directorio de usuario con el car�cter de la virgulilla\NT{tilde en ingl�s} ({\tt \~{}}). Por ejemplo, la orden

\begin{tscreen}
/home/larry\# more \~{}/papers/history-final
\end{tscreen}

es equivalente a

\begin{tscreen}
/home/larry\# more /home/larry/papers/history-final
\end{tscreen}

El int�rprete de �rdenes reemplaza el car�cter {\tt \~{}} por el nombre de su directorio de trabajo. 

Puede especificar tambi�n los directorios de usuario de otros usuarios con el car�cter virgulilla. La ruta {\tt \~{}karl/letters} es expandido
a {\tt /home/karl/letters} por el int�rprete de �rdenes (si {\tt /home/karl} es el directorio de usuario de Karl). El uso de la virgulilla es simplemente un atajo; no existe ning�n directorio llamado {\tt \~{}}---es s�lo una ayuda proporcionada por el int�rprete de �rdenes.

\index{{\linux}!conceptos b�sicos|)}

% Linux Installation and Getting Started    -*- TeX -*-
% basic.tex
% Copyright (c) 1992, 1993 by Matt Welsh, Larry Greenfield and Karl Fogel
%
% This file is freely redistributable, but you must preserve this copyright 
% notice on all copies, and it must be distributed only as part of "Linux 
% Installation and Getting Started". This file's use is covered by
% the copyright for the entire document, in the file "copyright.tex".
%
% Copyright (c) 1998 by Specialized Systems Consultants Inc. 
% <ligs@ssc.com>
%Revision 1 realizada el 9 de julio de 2002 por Fco. Javier Fernandez <serrador@arrakis.es>
%Revisi�n 2 Realizada el 17 de julio 2002 por Francisco Javier Fern�ndez <serrador@arrakis.es>

%\section{First steps into Linux.}
\section{Primeros Pasos en {\linux}}
\markboth{Tutorial {\linux }}{Primeros pasos en {\linux}}
Antes de que empecemos, es importante saber que todos 
los nombres de ficheros y �rdenes en un sistema {\linux} 
son {\bf case-sensitive} (que diferencian entre 
may�sculas y min�sculas a diferencia de sistemas 
operativos como MS-DOS). Por ejemplo, la orden {\tt make} 
es muy diferente de {\tt Make} o {\tt MAKE}. Lo mismo se 
cumple para nombres de ficheros y directorios.


\subsection{Movi�ndonos por la estructura de directorios.}
%\subsection{Moving around.}
Ahora que puede entrar en el sistema y que sabe c�mo referenciar los ficheros usando las rutas de los mismos, �c�mo puede cambiar el directorio de trabajo actual, para hacer la vida m�s f�cil?

\index{directorio!estructura!movi�ndose por ella con cd@movi�ndose por ella con cd {\tt cd}}
\index{cd@{\tt cd}|(}
La orden para moverse por la estructura de directorios es {\tt cd}, que es una abreviatura de ''cambiar directorio''.
Muchas de las �rdenes m�s usadas en {\linux} son de dos o tres letras.
La forma de usar la orden {\tt cd} es

\begin{tscreen}
cd \cparam{directorio}
\end{tscreen}

donde  \textsl{directorio} es el nombre del directorio que quiere que se convierta en el directorio de trabajo actual. 

Como se mencion� antes, cuando entra en el sistema, comienza en su directorio de usuario. Si Larry quisiera cambiar al subdirectorio {\tt papers}, usar�a la orden

\begin{tscreen}
/home/larry\# cd papers \\
/home/larry/papers\#
\end{tscreen}

Como puede ver, el indicador de Larry cambia para reflejar 
su directorio de trabajo actual (de forma que sabe d�nde 
se encuentra). Ahora que est� en el directorio  {\tt papers}, 
puede ver history-final con la orden

\begin{tscreen}
/home/larry/papers\# more history-final
\end{tscreen}

\index{directorio!padre!.. para referirse al@{\tt ..} para referirse al}
\index{directorio!. para referirse al@{\tt .} para referirse al}
\index{directorio padre!.. para referirse a@{\tt ..} para referirse a}

Ahora, Larry est� atascado en el subdirectorio {\tt papers}. Para regresar al directorio superior (o padre), ejecute la orden 

\begin{tscreen}
/home/larry/papers\# cd \ .. \\
/home/larry\#
\end{tscreen}

(Observe los espacios entre ``{\tt cd}'' y ``{\tt ..}''.)
Cada directorio tiene una entrada llamada ``{\tt ..}'' que se refiere al directorio padre. De forma similar, cada directorio tiene una entrada llamada ``{\tt .}''que se refiere a s� mismo. Por tanto, la orden 

\begin{tscreen}
/home/larry/papers\# cd \ .
\end{tscreen}

no lleva a ninguna parte. 

Con la orden {\tt cd} se pueden usar tambi�n rutas absolutas.
Para  cambiar  al directorio de usuario de Karl, se puede usar la orden

\begin{tscreen}
/home/larry/papers\# cd /home/karl \\
/home/karl\#
\end{tscreen}

Adem�s, la orden {\tt cd} sin argumentos le llevar� a su propio directorio de usuario. 

\begin{tscreen}
/home/karl\# cd \\
/home/larry\#
\end{tscreen}

\index{cd@{\tt cd}|)}
%\subsection{Looking at the contents of directories.}
\subsection{Mirando el contenido de los directorios}
\label{sec-ls}
\index{directorio!listar los contenidos de|(}
\index{ls@{\tt ls}|(}
\index{listando los contenidos de(}
\index{ficheros!listado|(}

Ahora que sabe c�mo moverse por los directorios, podr�a pensar, �y qu�?. Dar vueltas por los directorios no tiene mucho sentido por s� mismo, as� que introduzcamos una orden nueva,
 {\tt ls}. La orden {\tt ls} muestra un listado de ficheros y directorios, por omisi�n del directorio actual. Por ejemplo:

\begin{tscreen}
/home/larry\# ls \\
Mail \\
letters \\
papers \\
/home/larry\#
\end{tscreen}

Aqu� podemos darnos cuenta de que Larry tiene tres entradas en su directorio actual:  {\tt Mail}, {\tt letters}, y {\tt papers}. Esto no nos dice mucho -- ?` qu� son, directorios o ficheros? 
Podemos usar la opci�n {\tt -F} de la orden {\tt ls} para obtener informaci�n m�s detallada.

\begin{tscreen}
/home/larry\# ls\ --F \\
Mail/ \\
letters/ \\
papers/ \\
/home/larry\#
\end{tscreen}


Por la {\tt /} que aparece en cada nombre, sabemos que estas tres entradas son, de hecho, subdirectorios.

\index{ejecutable!definici�n}
\index{fichero!ejecutable!definici�n}

Ejecutando {\tt ls -F} puede tambi�n aparecer un ``{\tt *}'' al final de un nombre en la lista resultante, lo que indicar�a que el fichero es un {\bf ejecutable} o un programa que puede ejecutarse. 
Si no aparece nada al final de un nombre al usar {\tt ls -F}, el fichero es un ``plain old file'', es decir, no es ni un directorio ni un ejecutable.  

En general, cada orden UNIX puede tomar un cierto n�mero de opciones adem�s de otros argumentos. Estas opciones normalmente comienzan con un ``{\tt -}'', como se vi� arriba con la opci�n {\tt -F}.
 La opci�n {\tt -F} le dice a  {\tt ls} que d� m�s informaci�n acerca del tipo de ficheros involucrados --en nuestro caso, imprimiendo una {\tt /} despu�s de cada nombre de directorio. 

Si le da a {\tt ls} el nombre de un directorio, el sistema listar� los contenidos de ese directorio. 

\begin{tscreen}
/home/larry\# ls\ --F papers \\
english-lit \\
history-final \\
masters-thesis \\
notes/ \\
/home/larry\#
\end{tscreen}

O, para un listado m�s interesante, veamos qu� hay en el directorio de sistema {\tt /etc}. 

\begin{tscreen}
/home/larry\# ls /etc 
\begin{verbatim}
Images          ftpusers        lpc             rc.new          shells
adm             getty           magic           rc0.d           startcons
bcheckrc        gettydefs       motd            rc1.d           swapoff
brc             group           mount           rc2.d           swapon
brc~            inet            mtab            rc3.d           syslog.conf
csh.cshrc       init            mtools          rc4.d           syslog.pid
csh.login       init.d          pac             rc5.d           syslogd.reload
default         initrunlvl      passwd          rmt             termcap
disktab         inittab         printcap        rpc             umount
fdprm           inittab.old     profile         rpcinfo         update
fstab           issue           psdatabase      securetty       utmp
ftpaccess       lilo            rc              services        wtmp
/home/larry#
\end{verbatim}\end{tscreen}

Si es un usuario de MS-DOS, puede que se d� cuenta de que los nombres de ficheros pueden ser mayores de 8 caracteres, y pueden contener puntos en cualquier posici�n. 
Puede incluso usar m�s de un punto en un nombre de fichero.

Vayamos a la parte superior del �rbol de directorios, y luego bajemos a otro directorio con las �rdenes

\begin{tscreen}
/home/larry\# cd .. \\
/home\# cd .. \\
/\# cd usr \\
/usr\# cd bin \\
/usr/bin\#
\end{tscreen}

Puede tambi�n moverse a los directorio en un s�lo paso, haciendo {\tt cd /usr/bin}.

Pruebe a moverse por varios directorios, usando {\tt ls} y {\tt cd}. 
En algunos casos, puede que le aparezca el mensaje de error ``{\tt Permission denied}''\NT{Permiso denegado}. Esto es debido simplemente ``al sistema de seguridad UNIX'':
para poder usar las �rdenes {\tt ls} o {\tt cd}, debe tener permisos para hacerlo. Hablaremos m�s acerca de esto en
 \index{directorio!listado de los contenidos de|)}
 \index{ls@{\tt ls}|)}
 \index{listando los contenidos del directorio|)}
 \index{ficheros!listado|)}
la pagina~\pageref{sec-file-perms}.

%\subsection{Creating new directories.}
\subsection{Creaci�n de directorios nuevos}

\index{directorio!creaci�n}
\index{mkdir@{\tt mkdir}}
Es hora de aprender c�mo crear directorios. Esto requiere el uso de la orden {\tt mkdir}. Pruebe lo siguiente:

\begin{tscreen}
/home/larry\# mkdir foo \\
/home/larry\# ls -F \\
Mail/ \\
foo/ \\
letters/ \\
papers/ \\
/home/larry\# cd foo \\
/home/larry/foo\# ls \\
/home/larry/foo\# 
\end{tscreen}

!`Felicidades! Ha creado un nuevo directorio y se ha metido en �l. Como no hay 
ficheros en este nuevo directorio, aprendamos c�mo copiar ficheros de un lugar a otro.

%\subsection{Copying files.}
\subsection{Copiando ficheros}

\index{ficheros!copiar}
\index{copiar ficheros}
\index{cp@{\tt cp}}

Para copiar ficheros, use el orden {\tt cp}, como se muestra aqu�:

\begin{tscreen}
/home/larry/foo\# cp /etc/termcap\ \ . \\
/home/larry/foo\# cp /etc/shells\ \ . \\
/home/larry/foo\# ls\ --F \\
shells\ \ \ \ \ termcap  \\
/home/larry/foo\# cp shells bells \\
/home/larry/foo\# ls\ --F \\
bells\ \ \ \ \ shells\ \ \ \ \ termcap \\
/home/larry/foo\# 
\end{tscreen}

La orden {\tt cp} copia los ficheros escritos en la l�nea de �rdenes al fichero o directorio dados como �ltimo argumento. D�se cuenta de que usamos``{\tt .}'' para referirnos al directorio 
actual.

%\subsection{Moving files.}
\subsection{Moviendo ficheros}
\index{fichero!mover}
\index{mover ficheros}
\index{mv@{\tt mv}}

El orden {\tt mv} mueve ficheros, en vez de copiarlos. 
La sintaxis es muy parecida:

\begin{tscreen}
/home/larry/foo\# mv termcap sells \\
/home/larry/foo\# ls -F \\
bells\ \ \ \ \ sells\ \ \ \ \ shells \\
/home/larry/foo\# 
\end{tscreen}

D�se cuenta de que el fichero {\tt termcap} ha sido renombrado a {\tt shells}. Puede adem�s usar la orden {\tt mv} para mover un fichero a un directorio completamente 
nuevo.

\blackdiamond {\bf Nota:} {\tt mv} y {\tt cp} sobreescribir�n un fichero de destino que tiene el mismo nombre sin pregunt�rselo. Tenga cuidado cuando mueva un fichero a otro directorio. Puede que haya un fichero con el mismo nombre en ese directorio, !`que ser� sobreescrito!

%\subsection{Deleting ficheros and directories.}
\subsection{Borrando ficheros y directorios}
\index{fichero!borrar}
\index{borrar!ficheros}
\index{rm@{\tt rm}}
Usted tiene ahora una fea rima, que ha creado usando el orden {\tt ls}.
Para borrar un fichero, use el orden {\tt rm}, que proviene de ''remove'',\NT{quitar} como se muestra aqu�:

\begin{tscreen}
/home/larry/foo\# rm bells sells \\
/home/larry/foo\# ls -F \\
shells \\
/home/larry/foo\#
\end{tscreen}

No nos queda nada salvo shells, pero no nos quejaremos. D�se cuenta de que {\tt rm} no 
le preguntar� antes de borrar un fichero, as� que tenga cuidado.

\index{borrar!directorio}
\index{directorio!borrar}
\index{rmdir@{\tt rmdir}}
Una orden relacionada con {\tt rm} es {\tt rmdir}. Este orden borra un directorio, pero s�lo si 
el directorio est� vac�o. Si el directorio contiene alg�n fichero o 
subdirectorio, {\tt rmdir} protestar�.

%\subsection{Looking at ficheros.}
\subsection{Mirando en los ficheros}
\index{ficheros!ver el contenido de}
\index{more@{\tt more}}
\index{cat@{\tt cat}!para ver el contenido de un fichero}
Las �rdenes {\tt more} y {\tt cat} se usan para ver el contenido de los ficheros. El orden {\tt more} muestra un fichero, una pantalla completa cada vez, mientras que {\tt cat} muestra el fichero completo de una vez.

Para ver el contenido del fichero {\tt shells}, use la orden

\begin{tscreen}
/home/larry/foo\# more shells
\end{tscreen}

En caso de que est� interesado en lo que contiene {\tt shells}, es una lista de int�rpretes de �rdenes (shell) v�lidos en su sistema. En la mayor�a de los sistemas, esto incluye {\tt /bin/sh}, {\tt /bin/bash} y {\tt /bin/csh}. Hablaremos acerca de estos diferentes tipos de int�rpretes de �rdenes m�s adelante.

Mientras est� usando {\tt more}, presione \key{Space} para ver la siguiente p�gina de texto, y \key{b} para ver la p�gina anterior. Hay otras �rdenes disponibles en {\tt more},
�stos son s�lo los b�sicos. Podr� salir de {\tt more} pulsando \key{q}.

Salga de {\tt more} y pruebe {\tt cat /etc/termcap}. Probablemente el texto ir� demasiado r�pido para que pueda leerlo. 
El nombre ``{\tt cat}'' proviene de hecho de ``concatenate'', que es el verdadero uso del programa. El orden {\tt cat} puede usarse para "encadenar" los contenidos de varios ficheros y guardar el
 resultado en otro fichero.
Volveremos a esto en la secci�n ~\ref{sec-shell-script}.

%\subsection{Getting online help.}
\subsection{Obteniendo ayuda en l�nea}


\index{{\linux}!p�ginas de manual para}
\index{ayuda!en l�nea}
\index{p�ginas del manual}
\index{maunal de linux}
Casi todos los sistemas UNIX, incluyendo {\linux}, facilitan lo que se conoce como {\bf p�ginas del manual}. Estas p�ginas contienen documentaci�n acerca de �rdenes del sistema, recursos, ficheros
 de configuraci�n, etc.

\index{man@{\tt man}}

La orden usada para acceder a las p�ginas del manual es {\tt man}. Si est� interesado en aprender nuevas opciones de la orden {\tt ls} puede escribir

\begin{tscreen}
/home/larry\# man ls
\end{tscreen}

y se mostrar� la p�gina del manual para {\tt ls}.

Por desgracia, la mayor�a de las p�ginas de manual est�n escritas por personas que ya tienen alguna idea de lo que la orden o recurso hace. Por esta raz�n, las p�ginas del manual, a menudo,
 contienen s�lo los detalles t�cnicos de la orden, sin mucha explicaci�n. De todos modos, las p�ginas del manual pueden constituir un recurso muy valioso para refrescar su memoria si se le 
olvida la sintaxis de una orden. Las p�ginas del manual le hablar�n adem�s de �rdenes que no veremos en este libro.

Sugiero que pruebe {\tt man} para las �rdenes que ya hemos visto y cuando
veamos alguno nuevo. Algunas de estas �rdenes no tendr�n p�gina de manual,
por distintas razones. Primero, las p�ginas de manual puede que no se hayan
escrito todav�a. El proyecto de documentaci�n de {\linux} tambi�n es responsable de
las p�ginas de manual de {\linux}. Estamos acumulando poco a poco la
mayor�a de las p�ginas de manual disponibles para el sistema). Segundo, la
orden podr�a ser una orden interna del shell, o un alias (que se
discutir� en la p�gina ~\pageref{sec-shells-cmds}), la cual podr�a no tener
una p�gina de manual propia. Un ejemplo es {\tt cd}, que es una orden
interna del shell. El shell por s� s�lo procesa la orden {\tt cd} --- 
no hay un programa separado que implemente esta orden.

% {\linux} Installation and Getting Started    -*- TeX -*-
% msdos.tex
% Copyright (c) 1992, 1993 by Matt Welsh <mdw@sunsite.unc.edu>
%
% This file is freely redistributable, but you must preserve this copyright 
% notice on all copies, and it must be distributed only as part of "{\linux} 
% Installation and Getting Started". This file's use is covered by the 
% copyright for the entire document, in the file "copyright.tex".
%
% Copyright (c) 1998 by Specialized Systems Consultants Inc. 
% <ligs@ssc.com>

%\section{Accessing MS-DOS files.}
\section{Acceder a los ficheros MS-DOS\TM}
\markboth{Caracter�sticas Avanzadas}{Acceso a ficheros MS-DOS\tm}
\label{sec-msdos-mount}
\index{MS-DOS!acceder ficheros desde}
\index{ficheros!MS-DOS}
Si, por cualquier retorcida y extrafalaria raz�n, quiere acceder a ficheros
de MS-DOS, lo  podr� hacer f�cilmente desde {\linux}.

\index{MS-DOS!montando una partici�n bajo {\linux}}
\index{mount@{\tt mount}!para montar una partici�n MS-DOS}
La manera normal de acceder a los ficheros de MS-DOS es montar una partici�n MS-DOS o
un disco flexible bajo {\linux}, lo cual permite acceder a los ficheros directamente a trav�s
del sistema de ficheros. Por ejemplo, si tiene un disco flexible MS-DOS en
{\tt /dev/fd0}, la orden

\begin{tscreen}
\# mount -t msdos /dev/fd0 /mnt
\end{tscreen}

lo montar� en {\tt /mnt}. Consulte la Secci�n~\ref{sec-floppy} para m�s
informaci�n sobre c�mo montar discos flexibles.

Tambi�n puede montar una partici�n MS-DOS de su disco duro para que sea
accesible desde {\linux}. Si tiene una partici�n MS-DOS en {\tt /dev/hda1}, la orden

\begin{tscreen}
\# mount -t msdos /dev/hda1 /mnt
\end{tscreen}

la monta. Aseg�rese de desmontar ({\tt umount}) la partici�n cuando haya terminado
de usarla. Tambi�n se puede hacer que una partici�n MS-DOS se monte autom�ticamente
en el momento del arranque si incluye la entrada en {\tt /etc/fstab}. Consulte la
Secci�n~\ref{sec-manage-fs} para m�s detalles. La siguiente l�nea en {\tt
/etc/fstab} monta una partici�n MS-DOS en {\tt /dev/hda1} en el
directorio {\tt /dos}.

\begin{tscreen}
/dev/hda1\ \ \ \ \ /dos\ \ \ \ \ msdos \ \ \ \ \ defaults
\end{tscreen}

Tambi�n puede montar los sistemas de ficheros VFAT usados por Windows 95 y 98:

\begin{tscreen}
\# mount -t vfat /dev/hda1 /mnt
\end{tscreen}

Esto permite acceder a los nombres largos de ficheros de Windows 95\tm. Esto s�lo
se aplica a particiones que realmente tengan almacenados los nombres en formato
largo. No se puede montar un sistema de ficheros FAT16 normal y usarlo para
obtener nombres de ficheros largos.

\index{MS-DOS!uso de Mtools para acceder a ficheros}
El software Mtools tambi�n puede ser usado para acceder a ficheros MS-DOS\tm. Las
�rdenes {\tt mcd}, {\tt mdir}, y {\tt mcopy} se comportan todas como sus
equivalentes MS-DOS\tm. Si instala las Mtools, deber�a tener p�ginas del manual
disponibles para estas �rdenes.

\index{MS-DOS!ejecuci�n de programas bajo {\linux}}
\index{MS-DOS!emulador}
\index{Microsoft Windows!emulador} 
Acceder a ficheros MS-DOS es una cosa; ejecutar programas MS-DOS es
otra. Hay un emulador de MS-DOS\tm en desarrollo para {\linux}; es
f�cil de conseguir, y est� inclu�do en la mayor�a de las distribuciones.
Tambi�n se puede conseguir en muchos sitios, incluyendo los sitios
FTP para {\linux} listados en el Ap�ndice~\ref{app-ftp}. El emulador de MS-DOS est�
considerado como lo suficientemente potente para hacer funcionar un buen n�mero de
aplicaciones, incluyendo Wordperfect\tm, desde {\linux}. Sin embargo, {\linux} y MS-DOS son
sistemas operativos marcadamente diferentes. La potencia de cualquier emulador de MS-DOS
bajo UNIX est� limitada. Adem�s, est� en desarrollo un emulador de Microsoft Windows
que corra bajo X Window.











% \linux Installation and Getting Started    -*- TeX -*-
% commands.tex
% Copyright (c) 1992, 1993 by Matt Welsh, Larry Greenfield and Karl Fogel
%
% This fichero is freely redistributable, but you must preserve this copyright 
% notice on all copies, and it must be distributed only as part of "\linux 
% Installation and Getting Started". This fichero's use is covered by
% the copyright for the entire document, in the fichero "copyright.tex".
%
% Copyright (c) 1998 by Specialized Systems Consultants Inc. 
% <ligs@ssc.com>
% Revisi�n 1 por Fco. Javer Fern�ndez <serrador@arrakis.es> 9 de julio del 2002
%
%\section{Summary of basic UNIX commands.}\label{sec-command-summ}
\section{Sumario de �rdenes b�sicas}
\label{sec-command-summ}
\markboth{Tutorial de {\linux}}{Sumario de �rdenes b�sicas UNIX}

\index{�rdenes!sumario de b�sicas|(}
Esta secci�n introduce algunas de las m�s �tiles �rdenes b�sicas de un sistema UNIX, incluyendo aqu�llas que son cubiertas en la secci�n anterior.

\index{�rdenes!-@{\tt -} flag de opci�n de orden}
\index{flag@{\tt -} de opci�n de orden}
F�jese en que las opciones suelen empezar con ``{\tt -}'', y en la mayor�a de los casos es posible especificar m�s de una opci�n con un �nico ``{\tt -}''. Por ejemplo, en vez de usar {\tt ls -l -F}, 
se puede escribir {\tt ls -lF}.

En lugar de dar una lista de cada una de las opciones de una orden, ahora s�lo vamos a presentar �rdenes �tiles o importantes. De hecho, la mayor�a de estas �rdenes tienen muchas opciones 
que nunca usar�. Puede usar {\tt man} para echar un vistazo a las p�ginas de manual de cada orden, el cu�l lista todas las opciones disponibles.

D�se cuenta tambi�n de que muchas de estas �rdenes toman como argumento una lista de ficheros o directorios, indicados en esta tabla por ``\textsl{fichero1} \ldots \textsl{ficheroN}''. Por ejemplo,
la orden {\tt cp} toma como argumentos una lista de ficheros para copiar, seguido del fichero o directorio destino. Cuando va a copiar m�s de un fichero, el destino debe ser un directorio.

\begin{dispitems}

\index{cd@{\tt cd}}
\ditem {{\tt cd}}
Cambia el directorio de trabajo actual \\
Sintaxis: {\tt cd \cparam{directorio}} \\
Donde \textsl{directorio} es el directorio al que se quiere cambiar. ``{\tt .}''
hace referencia al directorio actual, ``{\tt ..}'' al directorio padre. Si no se especifica ning�n directorio le lleva, por omisi�n, a su directorio de usuario. \\

Ejemplo: {\tt cd ../foo} sube el directorio actual un nivel, y entonces, se introduce en el directorio {\tt foo}.

\index{ls@{\tt ls}}
\ditem {{\tt ls}}
Muestra informaci�n acerca de los ficheros y directorios nombrados. \\
Sintaxis: {\tt ls \cparam{ficheros} }\\
Donde \textsl{ficheros} consiste en los nombres de ficheros o directorios que se quieren listar. Las opciones que m�s se usan son {\tt -F} (para mostrar el tipo de fichero) y {\tt -l}
 (para mostrar una lista ''ampliada'' incluyendo el tama�o de los ficheros, propietario, permisos, etc.). \\
Ejemplo: {\tt ls -lF /home/larry} muestra los contenidos del directorio {\tt /home/larry}.

\index{cp@{\tt cp}}
\ditem {{\tt cp}} 
Copia uno o m�s ficheros a otro fichero o directorio. \\
Sintaxis: {\tt cp \cparam{ficheros}
\cparam{destino}} \\
Donde \textsl{ficheros} indica los ficheros que hay que copiar, y
\textsl{destino} es el fichero o directorio destino. \\

Ejemplo: {\tt cp ../frog joe} copia el fichero {\tt ../frog} al fichero o directorio {\tt joe}.

\index{mv@{\tt mv}}
\ditem {{\tt mv}} 
Mueve uno o m�s ficheros a otro directorio. Esta orden hace el equivalente de una copia seguido del borrado del fichero original.
Puede usar esto para renombrar ficheros, como con la orden de MS-DOS {\tt RENAME}. \\
Sintaxis: {\tt mv \cparam{ficheros}
\cparam{destino}} \\
Donde \textsl{ficheros} indica los arhivos que hay que mover, y
\textsl{destino} es el fichero o directorio destino. \\

Ejemplo: {\tt mv ../frog joe} mueve el fichero {\tt ../frog} al fichero o directorio {\tt joe}.

\index{rm@{\tt rm}}
\ditem {{\tt rm}} 
Borra ficheros. F�jese en que cuando borra un fichero bajo UNIX, 
son irrecuperables (al contrario que con MS-DOS, donde normalmente se puede ''desborrar'' el fichero). \\
Sintaxis: {\tt rm \cparam{ficheros}} \\
Donde \textsl{ficheros} describe el nombre de los ficheros que hay que borrar. \\
La opci�n {\tt -i} le pide confirmaci�n antes de borrar el fichero. \\

Ejemplo: {\tt rm -i /home/larry/joe /home/larry/frog} borra los ficheros {\tt joe} y {\tt frog} en {\tt /home/larry}.


\index{mkdir@{\tt mkdir}}
\ditem {{\tt mkdir}}
Crea nuevos directorios.\\
Sintaxis: {\tt mkdir \cparam{dirs} }\\
Donde \textsl{dirs} son los directorios que hay que crear. \\

Ejemplo: {\tt mkdir /home/larry/test} crea el directorio {\tt test}
en {\tt /home/larry}.

\index{rmdir@{\tt rmdir}}
\ditem {{\tt rmdir}} 
Borra directorios vac�os. Cuando use {\tt rmdir}, el directorio de trabajo actual no debe estar dentro del directorio que se pretende borrar. \\
Sintaxis: {\tt rmdir \cparam{dirs} }\\
Donde \textsl{dirs} define los directorios que hay que borrar. \\

Ejemplo: {\tt rmdir /home/larry/papers} borra el directorio {\tt /home/larry/papers}, si est� vac�o. 

\index{man@{\tt man}}
\ditem {{\tt man}} 
Muestra la p�gina de manual para la orden o recurso dado (es decir, no una utilidad del sistema que no sea 
una orden, como una funci�n de biblioteca.)\\

Sintaxis: {\tt man \cparam{command}} \\

Donde \textsl{command} es el nombre de la orden o recurso del que se quiere conseguir ayuda.\\

Ejemplo: {\tt man ls} le da informaci�n acerca de la orden {\tt ls}.

\index{more@{\tt more}}
\ditem {{\tt more}} 
Muestra informaci�n del contenido de los ficheros nombrados, pantalla por pantalla. \\
Sintaxis: {\tt more \cparam{ficheros}} \\
Donde \textsl{ficheros} indica los ficheros que se quieren mostrar. \\

Ejemplo: {\tt more papers/history-final} muestra el fichero {\tt papers/history-final}.

\index{cat@{\tt cat}}
\ditem {{\tt cat}} 
Oficialmente usado para concatenar ficheros, {\tt cat} tambi�n se usa para mostrar los contenidos de un fichero por pantalla. \\
Sintaxis: {\tt cat \cparam{ficheros}} \\
Donde \textsl{ficheros} indica los ficheros que se quieren mostrar. \\

Ejemplo: {\tt cat letters/from-mdw} muestra el fichero {\tt letters/from-mdw}.

\index{echo@{\tt echo}}
\ditem{{\tt echo}}
Muestra en la pantalla los argumentos que se le pasan a la orden. \\
Sintaxis: {\tt echo \cparam{args}} \\
Donde \textsl{args} indica los argumentos que se quieren mostrar. \\

Ejemplo: {\tt echo ``Hello world''} muestra la cadena ``{\tt Hello world}''.

\index{grep@{\tt grep}}
\ditem{{\tt grep}}
Muestra cada l�nea en uno o m�s ficheros que contiene un patr�n dado. \\
Sintaxis: {\tt grep \cparam{pattern} \cparam{ficheros}} \\
Donde \textsl{pattern} es un patr�n, y 
\textsl{ficheros} indica los ficheros donde se quiere buscar dicho patr�n. \\

Ejemplo: {\tt grep loomer /etc/hosts} muestra cada l�nea en el fichero {\tt /etc/hosts} que contiene el patr�n ``{\tt loomer}''.

\end{dispitems}

\index{�rdenes!sumario de las b�sicas|)}




% {\linux} Installation and Getting Started    -*- TeX -*-
% filesystem.tex
% Copyright (c) 1992, 1993 by Matt Welsh, Larry Greenfield and Karl Fogel
%
% This file is freely redistributable, but you must preserve this copyright 
% notice on all copies, and it must be distributed only as part of "{\linux} 
% Installation and Getting Started". This file's use is covered by
% the copyright for the entire document, in the file "copyright.tex".
%
% Copyright (c) 1998 by Specialized Systems Consultants Inc. 
% <ligs@ssc.com>

%\section{Exploring the file system.}\label{sec-filesystem-tour}
\section{Explorando el sistema de ficheros}\label{sec-filesystem-tour}
\markboth{Tutorial de {{\linux}}}{Explorando el sistema de ficheros}
\index{sistema de ficheros!exploraci�n|(}
Un {\bf sistema de ficheros} es la colecci�n de ficheros y la jerarqu�a de
directorios de un sistema. Ha llegado la hora de acompa�arle en un viaje
alrededor del sistema de ficheros.

% No ref to dirtree if ASCII
\iftex {
Usted ya tiene habilidad y conocimiento como para entender el sistema de ficheros
de {{\linux}}, y tiene un mapa de carreteras. (Ver figura en la p�gina~\pageref{dirtree}).  } 
\fi

Primero, cambie al directorio ra�z ({\tt cd /}), e introduzca {\tt
ls -F} para que aparezca una lista con su contenido. Probablemente ver�
los siguientes directorios\footnote{Puede que vea otros, y puede que no los
vea todos. Cada distribuci�n de {\linux} es diferente en ciertos aspectos.}:
{\tt bin}, {\tt dev}, {\tt etc}, {\tt home}, {\tt install}, {\tt lib},
{\tt mnt}, {\tt proc}, {\tt root}, {\tt tmp}, {\tt user}, {\tt usr},
y {\tt var}.

Ahora, veamos cada uno de estos directorios
\begin{dispitems}
\index{directorio!bin@{\tt /bin}}
\index{bin@{\tt /bin}}
\ditem{{\tt /bin}}
{\tt /bin} es la abreviatura de ``binarios'',
o ejecutables, y es donde residen muchos de los programas imprescindibles del sistema. 
Utilice {\tt ls -F /bin} para listar los ficheros que contiene.
Si repasa la lista, puede que reconozca algunas ordenes,
como {\tt cp}, {\tt ls}, y {\tt mv}. �stos son realmente los programas
que corresponden a esas ordenes. Cuando utiliza la orden {\tt cp},
est� ejecutando el programa {\tt /bin/cp}.


Usando {\tt ls -F}, ver� que muchos (si no todos) los ficheros en
{\tt /bin} tienen un asterisco (''{\tt *}'') a�adido a sus nombres de fichero.
Esto indica que los ficheros son ejecutables, como se describe en
p�gina~\pageref{sec-ls}.

\index{directorio!dev@{\tt /dev}}
\index{dev@{\tt /dev}}
\ditem{{\tt /dev}}
\index{device driver}
\index{controlador de dispositivo}
\index{fichero!dispositivo}
\index{dispositivos!acceso}
Los ''ficheros'' en {\tt /dev} son {\bf controladores de dispositivos}---acceden
a los dispositivos del sistema y a recursos como discos duros, m�dems y memoria.
Igual que su sistema puede leer datos de un fichero, tambi�n puede leer
la entrada del rat�n accediendo a {\tt /dev/mouse}.


\index{dispositivo!fd@{\tt fd}}
\index{dispositivo!disquete}
\index{dispositivo!floppy disk}
\index{floppy !nombres de dispositivo para}
\index{disquete!nombres de dispositivo para}
Los ficheros cuyos nombres comienzan por {\tt fd} son dispositivos de discos
flexibles. {\tt fd0} es la primera disquetera y {\tt fd1} es la segunda. Puede
que se haya dado cuenta de que hay m�s dispositivos de disco flexible que los
dos anteriores: �stos representan tipos espec�ficos de discos flexibles. Por ejemplo,
{\tt fd1H1440} accede a discos 3.5" de alta densidad en la disquetera 1.

Lo siguiente es una lista de algunos de los ficheros de dispositivo m�s
comunmente utilizados. Aunque puede que no tenga alguno de los dispositivos f�sicos
que se listan debajo, puede ocurrir que aun as� tenga controladores en {\tt dev} para
ellos.

\begin{itemize}
\index{dispositivos!consola}
\index{dispositivos!/dev/console@{\tt /dev/console}}
\index{consola!nombre para dispositivo}
\index{/dev/console@{\tt /dev/console}}
\item {\tt /dev/console} se refiere a la consola del sistema---es decir, al
monitor conectado directamente a su sistema.

\index{dispositivos!puertos serie}
\index{dispositivos!/dev/ttyS@{\tt /dev/ttyS}}
\index{dispositivos!/dev/cua@{\tt /dev/cua}}
\index{/dev/ttyS@{\tt /dev/ttyS}}
\index{/dev/cua@{\tt /dev/cua}}
\index{puertos serie!nombres de dispositivo para}
\item Los diversos dispositivos {\tt /dev/ttyS} y {\tt /dev/cua} se usan
para acceder a los puertos serie. {\tt /dev/ttyS0} se refiere a ''{\tt COM1}''
bajo MS-DOS. Los dispositivos {\tt /dev/cua} son dispositivos de ''llamada'' , y se
usan con un m�dem.\NT{En los n�cleos modernos a partir de la serie 2.2 los dispositivos
ttySx reemplazan a cuax en sus funciones}


\index{dispositivos!discos duros}
\index{dispositivos!/dev/hd@{\tt /dev/hd}}
\index{/dev/hd@{\tt /dev/hd}}
\index{discos duros!nombres de dispositivo}
\item Los dispositivos cuyos nombres comiencen por {\tt hd} acceden a discos duros.
{\tt /dev/hda} se refiere a {\em todo\/} el primer disco duro, mientras que {\tt
/dev/hda1} se refiere a la primera {\em partici�n\/} de {\tt /dev/hda}.



\index{dispositivos!SCSI}
\index{dispositivos SCSI!nombres para}
\index{dispositivos!/dev/sd@{\tt /dev/sd}}
\index{dispositivos!/dev/st@{\tt /dev/st}}
\index{dispositivos!/dev/sr@{\tt /dev/sr}}
\index{dev/sd@{\tt /dev/sd}}
\index{dev/st@{\tt /dev/st}}
\index{dev/sr@{\tt /dev/sr}}
\index{SCSI!nombres de dispositivos}
\item Los dispositivos cuyos nombres comienzan por {\tt sd} son discos SCSI.
Si tiene un disco duro SCSI, en lugar de acceder a �l a trav�s de {\tt
/dev/hda}, acceder�a con {\tt /dev/sda}. A las cintas SCSI se accede
v�a dispositivos {\tt st}, y a los CD-ROM SCSI v�a dispositivos {\tt sr}.



\index{dispositivos!puertos paralelos}
\index{dispositivos!/dev/lp@{\tt /dev/lp}}
\index{/dev/lp@{\tt /dev/lp}}
\index{puerto paralelo!nombre de dispositivo}
\item Los dispositivos cuyos nombres comienzan por {\tt lp} acceden a los
puertos paralelos. {\tt /dev/lp0} es lo mismo que ''{\tt LPT1}'' en el mundo MS-DOS.



\index{dispositivos!null}
\index{dispositivos!/dev/null@{\tt /dev/null}}
\index{/dev/null@{\tt /dev/null}}
\index{fichero null}
\item {\tt /dev/null} se utiliza como ''agujero negro''---los datos enviados
a este dispositivo se pierden para siempre. ?`Por qu� es �til esto? Bueno, si quiere
evitar que la salida de una orden salga por la pantalla, puede dirigir esa
salida a {\tt /dev/null}. Hablaremos de ello m�s adelante.



\index{dispositivos!consolas virtuales}
\index{dispositivos!/dev/tty@{\tt /dev/tty}}
\index{/dev/tty@{\tt /dev/tty}}
\index{consolas virtuales}
\item Los dispositivos cuyos nombres comienzan por {\tt /dev/tty} seguidos de un n�mero
se refieren a las ''consolas virtuales'' de su sistema (a las que se accede pulsando
\key{Alt-F1}, \key{Alt-F2}, y as� sucesivamente). {\tt /dev/tty1} se refiere
a la primera consola virtual, {\tt /dev/tty2} se refiere a la segunda, y as� sucesivamente.



\index{dispositivos!pseudo-terminales}
\index{dispositivos!/dev/pty@{\tt /dev/pty}}
\index{pseudo-terminales}
\index{/dev/pty@{\tt /dev/pty}}
\item Los dispositivos cuyos nombres comienzan por {\tt /dev/pty} son {\bf pseudo-terminales},
que se usan para proporcionar un ``terminal'' a las sesiones iniciadas remotamente.
Por ejemplo, si su m�quina est� en una red, las sesiones de {\tt telnet} entrantes
utilizar�n uno de los dispositivos {\tt /dev/pty}.
\end{itemize}



\index{directorio!/etc@{\tt /etc}}
\index{/etc@{\tt /etc}}
\ditem{{\tt /etc}}
{\tt /etc} contiene un buen n�mero de ficheros de configuraci�n del sistema.
Estos incluyen {\tt /etc/passwd} (la base de datos de usuarios), {\tt /etc/rc} 
(la macro de inicio del sistema), y as� sucesivamente. 



\index{directory!/sbin@{\tt /sbin}}
\index{/sbin@{\tt /sbin}}
\ditem{{\tt /sbin}}
{\tt /sbin} contiene binarios imprescindibles para el sistema que se usan para
su administraci�n.



\index{directorio!/inicio@{\tt /home}}
\index{/home@{\tt /home}}
\ditem{{\tt /home}}
{\tt /home} contiene los directorios de inicio de los usuarios. Por ejemplo, {\tt /home/larry}
es el directorio de inicio del usuario ''{\tt larry}''. En un sistema reci�n instalado,
puede que no haya ning�n usuario en este directorio.



\index{directorio!/lib@{\tt /lib}}
\index{/lib@{\tt /lib}}
\ditem{{\tt /lib}}
{\tt /lib} contiene las {\bf im�genes de las bibliotecas compartidas}, que son ficheros que
contienen c�digo que comparten muchos programas. Mejor que cada programa use sus propias
copias de estas rutinas compartidas, es que todas se guarden en un lugar com�n, en
{\tt /lib}. Esto hace que los ficheros ejecutables sean m�s peque�os y ahorra espacio
en el sistema.



\index{directorio!/proc@{\tt /proc}}
\index{/proc@{\tt /proc}}
\ditem{{\tt /proc}}
En {\tt /proc} se mantiene un ''sistema de ficheros virtual'', donde los ficheros
se guardan en memoria, no en disco. Estos ''ficheros'' hacen referencia a los
diversos {\bf procesos} que corren en el sistema, y permiten obtener informaci�n
sobre los procesos y programas en ejecuci�n en un instante dado.
Esto se discute con m�s detalle en
p�gina~\pageref{sec-process}.


\index{directorio!/tmp@{\tt /tmp}}
\index{/tmp@{\tt /tmp}}
\ditem{{\tt /tmp}}
Muchos programas guardan informaci�n temporalmente en un fichero
que se borra cuando el programa finaliza su ejecuci�n.
La localizaci�n est�ndar de estos ficheros es {\tt /tmp}.


\index{directorioy!/usr@{\tt /usr}}
\index{/usr@{\tt /usr}}
\ditem{{\tt /usr}}
{\tt /usr} es un directorio muy importante que contiene subdirectorios
que albergan algunos de los programas m�s importantes y �tiles usados
en el sistema.



Los diversos directorios descritos arriba son imprescindibles para
que el sistema funcione, pero muchos de los elementos que se encuentran en
{\tt /usr} son opcionales. Sin embargo, son esos elementos opcionales los que
hacen un sistema �til e interesante. Sin {\tt /usr}, se tendr�a un sistema
aburrido que s�lo soportar�a programas como {\tt cp} y {\tt ls}. {\tt /usr}
contiene muchos de los grandes paquetes de software y los ficheros de
configuraci�n que los acompa�an.


\index{directorio!/usr/X11R6@{\tt /usr/X11R6}}
\index{/usr/X11R6@{\tt /usr/X11R6}}
\ditem{{\tt /usr/X11R6}}
{\tt /usr/X11R6} contiene el sistema X Window, si se instal�. El sistema
X Window es un enorme y potente entorno gr�fico que proporciona un gran n�mero de
utilidades gr�ficas y programas, que aparecen en ''ventanas'' en la pantalla.
Si usted esta familiarizado con Microsoft Windows o el entorno Macintosh, X Window
le ser� muy familiar. El directorio {\tt /usr/X11R6} contiene todos los
ejecutables de X Window, ficheros de configuraci�n y ficheros de apoyo. Todo esto
se cubre con m�s detalle en el Cap�tulo~\ref{chap-advanced-xconfiguration}.

\index{directorio!/usr/bin@{\tt /usr/bin}}
\ditem{{\tt /usr/bin}}
{\tt /usr/bin} es el aut�ntico almac�n de software en cualquier sistema {\linux},
y contiene la mayor�a de los ejecutables de programas que no se encuentran en
otros sitios, como {\tt /bin}.



\index{directorio!/usr/etc@{\tt /usr/etc}}
\index{/usr/etc@{\tt /usr/etc}}
\ditem{{\tt /usr/etc}}
Como {\tt /etc} contiene diferentes ficheros de configuraci�n y programas
del sistema, {\tt /usr/etc} contiene incluso m�s que el anterior. En
general, los ficheros que se encuentran en {\tt /usr/etc/} no son esenciales
para el sistema, a diferencia de los que se encuentran en {\tt /etc}, que s�
lo son.


\index{directorio!/usr/include@{\tt /usr/include}}
\index{/usr/include@{\tt /usr/include}}
\ditem{{\tt /usr/include}}
{\tt /usr/include} contiene los {\bf ficheros de cabecera} para el compilador de C.
En estos ficheros (muchos de los cuales terminan en {\tt .h}, por ''header'')
se declaran nombres de estructuras de datos, subrutinas, y constantes usadas
al programar en el nivel de sistema UNIX. Si est� familiarizado con el lenguaje
de programaci�n C, aqu� encontrar� ficheros de cabecera como {\tt
stdio.h}, en el que se declaran funciones como {\tt printf()}.


\index{directorio!/usr/g++-include@{\tt /urs/g++-include}}
\index{/usr/g++-include@{\tt /urs/g++-include}}
\ditem{{\tt /usr/g++-include}}
{\tt /usr/g++-include} contiene ficheros de cabecera para el compilador de C++
(muy parecido a {\tt /usr/include}).

\index{directorio!/usr/lib@{\tt /usr/lib}}
\index{/usr/lib@{\tt /usr/lib}}
\ditem{{\tt /usr/lib}}
{\tt /usr/lib} contiene las bibliotecas ''stub''  y ''estatic'' equivalentes
a los ficheros situados en {\tt /lib}. Cuando se compila un programa, el programa
se ''enlaza'' con las bibliotecas situadas en {\tt /usr/lib}, que ordenar�n
al programa que mire en {\tt /lib} cuando necesite el c�digo real de la
librer�a. Por a�adidura, otros programas diversos guardan ficheros de configuraci�n en 
{\tt /usr/lib}.

\index{directorio!/usr/local@{\tt /usr/local}}
\index{/usr/local@{\tt /usr/local}}
\ditem{{\tt /usr/local}}
{\tt /usr/local} se parece mucho a {\tt /usr}---contiene diversos programas
y ficheros que no son imprescindibles para el sistema, pero que lo hacen divertido
y excitante. En general, los programas en {\tt /usr/local} son
espec�ficos de cada sistema---consecuentemente, {\tt /usr/local}
var�a mucho entre los diversos sistemas {\linux}.

\index{directorio!/usr/man@{\tt /usr/man}}
\index{/usr/man@{\tt /usr/man}}
\ditem{{\tt /usr/man}}
Este directorio contiene las p�ginas del manual. Hay dos subdirectorios en �l
para cada ''secci�n'' de p�ginas del manual (use la orden {\tt man man}
para m�s detalles).  Por ejemplo, {\tt /usr/man/man1} contiene las fuentes
(es decir, el original sin formatear) de la p�ginas del manual de la secci�n 1, y
{\tt /usr/man/cat1} contiene las p�ginas del manual formateadas de la secci�n 1.

\index{directorio!/usr/src@{\tt /usr/src}}
\index{/usr/src@{\tt /usr/src}}
\ditem{{\tt /usr/src}}

{\tt /usr/src} contiene el c�digo fuente (las instrucciones sin compilar)
de diversos programas del sistema. El directorio m�s importante aqu�
es {\tt /usr/src/linux}, que contiene el c�digo fuente del n�cleo de {\linux}.



\index{directorio!/var@{\tt /var}}
\index{/var@{\tt /var}}
\ditem{{\tt /var}}
En {\tt /var} se mantienen directorios que a veces cambian de tama�o o tienden
a crecer.  Muchos de estos directorios sol�an residir en {\tt /usr}, pero
desde que aqu�llos que mantienen {\linux} intentan conservarlo relativamente
sin cambios, los directorios que cambian a menudo se han pasado a
{\tt /var}.  Algunas distribuciones de {\linux} guardan las bases de datos de sus
paquetes de software en directorios bajo {\tt /var}.

\index{directorio!/var/log@{\tt /var/log}}
\index{/var/adm@{\tt /var/log}}
\ditem{{\tt /var/log}}
{\tt /var/log} contiene diversos ficheros de inter�s para el administrador
del sistema, espec�ficamente, los registros del sistema, que recogen errores
o problemas con el sistema. Otros ficheros recogen entradas e intentos
fallidos de entrar el sistema. Esto se cubrir� en el Cap�tulo~\ref{chap-sysadm}.

\index{directorio!/var/spool@{\tt /var spool}}
\index{/var/spool@{\tt /var spool}}
\ditem{{\tt /var/spool}}
{\tt /var/spool} contiene ficheros que son encolados para otro programa.

Por ejemplo, si su m�quina est� conectada a una red, el correo entrante
se guarda en {\tt /var/spool/mail} hasta que se lee o se borra.
Los art�culos de noticias entrantes o salientes est�n en
{\tt /var/spool/news}, y as� sucesivamente.

\end{dispitems}

\index{sistema de ficheros!exploraci�n|)}


% \linux: Instalaci�n y primeros pasos      -*- TeX -*-
% shells.tex
% Copyright (c) 1992, 1993 by Matt Welsh, Larry Greenfield and Karl Fogel
%
% Este fichero puede redistribuirse libremente, pero debe conservarse este distintivo 
% de copyright en todas las copias, y s�lo debe ser distribuido como parte de 
% "\linux: Instalaci�n y primeros pasos". El uso de este archvi est� cubierto por el 
% copyright para todo el documento, en el arhcivo "copyright.tex"
%
% Copyright (c) 1998 by Specialized Systems Consultants Inc. 
% <ligs@ssc.com>
%\section{Types of shells.}
%Revisado por Francisco javier Fern�ndez el 17 de julio de 2002

\section{Tipos de int�rpretes de �rdenes}
\markboth{Tutorial de {\linux}}{Las Shells o int�rpretes de �rdenes} 
 \index{shells|(}

Como se ha comentado antes, {\linux} es un sistema operativo multitarea y multiusuario.
La multitarea es {\em muy} �til, y una vez la haya comprendido, la usar� todo el
tiempo. Dentro de poco ejecutar� programas en segundo plano, cambiar� entre tareas
y redirigir� programas junto a resultados complicados con una sencilla orden.

Muchas de las caracter�sticas que se ver�n en esta secci�n son caracter�sticas
suministradas por el int�rprete de �rdenes. Se debe tener cuidado en no confundir
{\linux} (el actual sistema operativo) con el int�rprete de �rdenes. 
Un int�rprete de �rdenes es tan s�lo un interfaz con el sistema operativo que hay 
debajo. El int�rprete de �rdenes proporciona funcionalidad a�adida a {\linux}.

\index{shell scripts!definici�n}
Un int�rprete de �rdenes no es s�lo un int�rprete de las �rdenes interactivas que
se teclean en el indicador de �rdenes, sino tambi�n un potente lenguaje de programaci�n. Permite
escribir {\bf guiones (shell scripts)}, juntando varias �rdenes en un fichero. Si se conoce
MS-DOS, se reconocer� la similitud con los ficheros de procesamiento por lotes. Los
guiones del int�rprete de �rdenes son una herramienta muy potente, que le
permitir� automatizar y extender el uso de {\linux}. Mire la p�gina~\pageref{sec-shell-script} para m�s informaci�n.

\index{shells!Bourne shell}
\index{shells!C shell}
\index{Bourne shell}
\index{C Shell (csh)@C Shell ({\tt csh})}
\index{/bin/sh@{\tt /bin/sh}}
\index{/bin/csh@{\tt /bin/csh}}

Hay varios tipos de int�rprete de �rdenes \NT{{\em shell}, escudo en ingl�s} en el mundo de Unix.
Los m�s importantes son la ``{\em shell Bourne}'' y la ``{\em shell C}''. La 
{\em shell Bourne} utiliza una sintaxis de �rdenes como la {\em shell} original de 
los primeros sistemas UNIX, como System III. El nombre de la {\em shell Bourne} en 
la mayor�a de los sistemas {\linux} es {\tt /bin/sh} (donde {\tt sh} significa ``{\em 
shell}''. La {\em shell C} (no confundir con una concha marina) utiliza diferente 
sintaxis, parecida al lenguaje de programaci�n ``C'', y en la mayor�a de los 
sistemas {\linux} se llama {\tt/bin/csh}.

\index{shells!Bourne again shell}
\index{Bourne again shell}
\index{bash@{\tt bash}}
\index{/bin/bash@{\tt /bin/bash}}
\index{Tcsh}
\index{/bin/tcsh@{\tt /bin/tcsh}}
\index{tcsh@{\tt tcsh}}


Bajo {\linux}, hay disponibles muchas variaciones de int�rpretes de �rdenes. Las
dos m�s com�nmente utilizadas son {\em Bourne Again Shell}, o ``bash'' ({\tt/bin/bash}),
y ``Tcsh'' ({\tt /bin/tcsh}). La variante {\tt bash} es una forma de {\em shell Bourne} que incluye
muchas de las caracter�sticas avanzadas de la {\em shell C}. A causa de que {\tt bash}
soporta un superconjunto de sintaxis de la {\em shell Bourne}, los {\em guiones} de la 
{\em shell} escritos en el est�ndar de la shell Bourne podr�an trabajar con {\tt 
bash}. Si se prefiere la sintaxis de la {\em shell C}, {\linux} soporta {\tt tcsh}, 
que es una versi�n ampliada de la {\em shell C}.

El tipo de {\em shell} que usted decida utilizar ser� sobre todo una cuesti�n de fe. 
Algunas personas prefieren la sintaxis de la {\em shell Bourne} con las 
caracter�sticas avanzadas de {\tt bash}, y otros prefieren la sintaxis m�s estructurada 
de la {\em shell C}. Por lo que respecta a �rdenes normales como {\tt cp} y {\tt ls}, la
{\em shell} que se use no importa, la sintaxis es la misma. S�lo cuando se 
comienzan a escribir {\em guiones de �rdenes} o a usar las caracter�sticas avanzadas de
la {\em shell}, comienzan a importar las diferencias entre los tipos de {\em shell}.
Al discutir las caracter�sticas de varios int�rpretes, se notar�n las diferencias
entre las {\em shells} C y Bourne. Sin embargo, para los prop�sitos de este manual,
la mayor�a de estas diferencias son m�nimas (si realmente est�s interesado en
este punto, lea las p�ginas sobre {\tt bash} y {\tt tcsh}).\NT{Tambi�n puede leer
el ``Bash Scripting HOWTO''}
\index{shells|)}




% Linux: Instalaci�n y Primeros Pasos    -*- TeX -*-
% wildcard.tex
% Copyright (c) 1992, 1993 by Matt Welsh, Larry Greenfield and Karl Fogel
%
% Este archivo puede redistribuirse libremente, pero debe conservarse este distintivo 
% de copyright en todas las copias, y s�lo debe ser distribuido como parte de 
% "Linux: Instalaci�n y primeros pasos". El uso de este archvi est� cubierto por el 
% copyright para todo el documento, en el arhcivo "copyright.tex"
%
% Copyright (c) 1998 by Specialized Systems Consultants Inc. 
% <ligs@ssc.com>
% Revisi�n 1 por Francisco javier Fern�ndez 31 de agosto de 2002
%\section{Wildcards.}
\section{Caracteres comod�n}
\markboth{Tutorial de {\linux}}{Comodines}

\index{shells!caracteres comod�n para|(}
\index{caracteres comodin!en nombres de fichero|(}
\index{nombres de fichero!caracteres comod�n en|(}
\index{caracteres comod�nes!definici�n}
Una caracter�stica importante de la mayor�a de sistemas {\linux} es la 
posibilidad de referirse a m�s de un fichero usando caracteres especiales. 
Estos {\bf caracteres comod�n} le permiten referirse a todos los nombres de 
fichero que contengan el car�cter ``{\tt n}''.

\index{*@{\tt *}}
\index{caracteres comod�n!*@{\tt *}}
El comod�n `{\tt *}'' especifica cualquier car�cter o cadena de caracteres 
en el nombre de un fichero. Cuando usa el car�cter  ``{\tt *}'' en un
nombre de fichero, el int�rprete de �rdenes lo reemplaza con todas las posibles 
sustituciones de los nombres de fichero en el directorio al que est� 
haciendo referencia.

He aqu� un r�pido ejemplo. Suponga que Larry tiene los ficheros {\tt frog},
{\tt joe} y {\tt stuff} en su directorio actual.
\begin{tscreen}
/home/larry\# ls \\
frog\ \ \ \ \ joe\ \ \ \ \ stuff \\
/home/larry\#
\end{tscreen}

Para especificar todos los ficheros que contienen la letra ``o'' en el
nombre de fichero, use la instrucci�n
\begin{tscreen}
/home/larry\# ls *o* \\
frog\ \ \ \ \ joe \\
/home/larry\#
\end{tscreen}
Como puede ver, cada instancia de ``{\tt *}'' es reemplazada con todas
las sustituciones que coinciden con los nombres de fichero del directorio
actual.

El uso de ``{\tt *}'' s�lo, simplemente coincide con todos los nombres de 
fichero, porque todos los caracteres coinciden con el comod�n.
\begin{tscreen}
/home/larry\# ls * \\
frog\ \ \ \ \ joe\ \ \ \ \ stuff \\
/home/larry\#
\end{tscreen}

Aqu� hay algunos ejemplos m�s:
\begin{tscreen}
/home/larry\# ls f* \\
frog \\
/home/larry\# ls *ff \\
stuff \\
/home/larry\# ls *f* \\
frog\ \ \ \ \ stuff \\
/home/larry\# ls s*f \\
stuff \\
/home/larry\# 
\end{tscreen}

\index{expansion de comodines!definicion}
\index{shells!expansi�n de comodines}
El proceso de cambiar un ``{\tt *}'' en una serie de nombres de fichero se llama
{\bf expansi�n de comodines} y lo hace el int�rprete de �rdenes. Esto es importante: 
una orden individual, como {\tt ls}, {\em nunca} ve el ``{\tt *}'' en su lista de 
par�metros. El int�rprete de �rdenes expande el comod�n para incluir todos los
nombres de fichero que coinciden. As�, la orden
\begin{tscreen}
/home/larry\# ls *o*
\end{tscreen}
es expandido por el int�rprete de �rdenes a 
\begin{tscreen}
/home/larry\# ls frog joe
\end{tscreen}

\index{ficheros!ocultos!no hacen juego con los comodines}
Una nota importante del comod�n ``{\tt *}'' : {\em no} ve las coincidencias
de los nombres de fichero que empiezan con un �nico punto (``{\tt .}'').  
Estos ficheros se tratan como ficheros {\bf ocultos} --- aunque no est�n 
realmente escondidos, no aparecen en los listados normales con {\tt ls} y 
no son afectados por el uso del comod�n ``{\tt *}''.

He aqu� un ejemplo. Mencionamos antes que cada directorio  contiene dos 
entradas especiales: ``{\tt .}'' se refiere al directorio actual, y
``{\tt ..}'' , que se refiere al directorio padre. Sin embargo, cuando usa 
{\tt ls}, estas dos entradas no se muestran.
\begin{tscreen}
/home/larry\#  ls \\
frog\ \ \ \ \ joe\ \ \ \ \ stuff \\
/home/larry\# 
\end{tscreen}
Si usa el par�metro {\tt -a} con {\tt ls}, sin embargo, puede visualizar
los nombres de fichero que empiezan con ``{\tt .}''. Observe:
\begin{tscreen}
/home/larry\# ls -a \\
.\ \ \ \ \ ..\ \ \ \ \ .bash\_profile\ \ \ \ \ .bashrc\ \ \ \ \ frog
\ \ \ \ \ joe\ \ \ \ \ stuff \\
/home/larry\# 
\end{tscreen}
El listado contiene las dos entradas especiales, ``{\tt .}'' y ``{\tt
..}'', as� como otros dos ficheros ``ocultos'' --{\tt .bash\_profile}
y {\tt .bashrc}--. Estos dos ficheros son ficheros de inicio usados por
{\tt bash} cuando {\tt larry} entra en el sistema. Se describen en 
p�gina~\pageref{sec-init-scripts}.

Hay que fijarse en que cuando usa el comod�n ``{\tt *}'' , ninguno de los 
nombres de fichero que empiezan por ``{\tt .}'' son visualizados.
\begin{tscreen}
/home/larry\# ls * \\
frog\ \ \ \ \ joe\ \ \ \ \ stuff \\
/home/larry\# 
\end{tscreen}
Esto es una caracter�stica de seguridad: si el comod�n ``{\tt *}'' tiene 
coincidencias con nombres de fichero que empiezen por ``{\tt .}'', tambi�n 
tendr�a coincidencia con los nombres de directorios ``{\tt .}'' y 
``{\tt ..}''. Esto puede ser peligroso al usar ciertas �rdenes.

\index{caracteres comod�n!?@{\tt ?}}
\index{?@{\tt ?}}
Otro comod�n es ``{\tt ?}''.  El comod�n ``{\tt ?}'' s�lo se expande 
a un car�cter. As�, ``{\tt ls ?}'' muestra todos los nombres 
de fichero de un s�lo car�cter. Y ``{\tt ls termca?}'' mostrar�a 
``{\tt termcap}'' pero {\em no\/} ``{\tt termcap.backup}''.  Aqu� 
hay otros ejemplos:
\begin{tscreen}
/home/larry\# ls j?e \\
joe \\
/home/larry\# ls f??g \\
frog \\
/home/larry\# ls ????f \\
stuff \\
/home/larry\# 
\end{tscreen}

Como puede ver, los comodines le permiten especificar muchos ficheros 
a la vez. En el sumario de �rdenes que empieza en la p�gina~\pageref{sec-command-summ}, 
dijimos que las �rdenes {\tt cp} y {\tt mv} realmente pueden copiar o mover m�s 
de un fichero a la vez. Por ejemplo,
\begin{tscreen}
/home/larry\# cp /etc/s* /home/larry
\end{tscreen}
copia todos los ficheros de {\tt /etc} cuyo nombre empieza por ``{\tt s}'' al 
directorio {\tt /home/larry}. El formato de la orden {\tt cp} es realmente
\begin{tscreen}
cp \cparam{ficheros}   \cparam{destino}
\end{tscreen}
donde \textsl{ficheros} lista los nombres de fichero a copiar, y \textsl{destino} 
es el fichero o directorio destino.
La orden {\tt mv} tiene una sintaxis id�ntica.

Si est� copiando o moviendo m�s de un fichero, el \textsl{destino} tiene
que ser un directorio. S�lo puede copiar o mover un �nico fichero a otro fichero.

\index{shells!caracteres comod�n para|)}
\index{caracteres comod�n!en nombres de fichero|)}
\index{nombres de fichero!caracteres comod�n en|)}

% \linux: Instalaci�n y primeros pasos    -*- TeX -*-
% plumbing.tex
% Copyright (c) 1992, 1993 by Matt Welsh, Larry Greenfield and Karl Fogel
%
% Este fichero puede redistribuirse libremente, pero debe conservarse este distintivo 
% de copyright en todas las copias, y s�lo debe ser distribuido como parte de 
% "\linux: Instalaci�n y primeros pasos". El uso de este archvi est� cubierto por el 
% copyright para todo el documento, en el fichero "copyright.tex".
%
% Copyright (c) 1998 by Specialized Systems Consultants Inc. 
% <ligs@ssc.com>
%Revisi�n 1 por Francisco Javier Fern�ndez
%Revisi�n 2 por Fco. J Fern�ndez  el 9 de septiembre de 2002
%\section{\linux plumbing.} \label{sec-plumbing}
\section{Fontaner�a \linux} 
\label{sec-plumbing}
\markboth{Tutorial de  {\linux} }{Fontaner�a {\linux}}

%\subsection{Standard input and standard output.}
\subsection{Entrada y salida est�ndar.}
\index{standard input|(}
\index{standard output|(}
\index{stdin}
\index{stdout}
\index{entrada est�ndar}
\index{salida est�ndar}

Muchas instrucciones de {\linux} toman la entrada de lo que se llama {\bf standard input}
y mandan su salida a  {\bf standard output} (a menudo abreviados como
{\tt stdin} y {\tt stdout}). El int�rprete de �rdenes arregla las cosas de forma que la entrada
est�ndar es su teclado y la salida est�ndar es la pantalla.

He aqu� un ejemplo en el que se usa la orden {\tt cat}. Normalmente, 
{\tt cat} lee datos de todos los argumentos especificados por la l�nea de 
�rdenes y manda estos datos directamente a {\tt stdout}. Por tanto 
usando la orden 
\begin{tscreen}
/home/larry/papers\# cat history-final masters-thesis
\end{tscreen}
se muestra el contenido del fichero {\tt history-final} seguido por 
{\tt masters-thesis}. 

Sin embargo, si no especifica un nombre de fichero, {\tt cat} lee datos 
de {\tt stdin} y los devuelve a {\tt stdout}. Aqu� hay un ejemplo:
\begin{tscreen}
/home/larry/papers\# cat \\
Hello there. \\
Hello there. \\
Bye. \\
Bye. \\
\key{Ctrl-D} \\
/home/larry/papers\# 
\end{tscreen}
\index{se�al de fin-de-texto}
\index{EOT!se�al}
Cada l�nea que escriba ser� repetida inmediatamente por {\tt cat}. Cuando 
se lee de la entrada est�ndar, se le indica que la entrada ha "finalizado" 
enviando una se�al EOT (end-of-text , final de texto), que se genera 
pulsando \key{Ctrl-D}.

% By the way, guys, there's no such thing as an EOF character in UNIX. The
% terminal signal to signal EOT (end of text) is usually ^D. Files on
% disk don't have a terminating EOF as MS-DOS files do. The data just
% ends, and the read() call signals the end of data. :) --mdw

He aqu� otro ejemplo. La orden {\tt sort} lee l�neas de texto (de nuevo, 
de stdin, a no ser que le especifique uno o m�s nombres de ficheros) y manda
la salida ordenada a stdout. Pruebe lo siguiente.

\begin{tscreen}
/home/larry/papers\# sort \\
bananas \\
zanahorias \\
manzanas \\
\key{Ctrl-D} \\
bananas \\
manzanas \\
zanahorias \\
/home/larry/papers\# 
\end{tscreen}
Ahora ya podemos ordenar por orden alfab�tico la lista de la compra, 
�verdad que {\linux} es �til?

%subsection{Redirecting input and output.}
\subsection{Redireci�n de la  entrada y la salida}
\index{redirecci�n!entrada est�ndar}
\index{redirecci�n!salida est�ndar}
\index{salida!redirecci�n}
\index{salida est�ndar!redirecci�n}
\index{>@{\tt \verb'>'}}
Ahora, digamos que quiere mandar la salida de {\tt sort} a un fichero, 
para guardar nuestra lista de la compra en el disco. El int�rprete de �rdenes le permite 
{\bf redireccionar} la salida est�ndar a un nombre de fichero, usando el 
s�mbolo ``{\tt {\verb'>'}}''. Aqu� est� c�mo funciona:
\begin{tscreen}
/home/larry/papers\# sort $>$ listacompra \\
bananas \\
zanahorias \\
manzanas \\
\key{Ctrl-D} \\
/home/larry/papers\# 
\end{tscreen}
Como puede ver, el resultado de la orden {\tt sort} no se visualiza, 
pero se guarda en el fichero llamado {\tt listacompra}.
Veamos este fichero:
\begin{tscreen}
/home/larry/papers\# cat listacompra \\
bananas \\
manzanas \\
zanahorias \\
/home/larry/papers\# 
\end{tscreen}
Ahora puede ordenar su lista de la compra �y guardarla tambi�n!. Pero 
supongamos que est� guardando la lista de la compra original sin ordenar 
en el fichero {\tt items}. Un modo de ordenar la informaci�n y guardarla 
en un fichero ser�a darle a {\tt sort} el nombre del fichero a ser le�do, 
en lugar de la entrada est�ndar, y redireccionar la salida est�ndar como 
lo hicimos arriba, como sigue:
\begin{tscreen}
/home/larry/papers\# sort items $>$ listacompra \\
/home/larry/papers\# cat listacompra \\
bananas \\
manzanas \\
zanahorias \\
/home/larry/papers\# 
\end{tscreen}
\index{entrada!redirecci�n}
\index{entrada est�ndar!redirecci�n}
\index{<@{\tt \verb'<'}}
Sin embargo, hay otra forma de hacer esto. No s�lo puede 
redireccionar la salida est�ndar, tambi�n puede redireccionar la 
{\em entrada} est�ndar, usando el s�mbolo ``{\tt \verb'<'}''.
\begin{tscreen}
/home/larry/papers\# sort $<$ items \\
bananas \\
manzanas \\
zanahorias \\
/home/larry/papers\# 
\end{tscreen}
T�cnicamente, {\tt sort \verb'<' items} es equivalente a {\tt sort items}, pero
vamos a demostrar lo siguiente: {\tt sort \verb'<' items} se comporta como si los
datos del fichero {\tt items} fueran tecleados a la entrada est�ndar. El int�rprete de �rdenes 
maneja el redireccionamiento. A {\tt sort} no se le di� el nombre del fichero 
({\tt items}) a leer; en lo que concierne a {\tt sort}, �l todav�a lee de la 
entrada est�ndar como si hubiera tecleado los datos desde su teclado.

\index{filtros!definici�n}

Esto introduce el concepto de {\bf filtro}. Un filtro es un programa que 
lee datos de la entrada est�ndar, los procesa de alguna forma, y manda 
los datos procesados a la salida est�ndar. Usando la redirecci�n, la
entrada y salida est�ndar pueden ser referenciadas desde ficheros. Como se 
mencion� m�s arriba {\tt stdin} y {\tt stdout} son por omisi�n el teclado 
y la pantalla respectivamente. El programa {\tt sort} es un filtro simple. Ordena 
los datos entrantes y manda el resultado a la salida est�ndar. M�s sencillo 
a�n es {\tt cat}. No hace nada con los datos entrantes,  s�lo devuelve
todo lo que se le entrega.

%\subsection{Using pipes.}
\subsection{Uso de tuber�as}

\index{tuber�as (pipes)!uso|(}
Ya mostramos como usar {\tt sort} como un filtro. Sin embargo, 
estos ejemplos dan por hecho que usted tiene los datos guardados en
alguna parte o que teclear� los datos desde la entrada est�ndar. �Qu�
pasa si los datos que quiere ordenar vienen de la salida de otro programa,
como {\tt ls}? 

La opci�n {\tt -r} de {\tt sort} ordena los datos en orden alfab�tico 
inverso. Si quiere listar los ficheros de su directorio actual en orden 
inverso una forma de hacerlo es como sigue:
\begin{tscreen}
/home/larry/papers\# ls \\
english-list \\
history-final \\
masters-thesis \\
notes \\
\end{tscreen}
Ahora el redireccionamiento env�a la salida de la orden {\tt ls} a un fichero llamado 
{\tt file-list}:
\begin{tscreen}
/home/larry/papers\# ls $>$ file-list \\
/home/larry/papers\# sort -r file-list \\
notes \\
masters-thesis \\
history-final \\
english-list \\
/home/larry/papers\# 
\end{tscreen}
Aqu�, usted guarda la salida de un {\tt ls} en un fichero, y luego ejecuta 
{\tt sort -r} con ese fichero. Pero esto es inc�modo y usa un fichero 
temporal para guardar los datos de {\tt ls}.

\index{pipelining!definici�n}
\index{canales!creaci�n}
\index{canalizaci�n!definici�n}
La soluci�n es la {\bf canalizaci�n}\NT{pipelining}. �sta es una posibilidad del int�rprete de �rdenes, 
que conecta una serie de �rdenes mediante una ``tuber�a.''  La 
{\tt stdout} del primer programa se env�a a la {\tt stdin} del segundo 
programa. En este caso, queremos enviar la {\tt stdout} de {\tt ls} a la {\tt
stdin} de {\tt sort}.  Se utiliza el s�mbolo ``{\tt |}'' para crear una tuber�a, 
como sigue:
\begin{tscreen} 
/home/larry/papers\# ls $\mid$ sort -r \\
notes \\
masters-thesis \\
history-final \\
english-list \\
/home/larry/papers\# 
\end{tscreen}
Este programa es m�s corto y m�s f�cil de teclear.

He aqu� otro �til ejemplo, la orden
\begin{tscreen}
/home/larry/papers\# ls /usr/bin 
\end{tscreen}
muestra una lista larga de ficheros, la mayor�a de los cu�les
salen de la pantalla demasiado r�pido como para que lo pueda leer.
As� que, usamos {\tt more} para mostrar la lista de ficheros de 
{\tt /usr/bin}.
\begin{tscreen}
/home/larry/papers\# ls /usr/bin $\mid$ more 
\end{tscreen}
Ahora ya puede paginar las lista de ficheros c�modamente.

�Pero lo mejor no termina aqu�! Puede hacer canalizaciones entre m�s de dos 
programas juntos. El programa {\tt head} es un filtro que muestra las 
primeras l�neas de un flujo entrante (en este caso, entrada de una 
canalizaci�n). Si quiere mostrar el �ltimo nombre de fichero en orden 
alfab�tico del directorio actual, use estas �rdenes:
\begin{tscreen}
/home/larry/papers\# ls $\mid$ sort -r $\mid$ head -1 \\
notes \\
/home/larry/papers\# 
\end{tscreen}
donde {\tt head -1} muestra la primera l�nea de entrada que recibe (en 
este caso, el flujo de datos ordenados inversamente de {\tt ls}). 
\index{tuber�as (pipes)!uso|)}

%\subsection{Non-destructive redirection of output.}
\subsection{Redirecci�n no destructiva de la salida}
\index{ficheros!a�adiendo a}
\index{redirecci�n!no destructiva}
Usar ``{\tt {\verb'>'}}'' para redireccionar la salida a un fichero es destructivo: 
en otras palabras, la orden:
\begin{tscreen}
/home/larry/papers\# ls $>$ file-list
\end{tscreen}
sobreescribe el contenido del fichero {\tt file-list}. Si en su lugar, 
redirecciona con el s�mbolo ``{\tt {\verb'>>'}}'', la salida ser� concatenada 
al final del fichero, en vez de sobreescribirlo. Por ejemplo,
\begin{tscreen}
/home/larry/papers\# ls $>>$ file-list
\end{tscreen}
a�ade la salida de la orden {\tt ls} a {\tt file-list}.

Tenga presente que el redireccionamiento y las canalizaciones son caracter�sticas 
del int�rprete de �rdenes, que da soporte al uso de ``{\tt {\verb'>'}}'', ``{\tt {\verb'>>'}}'' y 
``{\tt {\verb'|'}}''. No tiene nada que ver con las �rdenes propiamente dichas.

\index{estrada est�ndar|)}
\index{salida est�ndar|)}
% Need to cover use of << and possibly use of stderr. Problem with 
% covering stderr is that it's different in different shells. Maybe later.


% \linux: Instalaci�n y primeros pasos    -*- TeX -*-
% perms.tex
% Copyright (c) 1992, 1993 by Matt Welsh, Larry Greenfield and Karl Fogel
%
% Este fichero puede redistribuirse libremente, pero debe conservarse este distintivo 
% de copyright en todas las copias, y s�lo debe ser distribuido como parte de 
% "\linux: Instalaci�n y primeros pasos". El uso de este fichero est� cubierto por el 
% copyright para todo  el documento, en el fichero "copyright.tex".
%
% Copyright (c) 1998 by Specialized Systems Consultants Inc. 
% <ligs@ssc.com>

%\section{File permisos.}
\section{Permisos de fichero}
\markboth{Tutorial de {\linux} }{Permisos de fichero}
\label{sec-file-perms}\label{sec-perms}

\index{ficheros!permisos de|(}
\index{permisos!de fichero|(}
\subsection{Conceptos de permisos de fichero}

\index{ficheros!permisos!definici�n}
\index{permisos!definici�n}
\index{ficheros!propiedad del usuario}
Como normalmente hay m�s de un usuario en un sistema {\linux}, �ste
proporciona un mecanismo conocido como {\bf permisos de fichero}, 
que protege los ficheros de los usuarios de las intromisiones de otros
usuarios.  Este mecanismo permite que los ficheros y directorios "sean 
propiedad" de un usuario en concreto. Por ejemplo, como Larry cre� los
ficheros en su directorio de usuario, Larry es el due�o de esos 
ficheros y tiene acceso a ellos.

{\linux} tambi�n permite que los ficheros sean compartidos por usuarios y 
grupos de usuarios. Si Larry quisiera, podr�a denegar el acceso a sus ficheros
de forma que ning�n otro usuario tuviera acceso a ellos. Sin embargo, en la
mayor�a de sistemas est� predefinido el permitir a otros usuarios la lectura de sus 
ficheros, pero nunca modificarlos o borrarlos.

\index{ficheros!propiedad del grupo}
Todo fichero es propiedad de un usuario particular. Sin embargo, los
ficheros tambi�n son propiedad de un {\bf grupo}, que es un grupo definido
de usuarios del sistema. Cada usuario se coloca en, al menos, un grupo al 
crearse su cuenta de usuario. Sin embargo, el administrador del sistema 
puede conceder al usuario el acceso a m�s de un grupo.

\index{usuarios!en grupos}
\index{grupos}
Los grupos se definen normalmente por el tipo de usuarios que accede a la
m�quina. Por ejemplo, en un sistema {\linux} universitario los usuarios pueden
ser situados en los grupos {\tt student}, {\tt staff}, {\tt faculty} o {\tt
guest}. Tambi�n hay unos pocos grupos definidos por el sistema (como {\tt bin} 
y {\tt admin}) usados por el propio sistema para controlar el acceso a 
los recursos --- es muy raro que usuarios de verdad pertenezcan a estos grupos
de sistemas.

Hay tres clases principales de permisos: de lectura, escritura y ejecuci�n.
Estos permisos pueden ser concedidos a tres tipos de usuarios: al propietario 
del fichero, al grupo al que pertenece el fichero y a todos los usuarios, 
independientemente del grupo.

\index{permisos!lectura}
\index{ficheros!permisos!lectura}
\index{directorio!permisos!lectura}
\index{permisos!escritura}
\index{ficheros!permisos!escritura}
\index{directorio!permisos!escritura}
\index{permisos!ejecuci�n}
\index{ficheros!permisos!ejecuci�n}
\index{directorio!permisos!ejecuci�n}
Los permisos de lectura permiten a un usuario leer el contenido de un
fichero, o, en el caso de un directorio, listar su contenido (usando 
{\tt ls}). Los permisos de escritura permiten a los usuarios escribir 
y modificar un fichero. Para directorios, los permisos de escritura 
permiten al usuario crear nuevos ficheros o borrar ficheros dentro de
ese directorio. Finalmente, los permisos de ejecuci�n permiten al usuario
ejecutar el fichero como un programa o gui�n de int�rprete de �rdenes (si el fichero es
un programa o un gui�n del int�rprete de �rdenes). En cuanto a los directorios, tener 
permisos de ejecuci�n permite al usuario hacer un {\tt cd} al directorio
en cuesti�n.


\subsection{Interpretando los permisos de fichero}
\index{permisos!interpretaci�n}
\index{ficheros!permisos!interpretaci�n}
\index{ficheros!listado de permisos con ls@listando permisos con {\tt ls}}
\index{ls@{\tt ls}!listado permisos de fichero con}
Veamos un ejemplo de demostraci�n de los permisos de fichero. Usando 
la orden {\tt ls} con la opci�n {\tt -l} se muestra un listado de ficheros 
en formato largo, incluyendo los permisos de los ficheros.
\begin{tscreen}
/home/larry/foo\# ls -l stuff
\begin{verbatim}
-rw-r--r--   1 larry    users         505 Mar 13 19:05 stuff
\end{verbatim}
/home/larry/foo\#
\end{tscreen}

El primer campo en el listado representa los permisos del fichero. El 
tercer campo es el propietario del fichero ({\tt larry}) y el cuarto
campo es el grupo al que pertenece el fichero ({\tt users}). Obviamente, 
el �ltimo campo es el nombre del fichero ({\tt stuff}). Explicaremos los 
dem�s campos despu�s.

El propietario de este fichero es {\tt larry}, y pertenece al grupo 
{\tt users}. La cadena {\tt -rw-r--r--} lista, en orden, los permisos 
concedidos al propietario del fichero, al grupo al que pertenece el fichero
y a todos los dem�s.

El primer car�cter de la cadena de permisos (``{\tt -}'') representa el tipo
de fichero. Un ``{\tt -}''  significa que es un fichero normal (a diferencia 
de un directorio o un controlador de dispositivo). Los tres caracteres 
siguientes (``{\tt rw-}'') representan  los permisos concedidos al due�o del
fichero, {\tt larry}. La ``{\tt r}'' viene de  ``read'' (lectura) y la ``{\tt w}''
viene de ``escritura'' (escritura). As�, {\tt larry} tiene permisos de lectura y 
escritura al fichero {\tt stuff}. 

Como ya se ha dicho, adem�s de los permisos de lectura y escritura, hay tambi�n 
un permiso de ejecuci�n, representado por una ``{\tt x}''. Sin embargo, 
un ``{\tt -}'' es listado aqu� en el lugar de una ``{\tt x}'', as� que Larry no
tiene permiso de ejecuci�n de este fichero. Esto est� bien, ya que el fichero
{\tt stuff} no es un programa de ning�n tipo. Naturalmente, como Larry es el 
propietario del fichero, se puede conceder a s� mismo el permiso de ejecuci�n 
si as� lo desea.
(Esto ser� descrito en breve)

Los tres caracteres siguientes,(``{\tt r--}''), representan los permisos del 
grupo sobre el fichero. El grupo al que pertenece este fichero es {\tt users}. 
Como s�lo aparece una `{\tt r}'' aqu�, cualquier usuario que pertenezca al grupo
{\tt users} podr� leer este fichero.

Los tres �ltimos caracteres, tambi�n (``{\tt r--}''), representan los permisos
concedidos al resto de usuarios en el sistema (otros que no sean el propietario 
del fichero ni los del grupo {\tt users}). De nuevo, como s�lo est� presente la
``{\tt r}'', los otros usuarios podr�n leer el fichero, pero no escribir en �l 
o ejecutarlo.

Aqu� hay algunos otros ejemplos de permisos:
\begin{dispitems}
\ditem{{\tt -rwxr-xr-x}}
El propietario del fichero puede leer, escribir, y ejecutar el fichero. Los 
usuarios del grupo del fichero, y todos los dem�s usuarios, pueden leer y 
ejecutar el fichero.

\ditem{{\tt -rw-------}}
El due�o del fichero puede leer y escribir en el fichero. Ning�n otro usuario 
puede acceder a este fichero.

\ditem{{\tt -rwxrwxrwx}}
Todos los usuarios pueden leer, escribir y ejecutar el fichero.
\end{dispitems}

\subsection{Dependencias}
\index{permisos!dependencias de}
\index{ficheros!permisos!dependencias de}
\index{directorio!permisos!dependencias de}
Los permisos concedidos a un fichero dependen tambi�n de los permisos del 
directorio en el que est� localizado el fichero. Por ejemplo, aunque un
fichero est� fijado a {\tt -rwxrwxrwx}, otros usuarios no podr�n acceder al
fichero si no tienen acceso de lectura y de ejecuci�n al directorio en el que
se encuentra el fichero. Por ejemplo si Larry quisiera restringir el acceso 
a todos sus ficheros, podr�a fijar los permisos de su directorio principal 
de usuario {\tt /home/larry} a {\tt -rwx------}. De esta forma, ning�n otro
usuario tendr� acceso a su directorio, ni a todos los ficheros y directorios 
dentro de �l. Larry no tiene que preocuparse de los permisos individuales de 
cada fichero.

En otras palabras, para que todos pueden acceder a un fichero, se debe 
tener acceso en ejecuci�n para todos los directorios a lo largo del camino 
del fichero, y acceso en lectura (o en ejecuci�n) para el propio fichero.

Normalmente, los usuarios de un sistema {\linux} son muy abiertos con sus
ficheros. Los permisos t�picos que se le dan a los ficheros son 
{\tt -rw-r--r--}, que permiten a otros usuarios leer el fichero pero nunca
cambiarlo. A los directorios se les suele dar los permisos 
{\tt -rwxr-xr-x}, que permiten a otros usuarios mirar por tus directorios, 
pero no crear o borrar ficheros dentro de ellos.

Sin embargo, muchos usuarios desean mantener a los dem�s lejos de sus ficheros.
Si se establecen los permisos de un fichero a {\tt -rw-------} se conseguir� 
que cualquier otro usuario no puede acceder al fichero. De la misma forma, al 
fijarse los permisos de un directorio como {\tt -rwx------} se mantiene a 
otros usuario fuera del directorio en cuesti�n.


\subsection{Cambio de permisos}
\index{permisos!cambiando}
\index{ficheros!permisos!cambiando}
\index{directorio!permisos!cambiando}
\index{chmod@{\tt chmod}}
La instrucci�n {\tt chmod} se usa para establecer los permisos de un fichero. S�lo 
el propietario de un fichero puede cambiar los permisos de ese fichero.
La sintaxis de {\tt chmod} es
\begin{tscreen}
chmod \{a,u,g,o\}\{+,-\}\{r,w,x\} \cparam{nombre\_fichero}
\end{tscreen}

Brevemente, puede poner uno o m�s de estos: {\bf a}ll (todos), {\bf u}ser 
(usuario), {\bf g}roup (grupo), u {\bf o}ther (otros).Despu�s especifica 
si est�s a�adiendo derechos ({\tt +}) o quit�ndolos ({\tt -}). Finalmente,
especifica uno o m�s de estos: {\bf r}ead (lectura), {\bf w}rite (escritura), 
y e{\bf x}ecute (ejecuci�n). Algunos ejemplos de instrucciones correctas son:

\begin{dispitems}
\ditem{{\tt chmod a+r stuff}}
Da a todos los usuarios permiso de lectura al fichero.
\ditem{{\tt chmod +r stuff}}
Lo mismo que arriba---si ninguno de {\tt a}, {\tt u}, {\tt g}, o {\tt o} se
especifica, se toma {\tt a} como predeterminado.
\ditem{{\tt chmod og-x stuff}}
Quita el permiso de ejecuci�n de todos los usuarios menos del propietario.
\ditem{{\tt chmod u+rwx stuff}}
Permite al propietario, leer, escribir y ejecutar el fichero.
\ditem{{\tt chmod o-rwx stuff}}
Quita los permisos de lectura, escritura y ejecuci�n de los usuarios que no
son el due�o ni los usuarios del grupo del fichero.
\end{dispitems}

\index{ficheros!permisos de|)}
\index{permisos!de ficheros|)}

% \linux Installation and Getting Started    -*- TeX -*-
% links.tex
% Copyright (c) 1992, 1993 by Matt Welsh <mdw@sunsite.unc.edu>
%
% This file is freely redistributable, but you must preserve this copyright 
% notice on all copies, and it must be distributed only as part of "\linux 
% Installation and Getting Started". This file's use is covered by the 
% copyright for the entire document, in the file "copyright.tex".
%
% Copyright (c) 1998 by Specialized Systems Consultants Inc. 
% <ligs@ssc.com>

%\section{Managing file links.}\label{sec-manage-links}
%Revisi�n 1 Francisco Javier Fern�ndez Serrador <serrador@arrakis.es>
%Gold
\section{Gesti�n  de enlaces a ficheros}
\label{sec-manage-links}
\markboth{Tutorial de {\linux}}{Gesti�n de enlaces a ficheros}
\index{ficheros!enlaces|(}
\index{enlaces|(}
\index{n�mero de inode!definici�n}
\index{ficheros!n�mero de inodo de}
\index{inodos!definici�n}
Los enlaces permiten darle a un fichero m�s de un nombre. Realmente, el
sistema identifica los ficheros por su {\bf n�mero de inodo}, que es
el �nico identificador del fichero para el sistema de ficheros.
Un directorio es en realidad una lista de n�meros de inodos con sus
correspondientes nombres de fichero. Cada nombre de fichero dentro de un directorio
es un {\bf enlace} a un inodo concreto.

%\subsection{Hard links.}
\subsection{Enlaces r�gidos}
\index{enlaces!duros}
\index{enlaces!r�gidos}
La orden {\tt ln} se utiliza para crear m�ltiples enlaces a un
fichero. Por ejemplo, digamos que tiene un fichero llamado {\tt foo} en 
un directorio. Usando {\tt ls -i}, puede ver el n�mero de inodo
de este fichero.
\begin{tscreen}
/home/larry\# ls -i foo \\
22192 foo \\
/home/larry\#
\end{tscreen}
Aqu�, {\tt foo} tiene un n�mero de inodo de 22192 en el
sistema de ficheros. Puede crear otro enlace a {\tt foo}, llamado {\tt bar}, como sigue:
\begin{tscreen}
/home/larry\# ln foo bar 
\end{tscreen}
Con {\tt ls -i}, puede comprobar que los dos ficheros tienen el mismo n�mero de inodo.
\begin{tscreen}
/home/larry\# ls -i foo bar \\
22192 bar\ \ \ 22192 foo \\
/home/larry\#
\end{tscreen}
Ahora, especificando tanto {\tt foo} como {\tt bar} se acceder�
al mismo fichero. Si hace cambios en {\tt foo}, esos cambios
aparecen tambi�n en {\tt bar}. A todos los efectos, {\tt foo}
y {\tt bar} son el mismo fichero.

A este tipo de enlaces se les conoce como {\bf enlaces r�gidos\/} porque directamente
crean el enlace al inodo. Tenga en cuenta que puede crear enlaces r�gidos s�lo
cuando est�n en el mismo sistema de ficheros; los enlaces simb�licos (ver debajo) no tienen
esta restricci�n.

Cuando borra un fichero con {\tt rm}, realmente s�lo est�
borrando uno de los enlaces a ese fichero. Si usa la orden
\begin{tscreen}
/home/larry\# rm foo
\end{tscreen}
entonces s�lo el enlace llamado {\tt foo} se borra, {\tt bar}
todav�a existir�. Un fichero s�lo se borra realmente del sistema
cuando no tiene enlaces. Normalmente, los ficheros tienen un
�nico enlace, por lo que usando la orden {\tt rm} se borra el fichero. Sin embargo,
si un fichero tiene m�ltiples enlaces, usando {\tt rm} s�lo se borrar�
un enlace simple; para borrar el fichero, deber� borrar todos los enlaces a �l.

\index{enlaces!mostrar n�mero de}
La orden {\tt ls -l} muestra el n�mero de enlaces a un fichero
(entre otra informaci�n).
\begin{tscreen}
/home/larry\# ls -l foo bar \\
\verb!-rw-r--r--   2 root     root          12 Aug  5 16:51 bar! \\
\verb!-rw-r--r--   2 root     root          12 Aug  5 16:50 foo! \\
/home/larry\#
\end{tscreen}
La segunda columna del listado, ''{\tt 2}'', especifica el n�mero de
enlaces al fichero.

Asi resulta que un directorio no es realmente m�s que un fichero que contiene informaci�n
sobre asociaciones enlaces-a-inodos. Adem�s, cada directorio contiene al menos dos
enlaces r�gidos: ''{\tt .}'' (un enlace apuntando a �l mismo) y 
''{\tt ..}'' (un enlace
apuntando a su directorio padre). El enlace ''{\tt ..}'' del directorio ra�z ({\tt /}) 
simplemente vuelve a apuntar a {\tt /}.  (En otras palabras, el directorio padre del
directorio ra�z es �l mismo.)

%\subsection{Symbolic links.}
\subsection{Enlaces simb�licos.}
\index{enlaces!simb�licos}
Los enlaces simb�licos son otro tipo de enlace, diferente
al enlace r�gido. Un enlace simb�lico permite dar otro nombre
a un fichero, pero no enlaza el fichero mediante el inodo.

La orden {\tt ln -s} crea un enlace simb�lico a un fichero.
Por ejemplo, si utiliza la orden
\begin{tscreen}
/home/larry\# ln -s foo bar
\end{tscreen} 
crear� un enlace simb�lico llamado {\tt bar} que apunte al fichero
{\tt foo}. Si utiliza {\tt ls -i}, ver� que los dos ficheros
tienen diferentes inodos.
\begin{tscreen}
/home/larry\# {\em ls -i foo bar} \\
22195 bar\ \ \ 22192 foo \\
/home/larry\#
\end{tscreen}
Sin embargo, usando {\tt ls -l}, vemos que el fichero {\tt bar}
es un enlace simb�lico apuntando a {\tt foo}.
\begin{tscreen}
/home/larry\# ls -l foo bar \\
\verb!lrwxrwxrwx   1 root     root           3 Aug  5 16:51 bar -> foo! \\
\verb!-rw-r--r--   1 root     root          12 Aug  5 16:50 foo! \\
/home/larry\#
\end{tscreen}

Los permisos de fichero de un enlace simb�lico no se utilizan (siempre
aparecen como {\tt rxwrxwrxw}). En su lugar, los permisos del enlace simb�lico
est�n determinados por los permisos del destino del enlace simb�lico (en nuestro
ejemplo, el fichero {\tt foo}).

Funcionalmente, los enlaces r�gidos y simb�licos son similares, pero hay diferencias.
Por un lado, se pueden crear enlaces simb�licos a ficheros que no existen, cosa
que no sucede con los enlaces r�gidos. Los enlaces simb�licos son
procesados de manera distinta a los r�gidos por el n�cleo, lo que constituye una
mera diferencia t�cnica pero que a veces puede resultar importante. Los enlaces
simb�licos son de ayuda porque identifican al fichero al que apuntan; con enlaces
r�gidos, no hay una manera f�cil de determinar qu� ficheros est�n enlazados al
mismo inodo.

Los enlaces se utilizan en muchos lugares dentro de un sistema \linux. Los
enlaces simb�licos son especialmente importantes para las bibliotecas compartidas
en {\tt /lib}. Consulte la p�gina~\pageref{sec-upgrade-libs} para m�s informaci�n.
\index{ficheros!enlaces|)}
\index{enlaces|)}

% \linux Installation and Getting Started    -*- TeX -*-
% job-control.tex
% Copyright (c) 1992, 1993 by Matt Welsh, Larry Greenfield and Karl Fogel
%
% This file is freely redistributable, but you must preserve this copyright 
% notice on all copies, and it must be distributed only as part of "\linux 
% Installation and Getting Started". This file's use is covered by
% the copyright for the entire document, in the file "copyright.tex".
%
% Copyright (c) 1998 by Specialized Systems Consultants Inc. 
% <ligs@ssc.com>

\section{Control de tareas.}\label{sec-job-control}
\markboth{Tutorial de {\linux}}{Control de Tareas}

\index{control de tareas|(}
\subsection{Tareas y procesos.}
\label{sec-process}\label{sec-processes}

\index{shells!controlde tareas proporcionado por}
El {\bf control de tareas} es una caracter�stica que incluyen muchos {\bf int�rpretes de �rdenes}
(incluyendo {\tt bash} y {\tt tcsh}) que permiten controlar m�ltiples �rdenes o {\bf tareas}
ejecut�ndose a la vez. Antes de ir m�s lejos, hay que hablar de los {\bf procesos}.

\index{proceso!definici�n}
\index{ps@{\tt ps}}
\index{proceso!ps para listar@{\tt ps} para listar}
Cada vez que se ejecuta un programa, se arranca lo que se denomina un proceso.
La orden {\tt ps} muestra una lista de los procesos actualmente en ejecuci�n,
como se ve aqu�:
\begin{tscreen}
/home/larry\# ps
\begin{verbatim}
  PID TT STAT  TIME COMMAND 
   24  3 S     0:03 (bash) 
  161  3 R     0:00 ps
\end{verbatim}
/home/larry\#
\end{tscreen}
\index{proceso!ID!definici�n}
En la primera columna aparece el {\tt PID} o {\bf identificador de proceso},
un n�mero �nico dado a cada proceso en ejecuci�n. La �ltima columna,
{\tt COMMAND}, es el nombre de la orden en ejecuci�n. Aqu�, estamos viendo
�nicamente los procesos que est� ejecutando el propio Larry. (Tambi�n hay otros
muchos procesos en ejecuci�n en el sistema---''{\tt ps -aux}''
los lista todos.) �stos son {\tt bash} (el {\bf int�rprete de �rdenes} de Larry)
y la propia orden {\tt ps}. Como puede ver,
{\tt bash} se ejecuta al mismo tiempo que la orden {\tt ps}.
{\tt bash} hizo que se ejecutara {\tt ps} cuando Larry escribi� la orden. Cuando
{\tt ps} ha finalizado su ejecuci�n (despu�s de haber mostrado la tabla de procesos),
el proceso {\tt bash} vuelve a tomar el control, y muestra el s�mbolo del sistema,
listo para recibir otra orden.

\index{tarea!definici�n}
A un proceso en ejecuci�n se le llama tambi�n {\em tarea\/}. Los t�rminos
{\em proceso\/} y {\em tarea\/} son intercambiables. Sin embargo, nos referimos a
un proceso como ''tarea'' cuando lo usamos en conjunci�n con {\bf
control de tareas} ---una caracter�stica del {\bf int�rprete de �rdenes} que permite conmutar entre
varios procesos independientes.

En muchos casos, los usuarios ejecutan una �nica tarea a la vez---cualquiera que
fuera la �ltima orden que escribieron. Sin embargo, usando el control de
tareas, se puede ejecutar varias tareas a la vez y conmutar entre ellas
cuando haga falta.

�Para qu� puede ser esto �til? Digamos que est� editando un fichero de texto y
quiere interrumpir la edici�n para hacer cualquier otra cosa. Mediante el control
de tareas, puede suspender temporalmente el editor, volver al s�mbolo del {\bf int�rprete de �rdenes}
y empezar a trabajar en otra cosa. Cuando haya terminado, puede volver al editor donde
lo dej�, como si no lo hubiera abandonado. Hay otros muchos usos pr�cticos
del control de tareas.

%\subsection{Foreground and background.}
\subsection{Primer plano y segundo plano.}

\index{proceso!foreground}
\index{proceso!background}
\index{tarea!foreground}
\index{tarea!background}
\index{proceso en primer plano}
\index{proceso en segundo plano}
\index{primer plano}
\index{segundo plano}
\index{proceso!primer plano}
\index{proceso!segundo plano}

Las tareas pueden estar tanto en {\bf primer plano} como en {\bf segundo plano}.
S�lo puede haber una tarea en primer plano cada vez. La tarea que est�
en primer plano es aquella con la que se interact�a --recibe la entrada
desde el teclado y env�a la salida a la pantalla, a menos que, por supuesto, se haya
redireccionado la entrada o la salida, como se describe en la
p�gina~\pageref{sec-plumbing}--. Por otro lado, las tareas que est�n en segundo
plano no reciben entradas desde el terminal --en general, se ejecutan
tranquilamente sin necesidad de interacci�n--.

Algunas tareas tardan mucho tiempo en acabar y no hacen nada interesante mientras
se est�n ejecutando. Compilar programas es una de esas tareas, como tambi�n
lo es comprimir un fichero grande. No hay ning�n motivo para estar sentado y aburrido
mientras espera a que estas tareas acaben; simplemente ejec�telos
en segundo plano. Mientras esas tareas corren en segundo plano, existe libertad
para ejecutar otros programas.

\index{tarea!suspendida}
Las tareas tambi�n pueden ser {\bf suspendidas}. Una tarea suspendida es una
tarea que est� detenida temporalmente. Despu�s de suspender una tarea,
se puede hacer que contin�e en primer o segundo plano cuando haga falta.
Reanudar una tarea suspendida no cambia el estado de la tarea de ninguna
manera --la tarea contin�a su ejecuci�n por donde se qued�--.

\index{proceso!interrumpido}
\index{tarea!interrupci�n}
Suspender una tarea no es lo mismo que interrumpirla. Cuando se {\bf
interrumpe} un proceso en ejecuci�n (pulsando la tecla de interrupci�n, que
suele ser \key{Ctrl-C})\footnote{Se puede establecer la tecla de
interrupci�n con la orden {\tt stty}.}, se mata ese proceso, para siempre. Una
vez que se mata el proceso, no hay manera de que se reanude. Hay que ejecutar
la orden otra vez. Adem�s, algunos programas capturan la interrupci�n, de manera
que pulsar \key{Ctrl-C} no matar� inmediatamente al proceso. Esto permite al 
programa llevar a cabo cualquier operaci�n de limpieza necesaria antes de salir.
De hecho, algunos programas no permitir�n de ning�n modo que se les mate mediante
interrupci�n.

Comencemos con un ejemplo simple. La orden {\tt yes} es una orden
in�til en apariencia que manda una cadena infinita de {\tt y}s a la
salida est�ndar. (En realidad s� es �til. Si se enlaza mediante una tuber�a la salida de
{\tt yes} a otra orden que realice una serie preguntas de s� o no, la cadena
de {\tt y}s confirmar� todas las preguntas.)

Intent�moslo:
\begin{tscreen}
/home/larry\# yes \\
y \\
y \\
y \\
y \\
y
\end{tscreen}
\index{proceso!matar}
\index{proceso!interrumpir}
\index{tarea!matar}
\index{tarea!interrumpir}
Las {\tt y}s continuar�n {\em ad infinitum}. Puede matar el proceso
pulsando la tecla de interrupci�n, que normalmente es \key{Ctrl-C}.
Para que no tengamos que aguantar la molesta cadena de {\tt y}s,
redirijamos la salida est�ndar de {\tt yes} a {\tt /dev/null}.
Si recuerda, {\tt /dev/null} act�a como un ''agujero negro'' para los datos.
Cualquier dato que se le env�e desaparece. Es un modo muy efectivo de
silenciar un programa charlat�n.
\begin{tscreen}
/home/larry\# yes $>$ /dev/null
\end{tscreen}
Ah, mucho mejor. No aparece nada, pero el s�mbolo del {\bf int�rprete de �rdenes} no vuelve.
Esto es porque {\tt yes} est� todav�a en ejecuci�n, y est� mandando
esas in�tiles {\tt y}s a {\tt /dev/null}. Para matar otra vez, la tarea,
pulse la tecla de interrupci�n.

Supongamos que quiere que la orden {\tt yes} contin�e su ejecuci�n pero
conservando el s�mbolo del {\bf int�rprete de �rdenes} para que pueda trabajar en otras cosas.
Puede pasar {\tt yes} a segundo plano, permitiendo su ejecuci�n, sin necesidad
de interactuar.

\index{tarea!paso a segundo plano}
\index{tarea!segundo plano}
Una manera de poner un proceso en segundo plano es a�adir un ''{\tt \&}''
al final de la orden.
\begin{tscreen}
/home/larry\# yes $>$ /dev/null \& \\
\verb+[1] 164+ \\
/home/larry\#
\end{tscreen}
Como puede ver, el s�mbolo del {\bf int�rprete de �rdenes} ha vuelto. Pero �qu� es este
''{\tt \verb+[1] 164+}''? Y �est� ejecut�ndose realmente la orden {\tt yes}?

''{\tt \verb+[1]+}'' representa el {\bf n�mero de tarea} para el proceso {\tt yes}.
El {\bf int�rprete de �rdenes} asigna un n�mero de tarea a cada tarea en ejecuci�n. Dado que
{\tt yes} es la �nica tarea que estamos ejecutando, tiene asignado el n�mero
de trabajo {\tt 1}. ''{\tt 164}'' es el identificador de proceso, o
PID, el n�mero otorgado a la tarea por el sistema. Se puede usar cualquiera de
los n�meros para referirse a la tarea, como se ver� m�s adelante.

\index{tareas@{\tt tareas}}
Ahora tenemos el proceso {\tt yes} ejecut�ndose en segundo plano, mandando
continuamente una cadena de {\tt y}s a {\tt /dev/null}. Para comprobar
el estado de este proceso, utilizamos la orden interna del {\bf int�rprete de �rdenes} {\tt jobs}.
\begin{tscreen}
/home/larry\# jobs \\
\verb-[1]+  Running                 yes >/dev/null  &- \\
/home/larry\#
\end{tscreen}
Efectivamente, ah� est�. Tambi�n se podr�a haber utilizado la orden {\tt ps} tal y como
se mostr� arriba para comprobar el estado de la tarea.

\index{kill@{\tt kill}}
\index{proceso!segundo plano!matar}
\index{tarea!segundo plano!matar}
Para acabar con la tarea, utilice la orden {\tt kill}.
Esta orden toma un n�mero de tarea o un identificador de proceso como
argumento. �sta era la tarea n�mero 1, as� que usando la orden
\begin{tscreen}
/home/larry\# kill \%1
\end{tscreen}
se mata la tarea. Cuando se identifica la tarea con su n�mero de tarea,
se debe anteponer al n�mero un car�cter de tanto por ciento (''{\tt \%}'').

Ahora que ha matado la tarea, utilice {\tt jobs} otra vez para comprobarlo:
\begin{tscreen}
/home/larry\# jobs
\begin{verbatim}
[1]+  Terminated              yes >/dev/null 
\end{verbatim}
/home/larry\# 
\end{tscreen}
Desde luego, la tarea est� muerta, y si utiliza la orden {\tt jobs} otra vez,
no deber�a aparecer ya nada.

Tambi�n se puede matar la tarea usando el n�mero de identificaci�n del proceso (PID),
que aparece junto con el n�mero de tarea cuando lo arranca. En nuestro ejemplo,
el identificador de proceso es 164, as� que la orden
\begin{tscreen}
/home/larry\# kill 164
\end{tscreen}
equivale a
\begin{tscreen}
/home/larry\# kill \%1
\end{tscreen}
No necesita usar ''{\tt \%}'' para hacer referencia a una tarea mediante su
identificador de proceso.

%\subsection{Stopping and restarting tareas.}
\subsection{Parando y relanzando tareas}
\index{tarea!detener}
Hay otra manera de pasar una tarea a segundo plano. Puede
arrancar la tarea normalmente (en primer plano), {\bf suspender} la tarea,
y reiniciarla en segundo plano.

Primero, arranque el proceso {\tt yes} en primer plano, como
hizo antes:
\begin{tscreen}
/home/larry\# yes $>$ /dev/null
\end{tscreen}
Otra vez, como {\tt yes} est� ejecut�ndose en primer plano, no deber�a ver
el s�mbolo del {\bf int�rprete de �rdenes}.

Ahora, mejor que interrumpir la tarea con \key{Ctrl-C}, {\bf susp�ndala}.
Suspender una tarea no la mata: s�lo la detiene temporalmente
hasta que se la reinicia de nuevo. Para hacer esto, pulse la tecla de suspensi�n, que
normalmente es \key{Ctrl-Z}.
\begin{tscreen}
/home/larry\# yes $>$ /dev/null \\
\key{ctrl-Z} \\
\verb-[1]+  Stopped                 yes >/dev/null- \\
/home/larry\#
\end{tscreen}
Mientras la tarea est� suspendida, simplemente no est� en ejecuci�n. No se
emplea tiempo de CPU para esa tarea. Sin embargo, puede reiniciar la tarea, lo que provoca
que se ejecute otra vez como si nada hubiera pasado. Continuar� su ejecuci�n
por donde se qued�.

\index{fg@{\tt fg}}
\index{tarea!recomenzar}
Para reiniciar la tarea en primer plano, utilice la orden
{\tt fg} (de ''foreground''). 
\begin{tscreen}
/home/larry\# fg \\
yes >/dev/null 
\end{tscreen}
\index{tarea!segundo plano}
\index{bg@{\tt bg}}
El {\bf int�rprete de �rdenes} muestra el nombre de la orden otra vez para que est� al tanto de qu� tarea
acaba de mandar a primer plano. Detenga la tarea otra vez con \key{Ctrl-Z}.
Esta vez, use la orden {\tt bg} para pasar la tarea a segundo plano.
Esto provoca que la orden se ejecute como si lo hubiera arrancado con
''{\tt \&}'', como en la �ltima secci�n.
\begin{tscreen}
/home/larry\# bg \\
\verb-[1]+- yes >/dev/null  \& \\
/home/larry\# 
\end{tscreen}
Y aqu� tiene su s�mbolo del {\bf int�rprete de �rdenes} de vuelta. {\tt jobs} deber�a informar que {\tt yes}
est� ciertamente ejecut�ndose, y puede matar la tarea con {\tt kill} como
hicimos antes.

�C�mo se puede detener la tarea otra vez? Utilizar \key{Ctrl-Z} no funcionar�, porque
la tarea est� en segundo plano. La respuesta es pasar la tarea a primer
plano con {\tt fg}, y luego detenerla. Tal y como parece, puede
utilizar {\tt fg} tanto en tareas detenidas como en tareas en segundo plano.

Hay una gran diferencia entre una tarea en segundo plano y una tarea
detenida. Una tarea detenida no est� en ejecuci�n ---no est� usando
tiempo de CPU, y no est� haciendo nada (la tarea todav�a ocupa memoria del
sistema, aunque puede haber sido volcada a disco). Una tarea en
segundo plano s� est� ejecut�ndose y usando memoria, al tiempo que
completa alguna acci�n mientras usted hace otra cosa.

Sin embargo, una tarea en segundo plano puede intentar mostrar texto por
el terminal, lo que puede resultar molesto si est� intentando trabajar en
otra cosa. Por ejemplo, si utiliz� la orden
\begin{tscreen}
/home/larry\# yes \&
\end{tscreen}
sin redirigir stdout a {\tt /dev/null}, una cadena de {\tt y}s
estar� apareciendo en la pantalla, sin posibilidad de interrumpirla.
(No se puede usar \key{Ctrl-C} para interrumpir tareas en segundo plano.)
Para detener las infinitas {\tt y}s, utilice la orden {\tt fg} para
pasar la tarea a primer plano, y luego utilice \key{Ctrl-C}
para matarla.

Otra nota. Las �rdenes {\tt fg} y {\tt bg} normalmente afectan a la �ltima
tarea detenida (indicado por un ''{\tt +}'' junto al n�mero de tarea
cuando se usa la orden {\tt jobs}).
Si est� ejecutando diversas tareas a la vez, puede pasar tareas a primer o
segundo plano pasando el n�mero de tarea como argumento a {\tt fg} o {\tt bg}, como en
\begin{tscreen}
/home/larry\# fg \%2
\end{tscreen}
(para pasar la tarea n�mero 2 a primer plano), o
\begin{tscreen}
/home/larry\# bg \%3
\end{tscreen}
(para pasar la tarea n�mero 3 a segundo plano). No se pueden usar identificadores
de proceso (PID) con {\tt fg} o {\tt bg}. 

Adem�s, usar el n�mero de tarea s�lamente, como en
\begin{tscreen}
/home/larry\# \%2
\end{tscreen}
equivale a
\begin{tscreen}
/home/larry\# fg \%2
\end{tscreen}

Recuerde que el control de tareas es una caracter�stica del {\bf int�rprete de �rdenes}. Las instrucciones
{\tt fg}, {\tt bg} y {\tt jobs} son internas del {\bf int�rprete de �rdenes}.
Si por cualquier motivo usted utiliza un {\bf int�rprete de �rdenes} que no soporte control de tareas,
no espere encontrar estas instrucciones disponibles.

Por a�adidura, hay algunos aspectos del control de tareas que var�an entre
{\tt bash} y {\tt tcsh}. De hecho, algunos {\bf int�rpretes de �rdenes} no proporcionan control
de tareas en absoluto---de cualquier manera, la mayor�a de los {\bf int�rpretes de �rdenes} disponibles
para {\linux} s� lo proporcionan.

\index{control de tareas|)}






% \linux: Instalaci�n y primeros pasos    -*- TeX -*-
% vi.tex
% Copyright (c) 1992, 1993 by Matt Welsh <mdw@sunsite.unc.edu>
%
% Este fichero puede redistribuirse libremente, pero debe conservarse este distintivo 
% de copyright en todas las copias, y s�lo debe ser distribuido como parte de 
% "\linux: Instalaci�n y primeros pasos". El uso de este archvi est� cubierto por el 
% copyright para todo el documento, en el arhcivo "copyright.tex"
%
% Copyright (c) 1998 by Specialized Systems Consultants Inc. 
% <ligs@ssc.com>

%\section{Using the {\tt vi} editor.}\label{sec-vi}
\section{Uso del editor {\tt vi}}\label{sec-vi}
\markboth{Tutorial de {\linux}}{Uso del editor {\tt vi}}

\index{vi@{\tt vi}|(}
\index{editor de texto!definici�n}
\index{editor!definici�n}
Un {\bf editor de texto} es un programa que se usa para editar ficheros que 
se componen con texto: una carta, un programa en C o un fichero de 
configuraci�n del sistema. Aunque para {\linux} hay disponibles muchos 
editores de texto, el �nico editor que es seguro que vas a encontrar en cualquier 
sistema UNIX o {\linux} es  {\tt vi} --- el ``editor visual\NT{visual editor}.'' 
El editor {\tt vi} no es el editor m�s f�cil de usar, ni es muy autoexplicativo.
 Sin embargo, como {\tt vi} es tan com�n en el mundo UNIX/Linux, y algunas veces 
necesario, merece un tratamiento aqu�.

\index{editor de texto!comparaci�n}
\index{Emacs}
\index{GNU/Emacs}
La elecci�n de su editor es m�s que nada una cuesti�n de gusto y 
estilo personal. Muchos usuarios prefieren el barroco, autoexplicativo y potente 
{\tt GNU emacs} ---un editor con m�s posibilidades que cualquier otro programa 
en el mundo UNIX. Por ejemplo, {\tt GNU emacs} tiene su propio dialecto interno del 
lenguaje de programaci�n LISP, y tiene muchas extensiones (una de las 
cuales es un programa de inteligencia artificial similar a Eliza). Sin 
embargo, como {\tt GNU emacs} y sus ficheros de soporte son relativamente extensos, 
puede que no est� instalado en algunos sistemas. Por otro lado,{\tt vi}
es peque�o y potente pero m�s dif�cil de usar. De todas formas, una vez 
que conozca la forma de funcionamiento de {\tt vi}, ser� realmente muy f�cil.

Esta secci�n presenta una introducci�n a {\tt vi} ---no hablaremos 
sobre todas sus caracter�sticas, s�lo las necesarias para 
empezar. Puede consultar la p�gina del manual de {\tt vi} si 
est� interesado en aprender m�s sobre las caracter�sticas del 
editor. Otra alternativa es leer el libro {\em Learning the vi Editor} 
de O'Reilly y asociados, o el {\em Vi Tutorial} de Specialized Systems 
Consultants (SSC) Inc. Mire el Ap�ndice~\ref{app-info} para informarse.

%\subsection{Concepts.}
\subsection{Conceptos}
Mientras se usa {\tt vi}, en cualquier instante estar� en uno de los tres modos 
de operaci�n. Estos modos se llaman {\em modo orden}, {\em modo inserci�n}, 
y {\em modo �ltima l�nea}. 

\index{vi@{\tt vi}!modo orden}
Cuando arranca {\tt vi}, est� en el {\em modo orden}. Este modo le 
permite usar �rdenes para editar ficheros o cambiar a otros modos. 
Por ejemplo, tecleando ``{\tt x}'' en modo �rdenes se borra el car�cter 
que est� sobre el cursor. Las teclas con flechas mueven el cursor por 
el documento que est� editando. Normalmente, las �rdenes usadas en este 
modo tienen uno o dos caracteres de longitud.

\index{vi@{\tt vi}!modo edici�n}
Usted realmente inserta o edita texto en el {\em modo inserci�n}. Cuando 
use {\tt vi}, probablemente pasar� la mayor parte del tiempo en 
este modo. Se comienza el modo de inserci�n usando una orden como 
``{\tt i}'' ( de ``insertar'') desde el modo �rdenes. Cuando est� 
en el modo de inserci�n, puede insertar texto en el documento en 
la posici�n actual del cursor. Para terminar el modo de inserci�n y 
volver al modo  �rdenes, presione \key{Esc}.

\index{vi@{\tt vi}!modo �ltima l�nea}
El {\em modo �ltima l�nea} es un modo especial usado para dar ciertas 
�rdenes extendidas a {\tt vi}. Mientras teclea estas �rdenes, 
aparecen en la �ltima l�nea de la pantalla (de aqu� su nombre). Por 
ejemplo, cuando teclea ``{\tt :}'' en modo orden, salta al modo 
�ltima l�nea y puedes usar �rdenes como ``{\tt wq}'' ( para escribir 
el fichero y salir de {\tt vi}), o ``{\tt q!}'' (para salir de 
{\tt vi} sin guardar los cambios). El modo �ltima l�nea es usado 
normalmente para las �rdenes de {\tt vi} que son m�s largas de un 
car�cter. En el modo �ltima l�nea, escribe una orden de una s�la 
l�nea y pulsa {\ret} para ejecutarla. 

%\subsection{Starting {\tt vi}.}
\subsection{Comenzando con {\tt vi}}
\index{vi@{\tt vi}!empezando}
La mejor forma de comprender estos conceptos es arrancar {\tt vi} y 
editar un fichero. El ejemplo ``screens'' de abajo muestra s�lo unas 
pocas l�neas de texto, como si la pantalla tuviera seis l�neas de 
longitud en vez de 24.

La sintaxis de {\tt vi} es 
\begin{tscreen}
vi \cparam{fichero}
\end{tscreen}
donde \textsl{fichero} es el nombre del fichero a editar.

Arranque {\tt vi} tecleando
\begin{tscreen}
/home/larry\# vi test
\end{tscreen}
para editar el fichero {\tt test}. Deber�a ser algo como:
\begin{tscreen}\fbox{\begin{minipage}{0.8\textwidth}
\underline{\~{}} \\
\~{} \\
\~{} \\
\~{} \\
\~{} \\
\~{} \\
"test" [New file]
\end{minipage}}\end{tscreen}

La columna de caracteres ``{\tt \~{}}'' le indica que est� al final 
del fichero. El  \underline{\ } representa el cursor.

%\subsection{Inserting text.}
\subsection{Insertando texto}

\index{vi@{\tt vi}!inserci�n de texto|(}
El programa {\tt vi} est� ahora en modo orden. Inserte texto en el 
fichero pulsando \key{i}, que pone al editor en el modo inserci�n, y 
empiece a teclear.
\begin{tscreen}\fbox{\begin{minipage}{0.8\textwidth}
Now is the time for all good men to come to the aid of the party\underline{.} \\
\~{} \\
\~{} \\
\~{} \\
\~{} \\
\~{} 
\end{minipage}}\end{tscreen}

Teclee cuantas l�neas quiera (pulsando {\ret} despu�s de cada una). 
Tal vez quiera corregir fallos con la tecla \key{Backspace}.

Para finalizar el modo de inserci�n y volver al modo orden pulse \key{Esc}.

En el modo orden puede usar las teclas de flecha para moverse 
por el fichero. (Si s�lo tiene una l�nea de texto, el intentar usar 
las teclas de flechas de arriba y abajo probablemente causar� que 
{\tt vi} emita un pitido.) 

Hay diversas formas de insertar texto adem�s de la orden  {\tt i}. 
La orden {\tt a} inserta texto empezando despu�s de la posici�n 
actual del cursor, en vez de en la posici�n actual del cursor. Por 
ejemplo, use la tecla de flecha izquierda para mover el cursor entre 
las palabras ``good'' y ``men.''
\begin{tscreen}\fbox{\begin{minipage}{0.8\textwidth}
Now is the time for all good\underline{\ }men to come to the aid of the party. \\
\~{} \\
\~{} \\
\~{} \\
\~{} \\
\~{} 
\end{minipage}}\end{tscreen}
Pulse \key{a} para empezar el modo inserci�n, teclee ``{\tt wo}'', y 
luego pulse \key{Esc} para volver al modo orden.
\begin{tscreen}\fbox{\begin{minipage}{0.8\textwidth}
Now is the time for all good wo\underline{m}en to come to the aid of the party. \\
\~{} \\
\~{} \\
\~{} \\
\~{} \\
\~{} 
\end{minipage}}\end{tscreen}

Para empezar insertando texto en la siguiente l�nea, use
la orden {\tt o}. Pulse \key{o} e inserte una o dos l�neas:
\begin{tscreen}\fbox{\begin{minipage}{0.8\textwidth}
Now is the time for all good humans to come to the aid of the party. \\
Afterwards, we'll go out for pizza and beer\underline{.} \\
\~{} \\
\~{} \\
\~{} \\
\~{}
\end{minipage}}\end{tscreen}

% Remember that at any time you're in command mode (where commands such
% as {\tt i}, {\tt a}, or {\tt o} are valid), in insert mode (where
% you're inserting text, followed by \key{esc} to return to command
% mode), or in last line mode (where you're entering extended commands,
% as discussed below).
\index{vi@{\tt vi}!inserci�n de texto|)}

%\subsection{Deleting text.}
\subsection{Borrando texto}
\index{vi@{\tt vi}!borrando texto|(}
En modo orden, la orden {\tt x} borra el car�cter debajo del cursor. 
Si pulsa \key{x} cinco veces, terminar� con:
\begin{tscreen}\fbox{\begin{minipage}{0.8\textwidth}
Now is the time for all good humans to come to the aid of the party. \\
Afterwards, we'll go out for pizza and\underline{\ } \\
\~{} \\
\~{} \\
\~{} \\
\~{} 
\end{minipage}}\end{tscreen}
Ahora pulse \key{a} e inserte algo de texto, seguido por \key{esc}:
\begin{tscreen}\fbox{\begin{minipage}{0.8\textwidth}
Now is the time for all good humans to come to the aid of the party. \\
Afterwards, we'll go out for pizza and Diet Coke\underline{.} \\
\~{} \\
\~{} \\
\~{} \\
\~{} 
\end{minipage}}\end{tscreen}
Puede borrar l�neas enteras usando la orden {\tt dd} (esto es, 
pulsar \key{d} dos veces en una fila). Si el cursor est� en la segunda 
l�nea y teclea {\tt dd}, ver�: 
\begin{tscreen}\fbox{\begin{minipage}{0.8\textwidth}
\underline{N}ow is the time for all good humans to come to the aid of the party. \\
\~{} \\
\~{} \\
\~{} \\
\~{} \\
\~{} 
\end{minipage}}\end{tscreen}

Para borrar la palabra sobre la que est� el cursor, use la orden {\tt dw}.
Coloque el cursor en la palabra ``good'', y teclee {\tt dw}.
\begin{tscreen}\fbox{\begin{minipage}{0.8\textwidth}
Now is the time for all \underline{h}umans to come to the aid of the party. \\
\~{} \\
\~{} \\
\~{} \\
\~{} \\
\~{}
\end{minipage}}\end{tscreen}
\index{vi@{\tt vi}!borrado de texto|)}

%\subsection{Changing text.}
\subsection{Cambiando texto}
\index{vi@{\tt vi}!cambio de texto|(}
Puede reemplazar secciones de texto usando la orden  {\tt R} . 
Ponga el cursor en la primera letra de ``party'', pulse \key{R}, y teclee la 
palabra ``hungry''.
\begin{tscreen}\fbox{\begin{minipage}{0.8\textwidth}
Now is the time for all humans to come to the aid of the hungry\underline{.} \\
\~{} \\
\~{} \\
\~{} \\
\~{} \\
\~{} 
\end{minipage}}\end{tscreen}
Usar {\tt R} para editar texto es como las �rdenes {\tt i} y {\tt a} , pero 
{\tt R} sobreescribe, mejor que insertar, texto.

La orden {\tt r} reemplaza el �nico car�cter debajo del cursor. Por ejemplo 
mueva el cursor al principio de la palabra ``Now'', y presione {\tt r} 
seguido de {\tt C}, ver�:
\begin{tscreen}\fbox{\begin{minipage}{0.8\textwidth}
\underline{C}ow is the time for all humans to come to the aid of the hungry. \\ 
\~{} \\
\~{} \\
\~{} \\
\~{} \\
\~{} 
\end{minipage}}\end{tscreen}

La orden ``{\tt \~{}}'' cambia la letra bajo el cursor de may�sculas a 
min�sculas y viceversa. Por ejemplo, si coloca el cursor en la ``o'' de 
``Cow'' arriba y presiona repetidamente \key{\~{}}, terminar� con:
\begin{tscreen}\fbox{\begin{minipage}{0.8\textwidth}
COW IS THE TIME FOR ALL WOMEN TO COME TO THE AID OF THE HUNGRY\underline{.}\\
\~{} \\
\~{} \\
\~{} \\
\~{} \\
\~{} 
\end{minipage}}\end{tscreen}
\index{vi@{\tt vi}!cambiar texto|)}


%\subsection{Commands for moving the cursor.}
\subsection{�rdenes para mover el cursor}
\index{vi@{\tt vi}!mover el cursor}
Ya conoce c�mo usar las teclas de las flechas para moverse por el 
documento. Adem�s puede usar las �rdenes {\tt h}, {\tt j}, {\tt k}, 
y {\tt l} para mover el cursor a la izquierda, abajo, arriba y 
derecha, respectivamente. Esto es �til cuando (por alguna raz�n) sus 
teclas de flechas no est�n funcionando correctamente.

La orden {\tt w} mueve el cursor al principio de la siguiente palabra. 
La orden {\tt b} lo mueve al principio de la palabra anterior. 

La orden {\tt 0} (la tecla cero) mueve el cursor al principio de la l�nea 
actual, y la orden {\tt \$} lo mueve hasta al final de la l�nea. 

Cuando se est� editando un fichero largo, querr� moverse hacia 
delante o hacia detr�s por el fichero una pantalla de una vez. 
Presionando \key{Ctrl-F} se mueve el cursor una pantalla hacia 
delante, y \key{Ctrl-B} lo mueve una pantalla hacia atr�s. 

Para mover el cursor al final del fichero, presione {\tt G}. Puede 
moverse tambi�n a una l�nea arbitraria; por ejemplo, tecleando la 
orden {\tt 10G} el cursor se mover� a la l�nea 10 del fichero. 
Para moverse al principio del fichero, use {\tt 1G}. 

Puede emparejar �rdenes de movimientos con otras �rdenes, tales como 
aquellas para borrar texto. Por ejemplo, la orden {\tt d\$} borra 
todo desde el cursor hasta el final de la l�nea; {\tt dG} borra todo 
desde el cursor hasta el final del fichero, y as� todas. 

%\subsection{Saving files and quitting {\tt vi}.}
\subsection{Guardandando ficheros y saliendo de {\tt vi}}
\index{vi@{\tt vi}!guardar cambios}
\index{vi@{\tt vi}!escribir cambios}
\index{vi@{\tt vi}!salir}
Para abandonar {\tt vi} sin hacer cambios al fichero, usa la orden 
{\tt :q!}. Cuando pulsa ``{\tt :}'', el cursor se mueve a la �ltima 
l�nea de la pantalla y estar� en el modo �ltima l�nea. 
\begin{tscreen}\fbox{\begin{minipage}{0.8\textwidth}
COW IS THE TIME FOR ALL WOMEN TO COME TO THE AID OF THE HUNGRY. \\
\~{} \\
\~{} \\
\~{} \\
\~{} \\
\~{} \\
:\underline{\ }
\end{minipage}}\end{tscreen}
En el modo �ltima l�nea, hay disponibles ciertas �rdenes extendidas 
Una de ellos es {\tt q!}, que sale de {\tt vi} sin guardar. La orden 
{\tt :wq} guarda el fichero y entonces sale de {\tt vi}. La orden 
{\tt ZZ} (desde el modo orden, sin el ``{\tt :}'') es equivalente a 
{\tt :wq}. Si el fichero no ha sido cambiado desde la �ltima vez que 
se guard�, simplemente sale, guardando la hora de modificaci�n del 
�ltimo cambio. Recuerde que debe presionar \ret despu�s de 
una orden introducida en el modo �ltima l�nea. 

Para guardar el fichero sin salir de vi, use {\tt :w}. 

%\subsection{Editing another file.}
\subsection{Edici�n de otro fichero}
\index{vi@{\tt vi}!cambiar a otros ficheros}
Para editar otro fichero, use la orden {\tt :e}. Por ejemplo, 
para parar de editar {\tt test} y editar el fichero {\tt foo} en su 
lugar, use la orden 
\begin{tscreen}\fbox{\begin{minipage}{0.8\textwidth}
COW IS THE TIME FOR ALL WOMEN TO COME TO THE AID OF THE HUNGRY. \\
\~{} \\
\~{} \\
\~{} \\
\~{} \\
\~{} \\
:e foo\underline{\ }
\end{minipage}}\end{tscreen}
Si usa {\tt :e} sin guardar el fichero primero, obtendr�s el mensaje 
de error 
\begin{tscreen}\fbox{\begin{minipage}{0.8\textwidth}
No write since last change (":edit!" overrides)
\end{minipage}}\end{tscreen}
que significa que {\tt vi} no quiere editar otro fichero hasta que 
guarde el primero. En este punto, puede usar {\tt :w} para guardar 
el fichero original, y entonces usar {\tt :e}, o puede usar la 
orden 
\begin{tscreen}\fbox{\begin{minipage}{0.8\textwidth}
COW IS THE TIME FOR ALL WOMEN TO COME TO THE AID OF THE HUNGRY. \\
\~{} \\
\~{} \\
\~{} \\
\~{} \\
\~{} \\
:e! foo\underline{\ }
\end{minipage}}\end{tscreen}
La orden ``{\tt !}'' le dice a {\tt vi} lo que realmente quiere hacer usted --- 
editar el nuevo fichero sin salvar los cambios del primero. 

%\subsection{Including other files.}
\subsection{Inclusi�n de otros ficheros}
\index{vi@{\tt vi}!incluir ficheros}
Si usa la orden {\tt :r}, puede incluir los contenidos de otro fichero 
en el fichero actual. Por ejemplo, la orden
\begin{tscreen}
:r foo.txt
\end{tscreen}
inserta los contenidos del fichero {\tt foo.txt} en el texto en la posici�n 
del cursor.

%\subsection{Running shell commands.}
\subsection{Ejecuci�n de �rdenes del int�rprete}
\index{vi@{\tt vi}!�rdenes shell desde  }
Tambi�n puede ejecutar �rdenes del int�rprete de ordenes dentro de {\tt vi}.
La orden {\tt :r!} funciona como {\tt :r}, pero en lugar de leer un fichero, 
inserta la salida de una determinada orden en la posici�n actual del cursor. Por 
ejemplo, si usa la orden 
\begin{tscreen}
:r! ls -F
\end{tscreen}
terminar� con
\begin{tscreen}\fbox{\begin{minipage}{0.8\textwidth}
COW IS THE TIME FOR ALL WOMEN TO COME TO THE AID OF THE HUNGRY. \\
letters/ \\
misc/ \\
papers\underline{/} \\
\~{} \\
\~{} 
\end{minipage}}\end{tscreen}
Tambi�n puede salir a un int�rprete de ordenes desde {\tt vi}, en otras 
palabras, ejecutar una orden desde dentro de {\tt vi}, y volver al editor cuando 
haya terminado. 
Por ejemplo, si usa la orden
\begin{tscreen}
:! ls -F
\end{tscreen}
la orden {\tt ls -F} ser� ejecutada y los resultados se mostrar�n en la 
pantalla, pero no insertados en el fichero que est� editando.
Si usa la orden 
\begin{tscreen}
:shell
\end{tscreen}
{\tt vi} empieza una instancia (copia) de la shell, permitiendole temporalmente 
poner vi ``en suspenso'' mientras ejecuta otras �rdenes. S�lo tiene que 
salir de la shell (usando la orden {\tt exit} ) para volver a {\tt vi}.

%\subsection{Getting {\tt vi} help.}
\subsection{Obtenci�n de  ayuda}
El editor {\tt vi} no proporciona demasiada ayuda interactiva (la mayor�a de 
los programas {\linux} no lo hacen), pero siempre puede leer las p�ginas 
del manual de {\tt vi}. 
Como {\tt vi} es un ``front-end'' visual del editor {\tt ex} que maneja muchas 
de las �rdenes del modo �ltima l�nea de {\tt vi}, adem�s de leer la 
p�gina del manual de {\tt vi}, mire tambi�n la de {\tt ex}.
\index{vi@{\tt vi}|)}

% Linux Installation and Getting Started    -*- TeX -*-
% environment.tex
% Copyright (c) 1992, 1993 by Matt Welsh, Larry Greenfield and Karl Fogel
%
% This file is freely redistributable, but you must preserve this copyright 
% notice on all copies, and it must be distributed only as part of "Linux 
% Installation and Getting Started". This file's use is covered by
% the copyright for the entire document, in the file "copyright.tex".
%
% Traducci�n al espa�ol por Eduardo Lluna Gil <elluna@aii.upv.es>
% $Log: environment.tex,v $
% Revision 1.8  2003/07/19 06:01:30  joseluis.ranz
% Correcciones varias.
%
% Revision 1.7  2002/10/12 19:53:24  montuno
% quitando defectos y comandos
%
% Revision 1.6  2002/09/09 16:50:47  pakojavi2000
% Correcci�n de fallos peque�os
%
% Revision 1.5  2002/07/20 17:41:17  pakojavi2000
% beta2
%
% Revision 1.4  2002/07/20 15:40:18  pakojavi2000
% Beta2
%
% Revision 1.3  2002/07/09 02:05:04  pakojavi2000
% Correcciones menores
%
% Revision 1.2  2000/11/13 08:45:31  amolina
% *** empty log message ***
%
% Revision 0.5.0.1  1996/02/10 23:45:14  rcamus
% Primera beta publica
% Revisi�n 2 2002/07/02 por Francisco Javier Fern�ndez <serrador@arrakis.es>
% Release Candidate 1


\section{Personalizando su entorno}

\index{entorno!personalizaci�n|(}

El int�rprete de �rdenes proporciona muchos mecanismos para 
personalizar su entorno de trabajo. Como hemos mencionado antes, el 
int�rprete de �rdenes es m�s que un mero int�rprete---es 
tambi�n un poderoso lenguaje de programaci�n. Aunque escribir guiones 
del int�rprete de �rdenes es una tarea extensa, nos gustar�a 
introducirle algunas formas en las que puede simplificar su trabajo en 
un sistema UNIX mediante el uso de caracter�sticas avanzadas del 
int�rprete.

Como mencionamos antes, diferentes int�rpretes usan diferentes sintaxis para
la ejecuci�n de guiones. Por ejemplo, Tcsh usa una notaci�n al estilo C,
mientras que Bourne usa otro tipo de sintaxis. En esta secci�n no nos
fijaremos en las diferencias entre los dos y supondremos que los guiones se
escriben con la sintaxis del int�rprete de �rdenes Bourne.

\subsection{Guiones del int�rprete de �rdenes}
\label{sec-shell-script}
\index{guiones de int�rprete de �rdenes!definici�n}
\index{�rdenes!agrupando con guiones}

Supongamos que usa una serie de �rdenes a menudo, y le gustar�a 
acortar el tiempo requerido para teclear agrup�ndolos en una �nica 
``orden''. Por ejemplo, las �rdenes
\begin{tscreen}
/home/larry\# {\em cat cap�tulo1 cap�tulo2 capitulo3 $>$ libro} \\
/home/larry\# {\em wc -l libro} \\
/home/larry\# {\em lp libro}
\end{tscreen}
concatenar�n los ficheros {\tt cap�tulo1}, {\tt cap�tulo2} y {\tt cap�tulo3} y
guardar� el resultado en el fichero {\tt libro}. Entonces, se mostrar� el
recuento del n�mero de l�neas del fichero {\tt libro} y finalmente se
imprimir� con la intrucci�n {\tt lp}.

En lugar de teclear todas esas �rdenes, podr�a agruparlas en un 
{\bf gui�n del int�rprete de �rdenes}. Describimos los guiones 
brevemente en la Secci�n~\ref{sec-shell-script}. El gui�n usado para 
ejecutar todas las �rdenes ser�a
\begin{tscreen}\begin{verbatim}
#!/bin/sh
# Un gui�n para crear e imprimir el libro

cat cap�tulo1 cap�tulo2 cap�tulo3 > libro
wc -l libro
lp libro
\end{verbatim}\end{tscreen}

Si el gui�n se salva en el fichero {\tt hacerlibro}, podr�a simplemente 
usar la orden
\begin{tscreen}
/home/larry\# {\em hacerlibro}
\end{tscreen}
para ejecutar todas las �rdenes del gui�n. Los guiones son simples 
ficheros de texto; puede crearlos con un editor como {\tt emacs} o {\tt vi}
% \ifodd\igsascii {} \else
\footnote{{\tt vi} se describe en la Secci�n~\ref{sec-vi}.}.
% \fi

Veamos este gui�n. La primera l�nea ``{\tt \#!/bin/sh}'', 
identifica el fichero como un gui�n y le dice al int�rprete de 
�rdenes c�mo ejecutarlo. Instruye al int�rprete a pasarle el gui�n 
a {\tt /bin/sh} para la ejecuci�n, donde {\tt /bin/sh} es el programa 
del int�rprete. �Por qu� es esto importante? En la mayor�a de los 
sistemas UNIX {\tt /bin/sh} es un int�rprete de �rdenes Bourne, como 
Bash. Forzando al gui�n a ejecutarse usando {\tt /bin/sh} nos estamos 
asegurando de que ser� interpretado seg�n la sintaxis de Bourne. Esto 
har� que el gui�n se ejecute usando la sintaxis Bourne aunque est� 
usando Tcsh como int�rprete de �rdenes.

\index{guiones del int�rprete de �rdenes!comentarios}
La segunda l�nea es un {\em comentario}. Estos comienzan con el 
car�cter ``{\tt \#}'' y contin�an hasta el final de la l�nea. Los 
comentarios son ignorados por el int�rprete de �rdenes---son 
habitualmente usados para identificar el gui�n con el programador.

El resto de las l�neas del gui�n son simplemente �rdenes como las 
que podr�a teclear directamente. En efecto, el int�rprete de 
�rdenes lee cada l�nea del gui�n y ejecuta la l�nea como si 
hubiese sido tecleada en la l�nea de �rdenes.

\index{guiones del int�rprete de �rdenes!permisos para}
\index{permisos!para los guiones del int�rprete de �rdenes}
Los permisos son importantes para los guiones. Si crea un gui�n, debe
asegurarse de que tiene permisos de ejecuci�n para poder
ejecutarlo\footnote{Cuando crea ficheros de texto, los permisos por omisi�n
usualmente no incluyen los de ejecuci�n.}.
La orden
\begin{tscreen}
/home/larry\# {\em chmod u+x hacerlibro}
\end{tscreen}
puede usarse para dar permisos de ejecuci�n al gui�n {\tt hacerlibro}.

\subsection{Variables del int�rprete de �rdenes y el entorno}
\index{int�rprete de �rdenes!variables!definici�n}
\index{variables!int�rprete de �rdenes}
\index{variables!en guiones}
\index{guiones del int�rprete de �rdenes!variables en}
El int�rprete de �rdenes le permite definir {\bf variables} como la 
mayor�a de los lenguajes de programaci�n. Una variable es simplemente 
un trozo de datos al que se le da un nombre.

\blackdiamond N�tese que Tcsh, as� como otros int�rpretes del estilo C, usan
un mecanismo diferente para inicializar variables del descrito aqu�. Esta
discusi�n supondr� el uso del int�rprete Bourne, como es Bash (el cual
probablemente est� usando). Vea la p�gina de manual de Tcsh para m�s
detalles.

Cuando asigna un valor a una variable (usando el operador ``{\tt =}''),
puede acceder a la variable anteponiendo a su nombre ``{\tt \$}'', como se 
ve a continuaci�n.
\begin{tscreen}
/home/larry\# {\em foo=``hola all�''} 
\end{tscreen}
A la variable {\tt foo} se le da el valor ``{\tt hola all�}''. Podemos
ahora hacer referencia a ese valor a trav�s del nombre de la variable con el
prefijo ``{\tt \$}''. La orden
\begin{tscreen}
/home/larry\# {\em echo \$foo} \\
hola all� \\
/home/larry\# 
\end{tscreen}
produce el mismo resultado que
\begin{tscreen}
/home/larry\# {\em echo ``hola all�''} \\
hola all� \\
/home/larry\# 
\end{tscreen}

Estas variables son internas al int�rprete. Esto significa que s�lo �ste
podr� acceder a las variables. Esto puede ser �til en los guiones; si
necesita mantener, por ejemplo, el nombre de un fichero, puede almacenarlo
en una variable. Usando la orden {\tt set} mostrar� una lista de todas las
variables definidas en el int�rprete de �rdenes.

\index{variables del int�rprete de �rdenes!exportando al entorno}
\index{exportar@{\tt exportar}}
De cualquier modo, el int�rprete de �rdenes permite {\bf exportar} 
variables al {\bf entorno}. El entorno es el conjunto de variables a las 
cu�les tienen acceso todas las �rdenes que ejecute. Una vez que se 
define una variable en el int�rprete, exportarla hace que se convierta 
tambi�n en parte del entorno. La orden {\tt export} se usa para 
exportar variables al entorno.

\index{setenv@{\tt setenv}}
\blackdiamond De nuevo, hemos de diferenciar entre Bash y Tcsh. Si est�
usando Tcsh, deber� usar una sintaxis diferente para las variables de
entorno (se usa la orden {\tt setenv}). Dir�jase a la p�gina de 
manual de Tcsh para m�s informaci�n.

\index{variables!entorno}
El entorno es muy importante en un sistema UNIX. Le permite configurar
ciertas �rdenes simplemente inicializando variables con las �rdenes ya
conocidas.

Veamos un ejemplo r�pido. La variable de entorno {\tt PAGER} se usa por la
orden {\tt man}. Especifica la orden que se usar� para mostrar las p�ginas
del manual una a una. Si inicializa {\tt PAGER} con el nombre del
programa, se usar� �ste para mostrar las p�ginas de manual en lugar de 
{\tt more} (el cu�l es usado por omisi�n).

Inicialice {\tt PAGER} a ``{\tt cat}''. Esto har� que la salida de 
{\tt man} sea mostrada de una vez, sin pausas entre p�ginas.
\begin{tscreen}
/home/larry\# {\em PAGER=``cat''}
\end{tscreen}

Ahora exportamos {\tt PAGER} al entorno.
\begin{tscreen}
/home/larry\# {\em export PAGER}
\end{tscreen}

Pruebe la orden {\tt man ls}. La p�gina deber�a volar por su 
pantalla sin detenerse entre p�ginas.

Ahora, si inicializa {\tt PAGER} a ``{\tt more}'', se usar� la orden 
{\tt more} para mostrar las p�ginas del manual.
\begin{tscreen}
/home/larry\# {\em PAGER=``more''}
\end{tscreen}

N�tese que no hemos de usar la orden {\tt export} despu�s del cambio 
de la variable {\tt PAGER}. Solo hemos de exportar las variables una vez; 
cualquier cambio efectuado con posterioridad ser� autom�ticamente 
propagado al entorno.

Las p�ginas de manual para una orden en particular, le informar�n acerca del
uso de alguna variable de entorno por parte de esa orden; por ejemplo, la
p�gina de manual de {\tt man} explica que {\tt PAGER} es usado para
especificar la orden de paginado.

Algunas �rdenes comparten variables de entorno; por ejemplo, muchas �rdenes
usan la variable {\tt EDITOR} para especificar el editor por omisi�n que se
usar� si es necesario.

El entorno es tambi�n usado para guardar informaci�n importante acerca de
la sesi�n en curso. Un ejemplo es la variable de entorno {\tt HOME}, que
contiene el nombre del directorio de origen del usuario.
\begin{tscreen}
/home/larry/papers\# {\em echo \$HOME} \\
/home/larry
\end{tscreen}

Otra variable de entorno interesante es {\tt PS1}, la cu�l define el
indicador (``prompt'') principal que usar� el int�rprete. Por ejemplo,
\begin{tscreen}
/home/larry\# {\em PS1=``Su instrucci�n, por favor: ''} \\
Su instrucci�n, por favor: 
\end{tscreen}

Para volver a inicializar el ``prompt'' a su valor habitual (el cual
contiene el directorio actual seguido por el s�mbolo ``{\tt \#}''),
\begin{tscreen}
Su instrucci�n , por favor: {\em PS1=``{\verb'-w#'} -''} \\
/home/larry\# 
\end{tscreen}
La p�gina de manual de {\tt bash} describe la sintaxis usada para
inicializar el indicador.

\subsubsection{La variable de entorno {\tt PATH}}
\index{entorno!variables!PATH@{\tt PATH}}

Cuando usa la orden {\tt ls} ?`c�mo encuentra el int�rprete el programa
ejecutable {\tt ls}?. De hecho, {\tt ls} se encuentra en {\tt /bin/ls} en la
mayor�a de los sistemas. El int�rprete usa la variable de entorno 
{\tt PATH} (``camino'') para localizar los ficheros ejecutables u �rdenes que tecleamos.

Por ejemplo, su variable {\tt PATH} puede inicializarse a:
\begin{tscreen}
/bin:/usr/bin:/usr/local/bin:.
\end{tscreen}

Esto es una lista de directorios en los que el int�rprete debe buscar. Cada
directorio est� separado por un ``{\tt :}''. Cuando usa la orden {\tt ls},
el int�rprete primero busca {\tt /bin/ls}, luego {\tt /usr/bin/ls} y 
as� hasta que lo localice o acabe la lista.

N�tese que {\tt PATH} no interviene en la localizaci�n de ficheros
regulares. Por ejemplo, si usa la orden
\begin{tscreen}
/home/larry\# {\em cp foo bar}
\end{tscreen}
El int�rprete no usar� {\tt PATH} para localizar los ficheros {\tt foo} y
{\tt bar} --esos nombres se suponen completos--. S�lo se usar� {\tt PATH} para
localizar el programa ejecutable {\tt cp}.

�sto le permitir� ahorrar mucho tiempo; significa que no deber� recordar
d�nde se guardan las instrucciones. En muchos sistemas los ficheros 
ejecutables se dispersan por muchos sitios, como {\tt /usr/bin}, {\tt 
/bin} o {\tt /usr/local/bin}. En lugar de dar el nombre completo con el 
camino (como {\tt /usr/bin/cp}), solo hemos de inicializar {\tt PATH} con 
la lista de los directorios donde queremos que se busquen autom�ticamente.

N�tese que {\tt PATH} contiene ``{\tt .}'', el cual es el directorio 
actual de trabajo. Esto le permite crear guiones o programas y 
ejecutarlos desde su directorio de trabajo actual sin tener que 
especificarlo directamente (como en {\tt ./makebool}). Si un directorio 
no est� en su {\tt PATH}, entonces el int�rprete no buscar� en �l 
�rdenes para ejecutar --�sto incluye al directorio de trabajo--.

\subsection{Guiones de inicializaci�n del int�rprete}\label{sec-init-scripts}
\index{guiones del int�rprete de �rdenes!inicializacion}
\index{guiones de inicializaci�n!para int�rpretes de ordenes}

Aparte de los guiones que puede crear, hay un n�mero de �stos que usa el
int�rprete de �rdenes para ciertos prop�sitos. Los m�s importantes 
son sus {\bf guiones de inicializaci�n}, guiones autom�ticamente 
ejecutados por el int�rprete al abrir una sesi�n.

Los guiones de inicializaci�n son eso, simples guiones como los descritos
arriba. De cualquier modo, son muy �tiles para la inicializaci�n de su
entorno al ejecutarse autom�ticamente. Por ejemplo, si siempre usa la orden
{\tt mail} para comprobar si tiene correo al iniciar una sesi�n, incluya en
su gui�n de inicializaci�n dicha orden y ser� ejecutada autom�ticamente.

\index{Int�rprete de presentaci�n!definici�n}
Tanto Bash como Tcsh distinguen entre un {\bf int�rprete de presentaci�n} y
otras invocaciones del int�rprete. Un int�rprete de presentaci�n es 
el que se ejecuta en el momento de la presentaci�n al sistema (login). 
Es el �nico que usar�. De cualquier modo, si ejecuta una opci�n de 
salir a un int�rprete desde alg�n programa, como {\tt vi}, inicializa 
otra instancia del int�rprete de �rdenes, el cual no es su 
int�rprete de presentaci�n. Adem�s, en cualquier momento que 
ejecuta un gui�n, autom�ticamente est� arrancando otro int�rprete 
que va a ser el encargado de ejecutar el gui�n.

\index{int�rpretes de orddenes!ficheros de inicializaci�n}
\index{ficheros de inicializaci�n!para int�rpretes de ordenes}
\index{/etc/profile@{\tt /etc/profile}}
\index{.bash\_profile@{\tt .bash\_profile}}
\index{.bashrc@{\tt .bashrc}}
\index{.profile@{\tt .profile}}
Los ficheros de inicializaci�n usados por Bash son: {\tt /etc/profile}
(configurado por el administrador del sistema, y ejecutado por todos los
usuarios de Bash en el momento de la presentaci�n al sistema), 
{\tt \$HOME/.bash\_profile} (ejecutado por una sesi�n de presentaci�n Bash)
y {\tt \$HOME/.bashrc} (ejecutadas por todas las sesiones Bash que no son de
presentaci�n). Si {\tt .bash\_profile} no est� presente, se usa en su lugar 
{\tt .profile}

\index{/etc/csh.login@{\tt csh.login}}
\index{.tcshrc@{\tt.tcshrc}}
Tcsh usa los siguientes guiones de inicializaci�n: {\tt /etc/csh.login}
(ejecutado por todos los usuarios de Tcsh en el momento de la presentaci�n
al sistema), {\tt \$HOME/.tcshrc} (ejecutado en la presentaci�n al sistema
por todas las instancias nuevas de Tcsh) y {\tt \$HOME/.login} (ejecutado en
la presentaci�n al sistema, seguido {\tt .tcshrc}).
Si {\tt .tcshrc} no est� presente, {\tt .cshrc} se usa en su lugar.

Para entender completamente la funci�n de estos ficheros, necesitar�
aprender m�s acerca del int�rprete de �rdenes. La programaci�n de 
guiones es una materia complicada, m�s all� del alcance de este 
libro. Lea las p�ginas de manual de {\tt bash} y/o {\tt tcsh} para 
aprender m�s sobre la configuraci�n de su entorno.

\index{entorno!personalizaci�n|)}

% {\linux} Installation and Getting Started    -*- TeX -*-
% other.tex
% Copyright (c) 1992, 1993 by Matt Welsh, Larry Greenfield and Karl Fogel
%
% This file is freely redistributable, but you must preserve this copyright 
% notice on all copies, and it must be distributed only as part of "{\linux} 
% Installation and Getting Started". This file's use is covered by
% the copyright for the entire document, in the file "copyright.tex".
%
% Copyright (c) 1998 by Specialized Systems Consultants Inc. 
% <ligs@ssc.com>

%\section{So you want to strike out on your own?}
\section{�Quiere seguir por su cuenta?}
\markboth{Tutorial de {\linux}}{Por tu cuenta}

Este cap�tulo deber�a proporcionarle informaci�n suficiente para un
uso b�sico de {\linux}.

% Keep in mind that many of the interesting and important aspects
% of {\linux} are not covered here---these are the very basics.
% With this foundation, before long
% you'll be up and running complicated applications and fulfilling the
% potential of your system. If things don't seem exciting at first,
% don't despair---there is much to be learned.
Las p�ginas del manual son herramientas indispensables para aprender
{\linux}. Pueden parecer confusas al principio, pero hay abundante informaci�n si 
indaga bajo la superficie.

Tambi�n le sugerimos que lea un libro sobre {\linux} en general. {\linux} tiene
otras caracter�sticas adem�s de las que aparecen a primera vista. Desafortunadamente, muchas
est�n m�s all� del alcance de este libro. Otros libros sobre {\linux} recomendados est�n
listados en el Ap�ndice~\ref{app-info}



