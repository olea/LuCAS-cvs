% \linux Installation and Getting Started    -*- TeX -*-
% tut-intro.tex
% Copyright (c) 1992, 1993 by Matt Welsh, Larry Greenfield and Karl Fogel
%
% This file is freely redistributable, but you must preserve this copyright 
% notice on all copies, and it must be distributed only as part of "\linux 
% Installation and Getting Started". This file's use is covered by
% the copyright for the entire document, in the file "copyright.tex".
%
% Copyright (c) 1998 by Specialized Systems Consultants Inc. 
% <ligs@ssc.com>
%Revisi�n 1 por Francisco Javier Fernandes <serrador@arrakis.es>
%gold
\section{Introducci�n}

Si es nuevo en UNIX y {\linux}, puede que est� un poco intimidado por el tama�o y
la aparente complejidad del sistema que tiene delante suya.
Este cap�tulo no profundiza en muchos detalles ni cubre
temas avanzados. Por contra, queremos que aterrice corriendo.

Aqu� se asume que posee pocos conocimientos, salvo quiz�s
algo de familiaridad con ordenadores personales, y MS-DOS. Sin embargo, incluso 
si no es un usuario de MS-DOS, deber�a ser capaz de entender todo esto. A primera 
vista, {\linux} se parece mucho a MS-DOS---despu�s de todo, hay partes de MS-DOS 
basadas en el sistema operativo CP/M, que a su vez se basaba en UNIX. Sin embargo,
s�lo las caracter�sticas m�s superficiales de {\linux} se parecen a MS-DOS. Incluso 
si es completamente nuevo en el mundo del PC, este tutorial deber�a serle de ayuda.

Y, antes de que comencemos: {\em No tenga miedo a experimentar \/} El sistema
no le morder�. No se puede destrozar todo s�lo por trabajar con el sistema.
{\linux} tiene incorporadas caracter�sticas de seguridad para evitar que usuarios 
''normales''da�en ficheros que sean imprescindibles para el sistema. Aun as�, 
lo peor que puede ocurrir es que borre todos o algunos de sus ficheros y tenga que
reinstalar el sistema. As� que, en este punto, no tiene nada
que perder.


