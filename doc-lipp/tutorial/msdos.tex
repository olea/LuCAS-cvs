% {\linux} Installation and Getting Started    -*- TeX -*-
% msdos.tex
% Copyright (c) 1992, 1993 by Matt Welsh <mdw@sunsite.unc.edu>
%
% This file is freely redistributable, but you must preserve this copyright 
% notice on all copies, and it must be distributed only as part of "{\linux} 
% Installation and Getting Started". This file's use is covered by the 
% copyright for the entire document, in the file "copyright.tex".
%
% Copyright (c) 1998 by Specialized Systems Consultants Inc. 
% <ligs@ssc.com>

%\section{Accessing MS-DOS files.}
\section{Acceder a los ficheros MS-DOS\TM}
\markboth{Caracter�sticas Avanzadas}{Acceso a ficheros MS-DOS\tm}
\label{sec-msdos-mount}
\index{MS-DOS!acceder ficheros desde}
\index{ficheros!MS-DOS}
Si, por cualquier retorcida y extrafalaria raz�n, quiere acceder a ficheros
de MS-DOS, lo  podr� hacer f�cilmente desde {\linux}.

\index{MS-DOS!montando una partici�n bajo {\linux}}
\index{mount@{\tt mount}!para montar una partici�n MS-DOS}
La manera normal de acceder a los ficheros de MS-DOS es montar una partici�n MS-DOS o
un disco flexible bajo {\linux}, lo cual permite acceder a los ficheros directamente a trav�s
del sistema de ficheros. Por ejemplo, si tiene un disco flexible MS-DOS en
{\tt /dev/fd0}, la orden

\begin{tscreen}
\# mount -t msdos /dev/fd0 /mnt
\end{tscreen}

lo montar� en {\tt /mnt}. Consulte la Secci�n~\ref{sec-floppy} para m�s
informaci�n sobre c�mo montar discos flexibles.

Tambi�n puede montar una partici�n MS-DOS de su disco duro para que sea
accesible desde {\linux}. Si tiene una partici�n MS-DOS en {\tt /dev/hda1}, la orden

\begin{tscreen}
\# mount -t msdos /dev/hda1 /mnt
\end{tscreen}

la monta. Aseg�rese de desmontar ({\tt umount}) la partici�n cuando haya terminado
de usarla. Tambi�n se puede hacer que una partici�n MS-DOS se monte autom�ticamente
en el momento del arranque si incluye la entrada en {\tt /etc/fstab}. Consulte la
Secci�n~\ref{sec-manage-fs} para m�s detalles. La siguiente l�nea en {\tt
/etc/fstab} monta una partici�n MS-DOS en {\tt /dev/hda1} en el
directorio {\tt /dos}.

\begin{tscreen}
/dev/hda1\ \ \ \ \ /dos\ \ \ \ \ msdos \ \ \ \ \ defaults
\end{tscreen}

Tambi�n puede montar los sistemas de ficheros VFAT usados por Windows 95 y 98:

\begin{tscreen}
\# mount -t vfat /dev/hda1 /mnt
\end{tscreen}

Esto permite acceder a los nombres largos de ficheros de Windows 95\tm. Esto s�lo
se aplica a particiones que realmente tengan almacenados los nombres en formato
largo. No se puede montar un sistema de ficheros FAT16 normal y usarlo para
obtener nombres de ficheros largos.

\index{MS-DOS!uso de Mtools para acceder a ficheros}
El software Mtools tambi�n puede ser usado para acceder a ficheros MS-DOS\tm. Las
�rdenes {\tt mcd}, {\tt mdir}, y {\tt mcopy} se comportan todas como sus
equivalentes MS-DOS\tm. Si instala las Mtools, deber�a tener p�ginas del manual
disponibles para estas �rdenes.

\index{MS-DOS!ejecuci�n de programas bajo {\linux}}
\index{MS-DOS!emulador}
\index{Microsoft Windows!emulador} 
Acceder a ficheros MS-DOS es una cosa; ejecutar programas MS-DOS es
otra. Hay un emulador de MS-DOS\tm en desarrollo para {\linux}; es
f�cil de conseguir, y est� inclu�do en la mayor�a de las distribuciones.
Tambi�n se puede conseguir en muchos sitios, incluyendo los sitios
FTP para {\linux} listados en el Ap�ndice~\ref{app-ftp}. El emulador de MS-DOS est�
considerado como lo suficientemente potente para hacer funcionar un buen n�mero de
aplicaciones, incluyendo Wordperfect\tm, desde {\linux}. Sin embargo, {\linux} y MS-DOS son
sistemas operativos marcadamente diferentes. La potencia de cualquier emulador de MS-DOS
bajo UNIX est� limitada. Adem�s, est� en desarrollo un emulador de Microsoft Windows
que corra bajo X Window.










