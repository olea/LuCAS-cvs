% Linux Installation and Getting Started    -*- TeX -*-
% problems.tex
% Copyright (c) 1992-1994 by Matt Welsh <mdw@sunsite.unc.edu>
%
% This file is freely redistributable, but you must preserve this copyright 
% notice on all copies, and it must be distributed only as part of "Linux 
% Installation and Getting Started". This file's use is covered by the 
% copyright for the entire document, in the file "copyright.tex".
%
% Copyright (c) 1998 by Specialized Systems Consultants Inc. 
% <ligs@ssc.com>
%Revisi�n 1 15/07/2002 por Francisco Javier Fern�ndez <serrador@arrakis.es>
%Para publicar

\section{Cuando se tienen problemas}
\markboth{Obtenci�n e instalaci�n de {\linux}}{Cuando se tienen problemas}
\namedlabel{sec-install-probs}{Problemas de instalaci�n}
\index{instalaci�n!problemas|(}

Casi todos hemos tenido alguna clase de problema o cuelgue cuando se intenta
instalar {\linux} la primera vez. La mayor�a de las veces, el problema es causado
por un simple malentendido. Sin embargo, a veces el problema puede ser
algo m�s serio, como un descuido de uno de los desarrolladores o un bug.

%Almost everyone runs into some kind of snag or hangup when attempting
%to install Linux the first time. Most of the time, the problem is caused
%by a simple misunderstanding. Sometimes, however, it can be something
%more serious, like an oversight by one of the developers, or a bug.

%Esta seccion describir� algunos de los problemas de instalaci�n m�s comunes
%y c�mo resolverlos (N del T Comentado en el original)
Si su instalaci�n parece ser exitosa pero se recibieron mensajes de error 
inesperados, �stos se describen aqu�.

%% This section will describe some of the
%% most common installation problems, and how to solve them. 
%If your installation
%appears to be successful, but you received unexpected error messages,
%these are described here as well.

%Si su instalaci�n aparenta haber sido exitosa pero se recibieron mensajes inesperados
%de error, �stos se describen aqu�.
% Linux Installation and Getting Started    -*- TeX -*-
% chap-sysadm.tex
% Copyright (c) 1993 by Matt Welsh and Lars Wirzenius
%
% This file is freely redistributable, but you must preserve this copyright 
% notice on all copies, and it must be distributed only as part of "Linux 
% Installation and Getting Started". This file's use is covered by
% the copyright for the entire document, in the file "copyright.tex".
%
% Copyright (c) 1998 by Specialized Systems Consultants Inc. 
% <ligs@ssc.com>
%
% Este fichero es de distribuci�n libre, pero debe mantenerse esta 
% informaci�n de Copyright en todas las copias, y debe distribuirse solo como
% parte de "Instalaci�n y Primeros Pasos en Linux". El uso de este fichero esta
% cubierto por el Copyright del documento completo, en el fichero "copyright.tex"
% Copyright (c) 1995 por Gerardo Izquierdo para la versi�n al Castellano

%
% Versi�n para lipp 2.0 por Alberto Molina. Comentarios a:
%            alberto@nucle.us.es 
%

\section{Iniciando el Sistema}
\markboth{Administraci�n del Sistema}{Iniciando el Sistema}
\label{sec-bootfloppy} %% legacy of deleted sub-hed -- rk

\index{arrancando|(}
\index{administraci�n del sistema!arrancando Linux|(}
Hay varias maneras de arrancar el sistema, bien sea desde disquete o 
bien desde el disco duro.

\subsection{Utilizando un disquete de arranque}

\index{arrancando Linux!con un disquete de arranque}
Mucha gente arranca Linux utilizando un disquete de inicio que contiene 
una copia del n�cleo de Linux. Este n�cleo tiene la partici�n ra�z de Linux 
codificada en �l, para que sepa d�nde buscar en el disco duro el 
sistema de ficheros ra�z. (La orden {\tt rdev} puede ser utilizada 
para poner la partici�n ra�z en la imagen del n�cleo; ver m�s 
adelante.) Por ejemplo, �ste es el tipo de disquete creado por Slackware 
durante la instalaci�n.

\index{/Image@{\tt /Image}}
\index{/etc/Image@{\tt /etc/Image}}
\index{/vmlinux@{\tt /vmlinux}}
\index{disquete de arranque!creando}
\index{n�cleo!nombre de fichero de la imagen del}
Para crear un disquete de arranque propio, hay que localizar en primer lugar la 
imagen del n�cleo en su disco duro. Debe estar en el fichero {\tt /Image} o
{\tt /etc/Image}. Algunas instalaciones utilizan el fichero {\tt /vmlinux}
para el n�cleo.

\index{n�cleo!imagen comprimida del}
\index{/zImage@{\tt /zImage}}
\index{/etc/zImage@{\tt /etc/zImage}}
\index{/vmlinuz@{\tt /vmlinuz}}
En su lugar, puede que haya un n�cleo comprimido. Un n�cleo comprimido se
descomprime a s� mismo en memoria en tiempo de arranque, y utiliza mucho 
menos 
espacio en el disco duro. Si se tiene un n�cleo comprimido, puede encontrarse
en el fichero {\tt /zImage} o {\tt /etc/zImage}. Algunas instalaciones utilizan el fichero
{\tt /vmlinuz} para el n�cleo comprimido.

\index{ra�z, dispositivo!poniendo el nombre de con} 
\index{rdev@poniendo el nombre de con {\tt rdev}}
\index{rdev@{\tt rdev}}
Una vez que se sabe d�nde est� el n�cleo, hay que poner el nombre de la 
partici�n ra�z de un dispositivo ra�z en la imagen del 
n�cleo, utilizando la orden {\tt rdev}. El formato de este orden es
\begin{tscreen}
rdev \cparam{nombre-de-n�cleo} \cparam{dispositivo-ra�z}
\end{tscreen}
donde \cparam{nombre-del-n�cleo} es el nombre de la imagen del n�cleo, y 
\cparam{dispositivo-ra�z} es el nombre de la partici�n ra�z de 
Linux. Por ejemplo, para hacer que el dispositivo ra�z en el n�cleo {\tt /etc/Image}
sea {\tt /dev/hda2}, utilice la orden
\begin{tscreen}
\# {\em rdev /etc/Image /dev/hda2}
\end{tscreen}

Con {\tt rdev} tambi�n se pueden poner otras opciones en el n�cleo, como puede ser el
modo SVGA predeterminado que se utilizar� en tiempo de arranque. Utilizando
{\tt rdev -h} se obtiene un mensaje de ayuda.

Una vez puesto el dispositivo ra�z, tan s�lo hay que copiar la 
imagen del n�cleo al disquete. Siempre que se copia datos a un 
disquete, es una buena idea formatear previamente el disquete, usando
el {\tt FORMAT.COM} en MS-DOS o el programa {\tt fdformat} de Linux. 
Esto establece la informaci�n de pista y sector en el disquete con la 
que puede detectarse como de alta o baja densidad.

El formateo de disquetes y las controladoras de los mismos se discuten
m�s tarde en la p�gina~\pageref{sec-backfloppy}.

Para copiar el n�cleo desde el fichero {\tt /etc/Image} al disquete
en {\tt /dev/fd0}, se puede utilizar la orden:
\begin{tscreen}
\# {\em cp /etc/Image /dev/fd0}
\end{tscreen}

Este disquete debe arrancar ahora Linux.

\subsection{Utilizando LILO}\label{sec-lilo}

\index{LILO|(}
\index{arrancando!con LILO|(}
Otro m�todo de arranque es utilizar LILO, un programa que reside en el 
sector de arranque del disco duro. Este programa se ejecuta cuando el 
sistema se inicia desde el disco duro, y puede arrancar autom�ticamente
Linux desde una imagen de n�cleo almacenada en el propio disco duro.

\index{LILO!como cargador de arranque}
\index{sistemas operativos!arrancando no-Linux}
\index{arrancando sistemas no-Linux}
LILO puede utilizarse tambi�n como una primera etapa de carga de varios
sistemas operativos, permitiendo seleccionar en tiempo de arranque qu� 
sistema operativo (como Linux o MS-DOS) arrancar. Cuando se arranca
utilizando LILO, se inicia el sistema operativo predeterminado, a menos que
pulse \key{shift} durante la secuencia de arranque o se especifique el
fichero {\tt /etc/lilo.conf}.
En cualquiera de estos casos, se presentar� un indicador de
arranque, donde debe teclear el nombre del sistema operativo a arrancar
(como puede ser ``{\tt linux}'' o ``{\tt msdos}''). Si se pulsa la tecla
\key{tab} en el indicador de arranque, se le presentar� una lista de los 
sistemas operativos disponibles.

\index{LILO!instalaci�n}
La forma m�s simple de instalar LILO es editar el fichero de 
configuraci�n, {\tt /etc/lilo.conf},
y ejecutar la instrucci�n
\begin{tscreen}
\# {\em /sbin/lilo}
\end{tscreen}

El fichero de configuraci�n de LILO contiene un p�rrafo para cada 
sistema operativo que se pueda querer arrancar. La mejor forma de mostrarlo
es con un ejemplo de un fichero de configuraci�n LILO. El ejemplo siguiente 
es para un sistema que tiene una partici�n ra�z Linux en {\tt /dev/hda1} y
una partici�n MS-DOS en {\tt /dev/hda2}.

\begin{tscreen}\begin{verbatim}
# Le indicamos a LILO que modifique el registro de arranque de
# /dev/hda (el primer disco duro no-SCSI). Si se quiere arrancar desde
# una unidad distinta de /dev/hda, se debe cambiar la siguiente l�nea
boot = /dev/hda

# Modo de v�deo
vga = normal

# Tiempo de respuesta en milisegundos. Tiempo del que se dispone para
# pulsar ``SHIFT''.
delay = 60

# Nombre del cargador de arranque. No hay raz�n para cambiarlo, a menos
# que se este haciendo una modificaci�n seria del LILO
install = /boot/boot.b

# Esto fuerza a LILO a solicitar el Sistema Operativo con el que se va
# a arrancar. Si se pulsa 'TAB' se presentan las distintas opciones,
# de acuerdo con los 'label=' siguientes.
indicador de �rdenes

# Dejemos a LILO efectuar alguna optimizaci�n.
compact

# Parrafo para la partici�n ra�z de Linux en /dev/hda1.
image = /etc/Image   # Ubicaci�n del n�cleo
   label = linux     # Nombre del SO (para el men� de arranque de LILO)
   root = /dev/hda1  # Ubicaci�n de la partici�n ra�z
   vga = ask         # Indicar al n�cleo que pregunte por modos SVGA
                     #   en tiempo de arranque

# P�rrafo para la partici�n MSDOS en /dev/hda2.
other = /dev/hda2    # Ubicaci�n de la partici�n
   table = /dev/hda  # Ubicaci�n de la tabla de partici�n para /dev/hda2 
   label = msdos     # Nombre del SO (para el men� de arranque)
\end{verbatim}\end{tscreen}

\index{LILO!seleccionando el sistema operativo predeterminado para}
\index{sistemas operativos!arrancando no-Linux}
El primer p�rrafo de sistema operativo en el men� del fichero de 
configuraci�n ser� el sistema operativo predeterminado que arrancar� LILO.
Se puede seleccionar otro sistema operativo en el indicador de arranque de
LILO, tal y como se indic� anteriormente.

El instalador de Microsoft Windows '95 sobreescribe el sector de
arranque. Si va a instalar Windows '95 en su sistema despu�s de
instalar LILO, debe asegurarse de crear un disquete de inicio antes,
ver~\ref{sec-bootfloppy}). Con el disquete de inicio, puede iniciar
Linux y reinstalar LILO tras la instalaci�n Windows '95. Simplemente
escribiendo como ``root'' la orden {\tt /sbin/lilo}. 
Las particiones con Windows '95 se pueden configurar de forma
totalmente equivalente a la vista anteriormente con particiones de MS-DOS.

Las FAQ (Preguntas frecuentemente formuladas) (ver Ap�ndice~\ref{app-info})
dan m�s informaci�n sobre LILO, incluyendo c�mo utilizar LILO con el 
``OS/2's~Boot~Manager''.
\index{arrancando!con LILO|)}
\index{LILO|)}
\index{arrancando|)}
\index{administraci�n del sistema!arrancando Linux|)}


% Linux Installation and Getting Started    -*- TeX -*-
% hardware.tex
% Copyright (c) 1992-1994 by Matt Welsh <mdw@sunsite.unc.edu>
%
% This file is freely redistributable, but you must preserve this copyright 
% notice on all copies, and it must be distributed only as part of "Linux 
% Installation and Getting Started". This file's use is covered by the 
% copyright for the entire document, in the file "copyright.tex".
%
% Copyright (c) 1998 by Specialized Systems Consultants Inc. 
% <ligs@ssc.com>
% Traducido por Francisco Javier Frenandez <serrador@arrakis.es>
% Revision 1 6 de julio por Fco. Javier Fern�ndez <serrador@arrakis.es>
%Revisi�n 2 15 julio 2002
%gold
\subsection{Problemas con el hardware}
\label{sec-install-probs-hardware}

\index{hardware!problemas}
\index{instalaci�n!problemas con el hardware}

%The most common form of problem when attempting to install or use Linux
%is an incompatibility with hardware. Even if all of your hardware is supported
%by Linux, a misconfiguration or hardware conflict can sometimes cause
%strange results---your devices may not be detected at boot time, or
%the system may hang. 
El problema m�s com�n cuando se intenta instalar o usar GNU/Linux es una incompatibilidad con el hardware. 
Incluso si todo su hardware est� soportado por GNU/Linux, una configuraci�n err�nea o un conflicto con otro
dispositivo puede algunas veces ocasionar resultados extra�os---los dispositivos pueden no ser detectados al arrancar,
o el sistema se puede colgar.

%It is important to isolate these hardware problems if you suspect 
%that they may be the source of your trouble. 
Es importante aislar estos problemas con el hardware si se sospecha que �stos pueden ser la fuente de sus problemas.
%Comentado en la versi�n original
% In the following sections 
% we will describe some common hardware problems and how to resolve them.

% Linux Installation and Getting Started    -*- TeX -*-
% conflicts.tex
% Copyright (c) 1992-1994 by Matt Welsh <mdw@sunsite.unc.edu>
%
% This file is freely redistributable, but you must preserve this copyright 
% notice on all copies, and it must be distributed only as part of "Linux 
% Installation and Getting Started". This file's use is covered by the 
% copyright for the entire document, in the file "copyright.tex".
%
% Copyright (c) 1998 by Specialized Systems Consultants Inc. 
% <ligs@ssc.com>

%Tradu por Fco. Javier Fern�ndez <serrador@arrakis.es>
%Revisi�n 1 16/7/2002  por Francisco Javier Fernandez

\subparagraph*{Aislando problemas de hardware}
\namedlabel{sec-install-probs-hardware-conflicts}{Conflictos con el hardware}
\index{hardware!conflictos}

Si experimentas alg�n problema que creas que est� relacionado con el hardware, 
lo primero que deber�as hacer es intentar aislar el problema.  
Esto significa  eliminar  todas las posibles variables y (usualmente) 
desmontar el sistema, pieza a pieza, hasta que el componente es aislado.

%If you experience a problem that you believe to be hardware-related, 
%the first thing that you should to do is attempt to isolate the problem.
%This means eliminating all possible variables and (usually) taking the
%system apart, piece-by-piece, until the offending piece of hardware is
%isolated.

Esto no es tan terrible como suena. B�sicamente, se deber� retirar todo
el hardware prescindible del equipo, y entonces determinar cu�l de los
dispositivos est� causando el problema, posiblemente reconectando cada
dispositivo, uno cada vez. Esto quiere decir que se deber� retirar todo el
hardware excepto la unidad de disquettes y la tarjeta de v�deo y por supuesto
el teclado. Incluso los dispositivos aparentemente inocentes como los ratones
pueden causar insospechados problemas a no ser que se les considere 
no esenciales.

%This is not as frightening as it may sound. Basically, you should
%remove all nonessential hardware from your system, and then determine
%which device is causing the trouble---possibly by reinserting each
%device, one at a time. This means that you should remove all hardware
%other than the floppy and video controllers, and of course the
%keyboard. Even innocent-looking devices such as mouse controllers can
%wreak unknown havoc on your peace of mind unless you consider them
%nonessential.

Por ejemplo, digamos que el sistema se cuelga durante la secuencia de 
detecci�n de la placa Ethernet en el arranque. Quiz� pueda hipotetizar
que hay un conflicto con la tarjeta Ethernet en su computadora. La manera
r�pida y f�cil de encontrarlo es extraer la tarjeta Ethernet e intentar
arrancar otra vez. Si todo va bien, entonces  sabe que o (a) la tarjeta 
Ethernet no tiene soporte en Linux (ver p�gina~\pageref{sec-intro-hardware}),
o (b) hay un conflicto de direcci�n o IRQ con la tarjeta.

%For example, let's say that the system hangs during the Ethernet board
%detection sequence at boot time. You might hypothesize that there is a
%conflict or problem with the Ethernet board in your machine. The quick
%and easy way to find out is to pull the Ethernet board, and try
%booting again. If everything goes well, then you know that either (a)
%the Ethernet board is not supported by Linux (see
%P�gina~\pageref{sec-intro-hardware}), or (b) there is an address or IRQ
%conflict with the board.

\index{IRQ}
``�Conflicto de direcci�n o IRQ?'' �Qu� diablos significa esto?
Todos los dispositivos en un computador usan una {\bf IRQ}, o 
{\em Interrupt ReQuest line}, \NT{l�nea de petici�n de interrupci�n}
para decirle al sistema que necesitan
algo hecho. Puedes pensar en la IRQ como un cordel del que el dispositivo tira
cuando necesita que el sistema se haga cargo de  alguna petici�n pendiente.
Si m�s de un dispositivo est� tirando del mismo cordel, el n�cleo no es capaz
de determinar cu�l de los dispositivos necesita su atenci�n. Desastre al instante.


%\index{IRQ}
%``Address or IRQ conflict?'' What on earth does that mean? 
%All devices in your machine use an {\bf IRQ}, or 
%{\em interrupt request line}, to tell the system that they need something
%done on their behalf. You can think of the IRQ as a cord that the device
%tugs when it needs the system to take care of some pending request.
%If more than one
%device is tugging on the same cord, the kernel won't be able to detemine
%which device it needs to service. Instant mayhem.

Entonces, hay que asegurarse de que todos los dispositivos instalados usan
una �nica IRQ. En general la IRQ de un dispositivo puede establecerse mediante
jumpers en la tarjeta; mira la documentaci�n para detalles espec�ficos 
del dispositivo.
Algunos dispositivos no requieren el uso de una IRQ, pero se sugiere
que si hay alguna disponible, se ponga. (Las controladoras SCSI Seagate
ST01 y ST02 son buenos ejemplos).

%Therefore, be sure that all of your installed devices use unique IRQ
%lines. In general, the IRQ for a device can be set by jumpers on the
%card; see the documentation for the particular device for details.
%Some devices do not require the use of an IRQ at all, but it is
%suggested that you configure them to use one if possible. (The Seagate
%ST01 and ST02 SCSI controllers are good examples).

En algunos casos, el n�cleo proporcionado por tu medio de instalaci�n est�
configurado para usar ciertas IRQs para ciertos dispositivos. Por ejemplo, la 
controladora SCSI TMC-950, la controladora de CD-ROM Mitsumi y el driver del bus del rat�n.
Si se quiere usar dos o m�s de estos dispositivos, habr� que instalar primero
{\linux} con s�lo uno de estos dispositivos activados, despu�s recompilar
el n�cleo para cambiar la IRQ  predeterminada de uno de ellos.
(Ver cap�tulo~\ref{chap-sysadm-num} para informaci�n acerca de recompilar el n�cleo.)

%In some cases, the kernel provided on your installation media is configured
%to use certain IRQs for certain devices. For example, on some distributions
%of Linux, the kernel is preconfigured to use IRQ 5 for the TMC-950 SCSI 
%controller, the Mitsumi CD-ROM controller, and the bus mouse driver. 
%If you want to use two or more of these devices, you'll need to first
%install Linux with only one of these devices enabled, then recompile the
%kernel in order to change the default IRQ for one of them.
%(See Chapter~\ref{chap-sysadm-num} for information
%on recompiling the kernel.) 


Otro �rea donde pueden aparecer conflictos de hardware es con los canales DMA
(Direct Memory Access)\NT{acceso directo a memoria}, direcciones de E/S y direcciones de
memoria compartida. Todos estos t�rminos describen mecanismos a trav�s de los cuales el sistema
se comunica con los dispositivos f�sicos. Algunas tarjetas Ethernet, por ejemplo,
usan una direcci�n compartida de memoria as� como una IRQ para comunicarse con el sistema.
Si cualquiera de �stas est� en conflicto con otros dispositivos, entonces el sistema puede comportarse
de manera err�tica.
Deber�as ser capaz de cambiar el canal DMA, las direcciones de E/S o memoria compartida para varios
dispositivos mediante los jumpers \NT{ unas clavijas en la placa} de las tarjetas. (Desafortunadamente, algunos
dispositivos no permiten cambiar estos ajustes.)




%Another area where hardware conflicts can arise is with DMA (direct
%memory access) channels, I/O addresses, and shared memory addresses. 
%All of these terms describe mechanisms through which the system interfaces 
%with hardware devices. Some Ethernet boards, for example, use a shared memory 
%address as well as an IRQ to interface with the system. If any of these
%are in conflict with other devices, then the system may behave unexpectedly.
%You should be able to change the DMA channel, I/O or shared
%memory addresses for your various devices with jumper settings. (Unfortunately,
%some devices don't allow you to change these settings.)

La documentaci�n para varios dispositivos hardware deber�a especificar la IRQ,
el canal DMA, direcci�n E/S o direcci�n de memoria compartida que el dispositivo
usa, y c�mo configurarlo. Otra vez, la manera m�s simple de evitar estos problemas
es deshabilitar temporalmente los dispositivos en conflicto hasta que se tenga
tiempo de determinar la causa del problema.

%The documentation for various hardware devices should specify the IRQ,
%DMA channel, I/O address, or shared memory address that the devices
%use, and how to configure them. Again, the simple way to get around
%these problems is to temporarily disable the conflicting devices until
%you have time to determine the cause of the problem.

En el cuadro se puede ver una lista de  las IRQ y canales DMA utilizados
por varios dispositivos ``est�ndar'' en la mayor�a de sistemas. Casi
todos los sistemas tienen alguno de estos dispositivos, as� que se puede
evitar el poner la IRQ o el DMA de otro dispositivo en conflicto con estos valores.

%The table below is a list of IRQ and DMA channels used by various
%``standard'' devices on most systems. Almost all systems have some of
%these devices, so you should avoid setting the IRQ or DMA of other
%devices in conflict with these values.

\begin{table}\begin{center}
\small\begin{tabular}{|l|l|l|l|}
\hline
Device                     &   I/O address  & IRQ & DMA \\
\hline
{\tt ttyS0} ({\tt COM1})   &   3f8          &  4  &  n/a \\
{\tt ttyS1} ({\tt COM2})   &   2f8          &  3  &  n/a \\
{\tt ttyS2} ({\tt COM3})   &   3e8          &  4  &  n/a \\
{\tt ttyS3} ({\tt COM4})   &   2e8          &  3  &  n/a \\

{\tt lp0} ({\tt LPT1})     &   378 - 37f    &  7  &  n/a \\
{\tt lp1} ({\tt LPT2})     &   278 - 27f    &  5  &  n/a \\

{\tt fd0}, {\tt fd1} (disqueteras 1 y 2) & 3f0 - 3f7 & 6 & 2 \\
{\tt fd2}, {\tt fd3} (disqueteras 3 y 4) & 370 - 377 & 10 & 3 \\
\hline
\end{tabular}\normalsize\rm
\caption{Preajustes por omisi�n de dispositivos est�ndar.}
\label{table-dev-settings}
\end{center}\end{table}

\index{hardware!conflictos|)}

% Linux Installation and Getting Started    -*- TeX -*-
% hd.tex
% Copyright (c) 1992-1994 by Matt Welsh <mdw@sunsite.unc.edu>
%
% This file is freely redistributable, but you must preserve this copyright 
% notice on all copies, and it must be distributed only as part of "Linux 
% Installation and Getting Started". This file's use is covered by the 
% copyright for the entire document, in the file "copyright.tex".
%
% Copyright (c) 1998 by Specialized Systems Consultants Inc. 
% <ligs@ssc.com>
% Traducci�n realizada por Francisco javier Fern�ndez <serrador@arrakis.es>
%Revisi�n 1 por FJFS 
%Gold
\subparagraph*{Problemas reconociendo la controladora de disco}%Problems recognizing hard drive or controller.}
\index{hardware!problemas con el disco duro}

%When Linux boots, you should see a series of messages on your screen such
%as: 
Cuando {\linux} arranca, se deber�a ver una serie de mensajes en la pantalla como:
\begin{tscreen}
Console: colour EGA+ 80x25, 8 virtual consoles \\
Serial driver version 3.96 with no serial options enabled \\
tty00 at 0x03f8 (irq = 4) is a 16450 \\
tty03 at 0x02e8 (irq = 3) is a 16550A \\
lp\_init: lp1 exists (0), using polling driver \\
\ldots
\end{tscreen}
%Here, the kernel is detecting the various hardware devices present on your
%system. At some point, you should see the line
Aqu�, el n�cleo est� detectando los distintos dispositivos hardware presentes en el sistema. En alg�n punto se deber�a ver la l�nea:
\begin{tscreen}
Partition check:
\end{tscreen}
%followed by a list of recognized partitions, for example:
seguida por una lista de las particiones reconocidas, por ejemplo:
\begin{tscreen}
Partition check: \\
\ \ hda: hda1 hda2 \\
\ \ hdb: hdb1 hdb2 hdb3
\end{tscreen}
%If, for some reason, your drives or partitions are not recognized, then
%you will not be able to access them in any way. 
Si por alguna raz�n, las unidades de disco o las particiones no se reconocen, entonces no se podr� acceder
a ellas de ninguna manera.
%There are several things that can cause this to happen:
Hay varias cosas que pueden causar que esto pase:
\begin{itemize}
\item {\bf La controladora del disco duro no est� soportada.}%Hard drive or controller not supported.} If you have a
%hard drive controller (IDE, SCSI, or otherwise) that is not supported
%by Linux, the kernel will not recognize your partitions at boot time.

Si se tiene una controladora de disco (IDE,SCSI, o lo que sea) que no tenga soporte en Linux, el
n�cleo no reconocer� las particiones al arrancar.
\index{disco duro!problemas}

\item {\bf Unidad o controladora configurada incorrectamente.}%Drive or controller improperly configured.}
%Even if your controller is supported by Linux, it may not be
%configured correctly. (This is particularly a problem for SCSI
%controllers. Most non-SCSI controllers should work fine without any
%additional configuration).

Incluso si la controladora est� soportada por Linux, quiz� no se haya configurado apropiadamente. (Este es un
problema particular para las controladoras SCSI. La mayor�a de las controladoras no SCSI deber�an funcionar bien sin
ninguna configuraci�n adicional).
%Refer to the documentation for your hard drive and/or controller. In
%particular, many hard drives need to have a jumper set to be used as a
%slave drive (the second device on either the primary or secondary IDE
%bus). The acid test of this kind of condition is to boot MS-DOS or
%some other operating system that is known to work with your drive and
%controller. If you can access the drive and controller from another
%operating system, then it is not a problem with your hardware
%configuration.
Echa un vistazo a la documentaci�n del disco duro o la controladora. En particular, 
muchos discos duros necesitan tener un jumper puesto para ser usado como unidad esclava 
(el segundo dispositivo en cualquiera del bus IDE primario o secundario)
Una prueba para esta clase de condici�n es arrancar MS-DOS o alg�n otro
sistema operativo que se sepa que funciona con la controladora y la unidad de disco.
Si se puede acceder al disco duro y la controladora desde otro sistema operativo,
entonces no es un problema de la configuraci�n de hardware.
%See P�gina~\pageref{sec-install-probs-hardware-conflicts}, above, for
%information on resolving possible device conflicts, and
%P�gina~\pageref{sec-install-probs-hardware-scsi}, below, for information
%on configuring SCSI devices.

Consulta la p�gina~\pageref{sec-install-probs-hardware-conflicts}, arriba, para
informarte acerca de la posible resoluci�n de conflictos de dispositivos, y la p�gina~\pageref{sec-install-probs-hardware-scsi} m�s abajo, para
m�s informaci�n acerca de la configuraci�n de dispositivos SCSI.

\item {\bf La controladora est� configurada apropiadamente, pero no es detectada.}%Controller properly configured, but not detected.}
%Some BIOS-less SCSI controllers require the user to specify
%information about the controller at boot time.  A description of how
%to force hardware detection for these controllers begins on
%P�gina~\pageref{sec-install-probs-hardware-scsi}.

Algunas BIOS de las controladoras SCSI requieren que el usuario especifique informaci�n
acerca de la controladora al inicio. Hay una descripci�n de c�mo
forzar la detecci�n de hardware para estas controladoras en la
p�gina~\pageref{sec-install-probs-hardware-scsi}.

\item {\bf No se reconoce la geometr�a del disco.} %Hard drive geometry not recognized.} Some systems, like
%the IBM PS/ValuePoint, do not store hard drive geometry information in
%the CMOS memory, where Linux expects to find it. Also, certain SCSI
%controllers need to be told where to find drive geometry in order for
%Linux to recognize the layout of your drive.

Algunos sistemas como los IBM PS/Valuepoint, no guardan la informaci�n de la geometr�a del disco duro en la memoria CMOS,
donde Linux espera encontrarla. Tambi�n ciertas controladoras SCSI necesitan que se las diga expl�citamente d�nde
encontrar la geometr�a de la unidad para que Linux reconozca la disposici�n del disco.

%Most distributions provide a bootup option to specify the 
%drive geometry. In general, when booting the installation
%media, you can specify the drive geometry at the LILO boot indicador de �rdenes with
%a command such as:
Muchas distribuciones proporcionan una opci�n de arranque para especificar la geometr�a del disco.
En general, cuando se arranca el medio de instalaci�n, se puede especificar la geometr�a de la unidad en
el indicador de LILO con una orden como:
\begin{tscreen}
boot: {\em linux hd=\cparam{cilindros},\cparam{cabezas},\cparam{sectores}}
\end{tscreen}
%where \cparam{cylinders}, \cparam{heads}, and \cparam{sectors} correspond
%to the number of cylinders, heads, and sectors per track for your hard
%drive. 
donde \cparam{cilindros}, \cparam{cabezas}, y \cparam{sectores} corresponden
al n�mero de cilindros, cabezas y sectores por pista del disco duro.

%After installing Linux, you will be able to install LILO, allowing you
%to boot from the hard drive. At that time, you can specify the drive
%geometry to LILO, making it unnecessary to enter the drive geometry
%each time you boot. See Chapter~\ref{chap-sysadm-num} for more
%information about LILO.

Tras instalar {\linux}, deber� instalar LILO, permiti�ndole arrancar
desde el disco duro. En este momento, se puede especificar la geometr�a de la unidad a LILO,
haciendo innecesario introducir la geometr�a del disco cada vez que arranca. Consulta el
Cap�tulo~\ref{chap-sysadm-num} para m�s informaci�n acerca de LILO.
\end{itemize}

\index{hardware!problemas con el disco duro}

% Linux Installation and Getting Started    -*- TeX -*-
% scsi.tex
% Copyright (c) 1992-1994 by Matt Welsh <mdw@sunsite.unc.edu>
%
% This file is freely redistributable, but you must preserve this copyright 
% notice on all copies, and it must be distributed only as part of "Linux 
% Installation and Getting Started". This file's use is covered by the 
% copyright for the entire document, in the file "copyright.tex".
%
% Copyright (c) 1998 by Specialized Systems Consultants Inc. 
% <ligs@ssc.com>
%Traducido por Francisco Javier Fernandez <serrador@arrakis.es>
%Revisado el 6 de julio de 2002 por Francisco Javier Fern�ndez
% Revisi�n 2 16 de julio 2002 por Francisco Javier Fernandez
%gold

\subparagraph*{Problemas con las controladoras y los dispositivos SCSI} %Problems with SCSI controllers and devices.}
\namedlabel{sec-install-probs-hardware-scsi}{}
\index{hardware!problemas con SCSI}
\index{SCSI!problemas}
%Presented here are some of the most common problems with SCSI
%controllers and devices like CD-ROMs, hard drives, and tape drives. If
%you have problems getting Linux to recognize your drive or controller,
%read on.

Aqu� se presentan algunos de los problemas m�s comunes con las controladoras SCSI
y los dispositivos como CD-ROMs, discos duros, y unidades de cinta. Si
se tiene alg�n problema con {\linux} reconociendo un disco o controladora, siga leyendo.

%The Linux SCSI HOWTO (see App�ndice~\ref{app-sources-num}) contains much useful
%information on SCSI devices in addition to that listed here. SCSI can be
%particularly tricky to configure at times.

El COMO de Linux SCSI (ver Ap�ndice~\ref{app-sources-num}) contiene mucha informaci�n
�til acerca de dispositivos SCSI en adici�n de  lo que se muestra aqu�. SCSI puede ser dif�cil de 
configurar a veces.


\begin{itemize}

\item {\bf Un dispositivo SCSI se detecta en todos los posibles IDs.}
%A SCSI device is detected at all possible IDs.} This is caused
%by strapping the device to the same address as the controller. You need to
%change the jumper settings so that the drive uses a different address than
%the controller.

Esto es causado al poner el dispositivo con el mismo identificador que la controladora. Es necesario cambiar el
ajuste del jumper para que el dispositivo use una direcci�n diferente que la controladora.


\item {\bf Linux informa de errores, incluso si se sabe que los dispositivos est�n libres de errores.} 
%Linux reports sense errors, even if the devices are known to be
%error-free.} This can be caused by bad cables or bad termination. If
%your SCSI bus is not terminated at both ends, you may have errors
%accessing SCSI devices. When in doubt, always check your cables.

Esto puede ser causado por cables defectuosos o de baja calidad o una mala terminaci�n de la cadena SCSI.
Si el bus SCSI no est� terminado a ambos extremos, se pueden producir errores accediendo a los dispositivos SCSI.
Si se tiene alguna duda, {\em siempre revise los cables}.

\item {\bf Los dispositivos SCSI informan de errores ``timeout''.}
%SCSI devices report timeout errors.} This is usually caused by 

Los errores de tiempo de espera agotado, normalmente son producidos por un conflicto con una IRQ, DMA o direcci�n de dispositivo. 
Revisa las interrupciones de la controladora, a ver si est�n en su sitio.
%a conflict with IRQ, DMA, or device addresses. Also check that interrupts
%are enabled correctly on your controller.

\item {\bf Las controladoras SCSI que usan BIOS no son detectadas.}
%SCSI controllers that use BIOS are not detected.} Detection of
%controllers that use BIOS will fail if the BIOS is disabled, or if
%your controller's signature is not recognized by the kernel. See the
%Linux SCSI HOWTO, available from the sources in
%App�ndice~\ref{app-sources-num}, for more information about this.

La detecci�n de las controladoras que usan BIOS fallar� si la BIOS est� deshabilitada, o si
la firma del controlador no la reconoce el n�cleo. Consulta el C�MO Linux SCSI, disponible
desde las fuentes de informaci�n disponibles en el Ap�ndice~\ref{app-sources-num}, para m�s informaci�n acerca de esto.

\item {\bf Las controladoras que usan  memoria de E/S mapeada no funcionan.} 
%Controllers using memory mapped I/O do not work.} This is caused
%when the memory-mapped I/O ports are incorrectly cached. Either mark the
%board's address space as uncacheable in the XCMOS settings, or disable
%cache altogether.

Esto ocurre cuando los puertos de E/S mapeados a memoria se cachean incorrectamente.
Hay dos soluciones: una es marcar el espacio de direccionamiento de la tarjeta como 
no cacheable en los ajustes de la CMOS. La segunda soluci�n es deshabilitar toda la cach�.

\item {\bf Mientras se particiona, se obtiene una advertencia tipo ``cylinders $>$ 1024'', o no se puede
iniciar desde una partici�n usando cilindros numerados por encima del 1023.}
%When partitioning, you get a warning that ``cylinders $>$ 1024'', or
%you are unable to boot from a partition using cylinders numbered above 1023.}

%BIOS limits the number of cylinders to 1024, and any partition using
%cylinders numbered above this won't be accessible from the BIOS. As far as
%Linux is concerned, this affects only booting; once the system has booted
%you should be able to access the partition. Your options are to either 
%boot Linux from a boot floppy, or boot from a partition using 
%cylinders numbered below 1024. 

La BIOS limita el numero de cilindros a 1024, y cualquier partici�n que use cilindros numerados por encima
de esto no ser� accesible por la BIOS. Esto s�lo afecta a Linux al arranque; una vez que el sistema ha arrancado
se podr� acceder a la partici�n. Las opciones son o arrancar Linux desde un disquete, o arrancar desde
una partici�n usando cilindros por debajo del 1024.

\item {\bf Al arrancar no se reconocen unidades de CD-ROM u otros dispositivos extra�bles.} %CD-ROM drive or other removeable media devices are not recognized
%at boot time.} Try booting with a CD-ROM (or disk) in the drive. This is 
%necessary for some devices. 

Intenta arrancando con un CD-ROM (o disco) en la unidad. Esto es necesario en algunos dispositivos.
\end{itemize}

%If your SCSI controller is not recognized, you may need to 
%force hardware detection at boot time. This is particularly important for
%BIOS-less SCSI controllers. Most distributions allow you to
%specify the controller IRQ and shared memory address when booting the
%installation media. For example, if you are using a TMC-8xx controller,
%you may be able to enter

Si no se reconoce su controladora SCSI, quiz� se necesite forzar la detecci�n de hardware al arrancar. Esto es particularmente importante
para las controladoras SCSI que carecen de BIOS. Muchas distribuciones permiten especificar la IRQ de la controladora y
la direcci�n de memoria compartida cuando se arranca el medio de instalaci�n. Por ejemplo, si se usa una controladora TMC-8xx,
se podr� introducir:

\begin{tscreen}
boot: linux tmx8xx=\cparam{interrupci�n},\cparam{direcci�n}
\end {tscreen}
%at the LILO boot indicador de �rdenes, where \textsl{interrupt} is the IRQ of
%controller, and \textsl{memory-address} is the shared memory
%address. Whether or not this is possible depends on the distribution
%of Linux; consult your documentation for details.
en el indicador de inicio de LILO, donde \textsl{interrupci�n} es la IRQ de la controladora, y 
\textsl{direcci�n} es la direcci�n de memoria compartida. Esto es o no posible dependiendo de la 
distribuci�n de {\linux}; consulta la documentaci�n para m�s detalles.

\index{hardware!problemas con SCSI}
\index{SCSI!problemas}


\index{hardware!problemas}
\index{instalaci�n!problemas con el hardware}

% Linux Installation and Getting Started    -*- TeX -*-
% install.tex
% Copyright (c) 1992-1994 by Matt Welsh <mdw@sunsite.unc.edu>
%
% This file is freely redistributable, but you must preserve this copyright 
% notice on all copies, and it must be distributed only as part of "Linux 
% Installation and Getting Started". This file's use is covered by the 
% copyright for the entire document, in the file "copyright.tex".
%
% Copyright (c) 1998 by Specialized Systems Consultants Inc. 
% <ligs@ssc.com>
%9 de julio de 2002 Traducci�n realizada por Francisco Javier Fern�ndez <serrador@arrakis.es>
%12 de julio de 2001 Revisi�n 1 por Francisco Javier Mart�nez <serrador@arrakis.es>
% 15 de julio revision 2 realizada por pakojavi2000
\subsection{Problemas instalando el software} %Problems installing the software.}
\namedlabel{sec-install-probs-install}

%Actually installing the Linux software should be quite trouble-free,
%if you're lucky. The only problems that you might experience would be
%related to corrupt installation media or lack of space on your Linux
%filesystems. Here is a list of these common problems.
Actualmente instalar el software {\linux} deber�a estar libre de problemas
si se tiene suerte. Los �nicos problemas que quiz� se puedan experimentar deber�an ser
los relacionados con medios de instalaci�n defectuosos o la falta de espacio en los sistemas de 
ficheros de su {\linux}. aqu� hay una lista de estos problemas comunes.

\begin{itemize}
\index{instalaci�n!errores en los medios}
%\item {\bf System reports ``{\tt read error}'' , ``{\tt file not found}'',
%or other errors while attempting to install the software.} This
%indicates a problem with the installation media. If you install from
%floppy, keep in mind that floppies are quite succeptible to media
%errors of this type. Be sure to use brand-new, newly formatted
%floppies. If you have an MS-DOS partition on your drive, many Linux
%distributions allow you to install the software from the hard
%drive. This may be faster and more reliable than using floppies.
\item  {\bf El sistema informa de  ``{\tt read error}\NT{ error de lectura}'', ``{\tt file not found}\NT{ fichero no encontrado}'', u otros errores mientras se intenta instalar el software}


Esto indica un problema con el medio de instalaci�n. Si se instala desde disquete, hay que tener
en cuenta que los disquetes son muy susceptibles de tener errores de este tipo.
Es conveniente asegurarse de utilizar un disquete nuevo de marca, reci�n formateado.
Si se tiene una particion de MS-DOS en el disco, muchas distribuciones de {\linux}
permiten la instalaci�n de software desde disco duro. Esto puede ser m�s r�pido y
seguro que usar disquetes.

%If you use a CD-ROM, be sure to check the disc for scratches, dust, or
%other problems that may cause media errors.

Si se utiliza un CD-ROM, hay que asegurarse de verificar el disco buscando
ara�azos, polvo o otros problemas que puedan causar errores en el medio.

%The cause of the problem may be that the media is in the incorrect
%format. Many Linux distributions require that the floppies be
%formatted in high-density MS-DOS format. (The boot floppy is the
%exception; it is not in MS-DOS format in most cases.) If all else
%fails, either obtain a new set of floppies, or recreate the floppies
%(using new diskettes) if you downloaded the software yourself.
La causa del problema puede ser que el medio est� en el formato incorrecto.
Muchas distribuciones de {\linux} requieren que los disquetes est�n formateados
en el formato MS-DOS de alta densidad.  El disquete de arranque es la
excepci�n; no est� en formato MS-DOS en la mayor�a de los casos. Si todo lo
dem�s falla, mejor obtener un nuevo conjunto de disquetes, o rehacer los
disquetes (utilizando disquetes {\em nuevos}) si se descarg� el software usted mismo.

% System reports errors such as ``{\tt tar: read error}'' or 
%``{\tt gzip: not in gzip format}''.}  This problem is usually caused
%by corrupt files on the installation media. In other words, your
%floppy may be error-free, but the data on the floppy is in some way
%corrupted. If you downloaded the Linux software using text mode,
%rather than binary mode, then your files will be corrupt, and
%unreadable by the installation software.
\item {\bf El sistema informa de errores como ``{\tt tar: read error}\NT{ error de lectura}'' o ``{\tt gzip: not in gzip format}\NT{ no en formato gzip}''} 


Este problema est� causado usualmente por ficheros corrompidos en el medio de instalaci�n. En otras palabras, 
los disquetes pueden estar libres de errores, pero los datos en los disquetes
est�n de alg�n modo corrompidos. Si se descarg� el software de {\linux} usando
modo texto en vez de modo binario, entonces los ficheros estar�n corrompidos, y
el programa de instalaci�n no podr� leerlos.

\item {\bf El sistema informa de errores como ``{\tt device full}\NT{ dispositivo lleno}''  mientras se instala}
%This is a clear-cut sign that you have run out of space
%when installing the software. Not all Linux distributions can pick up
%the mess cleanly; you shouldn't be able to abort the installation and expect
%the system to work.


Este es un s�ntoma claro de que se ha acabado el espacio cuando se instala el software.
No todas las distribuciones de {\linux} pueden solucionar el 
desastre limpiamente; no se podr� abortar la instalaci�n y esperar que el sistema funcione.
%The solution is usually to re-create your file systems (with {\tt
%mke2fs}) which deletes the partially installed software. You can
%attempt to re-install the software, this time selecting a smaller
%amount of software to install. In other cases, you may need to start
%completely from scratch, and rethink your partition and filesystem
%sizes.


La soluci�n pasa por volver a crear el sistema de ficheros (con {\tt
mke2fs}), que borra el software parcialmente instalado). Se puede intentar
una reinstalaci�n, seleccionando esta vez una menor cantidad de paquetes a
instalar. En otros casos, quiz� se necesite empezar completamente desde el principio,
y replantearse el particionado y los tama�os de los sistemas de ficheros.

%\item {\bf System reports errors such as ``{\tt read\_intr: 0x10}'' while
%accessing the hard drive.}  This usually indicates bad blocks on your
%drive. However, if you receive these errors while using {\tt mkswap}
%or {\tt mke2fs}, the system may be having trouble accessing your
%drive. This can either be a hardware problem (see
%P�gina~\pageref{sec-install-probs-hardware}), or it might be a case of
%poorly specified geometry. If you used the

\item {\bf El sistema informa de errores como ``{\tt read\_intr: 0x10}\NT{ lectura interrumpida}''  mientras se accede al disco duro}


Esto usualmente indica que hay bloques defectuosos en la unidad. Sin embargo,
si se reciben estos errores mientras se usa {\tt mkswap} o {\tt mke2fs}, el
sistema puede estar teniendo problemas accediendo a la unidad. Esto puede
ser o un problema de hardware (ver p�gina~\pageref{sec-install-probs-hardware}), o
puede quiz� ser un caso de una geometr�a mal especificada. Si se utiliz� la opci�n
\begin{tscreen}
hd=\cparam{cylindros},\cparam{cabezas},\cparam{sectores}
\end{tscreen}
%option at boot time to force detection of your drive geometry, and 
%incorrectly specified the geometry, you could be prone to this problem.
%This can also happen if your drive geometry is incorrectly specified in
%the system CMOS. 
en tiempo de arranque, para forzar la detecci�n de la geometr�a de la unidad,
y  se especific� incorrectamente la geometr�a, se puede ser candidato a este problema.
Esto tambi�n puede ocurrir cuando la geometr�a de la unidad est� especificada
incorrectamente en la CMOS del sistema.

%\item {\bf System reports errors like ``{\tt file not found}'' or 
%``{\tt permission denied}''.} This problem can occur if not all of the 
%necessary files are present on the installation media (see the next 
%paragraph) or if there is a permissions problem with the installation
%software. For example, some distributions of Linux have been known to
%have bugs in the installation software itself. These are usually fixed
%very rapidly, and are quite infrequent.
%If you suspect that the distribution software contains bugs, and
%you're sure that you have not done anything wrong, contact the maintainer
%of the distribution to report the bug. 
\item {\bf El sistema informa de errores como ``{\tt file not found}\NT{ fichero no encontrado}'' o ``{\tt permission denied}\NT{ permiso denegado}''} 


Este problema puede ocurrir cuando no todos los
ficheros necesarios se encuentran presentes en el medio de instalaci�n (ver el 
siguiente p�rrafo) o si hay un problema de permisos con el software de instalaci�n.
Por ejemplo, se sabe que algunas distribuciones de {\linux} tienen errores en el
programa de instalaci�n mismo. �stas normalmente se solucionan r�pidamente, y son 
muy infrecuentes.
Si se sospecha que el software de la distribuci�n contiene errores, y se est�
seguro de que no se ha hecho nada equivocadamente, contacte con el mantenedor
de la distribuci�n para informar del error.
\end{itemize}

%If you have other strange errors when installing Linux (especially if you 
%downloaded the software yourself), be sure that you actually obtained all
%of the necessary files when downloading. For example, some people use the
%FTP command 
Si se tienen otros errores extra�os cuando se instala {\linux} (especialmente si 
descarg� el software usted mismo), aseg�rese de que se obtuvo todos los ficheros necesarios
en la descarga. Por ejemplo, alguna gente utiliza la orden FTP

\begin{tscreen}
mget *.*
\end{tscreen}

%when downloading the Linux software via FTP. This will download only those
%files that contain a ``{\tt .}'' in their filenames; if there are any files
%without the ``{\tt .}'', you will miss them. The correct command to use
%in this case is
cuando descarga el software de {\linux} por FTP. Esto descargar� �nicamente aquellos
ficheros que contengan un punto ``{\tt .}'' en sus nombres de fichero; si
alg�n fichero no tiene el punto ``{\tt .}'', no se descargar�. La orden correcta
para descargarlo todo en este caso es: 

\begin{tscreen}
mget *
\end{tscreen}

%The best advice is to retrace your steps when something goes
%wrong. You may think that you have done everything correctly, when in
%fact you forgot a small but important step somewhere along the way. In
%many cases, re-downloading and re-installing the software can solve
%the problem. Don't beat your head against the wall any longer than you
%have to!
La mejor advertencia es repasar los pasos que se han dado cuando algo va mal.
Qu�z� se piense que se ha hecho todo correctamente, cuando de hecho
se olvid� un peque�o pero importante paso en alg�n lugar a lo largo del proceso.
En muchas ocasiones, volver a descargar y reinstalar el software puede 
resolver el problema. �No se d� coscorrones frente a un muro m�s de lo necesario!

%Also, if Linux unexpectedly hangs during installation, there may be a
%hardware problem of some kind. See page ~lala for hints
Adem�s, si {\linux} se cuelga sin esperarlo durante la instalaci�n, quiz� haya un
problema de hardware de alguna clase. Consulte la p�gina~\pageref{sec-install-probs-hardware}
para m�s datos.

% Linux Installation and Getting Started    -*- TeX -*-
% postinstall.tex
% Copyright (c) 1992-1994 by Matt Welsh <mdw@sunsite.unc.edu>
% This file is freely redistributable, but you must preserve this copyright 
% notice on all copies, and it must be distributed only as part of "Linux 
% Installation and Getting Started". This file's use is covered by the 
% copyright for the entire document, in the file "copyright.tex".
%
%Este fichero es redistribuido libremente, pero debes mantener este anuncio 
%del copyright en todas las copias, y solo debe ser distribuido como una parte de
%"Instalaci�n de Linux y Comenzar (a utilizar linux)". El uso de este fichero
%esta cubierto por el copyright para el documento entero, en el fichero "copyright.tex".
%
% Copyright (c) 1998 by Specialized Systems Consultants Inc. 
% <ligs@ssc.com>

%\section{Post-installation procedures.}
%\markboth{Obtaining and Installing Linux}{Post-installation Procedures}
%\namedlabel{sec-install-postinstall}{Postinstallation procedures}
% Traducido por Juan Garay Ram�rez rottwailer10@hotmail.com
% Primera correcci�n y subida al CVS por Antonio Rueda montuno@openbank.es
% Revisi�n 2 por Francisco javier Mart�nez <serrador@arrakis.es>
%Gold

\section {Procedimientos post-instalaci�n}
\markboth {Obteniendo e instalando {\linux}}{Procedimientos despu�s de la instalaci�n}
%\namedlabel {sec-instalar-despu�s de instalar}{Procedimientos despu�s de la instalaci�n}
\namedlabel{sec-install-postinstall}{Procedimientos despu�s de la instalaci�n}

% After you complete the Linux installation, there should be little left
% to do before you can use the system. In most cases, you should be able
% to reboot the system, login as {\tt root}, and begin exploring the
% system. (Each distribution has a slightly different method for doing
% this.)

% Este p�rrafo aparece comentado en el original.
%Despu�s de completar la instalaci�n de Linux, hay una peque�a cosa 
%que debes hacer antes de poder usar el sistema. En la mayor�a de los casos,
%debes reiniciar el sistema, y entrar como administrador {\tt root}, y empezar a examinar 
%el sistema. (Cada distribuci�n tiene ligeramente m�todos diferentes para
%hacer esto.)

% At this point it's a good idea to explain how to reboot and shutdown
% the system as you're using it. You should never reboot or shutdown
% your Linux system by pressing the reset switch.
% You shouldn't simply switch off the power, either. As with most UNIX
% systems, Linux caches disk writes in memory. Therefore, if you
% suddenly reboot the system without shutting down ``cleanly'', you can
% corrupt the data on your drives, causing untold damage.

En este punto es una buena idea explicar como reiniciar y apagar el sistema 
que est�s usando. Nunca debes reiniciar o apagar el sistema {\linux} 
presionando el bot�n de reset.
Tampoco debes desconectar la electricidad nunca. Como con la mayor�a de los 
sistemas UNIX, Linux guarda las escrituras en disco en memoria. Por lo tanto, 
si reinicias de repente el sistema sin apagarlo limpiamente (correctamente), 
puedes corromper los datos de los discos duros, causando un da�o incalculable.

% The easiest way to shut down the system is with the {\tt shutdown} command.
% As an example, to shutdown and reboot the system immediately, use the
% following command as {\tt root}:
% \begin{tscreen}
% \# shutdown --r now
% \end{tscreen}
% \index{shutdown@{\tt shutdown}}
% This cleanly reboots your system. The manual page for {\tt shutdown}
% describes the other command-line arguments that are available. Use
% the command {\tt man shutdown} to see the manual page for {\tt shutdown}.

La manera m�s f�cil de apagar el sistema es con la orden {\tt shutdown}.
Por ejemplo, para apagar y reiniciar el sistema inmediatamente, 
usa la orden siguiente {\tt root}:
\begin{tscreen}
\# shutdown --r now
\end{tscreen}
\index{shutdown@{\tt shutdown}}
\index{apagar}
Esto reinicia limpiamente el sistema. La p�gina del manual para {\tt shutdown}
describe las otras �rdenes y argumentos disponibles. Usa la orden
{\tt man shutdown} para ver la p�gina del manual para la orden {\tt shutdown}.

% Note, however, that many Linux distributions do not provide the {\tt
% shutdown} command on the installation media. This means that the first
% time you reboot your system after installation, you may need to use
% the \key{Ctrl}-\key{Alt}-\key{Del} combination. 

Observa, sin embargo, que muchas de las distribuciones de {\linux} no vienen
provistas de la orden {\tt shutdown} en la instalaci�n media (b�sica). Esto significa
que la primera vez que reinicies el sistema despu�s de la instalaci�n, 
necesitas usar la combinaci�n de teclas \key{Ctrl}-\key{Alt}-\key{Del}.

% After you have a chance to explore and use the system, there are
% several configuration chores that you should undertake. The first is
% to create a user account for yourself (and, optionally, any other
% users that might have access to the system). Creating user accounts is
% described in Chapter~\ref{chap-sysadm-num}. Usually, all that you have
% to do is {\tt login} as {\tt root}, and run the {\tt adduser} (sometimes
% {\tt useradd}) program. This leads you through several indicador de �rdeness to
% create new user accounts. 

Despu�s de tener la ocasi�n de explorar y usar el sistema, hay varias tareas de 
configuraci�n que debes emprender. La primera es crear una cuenta de 
usuario para ti mismo (y, opcionalmente, alguna otra para usuarios que 
vayan a tener acceso al sistema). La creaci�n de cuentas de usuario est� 
descrita en el cap�tulo ~\ref{chap-sysadm-num}. Normalmente, todo lo que 
tienes que hacer es registrarte como {\tt root} (administrador),
y ejecutar el programa {\tt adduser} (a veces  {\tt useradd}). �ste te lleva a 
trav�s de  varios indicadores para crear cuentas de usuario nuevas.

% If you create more than one filesystem for Linux, or if you're using a
% swap partition, you may need to edit the file {\tt /etc/fstab} in
% order for those filesystems to be available automatically after
% rebooting. If you're using a separate filesystem for {\tt /usr}, and
% none of the files that should be in {\tt /usr} appear to be present,
% you may simply need to mount that filesystem. See
% P�gina~\pageref{example-sysadm-fstab} for a description of the {\tt
% /etc/fstab} file.
% \ifodd\igsslack
% Note that the Slackware distribution of Linux automatically configures
% your filesystems and swap space at installation time, so this 
% usually isn't necessary.
% \fi

Si creas m�s de un sistema de ficheros para {\linux}, o si est�s usando una 
partici�n de intercambio, quiz� necesites editar el fichero {\tt /etc/fstab}  
para que esos sistemas de ficheros est�n disponibles autom�ticamente 
despu�s de reiniciar. Si est�s usando sistemas de ficheros separados para {\tt /usr},
y ninguno de esos ficheros que deber�an estar en  {\tt /usr} parecen estar presentes, 
simplemente necesitas montar esos sistemas de ficheros.
Consulta la p�gina~\pageref{example-sysadm-fstab} para una descripci�n del fichero {\tt /etc/fstab}.

% Este parrafo aparece comentado en el original.
%\ifodd\igsslack
%Observa que la distribuci�n Slackware de Linux configura autom�ticamente
%tus ficheros de sistema y el espacio swap a la hora de la instalaci�n, de este 
%modo esto normalmente no es necesario.
%\fi

\index{instalaci�n!problemas|)}

