% {\linux} Installation and Getting Started    -*- TeX -*-
% distributions.tex
% Copyright (c) 1992-1994 by Matt Welsh <mdw@sunsite.unc.edu>
%
% This file is freely redistributable, but you must preserve this copyright 
% notice on all copies, and it must be distributed only as part of "{\linux} 
% Installation and Getting Started". This file's use is covered by the 
% copyright for the entire document, in the file "copyright.tex".
%
% Copyright (c) 1998 by Specialized Systems Consultants Inc. 
% <ligs@ssc.com>
%
% Traducci�n realizada por Alfonso Belloso. Env�e sus comentarios a:
%            Alfonso Belloso, 2:344/17.2 (FidoNet)
%            alfon@bipv02.bi.ehu.es.es (InterNet)
%
% Versi�n revisada LIPP 2.0 por Carlos Valdivia Yag�e. Comentarios a:
%            valyag@teleline.es
%
% Revisado y listo para publicaci�n el 15 de julio de 2002 por Javier Fernandez <serrador@arrakis.es>

\section{Distribuciones de {\linux}}
\markboth{Obtenci�n e instalaci�n de Linux}{Distribuciones de {\linux}}
\namedlabel{sec-install-distributions}{Distribuciones de {\linux}}
\namedlabel{sec-install-distrib}{Distribuciones de {\linux}}

\index{distribuciones|(}

% Puesto que {\linux} es software libre, no hay ninguna organizaci�n o
% entidad responsable de mantenerlo y distribuirlo. Por tanto,
% cualquiera es libre de agrupar y distribuir el software, en tanto en
% cuanto se respeten las restricciones de la GPL. El resultado final de
% esto es que existen muchas distribuciones de {\linux}, disponibles a
% trav�s de FTP an�nimo o en CD-ROM.

Ud. se encuentra ahora con la tarea de decidirse por una distribuci�n en
particular de {\linux} que se ajuste a sus necesidades. No todas las
distribuciones son iguales. Muchas de ellas incluyen pr�cticamente todo
el software que Ud. necesitar�a para poner en marcha un sistema
completo--- y otras distribuciones son ``peque�as'' distribuciones
orientadas a usuarios sin grandes cantidades de espacio en disco. Muchas
distribuciones solamente contienen lo esencial del software de {\linux}, y
se espera que Ud. instale paquetes de software m�s grandes, tales como
el sistema X Window. (En el cap�tulo~\ref{chap-xwindow} le mostraremos
c�mo.)

El {\em Linux Distribution HOWTO\/} (vea el Ap�ndice~\ref{sec-getting-internet})
contiene una lista de las distribuciones de {\linux} disponibles a trav�s de
Internet as� como por correo.

Si tiene acceso a las noticias de USENET, o a otro sistema electr�nico de
informaci�n, puede querer pedir all� opiniones personales de gente que
haya instalado {\linux}. Adem�s, {\it Linux Journal} mantiene una tabla de
revisiones de distribuciones (consulte {\tt
http://www.linuxjournal.com/selected.html} para leer las versiones
on-line de la tabla y los art�culos). Incluso mejor, si conoce a alguien
que haya instalado {\linux}, p�dale ayuda y consejo. Existen muchos
factores a considerar al elegir una distribuci�n; sin embargo, las
necesidades y opiniones de cada uno son diferentes. En la actualidad, la
mayor parte de las distribuciones de {\linux} m�s populares incluyen
pr�cticamente el mismo software, por lo que la distribuci�n que elija es m�s o
menos arbitrario.


\index{distribuciones|)}
