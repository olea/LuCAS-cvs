% Linux Installation and Getting Started    -*- TeX -*-
% acknowledgments.tex
% Copyright (c) 1992-1994 by Matt Welsh <mdw@sunsite.unc.edu>
% Copyright (c) 1997 Specialized Systems Consultants, Inc.
% 
% This file is freely redistributable, but you must preserve this copyright 
% notice on all copies, and it must be distributed only as part of "Linux 
% Installation and Getting Started". This file's use is covered by
% the copyright for the entire document, in the file "copyright.tex".
%
% Copyright (c) 1998 by Specialized Systems Consultants Inc. 
% <ligs@ssc.com>
% Corregido y comparado con el original por Antonio Rueda (24/11/00)
% montuno@openbank.es
%
% Corregido por Antonio Rueda (02/06/2002)

\section*{Agradecimientos}
 \addcontentsline{toc}{section}{Agradecimientos}

Esta edici�n se realiza gracias al trabajo de los que estuvieron antes,
mencionados m�s atr�s en el agradecimiento original de Matt Welsh.
Adem�s de ellos, por nuestra parte (SSC) tenemos que dar gracias a Larry Ayers, 
Boris Beletsky, Sean Dreilinger, Evan Leibovitch, y Henry Pierce por
contribuir con la informaci�n que se encuentra en el cap�tulo 2 
acerca de S.u.S.E. Linux\tm, Debian GNU/Linux\tm, Linux Slackware\tm, 
Caldera OpenLinux\tm, y Red Hat Linux\tm, respectivamente. David Bandel 
ha actualizado el Cap�tulo~\ref{chap-install-num} y ha a�adido una secci�n que describe los pasos
gen�ricos de una instalaci�n de GNU/Linux.  Vernard Martin ha actualizado y a�adido
material al Cap�tulo~\ref{chap-xwindow}.  Asimismo hay que dar gracias a Belinda Frazier
por su edici�n y a Jay Painter por su actualizaci�n del cap�tulo ~\ref{chap-networking} sobre
administraci�n de sistemas.

\subsection*{Agradecimientos de la anterior edici�n}
% \addcontentsline{toc}{section}{Acknowledgments from the previous edition.}

Este libro ha estado realiz�ndose durante mucho tiempo, 
y en �l han contribuido muchas personas. Quiero dar las gracias particularmente a  Larry Greenfield
y a Karl Fogel por su trabajo en la primera versi�n del Cap�tulo~\ref{chap-tutorial}, y a Lars Wirzenius
por su trabajo en el Cap�tulo~\ref{chap-sysadm}. Gracias a Michael K. Johnson por su asistencia con el LDP 
y las convenciones \LaTeX\ que se han utilizado en este manual, y a Ed Chi, que me envi� una copia impresa del libro. 

Gracias a Melinda A. McBride de SSC, Inc., que hizo un excelente trabajo
completando el �ndice de los Cap�tulos~\ref{chap-tutorial},
\ref{chap-sysadm}, y~\ref{chap-xwindow}. Tambi�n quisiera agradecer a
Andy Oram, Lar Kaufman, y Bill Hahn de O'Reilly y Asociados su apoyo
al LDP.

Gracias a Linux Systems Labs\tm, Morse Telecommunications\tm e Yggdrasil Computing\tm,
por su apoyo al LDP mediante las ventas de este libro y otros trabajos.

Muchas gracias a los numerosos activistas de {\linux}, entre ellos (sin seguir
un orden concreto) Linus Torvalds, Donald Becker, Alan Cox, Remy Card, 
Ted T'so, H. J. Lu, Ross Biro, Drew Eckhardt, Ed Carp, Eric Youngdale, 
Fred van Kempen, y Steven Tweedie, por dedicarle tanto tiempo y energ�a a 
a este proyecto. Sin ellos no habr�a nada sobre lo que escribir en un libro como 
�ste.

Para terminar, damos las gracias a los innumerables lectores que nos han
hecho llegar sus correcciones y comentarios, y que son demasiados para 
enumerarlos aqu�.


