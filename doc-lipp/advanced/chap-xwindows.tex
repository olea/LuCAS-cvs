% Linux Installation and Getting Started    -*- TeX -*-
% chap-xwindows.tex.
% Copyright (c) 1992, 1993 by Matt Welsh <mdw@sunsite.unc.edu>
%
% This file is freely redistributable, but you must preserve this copyright 
% notice on all copies, and it must be distributed only as part of "Linux 
% Installation and Getting Started". This file's use is covered by the 
% copyright for the entire document, in the file "copyright.tex".
%
% Copyright (c) 1998 by Specialized Systems Consultants Inc. 
% <ligs@ssc.com>
%Revision 5 de julio 2002 por Francisco Javier Fern�ndez <serrador@arrakis.es>
%Gold

%\chapter{The X Window System}\label{chap-xwindows}
\chapter{El Sistema X-Window}
\label{chap-xwindow}
\markboth{El Sistema X-Window}{}

% derived from the monstrous "advanced features" chapter.
\index{X~Window~System|(}
\index{X~Window~System!definici�n}
\markboth{Caracter�sticas Avanzadas}{El Sistema X~Window}

%The X~Window System is a graphical user interface (GUI) that was
%originally developed at the Massachusetts Institute of
%Technology. Commercial vendors have since made X the industry standard
%GUI for UNIX platforms. Virtually every UNIX workstation in the world
%now runs some form of X.

El sistema X-Window es una interfaz gr�fica para usuario (``GUI''
en sus siglas en ingl�s), que se desarroll� originalmente en el Instituto de Tecnolog�a de Massachussetts (Massachusetts Institute of
 Technology, m�s conocido como MIT). X es la ``GUI'' est�ndar para plataformas UNIX
 comerciales. Pr�cticamente todas las estaciones de trabajo UNIX del
 mundo trabajan bajo alguna forma de X.


\index{X11R6}
\index{XFree86}
%A free port of the MIT X~Window System version 11, release 6 (X11R6)
%for 80386, 80486, and Pentium UNIX systems was developed by a team of
%programmers that was originally headed by David Wexelblat. This
%release, known as XFree86\footnote{XFree86 is a trademark of The
%XFree86 Project, Inc.}, is available for System V/386, 386BSD, and
%other Intel x86 UNIX implementations, including Linux. It provides all
%of the binaries, support files, libraries, and tools required for
%installation.

Un equipo  de programadores encabezados inicialmente por David Welxelblat 
desarrollaron un porte libre del sistema X-Window del MIT,
versi�n 11 y ``release'' 6 (X11R6) para sistemas UNIX con 80386, 80486
y Pentium. Esta ``release'', conocida como
XFree86\footnote{XFree86 es una marca registrada por XFree86 Project,
  Inc.}, est� disponible para sistemas V/386, 386BSD y otras
implementaciones UNIX de Intel x86, incluyendo {\linux}. Proporciona
todos los binarios, ficheros de soporte, bibliotecas y utilidades para
la instalaci�n. 

%Some features offered by this release are:
%\begin{itemize}
%\item complete inclusion of the X Consortium's X11R6.3 release;
%\item a new DPMS extension, donated by Digital Equipment Corporation;
%\item the Low Bandwidth X (LBX) extension in all X servers;
%\item Microsoft IntelliMouse support;
%\item support for {\tt gzip} font compression.
%\end{itemize}

Algunas de las caracter�sticas que ofrece esta versi�n son:
\begin{itemize}
\item Inclusi�n de la versi�n completa de X11R6.3 del {\em X Consortium}.
\item Una nueva extensi�n ``DPMS'', donada por Digital Equipment
  Corporation.
\item La extensi�n de {\em Low Bandwidth X}(LBX) en todos los
  servidores X.
\item Soporte {\em Microsoft Intellimouse}
\item Soporte para la compresi�n de fuentes {\tt gzip}
\end{itemize}

%To use the X Window System, you are encouraged to read {\em The
%X~Window System: A User's Guide\/} (see Appendix~\ref{app-info}).
%Here, we describe step-by-step an XFree86 installation under Linux.
%You still need to fill in some of the details by reading the XFree86
%documentation, which is discussed below.  The Linux {\em XFree86 HOW
%TO\/} is another good information source.

Para usar el sistema X-Window, se recomienda leer {\em The X-Window
  System: A User's Guide\/} (ver Ap�ndice). Aqu�
  describiremos paso a paso la instalaci�n de XFree86 bajo {\linux}. Debe
  fijarse en algunos detalles leyendo la documentaci�n de XFree86, que
  se discute m�s adelante. El {\em XFree86 C�MO\/} de {\linux} es otra fuente de
  informaci�n interesante.
%%% Input each section

%%% Parte de la Guia Completa a Linux ver 1.2.2
%%% de Jaime E. Gomez <jgomez@uniandes.edu.co>
%%% http://linuxcol.uniandes.edu.co/infolinux/docs/guia_linux/guia_linux.html
\porcion{Configuraci�n de X-window}
\autor{\jeg}
\colaborador{\NC}
\revisor{\LLC}
\traductor{}

El sistema gr�fico est�ndar en las m�quinas UNIX y
particularmente en GNU/Linux, es X-Window, y ahora se procede a 
su instalaci�n. El sistema intentar� detectar la tarjeta de 
v�deo y el monitor presente
en el equipo. En general se tiene �xito en la auto-detecci�n, pero
de no ser as� siempre se puede escoger de la lista de dispositivos
proveida por el sistema.

Las nuevas distribuciones incluyen cuatro opciones de servidor para
instalar: 
\begin{itemize}
\item {\sf XFree86 3.3.6}: Opci�n m�s segura y conocida. 
\item {\sf XFree86 3.3.6 con aceleraci�n 3D }: La tarjeta es soportada en esta versi�n en modo tridimensional.
\item {\sf XFree86 4.0.3}: La versi�n m�s reciente de X-Window.
\item {\sf XFree86 4.0.3 con aceleraci�n 3D }: Igual a la anterior con soporte para motor gr�fico 3D.
\end{itemize}

Existe una gran cantidad de tarjetas soportadas por XFree86, y por
lo tanto la lista de estos dispositivos es extensa (alrededor de 1000).
Si la tarjeta del ordenador no se encuentra en ella, y definitivamente
no est� soportada, se puede escoger 
{\sf Unsupported VGA compatible} o {\sf Generic VGA Compatible},
aunque �stas s�lo proveer�n 16 colores en una resoluci�n m�xima de 800x600 usando
est�ndar VESA-1 de VGA16 (figura~\ref{fig_config_Xwin_tarjeta}).
Con algunas tarjetas no soportadas tambi�n puede escogerse el
despliegue gr�fico a trav�s de {\it Framebuffer} usando el
estandar VESA-2; pero no es recomendado por su bajo desempe�o.

\figura
{Configuraci�n de tarjeta de video para X-window}
{fig_config_Xwin_tarjeta}
{width=11cm}{Mandrake/configura/install/pan_tarjeta_video.png}                        


La lista de monitores es tambi�n extensa, pero es posible que no
exista el que se posee (figura~\ref{fig_config_Xwin_monitor}).
Para esto existen varias opciones est�ndar con la capacidad m�xima
del monitor. Por ejemplo, 1024x768 a 70 Hz. Si se desconoce esta 
caracter�stica, a�n es posible escoger dos opciones gen�ricas: 

\begin{itemize}
\item{{\it Standard VGA }: Monitores algo obsoletos que soportan
					VGA a frecuencias predeterminadas }
\item{{\it Super VGA}: Monitores que soportan SVGA a diferentes
frecuencias ({\it Multisync})}
\end{itemize}

Se recomienda ser conservador en esta selecci�n. Si escoge err�neamente
la frecuencia de refresco y su monitor no est� protegido, es posible 
que pueda da�arlo permanentemente. Todos los monitores modernos 
se apagan autom�ticamente en cuanto se trata de utilizar una frecuencia
mayor a la soportada.


\figura
{Configuraci�n de monitor para X-window}
{fig_config_Xwin_monitor}
{width=11cm}{Mandrake/configura/install/pan_monitor.png}                        

Con el monitor, la tarjeta y la memoria de v�deo es posible determinar 
tanto la resoluci�n como la cantidad de colores (profundidad). El instalador sugiere una
combinaci�n, pero esta puede ser cambiada al gusto del usuario. Al 
presionar [{\sf Aceptar}] se pregunta si se {\sf desea probar la 
configuraci�n}. El autor recomienda que se acepte con cautela, ya que es
posible que su tarjeta de v�deo se infarte y tenga que reiniciar todo
nuevamente.

Una vez se realiza la prueba, aparece la pantalla de confirmaci�n en
donde se puede cambiar toda la configuraci�n, tarjeta, monitor,
resoluci�n, etc. Una vez se considere satisfactoria la configuraci�n
se presiona [{\sf Hecho}]. Puede tambi�n responder afirmativamente
a la pregunta si se desea que el {\sf computador lance autom�ticamente
X al iniciar}. 

Ahora se presenta la pantalla de felicitaciones, se tiene un Mandrake
Linux instalado, y se prepara para reiniciar. El disco de instalaci�n ser�
expulsado y la m�quina reiniciada una vez se presione [{\sf Aceptar}].








