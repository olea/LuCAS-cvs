% Linux Installation and Getting Started    -*- TeX -*-
% uucp.tex
% Copyright (c) 1992, 1993 by Matt Welsh <mdw@sunsite.unc.edu>
%
% This file is freely redistributable, but you must preserve this copyright 
% notice on all copies, and it must be distributed only as part of "Linux 
% Installation and Getting Started". This file's use is covered by the 
% copyright for the entire document, in the file "copyright.tex".
%
% Copyright (c) 1998 by Specialized Systems Consultants Inc. 
% <ligs@ssc.com>
%
%
%
% Traduccion realizada por Juan Jose Amor. Envie sus comentarios a:
%            Juan Jose Amor, 2:341/12.19 (FidoNet)
%            jjamor@gedeon.ls.fi.upm.es (InterNet)
% $Log: uucp.tex,v $
% Revision 1.7  2003/07/19 20:28:22  pakojavi2000
% Arreglando un peque�o destrozo de macros
%
% Revision 1.6  2003/07/19 07:14:41  joseluis.ranz
% Correcciones varias.
%
% Revision 1.5  2002/07/25 02:08:22  pakojavi2000
% Beta 2.1
%
% Revision 1.4  2002/07/19 16:53:55  pakojavi2000
% Beta 1
%
% Revision 1.3  2002/07/13 09:54:45  pakojavi2000
% Gold version.
%
% Revision 1.2  2002/07/07 10:19:11  pakojavi2000
% M�s correcciones
%
% Revision 1.1.1.1  2000/10/31 18:33:56  montuno
% no message
%
% Revision 0.5.0.1  1996/02/10 23:45:06  rcamus
% Primera beta publica
%
% Revision 1 14/03/2000 realizada por Fernando Peral Perez <ffddoo@openbank.es>
% Revisi�n 2 7 de julio 2002 realizada por Francisco Javier Fernandez <serrador@arrakis.es>
% Revisi�n 3 13 de julio 2002 por Franciosco Javier Fernandez <serrador@arrakis.es>
%Gold

\section{Red con UUCP}\label{sec-uucp}
\markboth{Redes}{UUCP Redes}
\index{UUCP|(}
\index{redes!UUCP|(}
\index{noticias!UUCP}
UUCP (UNIX-to-UNIX Copy) es un viejo mecanismo usado para transferir
informaci�n entre sistemas Unix. Mediante UUCP, los sistemas Unix se
comunican con otros (v�a m�dem), transfiriendo mensajes de correo,
art�culos de {\it noticias}, ficheros y dem�s. Si no tiene acceso TCP/IP o
SLIP, puede  usar UUCP para comunicarse con el mundo. Casi todo el software de
correo y {\it noticias} (ver Secciones~\ref{sec-mail} y \ref{sec-news}) se puede
configurar para usar transferencias UUCP. De
hecho, si tiene  alg�n servidor Internet cercano, puede recibir correo en su
sistema de  esa red mediante UUCP.

El libro {\em Linux Network Administrator's Guide} contiene informaci�n
completa para configurar y utilizar UUCP en {\linux}. Tambi�n encontrar�
informaci�n en el documento {\it UUCP-HOWTO}, que puede obtener por FTP
an�nimo de {\tt sunsite.unc.edu}. Otra fuente de informaci�n sobre UUCP
es el libro {\em Managing UUCP and USENET}, de Tim O'Reilly y Grace Todino.
Vea el ap�ndice~\ref{app-info} para m�s informaci�n.


\index{UUCP|)}
\index{redes!UUCP|)}

\section{Redes con sistemas Microsoft}\label{sec-microsoftnet}
\markboth{Redes}{Microsoft Redes}
\index{Microsoft|(}
\index{redes!Microsoft|(}
Samba es un conjunto de programas que trabajan juntos para permitir a clientes
el acceso a ficheros e impresoras de un servidor a trav�s del protocolo SMB
(Server Message Block). Escrito inicialmente para Unix, Samba se ejecuta ahora
tambi�n en Netware, OS/2 y VMS.

En la pr�ctica esto significa que se pueden redireccionar a {\linux} discos e
impresoras de clientes Lan Manager, Windows 3.11, Windows NT, {\linux} y OS/2.
Esto permite a estos sistemas operativos comportarse como un servidor LAN o un
servidor Windows NT, s�lo que con funcionalidad y flexibilidad a�adida para
hacer la vida m�s f�cil para los administradores.



{\em Samba: Integrando Unix y Windows} contiene informaci�n completa sobre la
configuraci�n y uso de Samba en {\linux}. Las p�ginas de Samba est�n en  http://samba.anu.edu.au/samba/ 
y el {\it SMB HOWTO} puede ser de ayuda tambi�n.


\index{Microsoft|)}
\index{redes!Microsoft|)}
