% linux Installation and Getting Started    -*- TeX -*-
% mail.tex
% Copyright (c) 1992, 1993 by Matt Welsh <mdw@sunsite.unc.edu>
%
% This file is freely redistributable, but you must preserve this copyright 
% notice on all copies, and it must be distributed only as part of "{\linux} 
% Installation and Getting Started". This file's use is covered by the 
% copyright for the entire document, in the file "copyright.tex".
%
% Copyright (c) 1998 by Specialized Systems Consultants Inc. 
% <ligs@ssc.com>
% Revision 1 7 de julio 2002 por Francisco Javier Fernandez <serrador@arrakis.es>
% Revision 2 12 de julio 2002 por francisco Javier Fern�ndez <serrador@arrakis.es>
% Revision 3 13 de julio 2002 por Francisco Javier Fern�ndez <serrador@arrakis.es>
%Gold
\section{Correo electr�nico}
\label{sec-mail}
\markboth{Caracter�sticas Avanzadas}{Correo Electr�nico}
\index{e-mail|(}
\index{e-mail!transporte!definici�n}
\index{e-mail!mailer!definici�n}
\index{transporte!para e-mail}
\index{mailer!for e-mail}
\index{elm@{\tt elm}}
\index{mailx@{\tt mailx}}
\index{correo electr�nico|(}
\index{correo electr�nico!transporte!definici�n}

Como casi todos los UNIX, {\linux} dispone de paquetes de software para tener
correo electr�nico. �ste puede ser tanto local (entre usuarios de su
sistema) como remoto (entre usuarios de otras m�quinas o redes mediante una 
red TCP/IP o UUCP). El software de {\it correo electr�nico} consta normalmente de dos
partes: un agente de usuario o {\em mailer\/} y un {\em programa de 
transporte}. El agente de usuario es el 
software que el usuario utiliza para crear mensajes, leerlos, etc. 
Podemos destacar aqu� los programas {\tt elm} y {\tt mailx}. El 
programa de transporte es quien se ocupa de entregar correo 
tanto remoto como local, conociendo protocolos de comunicaciones y 
dem�s. El usuario nunca interact�a directamente con este programa, 
sino que lo hace a trav�s del agente de usuario. Sin embargo, como 
administrador del sistema es importante que comprenda c�mo funciona el programa de 
transporte, con el fin de configurarlo seg�n sus necesidades.


\index{Smail@{\tt Smail}}
\index{sendmail@{\tt sendmail}}

En {\linux}, el m�s conocido de los programas de transporte es {\tt sendmail}.
Es capaz de enviar tanto correo local como remoto
v�a UUCP o TCP/IP. Una alternativa a {\tt sendmail} es {\tt Smail}, que es menos complicado de configurar.

En el documento {\it Linux Mail HOWTO} se expone m�s informaci�n sobre
el software disponible para correo y c�mo configurarlo. Si pretende tener
correo remoto, necesitar� entender los conceptos de TCP/IP o UUCP (seg�n
la red utilizada) (vea las secciones~\ref{sec-tcpip} y \ref{sec-uucp}). Los
documentos de UUCP y TCP/IP indicados en el ap�ndice~\ref{app-info} 
tambi�n le ayudar�n.

Casi todo el software de correo para {\linux} puede obtenerse mediante FTP
an�nimo de {\tt sunsite.unc.edu} en el directorio {\tt 
/pub/linux/system/Mail}. 
\index{e-mail|)}
