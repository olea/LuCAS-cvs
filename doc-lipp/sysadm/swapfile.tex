% Linux Installation and Getting Started    -*- TeX -*-
% swapfile.tex
% Copyright (c) 1992, 1993 by Matt Welsh <mdw@sunsite.unc.edu>
%
% This file is freely redistributable, but you must preserve this copyright 
% notice on all copies, and it must be distributed only as part of "Linux 
% Installation and Getting Started". This file's use is covered by
% the copyright for the entire document, in the file "copyright.tex".
%
% Copyright (c) 1998 by Specialized Systems Consultants Inc. 
% <ligs@ssc.com>
%Traducido por Sebasti�n Gurin, Cancerbero <anon@adinet.com.uy>, el 09/01/01 
%Revisi�n 1 7/7/2002 por Francisco Javier Fernandez <serrador@arrakis.es>

\section{Usando un fichero de intercambio}
\label{sec-swap-file}
\markboth{Administraci�n del Sistema}{Usando un fichero de intercambio}

En lugar de reservar una partici�n separada para el espacio de
intercambio, se puede usar un fichero de intercambio. Sin embargo,
ser� necesario instalar {\linux} y conseguir que todo funcione antes de crearlo.

Teniendo {\linux} ya instalado, se puede usar las siguientes
instrucciones para crear el fichero de intercambio. La orden de abajo,
crea un fichero de intercambio de 8208 bloques de tama�o, (aproximadamente 8 Mb).


\begin{tscreen}
\# dd if=/dev/zero of=/swap bs=1024 count=8208 
\end{tscreen}

Esta orden crea el fichero de intercambio, {\tt /swap}. El par�metro
``{\tt count=}'', es el tama�o del fichero de intercambio en bloques.
\begin{tscreen}
\# mkswap /swap 8208
\end{tscreen}
Esta orden inicia el fichero de intercambio. Una vez m�s, ser�
necesario reemplazar el nombre y el tama�o del fichero de intercambio
con los valores apropiados. 

\begin{tscreen}
\# sync \\
\# swapon /swap
\end{tscreen}
Ahora el sistema est� realizando el intercambio en el fichero {\tt
  /swap}. La instrucci�n {\tt sync} garantiza que el fichero haya sido escrito en el disco. 

Una desventaja importante de usar un fichero de intercambio, es que todo acceso al fichero, es hecho a trav�s del sistema de ficheros. Esto significa que
los bloques que constituyen el fichero de intercambio pueden no ser contiguos. Como consecuencia, el rendimiento puede no ser tan bueno como el de una partici�n de 
intercambio, en donde los bloques son siempre contiguos y las demandas de entrada/salida son realizadas directamente al dispositivo. 
Otra desventaja de los ficheros de intercambio largos es el gran peligro de que el sistema de ficheros se corrompa si algo sale mal. 
Conservar los ficheros normales, separados de las particiones de intercambio previene que esto pase. 
Los ficheros de intercambio pueden ser �tiles si, por ejemplo, se
necesita usar, temporalmente, m�s espacio de intercambio. Si se est�
compilando un programa extenso y se quisiera  acelerar las cosas un
tanto, se puede crear un fichero de intercambio temporal y usarlo
adem�s del espacio de intercambio regular. 
Para eliminar un fichero de intercambio, usa primero {\tt swapoff}, como en
\begin{tscreen}
\# swapoff /swap
\end{tscreen}
Luego, el fichero puede ser eliminado
\begin{tscreen}
\# rm /swap
\end{tscreen}


Cada fichero o partici�n de intercambio puede tener un tama�o m�ximo
de 128 megabytes, pero se puede usar hasta 8 ficheros o particiones de
intercambio en el sistema. 

%Traducido por Sebasti�n Gurin (Cancerbero), el 12/01/01