% Linux Installation and Getting Started    -*- TeX -*-
% hats.tex
% Copyright (c) 1993 by Matt Welsh and Lars Wirzenius
%
% This file is freely redistributable, but you must preserve this copyright 
% notice on all copies, and it must be distributed only as part of "Linux 
% Installation and Getting Started". This file's use is covered by
% the copyright for the entire document, in the file "copyright.tex".
%
% Copyright (c) 1998 by Specialized Systems Consultants Inc. 
% <ligs@ssc.com>

% Traducci�n realizada por Alberto Molina. Comentarios a:
%            alberto@nucle.us.es
% Revisi�n 1 7/7/2002 <serrador@arrakis.es>
% Revisi�n 2 30/7/2002 <serrador@arrakis.es>

\section{La cuenta {\tt root}} 
\markboth{Administraci�n del Sistema}{La cuenta {\tt root}}

GNU/Linux diferencia entre varios usuarios. Lo que puede hacer cada uno
con respecto a los dem�s est� regulado. Los permisos de ficheros est�n
regulados de manera que los usuarios normales no puedan borrar o
modificar ficheros de directorios como {\tt /bin} y {\tt
  /usr/bin}. Muchos usuarios protegen sus ficheros con los permisos
apropiados, para que otros usuarios no tengan acceso a ellos (uno no
querr�a que nadie leyese sus cartas de amor). Cada usuario tiene una
{\bf cuenta} que incluye su nombre de usuario y su directorio
``home''. Adem�s, hay cuentas especiales definidas por el sistema que
tienen privilegios especiales. La m�s importante de todas es la {\bf
  cuenta root}, que usa el administrador del sistema. Por convenio, el
administrador del sistema es el usuario {\tt root}.

No hay restricciones para {\tt root}. �l o ella puede leer, modificar
o borrar cualquier fichero del sistema, cambiar los permisos y la
propiedad de los ficheros y ejecutar programas especiales como los que
particionan un disco duro o crean sistemas de ficheros. La idea
fundamental es que es una persona que vigila los registros del sistema
y que realiza tareas que no pueden ejecutar los usuarios
normales. Puesto que {\tt root} puede hacer cualquier cosa, es f�cil
cometer errores con consecuencias catastr�ficas.

Si un usuario normal tratase inadvertidamente de borrar todos los
ficheros de {\tt /etc}, el sistema no se lo permitir�a. Sin embargo,
si lo intentase {\tt root} el sistema no se lo impedir�a. Es muy f�cil
destrozar un sistema GNU/Linux usando {\tt root}. La mejor manera de
prevenir accidentes es:

\begin{itemize}
\item Pens�rselo dos veces antes de pulsar \key{Enter} para una orden no
  reversible. Si se va a borrar un directorio, revisar la orden completa
  para estar seguro de que es correcta.

\item Usar un prompt diferente para la cuenta {\tt root}. En los
  ficheros {\tt .bashrc} o {\tt .login} de la cuenta {\tt root}
  deber�a especificarse el prompt con algo diferente al del resto de
  usuarios. Mucha gente reserva el car�cter ``{\tt \#}'' para el
  prompt de {\tt root} y usa ``{\tt \$}'' para el del resto de usuarios.

\item Entrar como {\tt root} s�lo cuando sea estrictamente
  necesario. Cuando se hayan finalizado las tareas como administrador del
  sistema, salir de dicha cuenta. Cuanto menos se utilice dicha cuenta,
  menos da�o podr� provocarle al sistema. 
\end{itemize}

Uno se puede imaginar la cuenta {\tt root} como un sombrero m�gico que le da
inmensos poderes y con el que se puede, simplemente moviendo las manos,
destruir ciudades enteras. Es una buena imagen para ser cuidadoso y
saber lo que se tiene entre manos. Puesto que es tan f�cil destruir
cosas con sus manos, no es una buena idea ponerse el sombrero cuando
no hace falta, a pesar de la magn�fica sensaci�n.

Comentaremos con m�s detalle las responsabilidades del administrador
del sistema a partir de la p�gina~\pageref{sec-manage-users}.




