% Linux Installation and Getting Started    -*- TeX -*-
% chap-sysadm.tex
% Copyright (c) 1993 by Matt Welsh and Lars Wirzenius
%
% This file is freely redistributable, but you must preserve this copyright 
% notice on all copies, and it must be distributed only as part of "Linux 
% Installation and Getting Started". This file's use is covered by
% the copyright for the entire document, in the file "copyright.tex".
%
% Copyright (c) 1998 by Specialized Systems Consultants Inc. 
% <ligs@ssc.com>

% Traducci�n realizada por Alberto Molina. Comentarios a:
%            alberto@nucle.us.es
% Revisi�n 1 por Fco. javier Fernandez <serrador@arrakis.es>
%Gold

\def\fn#1{{\tt #1}}
\def\cmd#1{{#1}}

\chapter{Administraci�n del Sistema}\label{chap-sysadm}
\markboth{Administraci�n del Sistema}{}
\label{chap-sysadm-num}

Este cap�tulo cubre las cosas m�s importantes que se necesitan saber
acerca de la administraci�n del sistema para comenzar a usarlo sin problemas. Para que el cap�tulo tenga un tama�o razonable,
cubre s�lo lo b�sico y omite muchos detalles importantes. El libro
{\em Linux System Administrator's Guide}, de Lars Wirzenius (ver
Ap�ndice~\ref{app-info}) proporciona m�s detalles sobre la
administraci�n del sistema. Adem�s ayudar� a entender mejor c�mo
trabajan y cuelgan las cosas entre s�.

% Linux Installation and Getting Started    -*- TeX -*-
% hats.tex
% Copyright (c) 1993 by Matt Welsh and Lars Wirzenius
%
% This file is freely redistributable, but you must preserve this copyright 
% notice on all copies, and it must be distributed only as part of "Linux 
% Installation and Getting Started". This file's use is covered by
% the copyright for the entire document, in the file "copyright.tex".
%
% Copyright (c) 1998 by Specialized Systems Consultants Inc. 
% <ligs@ssc.com>

% Traducci�n realizada por Alberto Molina. Comentarios a:
%            alberto@nucle.us.es
% Revisi�n 1 7/7/2002 <serrador@arrakis.es>
% Revisi�n 2 30/7/2002 <serrador@arrakis.es>

\section{La cuenta {\tt root}} 
\markboth{Administraci�n del Sistema}{La cuenta {\tt root}}

GNU/Linux diferencia entre varios usuarios. Lo que puede hacer cada uno
con respecto a los dem�s est� regulado. Los permisos de ficheros est�n
regulados de manera que los usuarios normales no puedan borrar o
modificar ficheros de directorios como {\tt /bin} y {\tt
  /usr/bin}. Muchos usuarios protegen sus ficheros con los permisos
apropiados, para que otros usuarios no tengan acceso a ellos (uno no
querr�a que nadie leyese sus cartas de amor). Cada usuario tiene una
{\bf cuenta} que incluye su nombre de usuario y su directorio
``home''. Adem�s, hay cuentas especiales definidas por el sistema que
tienen privilegios especiales. La m�s importante de todas es la {\bf
  cuenta root}, que usa el administrador del sistema. Por convenio, el
administrador del sistema es el usuario {\tt root}.

No hay restricciones para {\tt root}. �l o ella puede leer, modificar
o borrar cualquier fichero del sistema, cambiar los permisos y la
propiedad de los ficheros y ejecutar programas especiales como los que
particionan un disco duro o crean sistemas de ficheros. La idea
fundamental es que es una persona que vigila los registros del sistema
y que realiza tareas que no pueden ejecutar los usuarios
normales. Puesto que {\tt root} puede hacer cualquier cosa, es f�cil
cometer errores con consecuencias catastr�ficas.

Si un usuario normal tratase inadvertidamente de borrar todos los
ficheros de {\tt /etc}, el sistema no se lo permitir�a. Sin embargo,
si lo intentase {\tt root} el sistema no se lo impedir�a. Es muy f�cil
destrozar un sistema GNU/Linux usando {\tt root}. La mejor manera de
prevenir accidentes es:

\begin{itemize}
\item Pens�rselo dos veces antes de pulsar \key{Enter} para una orden no
  reversible. Si se va a borrar un directorio, revisar la orden completa
  para estar seguro de que es correcta.

\item Usar un prompt diferente para la cuenta {\tt root}. En los
  ficheros {\tt .bashrc} o {\tt .login} de la cuenta {\tt root}
  deber�a especificarse el prompt con algo diferente al del resto de
  usuarios. Mucha gente reserva el car�cter ``{\tt \#}'' para el
  prompt de {\tt root} y usa ``{\tt \$}'' para el del resto de usuarios.

\item Entrar como {\tt root} s�lo cuando sea estrictamente
  necesario. Cuando se hayan finalizado las tareas como administrador del
  sistema, salir de dicha cuenta. Cuanto menos se utilice dicha cuenta,
  menos da�o podr� provocarle al sistema. 
\end{itemize}

Uno se puede imaginar la cuenta {\tt root} como un sombrero m�gico que le da
inmensos poderes y con el que se puede, simplemente moviendo las manos,
destruir ciudades enteras. Es una buena imagen para ser cuidadoso y
saber lo que se tiene entre manos. Puesto que es tan f�cil destruir
cosas con sus manos, no es una buena idea ponerse el sombrero cuando
no hace falta, a pesar de la magn�fica sensaci�n.

Comentaremos con m�s detalle las responsabilidades del administrador
del sistema a partir de la p�gina~\pageref{sec-manage-users}.




  %% rak
% Linux Installation and Getting Started    -*- TeX -*-
% chap-sysadm.tex
% Copyright (c) 1993 by Matt Welsh and Lars Wirzenius
%
% This file is freely redistributable, but you must preserve this copyright 
% notice on all copies, and it must be distributed only as part of "Linux 
% Installation and Getting Started". This file's use is covered by
% the copyright for the entire document, in the file "copyright.tex".
%
% Copyright (c) 1998 by Specialized Systems Consultants Inc. 
% <ligs@ssc.com>
%
% Este fichero es de distribuci�n libre, pero debe mantenerse esta 
% informaci�n de Copyright en todas las copias, y debe distribuirse solo como
% parte de "Instalaci�n y Primeros Pasos en Linux". El uso de este fichero esta
% cubierto por el Copyright del documento completo, en el fichero "copyright.tex"
% Copyright (c) 1995 por Gerardo Izquierdo para la versi�n al Castellano

%
% Versi�n para lipp 2.0 por Alberto Molina. Comentarios a:
%            alberto@nucle.us.es 
%

\section{Iniciando el Sistema}
\markboth{Administraci�n del Sistema}{Iniciando el Sistema}
\label{sec-bootfloppy} %% legacy of deleted sub-hed -- rk

\index{arrancando|(}
\index{administraci�n del sistema!arrancando Linux|(}
Hay varias maneras de arrancar el sistema, bien sea desde disquete o 
bien desde el disco duro.

\subsection{Utilizando un disquete de arranque}

\index{arrancando Linux!con un disquete de arranque}
Mucha gente arranca Linux utilizando un disquete de inicio que contiene 
una copia del n�cleo de Linux. Este n�cleo tiene la partici�n ra�z de Linux 
codificada en �l, para que sepa d�nde buscar en el disco duro el 
sistema de ficheros ra�z. (La orden {\tt rdev} puede ser utilizada 
para poner la partici�n ra�z en la imagen del n�cleo; ver m�s 
adelante.) Por ejemplo, �ste es el tipo de disquete creado por Slackware 
durante la instalaci�n.

\index{/Image@{\tt /Image}}
\index{/etc/Image@{\tt /etc/Image}}
\index{/vmlinux@{\tt /vmlinux}}
\index{disquete de arranque!creando}
\index{n�cleo!nombre de fichero de la imagen del}
Para crear un disquete de arranque propio, hay que localizar en primer lugar la 
imagen del n�cleo en su disco duro. Debe estar en el fichero {\tt /Image} o
{\tt /etc/Image}. Algunas instalaciones utilizan el fichero {\tt /vmlinux}
para el n�cleo.

\index{n�cleo!imagen comprimida del}
\index{/zImage@{\tt /zImage}}
\index{/etc/zImage@{\tt /etc/zImage}}
\index{/vmlinuz@{\tt /vmlinuz}}
En su lugar, puede que haya un n�cleo comprimido. Un n�cleo comprimido se
descomprime a s� mismo en memoria en tiempo de arranque, y utiliza mucho 
menos 
espacio en el disco duro. Si se tiene un n�cleo comprimido, puede encontrarse
en el fichero {\tt /zImage} o {\tt /etc/zImage}. Algunas instalaciones utilizan el fichero
{\tt /vmlinuz} para el n�cleo comprimido.

\index{ra�z, dispositivo!poniendo el nombre de con} 
\index{rdev@poniendo el nombre de con {\tt rdev}}
\index{rdev@{\tt rdev}}
Una vez que se sabe d�nde est� el n�cleo, hay que poner el nombre de la 
partici�n ra�z de un dispositivo ra�z en la imagen del 
n�cleo, utilizando la orden {\tt rdev}. El formato de este orden es
\begin{tscreen}
rdev \cparam{nombre-de-n�cleo} \cparam{dispositivo-ra�z}
\end{tscreen}
donde \cparam{nombre-del-n�cleo} es el nombre de la imagen del n�cleo, y 
\cparam{dispositivo-ra�z} es el nombre de la partici�n ra�z de 
Linux. Por ejemplo, para hacer que el dispositivo ra�z en el n�cleo {\tt /etc/Image}
sea {\tt /dev/hda2}, utilice la orden
\begin{tscreen}
\# {\em rdev /etc/Image /dev/hda2}
\end{tscreen}

Con {\tt rdev} tambi�n se pueden poner otras opciones en el n�cleo, como puede ser el
modo SVGA predeterminado que se utilizar� en tiempo de arranque. Utilizando
{\tt rdev -h} se obtiene un mensaje de ayuda.

Una vez puesto el dispositivo ra�z, tan s�lo hay que copiar la 
imagen del n�cleo al disquete. Siempre que se copia datos a un 
disquete, es una buena idea formatear previamente el disquete, usando
el {\tt FORMAT.COM} en MS-DOS o el programa {\tt fdformat} de Linux. 
Esto establece la informaci�n de pista y sector en el disquete con la 
que puede detectarse como de alta o baja densidad.

El formateo de disquetes y las controladoras de los mismos se discuten
m�s tarde en la p�gina~\pageref{sec-backfloppy}.

Para copiar el n�cleo desde el fichero {\tt /etc/Image} al disquete
en {\tt /dev/fd0}, se puede utilizar la orden:
\begin{tscreen}
\# {\em cp /etc/Image /dev/fd0}
\end{tscreen}

Este disquete debe arrancar ahora Linux.

\subsection{Utilizando LILO}\label{sec-lilo}

\index{LILO|(}
\index{arrancando!con LILO|(}
Otro m�todo de arranque es utilizar LILO, un programa que reside en el 
sector de arranque del disco duro. Este programa se ejecuta cuando el 
sistema se inicia desde el disco duro, y puede arrancar autom�ticamente
Linux desde una imagen de n�cleo almacenada en el propio disco duro.

\index{LILO!como cargador de arranque}
\index{sistemas operativos!arrancando no-Linux}
\index{arrancando sistemas no-Linux}
LILO puede utilizarse tambi�n como una primera etapa de carga de varios
sistemas operativos, permitiendo seleccionar en tiempo de arranque qu� 
sistema operativo (como Linux o MS-DOS) arrancar. Cuando se arranca
utilizando LILO, se inicia el sistema operativo predeterminado, a menos que
pulse \key{shift} durante la secuencia de arranque o se especifique el
fichero {\tt /etc/lilo.conf}.
En cualquiera de estos casos, se presentar� un indicador de
arranque, donde debe teclear el nombre del sistema operativo a arrancar
(como puede ser ``{\tt linux}'' o ``{\tt msdos}''). Si se pulsa la tecla
\key{tab} en el indicador de arranque, se le presentar� una lista de los 
sistemas operativos disponibles.

\index{LILO!instalaci�n}
La forma m�s simple de instalar LILO es editar el fichero de 
configuraci�n, {\tt /etc/lilo.conf},
y ejecutar la instrucci�n
\begin{tscreen}
\# {\em /sbin/lilo}
\end{tscreen}

El fichero de configuraci�n de LILO contiene un p�rrafo para cada 
sistema operativo que se pueda querer arrancar. La mejor forma de mostrarlo
es con un ejemplo de un fichero de configuraci�n LILO. El ejemplo siguiente 
es para un sistema que tiene una partici�n ra�z Linux en {\tt /dev/hda1} y
una partici�n MS-DOS en {\tt /dev/hda2}.

\begin{tscreen}\begin{verbatim}
# Le indicamos a LILO que modifique el registro de arranque de
# /dev/hda (el primer disco duro no-SCSI). Si se quiere arrancar desde
# una unidad distinta de /dev/hda, se debe cambiar la siguiente l�nea
boot = /dev/hda

# Modo de v�deo
vga = normal

# Tiempo de respuesta en milisegundos. Tiempo del que se dispone para
# pulsar ``SHIFT''.
delay = 60

# Nombre del cargador de arranque. No hay raz�n para cambiarlo, a menos
# que se este haciendo una modificaci�n seria del LILO
install = /boot/boot.b

# Esto fuerza a LILO a solicitar el Sistema Operativo con el que se va
# a arrancar. Si se pulsa 'TAB' se presentan las distintas opciones,
# de acuerdo con los 'label=' siguientes.
indicador de �rdenes

# Dejemos a LILO efectuar alguna optimizaci�n.
compact

# Parrafo para la partici�n ra�z de Linux en /dev/hda1.
image = /etc/Image   # Ubicaci�n del n�cleo
   label = linux     # Nombre del SO (para el men� de arranque de LILO)
   root = /dev/hda1  # Ubicaci�n de la partici�n ra�z
   vga = ask         # Indicar al n�cleo que pregunte por modos SVGA
                     #   en tiempo de arranque

# P�rrafo para la partici�n MSDOS en /dev/hda2.
other = /dev/hda2    # Ubicaci�n de la partici�n
   table = /dev/hda  # Ubicaci�n de la tabla de partici�n para /dev/hda2 
   label = msdos     # Nombre del SO (para el men� de arranque)
\end{verbatim}\end{tscreen}

\index{LILO!seleccionando el sistema operativo predeterminado para}
\index{sistemas operativos!arrancando no-Linux}
El primer p�rrafo de sistema operativo en el men� del fichero de 
configuraci�n ser� el sistema operativo predeterminado que arrancar� LILO.
Se puede seleccionar otro sistema operativo en el indicador de arranque de
LILO, tal y como se indic� anteriormente.

El instalador de Microsoft Windows '95 sobreescribe el sector de
arranque. Si va a instalar Windows '95 en su sistema despu�s de
instalar LILO, debe asegurarse de crear un disquete de inicio antes,
ver~\ref{sec-bootfloppy}). Con el disquete de inicio, puede iniciar
Linux y reinstalar LILO tras la instalaci�n Windows '95. Simplemente
escribiendo como ``root'' la orden {\tt /sbin/lilo}. 
Las particiones con Windows '95 se pueden configurar de forma
totalmente equivalente a la vista anteriormente con particiones de MS-DOS.

Las FAQ (Preguntas frecuentemente formuladas) (ver Ap�ndice~\ref{app-info})
dan m�s informaci�n sobre LILO, incluyendo c�mo utilizar LILO con el 
``OS/2's~Boot~Manager''.
\index{arrancando!con LILO|)}
\index{LILO|)}
\index{arrancando|)}
\index{administraci�n del sistema!arrancando Linux|)}


% Linux Installation and Getting Started    -*- TeX -*-
% shutdown.tex
% Copyright (c) 1993 by Matt Welsh and Lars Wirzenius
%
% This file is freely redistributable, but you must preserve this copyright 
% notice on all copies, and it must be distributed only as part of "Linux 
% Installation and Getting Started". This file's use is covered by
% the copyright for the entire document, in the file "copyright.tex".
%
% Copyright (c) 1998 by Specialized Systems Consultants Inc. 
% <ligs@ssc.com>
%
% Este fichero es de distribuci�n libre, pero debe mantenerse esta 
% informaci�n de Copyright en todas las copias, y debe distribuirse solo como
% parte de "Instalaci�n y Primeros Pasos en Linux". El uso de este fichero esta
% cubierto por el Copyright del documento completo, en el fichero "copyright.tex"
% Copyright (c) 1995 por Gerardo Izquierdo para la versi�n al Castellano
% $Log: shutdown.tex,v $
% Revision 1.9  2003/07/19 20:28:24  pakojavi2000
% Arreglando un peque�o destrozo de macros
%
% Revision 1.8  2003/07/19 06:20:33  joseluis.ranz
% Correcciones varias.
%
% Revision 1.7  2002/07/30 16:23:05  pakojavi2000
% Beta 2.2 Formatos de p�rrafo
%
% Revision 1.6  2002/07/20 22:24:29  pakojavi2000
% Beta2
%
% Revision 1.5  2002/07/20 17:41:16  pakojavi2000
% beta2
%
% Revision 1.4  2002/07/13 12:50:07  pakojavi2000
% Gold
%
% Revision 1.3  2001/04/18 16:29:10  amolina
% Segunda revisi�n de los ficheros
%
% Revision 1.2  2000/12/20 16:51:28  amolina
%
% Primera versi�n traducida de sysadm/shutdown.tex
%
% Revision 0.5.0.1  1996/02/10 23:45:13  rcamus
% Primera beta publica
%

% Versi�n para lipp 2.0 por Alberto Molina
%         comentarios a alberto@nucle.us.es
%
%Revisi�n 1 13 de julio 2002 por JFS <serrador@arrakis.es
%Gold
\section{Parada del sistema}
\markboth{Administraci�n del Sistema}{Parada del Sistema}
\label{sec-sysadm-shutdown}

\index{administraci�n del sistema!cierre del sistema|(}
\index{cierre del sistema|(}
Cerrar un sistema {\linux} tiene algo de truco. Hay que recordar que nunca se debe
cortar la corriente o pulsar el bot�n de apagado mientras el sistema
est� ejecut�ndose. El n�cleo sigue la pista de la entrada/salida a disco
en ``buffers'' de memoria. Si se reinicializa el sistema sin darle al n�cleo
la oportunidad de escribir sus ``buffers'' a disco, puede corromper sus sistemas
de ficheros.

En tiempo de cierre se toman tambi�n otras precauciones. Todos los procesos
reciben una se�al que les permite morir airosamente (escribiendo y cerrando 
todos los ficheros y ese tipo de cosas). Los sistemas de ficheros se 
desmontan por seguridad. Si se desea, el sistema tambi�n puede alertar a los
usuarios de que se est� cerrando y darles la posibilidad de desconectarse.

\index{orden shutdown@comando {\tt shutdown}}
La forma m�s simple de cerrar el sistema es con la orden {\tt 
shutdown}. El formato es
\begin{tscreen}
shutdown \cparam{tiempo} \cparam{mensaje-de-aviso}
\end{tscreen}
El argumento \cparam{tiempo} es el momento de cierre del sistema (en el
formato {\em hh:mm:ss}), y \cparam{mensaje-de-aviso} es un mensaje mostrado
en todos los terminales de usuario antes de cerrar. Alternativamente, se
puede especificar el par�metro \cparam{tiempo} como ``{\tt now}'', para
cerrar inmediatamente. Se le puede suministrar la opci�n {\tt -r} a 
{\tt shutdown} para reinicializar el sistema tras el cierre.

Por ejemplo, para cerrar el sistema a las 8:00pm, se puede utilizar la
siguiente orden
\begin{tscreen}
\# {\em shutdown -r 20:00}
\end{tscreen}

\index{halt@{\tt halt}}
La orden {\tt halt} puede utilizarse para forzar un cierre inmediato, sin
ning�n mensaje de aviso ni periodo de gracia. {\tt halt} se utiliza si se
es el �nico usuario del sistema y se quiere cerrar el sistema y apagarlo.

\blackdiamond No apagar o reinicializar el sistema hasta que se vea el mensaje:
\begin{tscreen}
The system is halted
\end{tscreen}
Es muy importante que cierre el sistema ``limpiamente'' utilizando la
orden {\tt shutdown} o el {\tt halt}. En algunos sistemas, se reconocer� 
el pulsar \key{ctrl-alt-del}, que causar� un {\tt shutdown}; en otros 
sistemas, sin embargo, el utilizar el ``Apret�n de Cuello de Vulcano''
reinicializar� el sistema inmediatamente y puede causar un desastre.

\index{administraci�n del sistema!cierre del sistema|)}
\index{cierre del sistema|)}



% inittabl.tex. this is new to the SSC edition, 12/8/97 -- rak
%
% Copyright (c) 1998 by Specialized Systems Consultants Inc. 
% <ligs@ssc.com> 

% Traducci�n realizada por Alberto Molina. Comentarios a:
%            alberto@nucle.us.es
% 


\subsection{El fichero {\tt /etc/inittab}} \label{sec-inittab}
\markboth{Administraci�n del Sistema}{El fichero {\tt /etc/inittab}}

Despu�s de que {\linux} arranque y el n�cleo monte el sistema de
ficheros de root, el primer programa que ejecuta el sistema es {\tt
  init}. Este programa es el encargado de lanzar los guiones de
inicializaci�n del sistema y de modificar el sistema operativo de su
estado inicial de arranque al estado est�ndar multiusuario. Tambi�n
define los int�rpretes de �rdenes {\tt login:} de todos los dispositivos tty del
sistema y especifica otras caracter�sticas del arranque y apagado.

Tras el arranque, {\tt init} permanece latente en segundo plano,
``monitoreando'' y si fuera necesario alterando la ejecuci�n del
sistema. Hay muchos detalles que deben comentarse del programa {\tt
  init}. Todas las tareas que realiza se definen en el fichero {\tt
  /etc/inittab}. Un ejemplo de dicho fichero se muestra a continuaci�n.

\blackdiamond Modificar el fichero {\tt /etc/inittab} de forma
incorrecta, puede impedirle registrarse en el sistema. Por ello, cuando se
modifique dicho fichero, hay que guardar una copia del fichero
original, adem�s de tener a mano el disquete de inicio, para el caso
en que se cometiera alg�n error.

\begin{tscreen}\begin{verbatim}
#
# inittab       Este fichero describe como el proceso INIT debe 
#               ajustar el sistema en ciertos niveles de ejecuci�n.
#
# Version:      @(#)inittab             2.04    17/05/93        MvS
#                                       2.10    02/10/95        PV
#
# Author:       Miquel van Smoorenburg, <miquels@drinkel.nl.mugnet.org>
# Modified by:  Patrick J. Volkerding, <volkerdi@ftp.cdrom.com>
# Minor modifications by:
#               Robert Kiesling, <kiesling@terracom.net>
#
# Nivel de ejecuci�n asumido.
id:3:initdefault:

# Iniciaci�n del sistema (se ejecuta al arrancar el sistema).
si:S:sysinit:/etc/rc.d/rc.S

# Script para ejecutarse cuando el sistema vaya a un usuario 
# (nivel de ejecuci�n 1).  
su:1S:wait:/etc/rc.d/rc.K

# Script para ejecutarse cuando el sistema vaya a multiusuario.
rc:23456:wait:/etc/rc.d/rc.M

# Qu� hacer cuando se pulse Ctrl-Alt-Del
ca::ctrlaltdel:/sbin/shutdown -t5 -rfn now

# El nivel de ejecuci�n 0 para el sistema.
l0:0:wait:/etc/rc.d/rc.0

# El nivel de ejecuci�n 6 reinicia el sistema.
l6:6:wait:/etc/rc.d/rc.6

# Qu� hacer cuando se va el suministro el�ctrico (bajar al nivel de
# ejecuci�n de un usuario).
pf::powerfail:/sbin/shutdown -f +5 "EL SUMINISTRO EL�CTRICO SE EST� CORTANDO"

# Si el suministro vuelve antes de bajar, cancelar el proceso.
pg:0123456:powerokwait:/sbin/shutdown -c "El SUMINISTRO EL�CTRICO EST�
VOLVIENDO"

# Si vuelve el suministro cuando se est� en modo de un usuario, volver
# al modo multiusuario.
ps:S:powerokwait:/sbin/init 5

# Los ``gettys'' en el modo multiusuario y las l�neas serie.
#
# NOTA NOTA NOTA �ajuste esto a su ``getty'' o no ser� capaz de ingresar!
#
# Nota: Debe especificar la velocidad de l�nea para ``agetty''.
# para ``getty_ps'' se usa una l�nea, se especifica la velocidad de
# l�nea y tambi�n se utiliza ``gettydefs''
c1:1235:respawn:/sbin/agetty 38400 tty1 linux
c2:1235:respawn:/sbin/agetty 38400 tty2 linux
c3:1235:respawn:/sbin/agetty 38400 tty3 linux
c4:1235:respawn:/sbin/agetty 38400 tty4 linux
c5:1235:respawn:/sbin/agetty 38400 tty5 linux
c6:12345:respawn:/sbin/agetty 38400 tty6 linux

# L�neas serie
# s1:12345:respawn:/sbin/agetty -L 9600 ttyS0 vt100
s2:12345:respawn:/sbin/agetty -L 9600 ttyS1 vt100

# L�neas de marcado telef�nico
d1:12345:respawn:/sbin/agetty -mt60 38400,19200,9600,2400,1200 ttyS0 vt100
#d2:12345:respawn:/sbin/agetty -mt60 38400,19200,9600,2400,1200 ttyS1 vt100

# El nivel de ejecuci�n 4 deber�a usarse para un sistema con X-window
# �nicamente, hasta que nos dimos cuenta de
# que lanzaba init en un bucle que manten�a la carga al menos a 1 todo  
# el tiempo. As� que, ahora hay un getty abierto en tty6. Esperemos que
# nadie se de cuenta. ;^)
# Quiz� no sea malo tener una consola de texto por ah�, en caso de que 
# le ocurriera algo a X.
x1:4:wait:/etc/rc.d/rc.4

# Fin de /etc/inittab

\end{verbatim}\end{tscreen}

Al iniciar, este {\tt /etc/inittab} lanza seis consolas virtuales, un
prompt de ingreso para el m�dem en {\tt /dev/ttys0} y un prompt de
ingreso para una terminal de caracteres conectada a trav�s de la l�nea
serie RS-232 a {\tt /dev/ttyS1}.

Brevemente podr�amos decir que el programa {\tt init} pasa a trav�s de
una serie de {\bf niveles de ejecuci�n}, que corresponden a varios
estados del sistema. Al nivel de ejecuci�n 1 se entra inmediatamente
despu�s de iniciar el sistema, los niveles de ejecuci�n 2 y 3 son los
modos de operaci�n del sistema normal y multiusuario respectivamente,
el nivel de ejecuci�n 4 lanza el sistema X Window a trav�s del X
display manager {\tt xdm} y el nivel de ejecuci�n 6 reinicia el
sistema. Los niveles de ejecuci�n asociados a cada orden, son el
segundo t�rmino de cada l�nea del fichero {\tt /etc/inittab}.

Por ejemplo, la l�nea:
\begin{tscreen}
s2:12345:respawn:/sbin/agetty -L 9600 ttyS1 vt100
\end{tscreen}
mantendr� un prompt de ingreso en una terminal serie para los niveles
de ejecuci�n 1--5. El ``{\tt s2}'' antes de los primeros dos puntos es
un identificador simb�lico que usa internamente {\tt init}. {\tt
  respawn} es una clave de {\tt init} que se usa a veces junto con las
terminales serie. Si tras un cierto per�odo de tiempo, el programa
{\tt agetty}, que genera los prompt de ingreso en las terminales, no
recibe se�al alguna en la terminal, termina su ejecuci�n. ``{\tt
respawn}'' hace que {\tt init} vuelva a ejecutar {\tt agetty},
asegurando que haya siempre un prompt de ingreso en la terminal,
independientemente de que haya alg�n otro ingreso. El resto de
par�metros se pasan directamente a {\tt agetty} y le especifican c�mo
debe generar la shell de ingreso, la capacidad de transferencia de
datos de la l�nea, el dispositivo serie y el tipo de terminal, como se
define en {\tt /etc/termcap} o {\tt /etc/terminfo}.

%The {\tt /sbin/agetty} program handles many details related to
%terminal I/O on the system. There are several different versions that
%are commonly in use on Linux systems. They include {\tt mgetty}, {\tt
%psgetty}, or simply, {\tt getty}.
El programa {\tt /sbin/agetty} maneja muchos detalles acerca de la E/S por
terminal en el sistema. Hay varias versiones diferentes que se unen regularmente en sistemas GNU/Linux.
Se incluyen {\tt mgetty}, {\tt psgetty} y {\tt getty}.

%In the case of the {\tt /etc/inittab} line 
En el caso de la l�nea de {\tt /etc/inittab}

\begin{tscreen}
d1:12345:respawn:/sbin/agetty -mt60 38400,19200,9600,2400,1200 ttyS0 vt100
\end{tscreen}
%which allows users to log in via a modem connected to serial line {\tt
%/dev/ttyS0}, the {\tt /sbin/agetty} parameters ``{\tt -mt60}'' allow
%the system to step through all of the modem speeds that a caller
%dialing into the system might use, and to shut down {\tt /sbin/agetty}
%if there is no connection after 60 seconds. This is called {\bf
%negotiating} a connection. The supported modem speeds are enumerated
%on the command line also, as well as the serial line to use, and the
%terminal type. Of course, both of the modems must support the data
%rate which is finally negotiated by both machines.

que permite a los usuarios ingresar usando un m�dem conectado a una l�nea serie
{\tt /dev/ttyS0}, los par�metros de {\tt /sbin/agetty} ``{\tt -m60}'' permiten
al sistema ir paso a paso por todas las velocidades del m�dem que un usuario
llamando al sistema puede usar, y apagar {\tt /sbin/getty}
si no hay ninguna conexi�n en 60 segundos. Esto se llama {\tt negociar} una
conexi�n. Las velocidades de modem soportadas se enumeran en la l�nea de �rdenes
tambi�n, as� como la l�nea serie a usar y el tipo de terminal. Desde luego, ambos m�dems
deben soportar el flujo de datos que se negocie finalmente por ambas m�quinas.


Se han pasado por alto muchos detalles importantes en esta
secci�n. Las tareas de {\tt /etc/inittab} ocupar�an un libro
completo. Para m�s informaci�n, pueden consultarse las p�ginas del
manual de {\tt init} y {\tt agetty} y los ``HOWTO'' del Proyecto de
Documentaci�n de Linux, disponibles en los lugares que se presentan en
el ap�ndice~\ref{app-sources-num}.



















% {\linux} Installation and Getting Started    -*- TeX -*-
% filesystem.tex
% Copyright (c) 1992, 1993 by Matt Welsh, Larry Greenfield and Karl Fogel
%
% This file is freely redistributable, but you must preserve this copyright 
% notice on all copies, and it must be distributed only as part of "{\linux} 
% Installation and Getting Started". This file's use is covered by
% the copyright for the entire document, in the file "copyright.tex".
%
% Copyright (c) 1998 by Specialized Systems Consultants Inc. 
% <ligs@ssc.com>

%\section{Exploring the file system.}\label{sec-filesystem-tour}
\section{Explorando el sistema de ficheros}\label{sec-filesystem-tour}
\markboth{Tutorial de {{\linux}}}{Explorando el sistema de ficheros}
\index{sistema de ficheros!exploraci�n|(}
Un {\bf sistema de ficheros} es la colecci�n de ficheros y la jerarqu�a de
directorios de un sistema. Ha llegado la hora de acompa�arle en un viaje
alrededor del sistema de ficheros.

% No ref to dirtree if ASCII
\iftex {
Usted ya tiene habilidad y conocimiento como para entender el sistema de ficheros
de {{\linux}}, y tiene un mapa de carreteras. (Ver figura en la p�gina~\pageref{dirtree}).  } 
\fi

Primero, cambie al directorio ra�z ({\tt cd /}), e introduzca {\tt
ls -F} para que aparezca una lista con su contenido. Probablemente ver�
los siguientes directorios\footnote{Puede que vea otros, y puede que no los
vea todos. Cada distribuci�n de {\linux} es diferente en ciertos aspectos.}:
{\tt bin}, {\tt dev}, {\tt etc}, {\tt home}, {\tt install}, {\tt lib},
{\tt mnt}, {\tt proc}, {\tt root}, {\tt tmp}, {\tt user}, {\tt usr},
y {\tt var}.

Ahora, veamos cada uno de estos directorios
\begin{dispitems}
\index{directorio!bin@{\tt /bin}}
\index{bin@{\tt /bin}}
\ditem{{\tt /bin}}
{\tt /bin} es la abreviatura de ``binarios'',
o ejecutables, y es donde residen muchos de los programas imprescindibles del sistema. 
Utilice {\tt ls -F /bin} para listar los ficheros que contiene.
Si repasa la lista, puede que reconozca algunas ordenes,
como {\tt cp}, {\tt ls}, y {\tt mv}. �stos son realmente los programas
que corresponden a esas ordenes. Cuando utiliza la orden {\tt cp},
est� ejecutando el programa {\tt /bin/cp}.


Usando {\tt ls -F}, ver� que muchos (si no todos) los ficheros en
{\tt /bin} tienen un asterisco (''{\tt *}'') a�adido a sus nombres de fichero.
Esto indica que los ficheros son ejecutables, como se describe en
p�gina~\pageref{sec-ls}.

\index{directorio!dev@{\tt /dev}}
\index{dev@{\tt /dev}}
\ditem{{\tt /dev}}
\index{device driver}
\index{controlador de dispositivo}
\index{fichero!dispositivo}
\index{dispositivos!acceso}
Los ''ficheros'' en {\tt /dev} son {\bf controladores de dispositivos}---acceden
a los dispositivos del sistema y a recursos como discos duros, m�dems y memoria.
Igual que su sistema puede leer datos de un fichero, tambi�n puede leer
la entrada del rat�n accediendo a {\tt /dev/mouse}.


\index{dispositivo!fd@{\tt fd}}
\index{dispositivo!disquete}
\index{dispositivo!floppy disk}
\index{floppy !nombres de dispositivo para}
\index{disquete!nombres de dispositivo para}
Los ficheros cuyos nombres comienzan por {\tt fd} son dispositivos de discos
flexibles. {\tt fd0} es la primera disquetera y {\tt fd1} es la segunda. Puede
que se haya dado cuenta de que hay m�s dispositivos de disco flexible que los
dos anteriores: �stos representan tipos espec�ficos de discos flexibles. Por ejemplo,
{\tt fd1H1440} accede a discos 3.5" de alta densidad en la disquetera 1.

Lo siguiente es una lista de algunos de los ficheros de dispositivo m�s
comunmente utilizados. Aunque puede que no tenga alguno de los dispositivos f�sicos
que se listan debajo, puede ocurrir que aun as� tenga controladores en {\tt dev} para
ellos.

\begin{itemize}
\index{dispositivos!consola}
\index{dispositivos!/dev/console@{\tt /dev/console}}
\index{consola!nombre para dispositivo}
\index{/dev/console@{\tt /dev/console}}
\item {\tt /dev/console} se refiere a la consola del sistema---es decir, al
monitor conectado directamente a su sistema.

\index{dispositivos!puertos serie}
\index{dispositivos!/dev/ttyS@{\tt /dev/ttyS}}
\index{dispositivos!/dev/cua@{\tt /dev/cua}}
\index{/dev/ttyS@{\tt /dev/ttyS}}
\index{/dev/cua@{\tt /dev/cua}}
\index{puertos serie!nombres de dispositivo para}
\item Los diversos dispositivos {\tt /dev/ttyS} y {\tt /dev/cua} se usan
para acceder a los puertos serie. {\tt /dev/ttyS0} se refiere a ''{\tt COM1}''
bajo MS-DOS. Los dispositivos {\tt /dev/cua} son dispositivos de ''llamada'' , y se
usan con un m�dem.\NT{En los n�cleos modernos a partir de la serie 2.2 los dispositivos
ttySx reemplazan a cuax en sus funciones}


\index{dispositivos!discos duros}
\index{dispositivos!/dev/hd@{\tt /dev/hd}}
\index{/dev/hd@{\tt /dev/hd}}
\index{discos duros!nombres de dispositivo}
\item Los dispositivos cuyos nombres comiencen por {\tt hd} acceden a discos duros.
{\tt /dev/hda} se refiere a {\em todo\/} el primer disco duro, mientras que {\tt
/dev/hda1} se refiere a la primera {\em partici�n\/} de {\tt /dev/hda}.



\index{dispositivos!SCSI}
\index{dispositivos SCSI!nombres para}
\index{dispositivos!/dev/sd@{\tt /dev/sd}}
\index{dispositivos!/dev/st@{\tt /dev/st}}
\index{dispositivos!/dev/sr@{\tt /dev/sr}}
\index{dev/sd@{\tt /dev/sd}}
\index{dev/st@{\tt /dev/st}}
\index{dev/sr@{\tt /dev/sr}}
\index{SCSI!nombres de dispositivos}
\item Los dispositivos cuyos nombres comienzan por {\tt sd} son discos SCSI.
Si tiene un disco duro SCSI, en lugar de acceder a �l a trav�s de {\tt
/dev/hda}, acceder�a con {\tt /dev/sda}. A las cintas SCSI se accede
v�a dispositivos {\tt st}, y a los CD-ROM SCSI v�a dispositivos {\tt sr}.



\index{dispositivos!puertos paralelos}
\index{dispositivos!/dev/lp@{\tt /dev/lp}}
\index{/dev/lp@{\tt /dev/lp}}
\index{puerto paralelo!nombre de dispositivo}
\item Los dispositivos cuyos nombres comienzan por {\tt lp} acceden a los
puertos paralelos. {\tt /dev/lp0} es lo mismo que ''{\tt LPT1}'' en el mundo MS-DOS.



\index{dispositivos!null}
\index{dispositivos!/dev/null@{\tt /dev/null}}
\index{/dev/null@{\tt /dev/null}}
\index{fichero null}
\item {\tt /dev/null} se utiliza como ''agujero negro''---los datos enviados
a este dispositivo se pierden para siempre. ?`Por qu� es �til esto? Bueno, si quiere
evitar que la salida de una orden salga por la pantalla, puede dirigir esa
salida a {\tt /dev/null}. Hablaremos de ello m�s adelante.



\index{dispositivos!consolas virtuales}
\index{dispositivos!/dev/tty@{\tt /dev/tty}}
\index{/dev/tty@{\tt /dev/tty}}
\index{consolas virtuales}
\item Los dispositivos cuyos nombres comienzan por {\tt /dev/tty} seguidos de un n�mero
se refieren a las ''consolas virtuales'' de su sistema (a las que se accede pulsando
\key{Alt-F1}, \key{Alt-F2}, y as� sucesivamente). {\tt /dev/tty1} se refiere
a la primera consola virtual, {\tt /dev/tty2} se refiere a la segunda, y as� sucesivamente.



\index{dispositivos!pseudo-terminales}
\index{dispositivos!/dev/pty@{\tt /dev/pty}}
\index{pseudo-terminales}
\index{/dev/pty@{\tt /dev/pty}}
\item Los dispositivos cuyos nombres comienzan por {\tt /dev/pty} son {\bf pseudo-terminales},
que se usan para proporcionar un ``terminal'' a las sesiones iniciadas remotamente.
Por ejemplo, si su m�quina est� en una red, las sesiones de {\tt telnet} entrantes
utilizar�n uno de los dispositivos {\tt /dev/pty}.
\end{itemize}



\index{directorio!/etc@{\tt /etc}}
\index{/etc@{\tt /etc}}
\ditem{{\tt /etc}}
{\tt /etc} contiene un buen n�mero de ficheros de configuraci�n del sistema.
Estos incluyen {\tt /etc/passwd} (la base de datos de usuarios), {\tt /etc/rc} 
(la macro de inicio del sistema), y as� sucesivamente. 



\index{directory!/sbin@{\tt /sbin}}
\index{/sbin@{\tt /sbin}}
\ditem{{\tt /sbin}}
{\tt /sbin} contiene binarios imprescindibles para el sistema que se usan para
su administraci�n.



\index{directorio!/inicio@{\tt /home}}
\index{/home@{\tt /home}}
\ditem{{\tt /home}}
{\tt /home} contiene los directorios de inicio de los usuarios. Por ejemplo, {\tt /home/larry}
es el directorio de inicio del usuario ''{\tt larry}''. En un sistema reci�n instalado,
puede que no haya ning�n usuario en este directorio.



\index{directorio!/lib@{\tt /lib}}
\index{/lib@{\tt /lib}}
\ditem{{\tt /lib}}
{\tt /lib} contiene las {\bf im�genes de las bibliotecas compartidas}, que son ficheros que
contienen c�digo que comparten muchos programas. Mejor que cada programa use sus propias
copias de estas rutinas compartidas, es que todas se guarden en un lugar com�n, en
{\tt /lib}. Esto hace que los ficheros ejecutables sean m�s peque�os y ahorra espacio
en el sistema.



\index{directorio!/proc@{\tt /proc}}
\index{/proc@{\tt /proc}}
\ditem{{\tt /proc}}
En {\tt /proc} se mantiene un ''sistema de ficheros virtual'', donde los ficheros
se guardan en memoria, no en disco. Estos ''ficheros'' hacen referencia a los
diversos {\bf procesos} que corren en el sistema, y permiten obtener informaci�n
sobre los procesos y programas en ejecuci�n en un instante dado.
Esto se discute con m�s detalle en
p�gina~\pageref{sec-process}.


\index{directorio!/tmp@{\tt /tmp}}
\index{/tmp@{\tt /tmp}}
\ditem{{\tt /tmp}}
Muchos programas guardan informaci�n temporalmente en un fichero
que se borra cuando el programa finaliza su ejecuci�n.
La localizaci�n est�ndar de estos ficheros es {\tt /tmp}.


\index{directorioy!/usr@{\tt /usr}}
\index{/usr@{\tt /usr}}
\ditem{{\tt /usr}}
{\tt /usr} es un directorio muy importante que contiene subdirectorios
que albergan algunos de los programas m�s importantes y �tiles usados
en el sistema.



Los diversos directorios descritos arriba son imprescindibles para
que el sistema funcione, pero muchos de los elementos que se encuentran en
{\tt /usr} son opcionales. Sin embargo, son esos elementos opcionales los que
hacen un sistema �til e interesante. Sin {\tt /usr}, se tendr�a un sistema
aburrido que s�lo soportar�a programas como {\tt cp} y {\tt ls}. {\tt /usr}
contiene muchos de los grandes paquetes de software y los ficheros de
configuraci�n que los acompa�an.


\index{directorio!/usr/X11R6@{\tt /usr/X11R6}}
\index{/usr/X11R6@{\tt /usr/X11R6}}
\ditem{{\tt /usr/X11R6}}
{\tt /usr/X11R6} contiene el sistema X Window, si se instal�. El sistema
X Window es un enorme y potente entorno gr�fico que proporciona un gran n�mero de
utilidades gr�ficas y programas, que aparecen en ''ventanas'' en la pantalla.
Si usted esta familiarizado con Microsoft Windows o el entorno Macintosh, X Window
le ser� muy familiar. El directorio {\tt /usr/X11R6} contiene todos los
ejecutables de X Window, ficheros de configuraci�n y ficheros de apoyo. Todo esto
se cubre con m�s detalle en el Cap�tulo~\ref{chap-advanced-xconfiguration}.

\index{directorio!/usr/bin@{\tt /usr/bin}}
\ditem{{\tt /usr/bin}}
{\tt /usr/bin} es el aut�ntico almac�n de software en cualquier sistema {\linux},
y contiene la mayor�a de los ejecutables de programas que no se encuentran en
otros sitios, como {\tt /bin}.



\index{directorio!/usr/etc@{\tt /usr/etc}}
\index{/usr/etc@{\tt /usr/etc}}
\ditem{{\tt /usr/etc}}
Como {\tt /etc} contiene diferentes ficheros de configuraci�n y programas
del sistema, {\tt /usr/etc} contiene incluso m�s que el anterior. En
general, los ficheros que se encuentran en {\tt /usr/etc/} no son esenciales
para el sistema, a diferencia de los que se encuentran en {\tt /etc}, que s�
lo son.


\index{directorio!/usr/include@{\tt /usr/include}}
\index{/usr/include@{\tt /usr/include}}
\ditem{{\tt /usr/include}}
{\tt /usr/include} contiene los {\bf ficheros de cabecera} para el compilador de C.
En estos ficheros (muchos de los cuales terminan en {\tt .h}, por ''header'')
se declaran nombres de estructuras de datos, subrutinas, y constantes usadas
al programar en el nivel de sistema UNIX. Si est� familiarizado con el lenguaje
de programaci�n C, aqu� encontrar� ficheros de cabecera como {\tt
stdio.h}, en el que se declaran funciones como {\tt printf()}.


\index{directorio!/usr/g++-include@{\tt /urs/g++-include}}
\index{/usr/g++-include@{\tt /urs/g++-include}}
\ditem{{\tt /usr/g++-include}}
{\tt /usr/g++-include} contiene ficheros de cabecera para el compilador de C++
(muy parecido a {\tt /usr/include}).

\index{directorio!/usr/lib@{\tt /usr/lib}}
\index{/usr/lib@{\tt /usr/lib}}
\ditem{{\tt /usr/lib}}
{\tt /usr/lib} contiene las bibliotecas ''stub''  y ''estatic'' equivalentes
a los ficheros situados en {\tt /lib}. Cuando se compila un programa, el programa
se ''enlaza'' con las bibliotecas situadas en {\tt /usr/lib}, que ordenar�n
al programa que mire en {\tt /lib} cuando necesite el c�digo real de la
librer�a. Por a�adidura, otros programas diversos guardan ficheros de configuraci�n en 
{\tt /usr/lib}.

\index{directorio!/usr/local@{\tt /usr/local}}
\index{/usr/local@{\tt /usr/local}}
\ditem{{\tt /usr/local}}
{\tt /usr/local} se parece mucho a {\tt /usr}---contiene diversos programas
y ficheros que no son imprescindibles para el sistema, pero que lo hacen divertido
y excitante. En general, los programas en {\tt /usr/local} son
espec�ficos de cada sistema---consecuentemente, {\tt /usr/local}
var�a mucho entre los diversos sistemas {\linux}.

\index{directorio!/usr/man@{\tt /usr/man}}
\index{/usr/man@{\tt /usr/man}}
\ditem{{\tt /usr/man}}
Este directorio contiene las p�ginas del manual. Hay dos subdirectorios en �l
para cada ''secci�n'' de p�ginas del manual (use la orden {\tt man man}
para m�s detalles).  Por ejemplo, {\tt /usr/man/man1} contiene las fuentes
(es decir, el original sin formatear) de la p�ginas del manual de la secci�n 1, y
{\tt /usr/man/cat1} contiene las p�ginas del manual formateadas de la secci�n 1.

\index{directorio!/usr/src@{\tt /usr/src}}
\index{/usr/src@{\tt /usr/src}}
\ditem{{\tt /usr/src}}

{\tt /usr/src} contiene el c�digo fuente (las instrucciones sin compilar)
de diversos programas del sistema. El directorio m�s importante aqu�
es {\tt /usr/src/linux}, que contiene el c�digo fuente del n�cleo de {\linux}.



\index{directorio!/var@{\tt /var}}
\index{/var@{\tt /var}}
\ditem{{\tt /var}}
En {\tt /var} se mantienen directorios que a veces cambian de tama�o o tienden
a crecer.  Muchos de estos directorios sol�an residir en {\tt /usr}, pero
desde que aqu�llos que mantienen {\linux} intentan conservarlo relativamente
sin cambios, los directorios que cambian a menudo se han pasado a
{\tt /var}.  Algunas distribuciones de {\linux} guardan las bases de datos de sus
paquetes de software en directorios bajo {\tt /var}.

\index{directorio!/var/log@{\tt /var/log}}
\index{/var/adm@{\tt /var/log}}
\ditem{{\tt /var/log}}
{\tt /var/log} contiene diversos ficheros de inter�s para el administrador
del sistema, espec�ficamente, los registros del sistema, que recogen errores
o problemas con el sistema. Otros ficheros recogen entradas e intentos
fallidos de entrar el sistema. Esto se cubrir� en el Cap�tulo~\ref{chap-sysadm}.

\index{directorio!/var/spool@{\tt /var spool}}
\index{/var/spool@{\tt /var spool}}
\ditem{{\tt /var/spool}}
{\tt /var/spool} contiene ficheros que son encolados para otro programa.

Por ejemplo, si su m�quina est� conectada a una red, el correo entrante
se guarda en {\tt /var/spool/mail} hasta que se lee o se borra.
Los art�culos de noticias entrantes o salientes est�n en
{\tt /var/spool/news}, y as� sucesivamente.

\end{dispitems}

\index{sistema de ficheros!exploraci�n|)}

  %% rak
% Linux Installation and Getting Started    -*- TeX -*-
% swapfile.tex
% Copyright (c) 1992, 1993 by Matt Welsh <mdw@sunsite.unc.edu>
%
% This file is freely redistributable, but you must preserve this copyright 
% notice on all copies, and it must be distributed only as part of "Linux 
% Installation and Getting Started". This file's use is covered by
% the copyright for the entire document, in the file "copyright.tex".
%
% Copyright (c) 1998 by Specialized Systems Consultants Inc. 
% <ligs@ssc.com>
%Traducido por Sebasti�n Gurin, Cancerbero <anon@adinet.com.uy>, el 09/01/01 
%Revisi�n 1 7/7/2002 por Francisco Javier Fernandez <serrador@arrakis.es>

\section{Usando un fichero de intercambio}
\label{sec-swap-file}
\markboth{Administraci�n del Sistema}{Usando un fichero de intercambio}

En lugar de reservar una partici�n separada para el espacio de
intercambio, se puede usar un fichero de intercambio. Sin embargo,
ser� necesario instalar {\linux} y conseguir que todo funcione antes de crearlo.

Teniendo {\linux} ya instalado, se puede usar las siguientes
instrucciones para crear el fichero de intercambio. La orden de abajo,
crea un fichero de intercambio de 8208 bloques de tama�o, (aproximadamente 8 Mb).


\begin{tscreen}
\# dd if=/dev/zero of=/swap bs=1024 count=8208 
\end{tscreen}

Esta orden crea el fichero de intercambio, {\tt /swap}. El par�metro
``{\tt count=}'', es el tama�o del fichero de intercambio en bloques.
\begin{tscreen}
\# mkswap /swap 8208
\end{tscreen}
Esta orden inicia el fichero de intercambio. Una vez m�s, ser�
necesario reemplazar el nombre y el tama�o del fichero de intercambio
con los valores apropiados. 

\begin{tscreen}
\# sync \\
\# swapon /swap
\end{tscreen}
Ahora el sistema est� realizando el intercambio en el fichero {\tt
  /swap}. La instrucci�n {\tt sync} garantiza que el fichero haya sido escrito en el disco. 

Una desventaja importante de usar un fichero de intercambio, es que todo acceso al fichero, es hecho a trav�s del sistema de ficheros. Esto significa que
los bloques que constituyen el fichero de intercambio pueden no ser contiguos. Como consecuencia, el rendimiento puede no ser tan bueno como el de una partici�n de 
intercambio, en donde los bloques son siempre contiguos y las demandas de entrada/salida son realizadas directamente al dispositivo. 
Otra desventaja de los ficheros de intercambio largos es el gran peligro de que el sistema de ficheros se corrompa si algo sale mal. 
Conservar los ficheros normales, separados de las particiones de intercambio previene que esto pase. 
Los ficheros de intercambio pueden ser �tiles si, por ejemplo, se
necesita usar, temporalmente, m�s espacio de intercambio. Si se est�
compilando un programa extenso y se quisiera  acelerar las cosas un
tanto, se puede crear un fichero de intercambio temporal y usarlo
adem�s del espacio de intercambio regular. 
Para eliminar un fichero de intercambio, usa primero {\tt swapoff}, como en
\begin{tscreen}
\# swapoff /swap
\end{tscreen}
Luego, el fichero puede ser eliminado
\begin{tscreen}
\# rm /swap
\end{tscreen}


Cada fichero o partici�n de intercambio puede tener un tama�o m�ximo
de 128 megabytes, pero se puede usar hasta 8 ficheros o particiones de
intercambio en el sistema. 

%Traducido por Sebasti�n Gurin (Cancerbero), el 12/01/01  %% rak
% \linux Installation and Getting Started    -*- TeX -*-
% users.tex
% Copyright (c) 1993 by Matt Welsh and Lars Wirzenius
%
% This file is freely redistributable, but you must preserve this copyright 
% notice on all copies, and it must be distributed only as part of "\linux 
% Installation and Getting Started". This file's use is covered by
% the copyright for the entire document, in the file "copyright.tex".
%
% Este fichero es de distribuci�n libre, pero debe mantenerse esta 
% informaci�n de Copyright en todas las copias, y debe distribuirse solo como
% parte de "Instalaci�n y Primeros Pasos en \linux". El uso de este fichero esta
% cubierto por el Copyright del documento completo, en el fichero "copyright.tex"
% Copyright (c) 1995 por Gerardo Izquierdo para la versi�n al Castellano
%

% 
% Versi�n para lipp 2.0 por Alberto Molina. Comentarios a:
%            alberto@nucle.us.es 
%Revisi�n1 por Javier Fernandez <serrador@arrakis.es>
%Gold


\section{Gesti�n de Usuarios}
\label{sec-manage-users}
\label{sec-add-user}

\index{administraci�n de usuarios!a�adiendo usuarios}
\index{a�adiendo usuarios}
\index{usuarios!a�adiendo}
Independientemente de que haya muchos usuarios o no en el sistema, es
importante comprender los aspectos de la gesti�n de usuarios bajo \linux.
Incluso si se es el �nico usuario, se debe tener, presumiblemente, una cuenta
distinta de {\tt root} para hacer la mayor parte del trabajo.

        Cada persona que utilice el sistema debe tener su propia cuenta.
        Raramente es una buena idea el que varias personas compartan la misma
        cuenta. No s�lo es un problema de seguridad, sino que las cuentas
        se utilizan para identificar un�vocamente a los usuarios al sistema.
        Es necesario saber qui�n est� haciendo qu� en cada momento.

\subsection{Conceptos de gesti�n de usuarios}
El sistema mantiene una cierta cantidad de informaci�n acerca de cada usuario.
Dicha informaci�n se resume a continuaci�n.
\begin{dispitems}

\index{usuarios!nombre de }
\index{nombre de usuario!definici�n}
\ditem{{\bf nombre de usuario}}
El nombre de usuario es el identificador �nico dado a cada usuario del 
sistema. Ejemplos de nombres de usuario son {\tt manolo}, {\tt pepe} y 
{\tt mdw}. Se pueden utilizar letras y d�gitos junto a los car�cteres 
``{\tt \_}'' (subrayado) y ``{\tt .}'' (punto). Los nombres de usuario se
limitan normalmente a 8 car�cteres de longitud.

\index{usuarios!user ID de}
\index{user ID!definici�n}
\index{UID!definici�n}
\ditem{{\bf ID de usuario}}
El ID de usuario, o UID en sus siglas en ingl�s, es un n�mero
�nico dado a cada usuario del sistema. El sistema normalmente
le sigue la pista a los usuarios por su UID, no por el nombre de usuario.

\index{usuarios!group ID de}
\index{group ID!definici�n}
\index{GID!definici�n}
\ditem{{\bf ID de grupo}}
El ID de grupo, o GID en sus siglas en ingl�s, es la identificaci�n
del grupo del usuario predeterminado. En la secci�n~\ref{sec-perms}
discutimos los permisos de grupo; cada usuario pertenece a uno o m�s
grupos definidos por el administrador del sistema.

\index{usuarios!clave de}
\ditem{{\bf clave}}
El sistema tambi�n almacena la clave cifrada del usuario. La orden
{\tt passwd} se utiliza para poner y cambiar las claves de los usuarios.


\index{usuarios!nombre completo de}
\ditem{{\bf nombre completo}}
El ``nombre real'' o ``nombre completo'' del usuario se almacena junto con el
nombre de usuario. Por ejemplo, el usuario {\tt jperez} puede tener el nombre
``Jos� P�rez'' en la vida real.

\index{usuarios!directorio inicial de}
\index{directorio inicial!definido}
\ditem{{\bf directorio inicial}}
El directorio inicial es el directorio en el que se coloca inicialmente al
usuario en tiempo de conexi�n. Cada usuario debe tener su propio directorio
inicial, normalmente situado bajo {\tt /home}.

\index{usuarios!Int�rprete de conexi�n de}
\index{int�rprete de conexi�n!definici�n}
\ditem{{\bf int�rprete al registrarse}}
El int�rprete al registrarse  es el int�rprete de �rdenes que se ejecutar�
cuando se registre el usuario. Ejemplos pueden ser {\tt /bin/bash} y {\tt /bin/tcsh}.
\end{dispitems}

\index{/etc/passwd@{\tt /etc/passwd}}
\index{fichero de constrase�as!formato de}
El fichero {\tt /etc/passwd} contiene la informaci�n anterior acerca de los
usuarios. Cada l�nea del fichero contiene informaci�n acerca de un �nico 
usuario; el formato de cada l�nea es
\begin{tscreen}
nombre:clave cifrada:UID:GID:nombre completo:dir.inicio:int�rprete
\end{tscreen}
Un ejemplo puede ser:
\begin{tscreen}
kiwi:Xv8Q981g71oKK:102:100:Laura Villa:/home/kiwi:/bin/bash
\end{tscreen}

Como puede verse, el primer campo , ``{\tt kiwi}'', es el nombre de usuario.

El siguiente campo, ``{\tt Xv8Q981g71oKK}'', es la clave cifrada.
Las claves no se almacenan en el sistema en ning�n formato legible por
el hombre. Las claves se cifran utiliz�ndose ellas mismas como clave
secreta. En otras palabras, s�lo si se conoce la clave, �sta puede ser
descifrada. Esta forma de cifrado es bastante segura.

Algunos sistemas utilizan ``claves en sombra'' en la que la informaci�n
de las claves se relega al fichero {\tt /etc/shadow}. Puesto que 
{\tt /etc/passwd} es legible por todo el mundo, {\tt /etc/shadow} suministra
un grado extra de seguridad, puesto que �ste no lo es. Las claves en 
sombra suministran algunas otras funciones como puede ser la expiraci�n de 
claves; no entraremos a detallar estas funciones aqu� .

El tercer campo ``{\tt 102}'', es el UID. Este debe ser �nico para cada 
usuario. El cuarto campo, ``{\tt 100}'', es el GID. Este usuario pertenece
al grupo numerado 100. La informaci�n de grupos, como la informaci�n de
usuarios, se almacena en el fichero {\tt /etc/group}. V�ase la 
secci�n~\ref{sec-manage-groups} para m�s informaci�n.

El quinto campo es el nombre completo del usuario. ``{\tt Laura Villa}''. Los
dos �ltimos campos son el directorio inicial del usuario ({\tt /home/kiwi}) y
la shell de ingreso ({\tt /bin/bash}), respectivamente. No es necesario
que el directorio inicial de un usuario tenga el mismo nombre que el del
nombre de usuario. Sin embargo, ayuda a identificar el directorio.

\subsection{A�adir usuarios}
Cuando se a�ade un usuario hay varios pasos a seguir. Primero, 
se le debe crear una entrada en {\tt /etc/passwd}, con un nombre de
usuario y UID �nicos. Se debe especificar el GID, nombre completo y resto
de informaci�n. Se debe crear el directorio inicial, y poner los permisos
en el directorio para que el usuario sea el due�o. Se deben suministrar 
ficheros de ordenes de inicializaci�n en el nuevo directorio y se debe 
hacer alguna otra configuraci�n del sistema (por ejemplo, preparar
un buz�n para el correo electr�nico entrante para el nuevo usuario).

Aunque no es dif�cil el a�adir usuarios a mano (yo lo hago), cuando 
se est� ejecutando un sistema con muchos usuarios, es f�cil el 
olvidarse de algo. La manera m�s simple de a�adir usuarios es 
utilizar un programa interactivo que vaya preguntando por la 
informaci�n necesaria y actualice todos los ficheros del sistema 
autom�ticamente. El nombre de este programa es {\tt useradd} o {\tt 
adduser} dependiendo del software que est� instalado.

Un fichero t�pico {\tt /etc/adduser.conf} se muestra a continuaci�n: 
\begin{tscreen}\begin{verbatim}
# /etc/adduser.conf: Configuraci�n de `adduser'.
# Vea adduser(8) y adduser.conf(5) para m�s informaci�n.

# La variable DSHELL especifica la shell de ingreso asumida en 
# el sistema.
DSHELL=/bin/bash

# La variable DHOME especifica el directorio que contendr� los
# directorios iniciales de los usuarios.
DHOME=/home

# Si en GROUPHOMES pone "yes", entonces los directorios iniciales
# estar�n en /home/nombre_grupo/usuario.
GROUPHOMES=no

# Si en LETTERHOMES pone "yes", entonces los directorios iniciales
# tendr�n un directorio extra correspondiente a la primera letra del
# nombre de usuario, como por ejemplo: /home/u/user.
LETTERHOMES=no

# La variable SKEL especifica el directorio que contiene los ficheros
# inciales configurables de cada usuario, como el fichero .profile que
# se copiar� al directorio de inicio de un usuaio cuando sea creado.
SKEL=/etc/skel

# De FIRST_SYSTEM_UID a LAST_SYSTEM_UID ambos inclusive, va el rango de
# UID para cuentas del sistema y administraci�n.
FIRST_SYSTEM_UID=100
LAST_SYSTEM_UID=999

# DE FIRST_UID a LAST_UID ambos inclusive, va el rango de UID para
# cuentas de usuarios. 
FIRST_UID=1000
LAST_UID=29999

# La variable USERGROUPS puede estar en "yes" o "no". Si est� en "yes"
# cada usuario tendr� que usar como asumido su propio grupo y su
# directorio inicial ser� g+s. Si est� en "no", cada usuario a�adido
# ser� colocado en el grupo con gid igual a USERS_GID (ver m�s abajo).
USERGROUPS=yes

# Si USERGROUPS est� en "no", entonces USERS_GID ser� el GID del grupo
# `users' del sistema.
USERS_GID=100

# Si se especifica QUOTAUSER, se limitar� el espacio para el
# directorio inicial de un usuario (cuota) mediante:
# `edquota -p QUOTAUSER newuser'
QUOTAUSER=""
\end{verbatim}\end{tscreen}

Adem�s de definir las variables predefinidas que la orden adduser
utiliza, {\tt /etc/adduser.conf} tambi�n especifica d�nde se localizan
los ficheros de configuraci�n del sistema de cada usuario. En este
ejemplo, est�n en {\tt /etc/skel}, definido por la l�nea {\tt
  SKEL=}. Los ficheros que se coloquen en este directorio, como {\tt
  .profile}, {\tt .tcshrc} o {\tt .bashrc} se copiar�n
autom�ticamente al directorio de inicio de un usuario al a�adirlo con
la orden {\tt adduser}.

\subsection{Borrando usuarios}

De forma parecida, se pueden borrar usuarios mediante la orden 
{\tt userdel} o {\tt deluser} dependiendo de qu� software est� instalado 
en el sistema.

\index{usuarios!deshabilitando}
\index{deshabilitando usuarios}
Si se desea ``deshabilitar'' temporalmente un usuario para que no se conecte
al sistema (sin borrar la cuenta del usuario), se puede anteponer un
asterisco (``{\tt *}'') al campo de la clave en {\tt /etc/passwd}. Por 
ejemplo, cambiando la l�nea de {\tt /etc/passwd} correspondiente a 
{\tt kiwi} a 
\begin{tscreen}
kiwi:*Xv8Q981g71oKK:102:100:Laura Villa:/home/kiwi:/bin/bash
\end{tscreen}
evitar� que {\tt kiwi} se conecte.

\subsection{Poniendo atributos de usuario}

Despu�s de que haya creado un usuario, puede necesitar cambiar alg�n 
atributo de dicho usuario, como puede ser el directorio inicial o la 
clave. La forma m�s simple de hacer �sto es cambiar los valores 
directamente en {\tt /etc/passwd}. Para poner clave a un usuario, utilice 
la orden {\tt passwd}. Por ejemplo,
\begin{tscreen}
\# {\em passwd manuel}
\end{tscreen}
cambiar� la clave de {\tt manuel}. S�lo el administrador `` {\tt root}'' puede cambiar la 
clave de otro usuario de esta forma. Los usuarios pueden cambiar su propia
clave con {\tt passwd} tambi�n.

En algunos sistemas, las instrucciones {\tt chfn} y {\tt chsh} est�n 
disponibles, permitiendo a los usuarios cambiar su nombre completo y
la shell de ingreso. Si no, deben pedirle al administrador del sistema
que los modifique.

\subsection{Grupos}\label{sec-manage-groups}

Como hemos citado anteriormente, cada usuario pertenece a uno o m�s grupos.
La �nica importancia real de las relaciones de grupo es la perteneciente a
los permisos de ficheros, como dijimos en la secci�n~\ref{sec-perms}, cada
fichero tiene un ``grupo propietario'' y un conjunto de permisos de grupo
que define de qu� forma pueden acceder al fichero los usuarios del grupo.

Hay varios grupos definidos en el sistema, como pueden ser {\tt bin}, 
{\tt mail}, y {\tt sys}. Los usuarios no deben pertenecer a ninguno de estos
grupos; se utilizan para permisos de ficheros del sistema. En su lugar, los
usuarios deben pertenecer a un grupo individual, como {\tt users}. Si se 
quiere ser detallista, se pueden mantener varios grupos de usuarios como por
ejemplo {\tt estudiantes}, {\tt mantenimiento} y {\tt secretar�a}.

\index{/etc/group@{\tt /etc/group}!formato de}
El fichero {\tt /etc/group} contiene informaci�n acerca de los grupos.
El formato de cada l�nea es
\begin{tscreen}
nombre de grupo:clave:GID:otros miembros
\end{tscreen}
Algunos ejemplos de grupos pueden ser:
\begin{tscreen}
root:*:0: \\
usuarios:*:100:mdw,pepe \\
invitados:*:200: \\
otros:*:250:kiwi
\end{tscreen}

El primer grupo, {\tt root}, es un grupo especial del sistema reservado para
la cuenta {\tt root}. El siguiente grupo, {\tt users}, es para usuarios 
normales. Tiene un GID de 100. Los usuarios {\tt mdw} y {\tt pepe} tienen
acceso a este grupo. Recu�rdese que en {\tt /etc/passwd} cada usuario tiene
un GID predeterminado. Sin embargo, los usuarios pueden pertenecer a m�s de un
grupo, a�adiendo sus nombres de usuario a otras l�neas de grupo en 
{\tt /etc/group}. La orden {\tt groups} nos dice a qu� grupos se tiene 
acceso.

El tercer grupo, {\tt invitados}, es para usuarios invitados, y {\tt otros}
es para ``otros'' usuarios. El usuario {\tt kiwi} tiene acceso a �ste 
grupo.

Como se puede ver, el campo ``clave'' de {\tt /etc/group} raramente se 
utiliza. A veces se utiliza para dar una clave para acceder a un grupo.
Esto es raras veces necesario. Para evitar el que los usuarios cambien a
grupos privilegiados (con la orden {\tt newgroup}), se pone el campo de
la clave a ``{\tt *}''.

Se pueden usar las ordenes {\tt addgroup} o {\tt groupadd} para a�adir
grupos a su sistema. Normalmente es m�s sencillo a�adir l�neas a 
{\tt /etc/group} uno mismo, puesto que no se necesitan m�s 
configuraciones para a�adir un grupo. Para borrar un grupo, s�lo hay 
que borrar su entrada de {\tt /etc/group}.



% Linux Installation and Getting Started    -*- TeX -*-
% hats.tex
% Copyright (c) 1993 by Matt Welsh and Lars Wirzenius
%
% This file is freely redistributable, but you must preserve this copyright 
% notice on all copies, and it must be distributed only as part of "Linux 
% Installation and Getting Started". This file's use is covered by
% the copyright for the entire document, in the file "copyright.tex".
%
% Este fichero es de distribuci\'on libre, pero debe mantenerse esta 
% informaci\'on de Copyright en todas las copias, y debe distribuirse solo como
% parte de "Instalaci\'on y Primeros Pasos en Linux". El uso de este fichero esta
% cubierto por el Copyright del documento completo, en el fichero "copyright.tex"
% Copyright (c) 1995 por Gerardo Izquierdo para la versi\'on al Castellano
% $Log: hats.tex,v $
% Revision 1.6  2003/07/19 06:32:34  joseluis.ranz
% Correcciones varias.
%
% Revision 1.5  2002/07/20 22:24:29  pakojavi2000
% Beta2
%
% Revision 1.4  2002/07/07 21:40:49  pakojavi2000
%  Traducci�n de fragmentos incompletos
%
% Revision 1.3  2001/04/18 16:29:10  amolina
% Segunda revisi�n de los ficheros
%
% Revision 1.2  2001/01/16 15:10:36  amolina
%
% Primera traducci�n de sysadm/hats,tex
%
% Revision 0.5.0.1  1996/02/10 23:45:12  rcamus
% Primera beta publica

% 
% Versi�n para lipp 2.0 por Alberto Molina. Comentarios a:
%            alberto@nucle.us.es 
%
%

\subsection{Responsabilidades de la Administraci�n del Sistema}

Puesto que el administrador de sistema tiene mucho m�s poder y
responsabilidad, cuando algunos usuarios tienen la oportunidad de
ingresar por primera vez como {\tt root}, tanto en sistemas GNU/Linux como
en otros, tienden a abusar de los privilegios de {\tt root}. Existen
``administradores de sistema'' que leen el correo de otros usuarios,
borran ficheros sin avisar y se comportan como ni�os con un poderoso
juguete entre sus manos.

Como el administrador tiene tanto poder sobre el sistema, se requiere
cierta madurez y autocontrol para utilizar la cuenta {\tt
  root}. Existe un c�digo de honor no escrito que establece las normas
de comportamiento del administrador del sistema para con el resto de
usuarios. �C�mo se sentir�a si el administrador de su sistema se
dedicase a leer su correo electr�nico o a mirar en sus
ficheros?. Existe un cierto vac�o legal en estos asuntos. En los
sistemas UNIX, el usuario {\tt root} tiene la posibilidad de saltarse
todos los mecanismos de seguridad y privacidad. Es importante que el
administrador de sistema establezca una relaci�n de confianza con sus
usuarios.

\subsection{C�mo proceder con los usuarios}

Los administradores de sistemas pueden tomar dos posturas cuando traten con
usuarios abusivos: ser paranoicos o confiados. El administrador de 
sistemas paranoico normalmente causa m\'as da\~no que el que previene. Una de
mis citas favoritas es: ``Nunca atribuyas a la malicia nada que pueda ser
atribuido a la estupidez''. Dicho de otra forma, muchos usuarios no tienen
la habilidad o el conocimiento para hacer da\~no real al sistema. El 90\% del
tiempo, cuando un usuario causa problemas en el sistema (por ejemplo, 
rellenando la partici\'on de usuarios con grandes ficheros, o ejecutando
m\'ultiples veces simult�neamente un gran programa), el usuario simplemente desconoce 
que est\'a causando un problema. He ido a ver a usuarios que estaban
causando una gran cantidad de problemas, pero su actitud estaba causada por 
la ignorancia, no por la malicia.

Cuando se encuentre con usuarios que puedan causar problemas potenciales
no sea hostil. La antigua regla de ``inocente hasta que se demuestre lo
contrario'' sigue siendo v\'alida. Es mejor una simple charla con el usuario,
pregunt\'andole acerca del problema, en lugar de causar una confrontaci\'on. Lo
\'ultimo que se desea es estar entre los malos desde el punto de vista del
usuario. Esto levantar\'\i a un mont\'on de sospechas acerca de si el
administrador de sistemas tiene el sistema correctamente 
configurado. Si un usuario cree que uno le disgusta o no le tiene 
confianza, le puede acusar de borrar ficheros o romper la privacidad del 
sistema. Esta no es, ciertamente, el tipo de situaci\'on en la que se
quisiera estar.

Si se encuentra que un usuario ha estado intentando ``romper'' el sistema,
o ha estado haciendo da\~no al sistema de forma intencionada, no hay
que devolver el comportamiento malicioso a su vez. En vez de ello,
simplemente, es recomendable darle un  aviso ---pero siendo
flexible. En muchos casos, se puede cazar a un usuario
``con las manos en la masa'', da\~nando al sistema, lo correcto es
advertirle y decirle que no lo vuelva a repetir. Sin embargo, si le
{\em vuelve\/} a cazar haciendo da\~no, entonces se puede estar
absolutamente seguro de que es intencionado.
Ni siquiera puedo empezar a describir la cantidad de veces que parec\'\i a que
hab\'\i a un usuario causando problemas al sistema, cuando de hecho, era o un
accidente o un fallo m\'\i o.

\subsection{Fijando las reglas}

La mejor forma de administrar un sistema no es con un pu\~no de hierro. 
As\'{\i} puede ser como se haga lo militar, pero UNIX no fue dise\~nado para 
ese tipo de disciplinas. Tiene sentido el escribir un conjunto sencillo y 
flexible de reglas para los usuarios, pero hay que recordar que cuantas menos 
reglas tenga, menos posibilidades habr\'a de romperlas. Incluso si las 
reglas para utilizar el sistema son perfectamente razonables y claras, 
siempre habr\'a momentos en que los usuarios romper\'an dichas reglas sin 
pretenderlo. Esto es especialmente cierto en el caso de usuarios UNIX 
nuevos, que est\'an aprendiendo los entresijos del sistema. No esta 
suficientemente claro, por ejemplo, que uno no debe bajarse un gigabyte de 
ficheros y envi\'arselo por correo a todos los usuarios del sistema. Los 
usuarios necesitan comprender las reglas y por qu� est\'an establecidas.

Si especifica reglas de uso para su sistema, hay que asegurarse de que el motivo 
detr\'as de cada regla particular est\'e claro. Si no se hace, los usuarios
encontrar\'an toda clase de formas creativas de salt\'arsela y no saber que en
realidad la est\'an rompiendo.

\subsection{Lo que todo esto significa}

No podemos decir c�mo ejecutar su sistema al \'ultimo detalle. Mucha de la
filosof\'\i a depende de c�mo se use el sistema. Si se tienen muchos 
usuarios, las cosas son muy diferentes de si s�lo tiene unos pocos o si 
se es el \'unico usuario del sistema. Sin embargo, siempre es una buena 
idea, en cualquier situaci\'on, comprender lo que ser administrador 
de sistema significa en realidad.

Ser el administrador de un sistema no le hace a uno un mago del UNIX. Hay
muchos administradores de sistemas que conocen muy poco acerca de UNIX.
Igualmente, hay muchos usuarios ``normales'' que saben m\'as acerca de 
UNIX que cualquier administrador de sistema. Tambi\'en, ser 
administrador de sistemas no le permite el utilizar la malicia contra sus 
usuarios. Aunque el sistema le d\'e el privilegio de enredar en los 
ficheros de los usuarios, no significa que se tenga ning\'un derecho a 
hacerlo.

Por \'ultimo, ser el administrador del sistema no es realmente una gran cosa.
No importa si el sistema es un peque\~no 386 o un super ordenador Cray. La
ejecuci\'on del sistema es la misma. El saber la clave de {\tt root} no
significa ganar dinero o fama. Tan solo le permitir\'a ejecutar el sistema
y mantenerlo funcionando. Eso es todo.

  %% rak
% Linux Installation and Getting Started    -*- TeX -*-
% tar.tex
% Copyright (c) 1993 by Matt Welsh and Lars Wirzenius
%
% This file is freely redistributable, but you must preserve this copyright 
% notice on all copies, and it must be distributed only as part of "Linux 
% Installation and Getting Started". This file's use is covered by
% the copyright for the entire document, in the file "copyright.tex".
%
% Copyright (c) 1998 by Specialized Systems Consultants Inc. 
% <ligs@ssc.com>
%

%Traducido por Sebasti�n Gurin, Cancerbero <anon@adinet.com.uy> el 12/01/01

%Revisado por Sebasti�n Gurin, Cancerbero <anon@adinet.com.uy> el 18/01/01
%Revisi�n 1 por Francisco Javier Mart�nez <serrador@arrakis.es>   el 7/7/02


 

\section{Almacenamiento y compresi�n de ficheros}
\markboth{Administraci�n de Sistema}{Almacenando y Comprimiendo ficheros}
\subsection*{Pre�mbulo a la traducci�n al castellano}
{\em En espa�ol existe cierta confusi�n entre los t�rminos fichero y archivo, los cu�les se toman como sin�nimos. En inform�tica y en este texto, 
existe una sutil pero importante diferencia entre los t�rminos. Cuando nos referimos a un fichero, nos referimos a cualquier tipo de documento,
imagen, sonido almacenado en un soporte l�gico. Sin embargo, un archivo es una clase especial de fichero que contiene otros ficheros. El origen
de dicha confusi�n parece ser la traducci�n err�nea de fichero por archivo en los sistemas operativos de Microsoft. No cometeremos el mismo error aqu�,
Por lo tanto aqu� se llamar�n archivos a los ficheros .tar y similares cuyo prop�sito es contener otros ficheros.}
{\em(Nota del Revisor)}

\subsection{Usando {\tt tar}}
Antes de que podamos hablar de copias de seguridad, necesitamos realizar una presentaci�n de las herramientas utilizadas para almacenar ficheros en sistemas UNIX. 

La orden {\tt tar} es la m�s usada para almacenar ficheros. Su sintaxis es:
\begin{tscreen}
tar \cparam{opciones} \cparam{ficheros} 
\end{tscreen}
en donde \textsl{opciones} es la lista de opciones para {\tt tar},
y \textsl{ficheros} es la lista de ficheros a agregar o extraer del archivo tar.  
Por ejemplo, la orden 
\begin{tscreen}
\# tar cvf backup.tar /etc
\end{tscreen}
empaqueta todos los ficheros del directorio {\tt /etc} en el archivo tar {\tt
backup.tar}. El primer par�metro que se le entrega a {\tt tar}, ``{\tt cvf}'', es la orden que le transmitimos a {\tt tar}.
``{\tt c}'' le dice a tar que cree un nuevo archivo. La opci�n ``{\tt v}''  fuerza a tar en el modo detallado,
imprimiendo los nombres de los ficheros seg�n se archivan. La opci�n ``{\tt f}'' le informa a {\tt tar}, que el pr�ximo argumento, 
{\tt backup.tar}, es el nombre del archivo a crear. El resto de los argumentos para {\tt tar} son el/los nombre(s) de ficheros(s) y
directorio(s) para agregar al archivo tar.

La instrucci�n
\begin{tscreen}
\# tar xvf backup.tar
\end{tscreen}
extraer� todos los ficheros archivados dentro de {\tt backup.tar} en el directorio actual.

\blackdiamond Los ficheros antiguos con el mismo nombre son sobrescritos cuando se extraen en un directorio existente. 
Antes de extraer ficheros de un archivo tar, es importante saber d�nde deben ser desempaquetados los ficheros.
Digamos que se han archivado los siguientes ficheros: {\tt /etc/hosts}, {\tt /etc/group}, y {\tt /etc/passwd}. Si se us� la orden

\begin{tscreen}
\# tar cvf backup.tar /etc/hosts /etc/group /etc/passwd
\end{tscreen}
el nombre del directorio {\tt /etc/} se a�adir� al principio de los nombres de cada fichero. Para extraer los ficheros en su ubicaci�n correcta, debe usarse
\begin{tscreen}
\# cd / \\
\# tar xvf backup.tar
\end{tscreen}
porque los ficheros son extra�dos con el nombre de ruta, guardado, en el archivo tar.
Sin embargo, si se han archivado los ficheros con la orden
\begin{tscreen}
\# cd /etc \\
\# tar cvf hosts group passwd
\end{tscreen}
el nombre del directorio no se conserva en el archivo tar. En consecuencia, necesitar�s hacer un ``{\tt cd /etc}'', 
antes de extraer los ficheros. Como puedes ver, el c�mo haya sido creado un fichero tar, marca una gran diferencia en c�mo se extrae; 
o dicho de otra modo: la manera en la que ser�n extra�dos los ficheros de un archivo tar, est� estrechamente relacionada con la manera en c�mo han sido archivados.
La orden
\begin{tscreen}
\# tar tvf backup.tar
\end{tscreen}
se puede usar para mostrar una lista de los ficheros del archivo tar, pero sin extraerlos. De esta forma se puede ver qu�
directorio se utiliz� como origen de los nombres de los ficheros, y se puede extraer el fichero desde la localizaci�n correcta.

\subsection{{\tt gzip} y {\tt compress}}
A diferencia de los de archivado para MS-DOS, {\tt tar} no comprime
los ficheros autom�ticamente seg�n los archiva. Por ejemplo: si se
archivan  dos ficheros de 1 Mega byte cada uno, en un archivo tar, el
tama�o de este �ltimo ser� de 2 Mega bytes. En \linux, la orden {\tt
gzip}, puede utilizarse para comprimir un archivo, (no tiene por que
ser un archivo tar). La instrucci�n
\begin{tscreen}
\# gzip -9 backup.tar
\end{tscreen}
comprime {\tt backup.tar}, dej�ndonos el fichero {\tt backup.tar.gz}, una versi�n
comprimida del archivo. El par�metro {\tt -9}, hace que {\tt gzip}, utilice el mayor factor de compresi�n.
La orden gunzip puede ser utilizado para descomprimir un fichero comprimido con gzip. La orden {\tt gzip -d} es equivalente a {\tt gunzip}.
{\tt gzip} es una herramienta relativamente nueva en la comunidad
UNIX. Durante muchos a�os, se utiliz� en su lugar {\tt compress}. Sin
embargo, debido a varios factores, incluyendo una disputa por una
patente de software contra su algoritmo de compresi�n, y el hecho de
que {\tt gzip} es mucho m�s eficiente, {\tt compress} se est�
volviendo anticuado.



\subsection{Aplic�ndolos en conjunto}
Para archivar un grupo de ficheros y comprimir el resultado, use las �rdenes
\begin{tscreen}
\# tar cvf backup.tar /etc \\
\# gzip -9 backup.tar
\end{tscreen}

Como resultado obtendr� {\tt backup.tar.gz}. Para descomprimir este
archivo, use las �rdenes inversas:

\begin{tscreen}
\# gunzip backup.tar.gz \\
\# tar xvf backup.tar
\end{tscreen}

Recordatorio: Siempre hay que estar seguro de encontrarse en el
directorio correcto antes de descomprimir un archivo tar.
Tambi�n se puede usar algunas de las ingeniosidades de {\linux} para
realizar esto, pero en una sola l�nea de ordenes: 
\begin{tscreen}
\# tar cvf - /etc $\mid$ gzip -9c $>$ backup.tar.gz
\end{tscreen}
Aqu�, enviamos el fichero tar a ``{\tt -}'', que representa la salida est�ndar de {\tt
tar}. Esto es canalizado a {\tt gzip}, quien comprime el archivo tar entrante. El producto es guardado en {\tt backup.tar.gz}.
La opci�n {\tt -c} le ordena a {\tt gzip} que env�e su salida a la salida est�ndar, que es reencauzada a {\tt backup.tar.gz}.
Una simple orden para descomprimir este archivo ser�a:
\begin{tscreen}
\# gunzip -c backup.tar.gz $\mid$ tar xvf -
\end{tscreen}
Nuevamente, {\tt gunzip} descomprime el contenido de {\tt backup.tar.gz} y env�a el archivo tar resultante a la salida est�ndar. �sta es canalizada a {\tt tar},
quien lee ``{\tt -}'', lo cual representa, esta vez, la entrada est�ndar de {\tt tar}.

Felizmente, la orden {\tt tar} incluye tambi�n la opci�n {\tt z} que, autom�ticamente realiza los procesos  de comprimir/descomprimir ficheros, e
invoca el programa, usando el algoritmo de compresi�n de {\tt gzip}.
La orden
\begin{tscreen}
\# tar cvfz backup.tar.gz /etc
\end{tscreen}
es equivalente a 
\begin{tscreen}
\# tar cvf backup.tar /etc \\
\# gzip backup.tar
\end{tscreen}
Tal como la orden
\begin{tscreen}
\# tar xvfz backup.tar.Z
\end{tscreen}
puede ser usado en lugar de 
\begin{tscreen}
\# uncompress backup.tar.Z  \\
\# tar xvf backup.tar
\end{tscreen}
Indagando en las p�ginas man se puede obtener mas informaci�n acerca de tar y gzip. 

\chapter{Copias de seguridad}
\section{Introducci\'on}
Las copias de seguridad del sistema son con frecuencia el \'unico mecanismo
de recuperaci\'on que poseen los administradores para restaurar una m\'aquina
que por cualquier motivo -- no siempre se ha de tratar de un pirata que borra
los discos -- ha perdido datos. Por tanto, una correcta pol\'{\i}tica para 
realizar, almacenar y, en caso de ser necesario, restaurar los {\it backups} es
vital en la planificaci\'on de seguridad de todo sistema.\\
\\Asociados a los {\it backups} suelen existir unos problemas de seguridad 
t\'{\i}picos en muchas organizaciones. Por ejemplo, uno de estos problemas es la
no verificaci\'on de las copias realizadas: el administrador ha dise\~nado una
pol\'{\i}tica de copias de seguridad correcta, incluso exhaustiva en muchas
ocasiones, pero nadie se encarga de verificar estas copias\ldots hasta que es
necesario restaurar ficheros de ellas. Evidentemente, cuando llega ese momento
el responsable del sistema se encuentra ante un gran problema, problema que se
podr\'{\i}a haber evitado simplemente teniendo la precauci\'on de verificar el
correcto funcionamiento de los {\it backups}; por supuesto, restaurar una 
copia completa para comprobar que todo es correcto puede ser demasiado trabajo
para los m\'etodos habituales de operaci\'on, por lo que lo que se suele hacer
es tratar de recuperar varios ficheros aleatorios del {\it backup}, asumiendo
que si esta recuperaci\'on funciona, toda la copia es correcta.\\
\\Otro problema cl\'asico de las copias de seguridad es la pol\'{\i}tica de 
etiquetado a seguir. Son pocos los administradores que no etiquetan los 
dispositivos de {\it backup}, algo que evidentemente no es muy \'util: si llega
el momento de recuperar ficheros, el operador ha de ir cinta por cinta (o
disco por disco, o CD-ROM por CD-ROM\ldots) tratando de averiguar d\'onde se
encuentran las \'ultimas versiones de tales archivos. No obstante, muchos 
administradores siguen una pol\'{\i}tica de etiquetado exhaustiva, 
proporcionando todo tipo de detalles sobre el contenido exacto de cada medio;
esto, que en principio puede parecer una posici\'on correcta, no lo es tanto:
si por cualquier motivo un atacante consigue sustraer una cinta, no tiene que
investigar mucho para conocer su contenido exacto, lo que le proporciona 
acceso a informaci\'on muy concreta (y muy valiosa) de nuestros sistemas sin ni 
siquiera penetrar en ellos. La pol\'{\i}tica correcta para etiquetar los {\it
backups} ha de ser tal que un administrador pueda conocer la situaci\'on exacta 
de cada fichero, pero que no suceda lo mismo con un atacante que roba el medio
de almacenamiento; esto se consigue, por ejemplo, con c\'odigos impresos en 
cada etiqueta, c\'odigos cuyo significado sea conocido por los operadores de
copias de seguridad pero no por un potencial atacante.\\
\\La ubicaci\'on final de las copias de seguridad tambi\'en suele ser err\'onea
en muchos entornos; generalmente, los operadores tienden a almacenar los {\it
backups} muy cerca de los sistemas, cuando no en la misma sala. Esto, que se
realiza para una mayor comodidad de los t\'ecnicos y para recuperar ficheros
f\'acilmente, es un grave error: no hay m\'as que imaginar cualquier desastre
del entorno, como un incendio o una inundaci\'on, para hacerse una idea de lo
que les suceder\'{\i}a a los {\it backups} en esos casos. Evidentemente, se
destruir\'{\i}an junto a los sistemas, por lo que nuestra organizaci\'on
perder\'{\i}a toda su informaci\'on; no obstante, existen voces que reivindican
como correcto el almacenaje de las copias de seguridad junto a los propios 
equipos, ya que as\'{\i} se consigue centralizar un poco la seguridad 
(protegiendo una \'unica estancia se salvaguarda tanto las m\'aquinas como las
copias). Lo habitual en cualquier organizaci\'on suele ser un t\'ermino 
medio entre ambas aproximaciones: por ejemplo, podemos tener un juego de copias
de seguridad completas en un lugar diferente a la sala de operaciones, pero
protegido y aislado como esta, y un juego para uso diario en la propia sala,
de forma que los operadores tengan f\'acil la tarea de recuperar ficheros;
tambi\'en podemos utilizar armarios ign\'{\i}fugos que requieran de ciertas 
combinaciones para su apertura (combinaciones que s\'olo determinado personal 
ha de conocer), si decidimos almacenar todos los {\it backups} en la misma 
estancia que los equipos.\\
\\Por \'ultimo, >qu\'e almacenar? Obviamente debemos realizar copias de 
seguridad de los
archivos que sean \'unicos a nuestro sistema; esto suele incluir directorios
como {\tt /etc/}, {\tt /usr/local/} o la ubicaci\'on de los directorios de
usuario (dependiendo del Unix utilizado, {\tt /export/home/}, {\tt /users/}, 
{\tt /home/}\ldots). Por supuesto, realizar una copia de seguridad de
directorios como {\tt /dev/} o {\tt /proc/} no tiene ninguna utilidad, de la
misma forma que no la tiene realizar {\it backups} de directorios del sistema
como {\tt /bin/} o {\tt /lib/}: su contenido est\'a almacenado en la 
distribuci\'on original del sistema operativo (por ejemplo, los CD-ROMs que 
utilizamos para instalarlo).
\section{Dispositivos de almacenamiento}
Existen multitud de dispositivos diferentes donde almacenar nuestras copias de
seguridad, desde un simple disco flexible hasta unidades de cinta de \'ultima
generaci\'on. Evidentemente, cada uno tiene sus ventajas y sus inconvenientes,
pero utilicemos el medio que utilicemos, \'este ha de cumplir una norma
b\'asica: ha de ser {\bf est\'andar}. Con toda probabilidad muchos 
administradores pueden presumir de poseer los {\it streamers} m\'as modernos,
con unidades de cinta del tama\~no de una cajetilla de tabaco que son capaces
de almacenar gigas y m\'as gigas de informaci\'on; no obstante, utilizar 
dispositivos de \'ultima generaci\'on para guardar los {\it backups} de nuestros
sistemas puede convertirse en un problema: >qu\'e sucede si necesitamos 
recuperar datos y no disponemos de esa unidad lectora tan avanzada? Imaginemos
simplemente que se produce un incendio y desaparece una m\'aquina, y con ella
el dispositivo que utilizamos para realizar copias de seguridad. En esta 
situaci\'on, o disponemos de otra unidad id\'entica a la perdida, o recuperar
nuestra informaci\'on va a ser algo dif\'{\i}cil. Si en lugar de un dispositivo
moderno, r\'apido y seguramente muy fiable, pero incompatible con el resto,
hubi\'eramos utilizado algo m\'as habitual (una cinta de 8mm., un CD-ROM, o 
incluso un disco duro) no tendr\'{\i}amos problemas en leerlo desde cualquier
sistema Unix, sin importar el {\it hardware} sobre el que trabaja.\\
\\Aqu\'{\i} vamos a comentar algunos de los dispositivos de copia de seguridad
m\'as utilizados hoy en d\'{\i}a; de todos ellos (o de otros, no listados
aqu\'{\i}) cada administrador ha de elegir el que m\'as se adapte a sus 
necesidades. En la tabla \ref{devices} se muestra una comparativa de todos
ellos.\\
\\{\bf Discos flexibles}\\
S\'{\i}, aunque los cl\'asicos {\it diskettes} cada d\'{\i}a se utilicen menos,
a\'un se pueden considerar un dispositivo donde almacenar copias de seguridad.
Se trata de un medio muy barato y portable entre diferentes operativos 
(evidentemente, esta portabilidad existe si utilizamos el disco como un 
dispositivo secuencial, sin crear sistemas de ficheros). Por contra, su 
fiabilidad es muy baja: la informaci\'on almacenada se puede borrar f\'acilmente
si el disco se aproxima a aparatos que emiten cualquier tipo de radiaci\'on,
como un tel\'efono m\'ovil o un detector de metales. Adem\'as, la capacidad de
almacenamiento de los {\it floppies} es muy baja, de poco m\'as de 1 MB por
unidad; esto hace que sea casi imposible utilizarlos como medio de 
{\it backup} de grandes cantidades de datos, restringiendo su uso a ficheros
individuales.\\
\\Un {\it diskette} puede utilizarse creando en \'el un sistema de ficheros,
mont\'andolo bajo un directorio, y copiando en los archivos a guardar. Por
ejemplo, podemos hacer un {\it backup} de nuestro fichero de claves en un disco 
flexible de esta forma.
\begin{quote}
\begin{verbatim}
luisa:~# mkfs -t ext2 /dev/fd0
mke2fs 1.14, 9-Jan-1999 for EXT2 FS 0.5b, 95/08/09
Linux ext2 filesystem format
Filesystem label=
360 inodes, 1440 blocks
72 blocks (5.00%) reserved for the super user
First data block=1
Block size=1024 (log=0)
Fragment size=1024 (log=0)
1 block group
8192 blocks per group, 8192 fragments per group
360 inodes per group

Writing inode tables: done                            
Writing superblocks and filesystem accounting information: done
luisa:~# mount -t ext2 /dev/fd0 /mnt/
luisa:~# cp /etc/passwd /mnt/
luisa:~# umount /mnt/
luisa:~# 
\end{verbatim}
\end{quote}
Si quisi\'eramos recuperar el archivo, no tendr\'{\i}amos m\'as que montar de
nuevo el {\it diskette} y copiar el fichero en su ubicaci\'on original. No 
obstante, este uso de los discos flexibles es minoritario; es m\'as habitual
utilizarlo como un dispositivo secuencial (como una cinta), sin crear en \'el
sistemas de ficheros -- que quiz\'as son incompatibles entre diferentes clones
de Unix -- sino accediendo directamente al dispositivo. Por ejemplo, si de
nuevo queremos hacer un {\it backup} de nuestro fichero de {\it passwords}, pero
siguiendo este modelo de trabajo, podemos utilizar la orden {\tt tar} 
(comentada m\'as adelante) para conseguirlo:
\begin{quote}
\begin{verbatim}
luisa:~# tar cvf /dev/fd0 /etc/passwd
tar: Removing leading `/' from absolute path names in the archive
etc/passwd
luisa:~#
\end{verbatim}
\end{quote}
Para recuperar ahora el archivo guardado, volvemos a utilizar la orden {\tt
tar} indicando como contenedor la unidad de disco correspondiente:
\begin{quote}
\begin{verbatim}
luisa:~# tar xvf /dev/fd0 
etc/passwd
luisa:~#
\end{verbatim}
\end{quote}
{\bf Discos duros}\\
Es posible utilizar una unidad de disco duro completa (o una partici\'on) 
para realizar copias de seguridad; como suced\'{\i}a con los discos flexibles,
podemos crear un sistema de ficheros sobre la unidad o la partici\'on 
correspondiente, montarla, y copiar los ficheros que nos interese guardar en 
ella (o recuperarlos). De la misma forma, tambi\'en podemos usar la unidad como
un dispositivo secuencial y convertirlo en un contenedor {\tt tar} o {\tt cpio};
en este caso hemos de estar muy atentos a la hora de especificar la unidad, ya
que es muy f\'acil equivocarse de dispositivo y machacar completamente la
informaci\'on de un disco completo (antes tambi\'en pod\'{\i}a suceder, pero
ahora la probabilidad de error es m\'as alta). Por ejemplo, si en lugar del
nombre del dispositivo correcto (supongamos {\tt /dev/hdc}) especificamos otro
(como {\tt /dev/hdd}), estaremos destruyendo la informaci\'on guardada en 
este \'ultimo.\\
\\Algo muy interesante en algunas situaciones es utilizar como dispositivo de
copia un disco duro id\'entico al que est\'a instalado en nuestro sistema, y 
del que deseamos hacer el {\it backup}; en este caso es muy sencillo hacer una
copia de seguridad completa. Imaginemos por ejemplo que {\tt /dev/hda} y {\tt 
/dev/hdc} son
dos discos exactamente iguales; en este caso, si queremos conseguir una
imagen especular del primero sobre el segundo, no tenemos m\'as que utilizar
la orden {\tt dd} con los par\'ametros adecuados:
\begin{quote}
\begin{verbatim}
luisa:~# dd if=/dev/hda of=/dev/hdc bs=2048
1523+0 records in
1523+0 records out
luisa:~#
\end{verbatim}
\end{quote}
{\bf Cintas magn\'eticas}\\
Las cintas magn\'eticas han sido durante a\~nos (y siguen siendo en la 
actualidad) el dispositivo de {\it backup} por excelencia. Las m\'as antiguas,
las cintas de nueve pistas, son las que mucha gente imagina al hablar de este
medio: un elemento circular con la cinta enrollada en \'el; este tipo de 
dispositivos se utiliz\'o durante mucho tiempo, pero en la actualidad est\'a en
desuso, ya que a pesar de su alta fiabilidad y su relativa velocidad de trabajo,
la capacidad de este medio es muy limitada (de hecho, las m\'as avanzadas son 
capaces de almacenar menos de 300 MB., algo que no es suficiente en la mayor
parte de sistemas actuales).\\
\\Despu\'es de las cintas de 9 pistas aparecieron las cintas de un cuarto de
pulgada (denominadas {\sc qic}), mucho m\'as peque\~nas en tama\~no que las
anteriores y con una capacidad m\'axima de varios {\it Gigabytes} (aunque la
mayor parte de ellas almacenan menos de un {\it Giga}); se trata de cintas
m\'as baratas que las de 9 pistas, pero tambi\'en m\'as lentas. El medio ya no
va descubierto, sino que va cubierto de una envoltura de pl\'astico.\\
\\A finales de los ochenta aparece un nuevo modelo de cinta que releg\'o a las
cintas {\sc qic} a un segundo plano y que se ha convertido en el medio m\'as
utilizado en la actualidad: se trata de las cintas de 8mm., dise\~nadas en su
origen para almacenar v\'{\i}deo. Estas cintas, del tama\~no de una {\it 
cassette} de audio, tienen una capacidad de hasta cinco {\it Gigabytes}, lo que
las hace perfectas para la mayor\'{\i}a de sistemas: como toda la informaci\'on
a salvaguardar cabe en un mismo dispositivo, el operador puede introducir la
cinta en la unidad del sistema, ejecutar un sencillo {\it shellscript}, y dejar
que el {\it backup} se realice durante toda la noche; al d\'{\i}a siguiente no
tiene m\'as que verificar que no ha habido errores, retirar la cinta de la 
unidad, y etiquetarla correctamente antes de guardarla. De esta forma se
consigue que el proceso de copia de seguridad sea sencillo y efectivo.\\
\\No obstante, este tipo de cintas tiene un grave inconveniente: como hemos 
dicho, originalmente estaban dise\~nadas para almacenar v\'{\i}deo, y se basan
en la misma tecnolog\'{\i}a para registrar la informaci\'on. Pero con una 
importante diferencia (\cite{kn:pep94}): mientras que perder unos {\it bits} de
la cinta donde hemos grabado los mejores momentos de nuestra \'ultima fiesta no 
tiene mucha importancia, si esos mismos {\it bits} los perdemos de una cinta de
{\it backup} el resto de su contenido puede resultar inservible. Es m\'as, es
probable que despu\'es de unos cuantos usos (incluidas las lecturas) la cinta
se da\~ne irreversiblemente. Para intentar solucionar estos problemas 
aparecieron las cintas {\sc dat}, de 4mm., dise\~nadas ya en origen para 
almacenar datos; estos dispositivos, algo m\'as peque\~nos que las cintas de
8mm. pero con una capacidad similar, son el mejor sustituto de las cintas
antiguas: son mucho m\'as resistentes que \'estas, y adem\'as relativamente
baratas (aunque algo m\'as caras que las de 8mm.).\\
\\Hemos dicho que en las cintas de 8mm. (y en las de 4mm.) se pueden almacenar 
hasta 5 GB. de informaci\'on. No obstante, algunos fabricantes 
anuncian capacidades de hasta 14 GB. utilizando compresi\'on {\it hardware}, 
sin dejar muy claro si las cintas utilizadas son est\'andar o no 
(\cite{kn:fri95}); evidentemente, esto puede llevarnos a problemas de los que
antes hemos comentado: >qu\'e sucede si necesitamos recuperar datos y no 
disponemos de la unidad lectora original? Es algo vital que nos aseguremos la
capacidad de una f\'acil recuperaci\'on en caso de p\'erdida de nuestros datos
(este es el objetivo de los {\it backups} al fin y al cabo), por lo que 
quiz\'as no es conveniente utilizar esta compresi\'on {\it hardware} a no ser 
que sea estrictamente necesario y no hayamos podido aplicar otra soluci\'on.\\
\\{\bf CD-ROMs}\\
En la actualidad s\'olo se utilizan cintas magn\'eticas en equipos antiguos
o a la hora de almacenar grandes cantidades de datos -- del orden de {\it
Gigabytes}. Hoy en d\'{\i}a, muchas m\'aquinas Unix poseen unidades grabadoras
de CD-ROM, un {\it hardware} barato y, lo que es m\'as importante, que utiliza
dispositivos de muy bajo coste y con una capacidad de almacenamiento suficiente
para muchos sistemas: con una unidad grabadora, podemos almacenar m\'as de 650
{\it Megabytes} en un CD-ROM que cuesta menos de 150 pesetas. Por estos motivos,
muchos administradores se decantan por realizar sus copias de seguridad en uno
o varios CD-ROMs; esto es especialmente habitual en estaciones de trabajo o en
PCs de sobremesa corriendo alg\'un clon de Unix (Linux, Solaris o FreeBSD por
regla general), donde la cantidad de datos a salvaguardar no es muy elevada y
se ajusta a un par de unidades de CD, cuando no a una sola.\\
\\En el punto \ref{cdrom} se comenta el mecanismo para poder grabar en un
CD-ROM; aunque los ejemplos que comentaremos son b\'asicos, existen multitud
de posibilidades para trabajar con este medio. Por ejemplo, podemos utilizar
dispositivos CD-RW, similares a los anteriores pero que permiten borrar la
informaci\'on almacenada y volver a utilizar el dispositivo (algo muy \'util en 
situaciones donde reutilizamos uno o varios juegos de copias), o
utilizar medios con una mayor capacidad de almacenamiento (CD-ROMs de 80 
minutos, capaces de almacenar hasta 700 MB.); tambi\'en es muy \'util lo
que se conoce como la grabaci\'on multisesi\'on, algo que nos va a permitir
ir actualizando nuestras copias de seguridad con nuevos archivos sin perder la
informaci\'on que hab\'{\i}amos guardado previamente.
\begin{table}
\begin{center}
\begin{tabular}{|c|c|c|c|}
\hline
Dispositivo & Fiabilidad & Capacidad & Coste/MB\\
\hline\hline
{\it Diskette} & Baja & Baja & Alto\\
\hline
CD-ROM & Media & Media & Bajo\\
\hline
Disco duro & Alta & Media/Alta & Medio.\\
\hline
Cinta 8mm. & Media & Alta & Medio.\\
\hline
Cinta DAT & Alta & Alta & Medio.\\
\hline
\end{tabular}
\caption{Comparaci\'on de diferentes medios de almacenamiento secundario.}
\label{devices}
\end{center}
\end{table}
\section{Algunas \'ordenes para realizar copias de seguridad}
Aunque muchos clones de Unix ofrecen sus propias herramientas para realizar
copias de seguridad de todo tipo (por ejemplo, tenemos {\tt mksysb} y {\tt 
savevg/restvg} en AIX, {\tt fbackup} y {\tt frecover} en HP-UX, {\tt bru} en
IRIX, {\tt fsphoto} en SCO Unix, {\tt ufsdump/ufsrestore} en Solaris\ldots), 
casi todas estas herramientas suelen presentar un
grave problema a la hora de recuperar archivos: se trata de {\it software}
propietario, por lo que si queremos restaurar total o parcialmente archivos 
almacenados con este tipo de programas, necesitamos el propio programa para 
hacerlo. En determinadas situaciones, esto no es posible o es muy dif\'{\i}cil: 
imaginemos un departamento que dispone de s\'olo una estaci\'on Silicon 
Graphics corriendo IRIX y pierde todos los datos de un disco, incluida la 
utilidad {\tt bru}; si ha utilizado esta herramienta para realizar {\it 
backups}, necesitar\'a otra estaci\'on con el mismo operativo para poder 
restaurar estas copias, lo que obviamente puede ser problem\'atico.\\
\\Por este motivo, muchos administradores utilizan herramientas est\'andar para
realizar las copias de seguridad de sus m\'aquinas; estas herramientas suelen
ser tan simples como un {\it shellscript} que se planifica para que 
autom\'aticamente haga {\it backups} utilizando \'ordenes como {\tt tar} o {\tt 
cpio}, programas habituales en cualquier clon de Unix y que no presentan 
problemas de interoperabilidad entre diferentes operativos. De esta forma, si en
la estaci\'on Silicon Graphics del ejemplo anterior se hubiera utilizado {\tt
tar} para realizar las copias de seguridad, \'estas se podr\'{\i}an restaurar
sin problemas desde una m\'aquina {\sc sparc} corriendo Solaris, y transferir
los ficheros de nuevo a la Silicon.
\subsection{{\tt dump}/{\tt restore}}
La herramienta cl\'asica para realizar {\it backups} en entornos Unix es desde
hace a\~nos {\tt dump}, que vuelca sistemas de ficheros completos (una 
partici\'on o una partici\'on virtual en los sistemas que las soportan, como
Solaris); {\tt restore} se utiliza para recuperar archivos de
esas copias. Se trata de una utilidad disponible en la mayor\'{\i}a de clones
del sistema operativo\footnote{HP-UX, IRIX, SunOS, Linux\ldots en Solaris se 
llama {\tt ufsdump} y en AIX {\tt backup}.}, potente (no diremos `sencilla') y 
lo m\'as importante: las 
copias son completamente compatibles entre Unices, de forma que por ejemplo 
podemos restaurar un {\it backup} realizado en IRIX en un sistema HP-UX. 
Adem\'as, como veremos luego, la mayor parte de las versiones de {\tt dump} 
permiten realizar copias
de seguridad sobre m\'aquinas remotas directamente desde l\'{\i}nea de \'ordenes
(en el caso que la variante de nuestro sistema no lo permita, podemos utilizar
{\tt rdump}/{\tt rrestore}) sin m\'as que indicar el nombre de m\'aquina 
precediendo al dispositivo donde se ha de realizar la copia.\\
\\La sintaxis general de la orden {\tt dump} es
\tt
\begin{center}
dump opciones argumentos fs
\end{center}
\rm
donde {\tt `opciones'} son las opciones de la copia de seguridad, {\tt 
`argumentos'} son los argumentos de dichas opciones, y {\tt `fs'} es el sistema 
de ficheros a salvaguardar. Se trata de una sintaxis algo peculiar: mientras
que lo habitual en Unix es especificar cada argumento a continuaci\'on de
la opci\'on adecuada (por ejemplo, {\tt `find . -perm 700 -type f'} indica un
argumento {\tt `700'} para la opci\'on {\tt `perm'} y uno {\tt `f'} para {\tt
`type'}), en la orden {\tt dump} primero especificamos toda la lista de
opciones y a continuaci\'on todos sus argumentos; no todas las opciones 
necesitan un argumento, y adem\'as la lista de argumentos tiene que
corresponderse exactamente, en orden y n\'umero, con las opciones que los
necesitan (por ejemplo, si {\tt `find'} tuviera una sintaxis similar, la orden 
anterior se habr\'{\i}a tecleado como {\tt `find . -perm -type 700 f'}). AIX
y Linux son los \'unicos Unices donde la sintaxis de {\tt dump} (recordemos que en el primero se denomina {\tt backup}) es la habitual.\\
\\Las opciones de {\tt `dump'} m\'as utilizadas son las que se muestran en la
tabla \ref{dumpops}; en las p\'aginas {\tt man} de cada clon de Unix se suelen
incluir recomendaciones sobre par\'ametros espec\'{\i}ficos para modelos de
cintas determinados, por lo que como siempre es m\'as que recomendable su
consulta. Fij\'andonos en la tabla, podemos ver que la opci\'on {\tt `u'}
actualiza el archivo {\tt /etc/dumpdates} tras realizar una copia de seguridad
con \'exito; es conveniente que este archivo exista antes de utilizar {\tt dump}
por primera vez (podemos crearlo con la orden {\tt touch}), ya que si no existe
no se almacenar\'a informaci\'on sobre las copias de seguridad de cada sistema
de ficheros (informaci\'on necesaria, por ejemplo, para poder realizar {\it
backups} progresivos). En este archivo {\tt dump} -- la propia orden lo hace,
el administrador no necesita modificar el archivo a mano\ldots y no debe 
hacerlo -- registra informaci\'on de las copias de cada sistema de archivos, su 
nivel, y la fecha de realizaci\'on, de forma que su aspecto puede ser similar 
al siguiente:
\begin{quote}
\begin{verbatim}
anita:~# cat /etc/dumpdates
/dev/dsk/c0d0s6   0 Thu Jun 22 05:34:20 CEST 2000
/dev/dsk/c0d0s7   2 Wed Jun 21 02:53:03 CEST 2000
anita:~#
\end{verbatim}
\end{quote}
\begin{table}
\begin{center}
\begin{tabular}{|c|c|c|}
\hline
Opci\'on & Acci\'on realizada & Argumento\\
\hline\hline
0--9 & Nivel de la copia de seguridad & NO\\
\hline
u & Actualiza {\tt /etc/dumpdates} al finalizar el {\it backup} & NO\\
\hline
f & Indica una cinta diferente de la usada por defecto & S\'I\\
\hline
b & Tama\~no de bloque & S\'I\\
\hline
c & Indica que la cinta destino es un cartucho & NO\\
\hline
W & Ignora todas las opciones excepto el nivel del {\it backup} & NO\\
\hline
\end{tabular}
\caption{Opciones de la orden {\tt dump}}
\label{dumpops}
\end{center}
\end{table}
El uso de {\tt dump} puede ser excesivamente complejo, especialmente en sistemas
antiguos donde es incluso necesario especificar la densidad de la cinta en
{\it bytes} por pulgada o su longitud en pies; no obstante, hoy en d\'{\i}a la 
forma m\'as habitual
de invocar a esta orden es {\tt `dump [1-9]ucf cinta fs'}, es decir, una
copia de seguridad del sistema de ficheros recibido como argumento, de un 
determinado nivel y sobre la unidad de cinta especificada. Por ejemplo para
realizar una copia de seguridad completa sobre la unidad de cinta {\tt 
/dev/rmt} de la partici\'on l\'ogica {\tt /dev/dsk/c0d0s7}, en Solaris podemos
utilizar la orden siguiente (podemos ver que nos muestra mucha informaci\'on
sobre el progreso de nuestra copia de seguridad en cada momento):
\begin{quote}
\begin{verbatim}
anita:~# ufsdump 0cuf /dev/rmt /dev/dsk/c0d0s7
DUMP: Date of this level 0 dump: Thu Jun 22 10:03:28 2000
DUMP: Date of last level 0 dump: the epoch
DUMP: Dumping /dev/dsk/c0d0s7 (/export/home) to /dev/rmt
DUMP: mapping (Pass I) [regular files]
DUMP: mapping (Pass II) [directories]
DUMP: estimated 24523 blocks (118796KB)
DUMP: Writing 63 Kilobyte records
DUMP: dumping (Pass III) [directories]
DUMP: dumping (Pass IV) [regular files]
DUMP: level 0 dump on Thu Jun 22 10:05:31 CEST 2000
DUMP: 24550 blocks (118927KB) on 1 volume
DUMP: DUMP IS DONE
anita:~#
\end{verbatim}
\end{quote}
Para realizar copias remotas, como hemos dicho antes, no tenemos m\'as que
anteponer el nombre del sistema donde deseemos realizar el volcado al nombre
del dispositivo donde se va a almacenar, separado de \'este por el car\'acter
{\tt `:'}; opcionalmente se puede indicar el nombre de usuario en el sistema
remoto, separ\'andolo del nombre de m\'aquina por {\tt `@'}:
\begin{quote}
\begin{verbatim}
anita:~# ufsdump 0cuf toni@luisa:/dev/st0 /dev/dsk/c0d0s7
\end{verbatim}
\end{quote}
Si estamos utilizando {\tt rdump}, hemos de tener definido un nombre de 
m\'aquina denominado {\tt \\`dumphost'} en nuestro archivo {\tt /etc/hosts}, que
ser\'a el sistema donde se almacene la copia remota. De cualquier forma (usemos
{\tt dump}, {\tt ufsdump} o {\tt rdump}), el {\it host} remoto ha de 
considerarnos como una m\'aquina de confianza (a trav\'es de {\tt 
/etc/hosts.equiv} o {\tt .rhosts}), con las consideraciones de seguridad que 
esto implica.\\
\\>C\'omo restaurar los {\it backups} realizados con {\tt dump}? Para esta
tarea se utiliza la utilidad {\tt restore} ({\tt ufsrestore} en Solaris), capaz
de extraer ficheros individuales, directorios o sistemas de archivos completos.
La sintaxis de esta orden es 
\tt
\begin{center}
restore opciones argumentos archivos
\end{center}
\rm
donde {\tt `opciones'} y {\tt `argumentos'} tienen una forma similar a
{\tt `dump'} (es decir, toda la lista de opciones seguida de toda la lista de
argumentos de las mismas, excepto en AIX y Linux, donde la notaci\'on es la
habitual), y {\tt `archivos'} evidentemente representa una
lista de directorios y ficheros para restaurar. En la tabla \ref{restoreops}
se muestra un resumen de las opciones m\'as utilizadas.
\begin{table}
\begin{center}
\begin{tabular}{|c|c|c|}
\hline
Opci\'on & Acci\'on realizada & Argumento\\
\hline\hline
r & Restaura la cinta completa & NO\\
\hline
f & Indica el dispositivo o archivo donde est\'a el {\it backup} & S\'I\\
\hline
i & Modo interactivo & NO\\
\hline
x & Extrae los archivos y directorios desde el directorio actual & NO\\
\hline
t & Imprime los nombres de los archivos de la cinta & NO\\
\hline
\end{tabular}
\caption{Opciones de la orden {\tt restore}}
\label{restoreops}
\end{center}
\end{table}
Por ejemplo, imaginemos que deseamos restaurar varios archivos de un {\it 
backup} guardado en el fichero {\tt `backup'}; en primer lugar podemos consultar
el contenido de la cinta con una orden como la siguiente (en Linux):
\begin{quote}
\begin{verbatim}
luisa:~# restore -t -f backup>contenido
Level 0 dump of /home on luisa:/dev/hda3
Label: none
luisa:~# cat contenido|more
Dump   date: Fri Jun 23 06:01:26 2000
Dumped from: the epoch
         2      .
        11      ./lost+found
     30761      ./lost+found/#30761
     30762      ./lost+found/#30762
     30763      ./lost+found/#30763
     30764      ./lost+found/#30764
     30765      ./lost+found/#30765
     30766      ./lost+found/#30766
     30767      ./lost+found/#30767
      4097      ./ftp
      8193      ./ftp/bin
      8194      ./ftp/bin/compress
      8195      ./ftp/bin/cpio
      8196      ./ftp/bin/gzip
      8197      ./ftp/bin/ls
      8198      ./ftp/bin/sh
      8199      ./ftp/bin/tar
      8200      ./ftp/bin/zcat
     12289      ./ftp/etc
     12290      ./ftp/etc/group
Broken pipe
luisa:~#
\end{verbatim}
\end{quote}
Una vez que conocemos el contenido de la copia de seguridad -- y por tanto el
nombre del archivo o archivos a restaurar -- podemos extraer el fichero que nos
interese con una orden como
\begin{quote}
\begin{verbatim}
luisa:~# restore -x -f backup ./ftp/bin/tar     
You have not read any tapes yet.
Unless you know which volume your file(s) are on you should start
with the last volume and work towards the first.
Specify next volume #: 1
set owner/mode for '.'? [yn] n
luisa:~# ls -l ftp/bin/tar 
---x--x--x   1 root     root       110668 Mar 21  1999 ftp/bin/tar
luisa:~#
\end{verbatim}
\end{quote}
Como podemos ver, la extracci\'on se ha realizado a partir del directorio 
de trabajo actual; si quisi\'eramos extraer archivos en su ubicaci\'on original
deber\'{\i}amos hacerlo desde el directorio adecuado, o, en algunas versiones
de {\tt restore}, especificar dicho directorio en la l\'{\i}nea de \'ordenes.\\
\\Una opci\'on muy interesante ofrecida por {\tt restore} es la posibilidad de
trabajar en modo interactivo, mediante la opci\'on {\tt `i'}; en este modo, al
usuario se le ofrece un {\it prompt} desde el cual puede, por ejemplo, listar
el contenido de una cinta, cambiar de directorio de trabajo o extraer archivos.
El siguiente ejemplo (tambi\'en sobre Linux) ilustra esta opci\'on:
\begin{quote}
\begin{verbatim}
luisa:~# restore -i -f backup
restore > help
Available commands are:
        ls [arg] - list directory
        cd arg - change directory
        pwd - print current directory
        add [arg] - add `arg' to list of files to be extracted
        delete [arg] - delete `arg' from list of files to be extracted
        extract - extract requested files
        setmodes - set modes of requested directories
        quit - immediately exit program
        what - list dump header information
        verbose - toggle verbose flag (useful with ``ls'')
        help or `?' - print this list
If no `arg' is supplied, the current directory is used
restore > ls
.:
ftp/        httpd/      httpsd/     lost+found/ samba/      toni/

restore > add httpd
restore > extract
You have not read any tapes yet.
Unless you know which volume your file(s) are on you should start
with the last volume and work towards the first.
Specify next volume #: 1
set owner/mode for '.'? [yn] n
restore > quit
luisa:~# 
\end{verbatim}
\end{quote}
Como podemos ver, hemos consultado el contenido de la copia de seguridad, 
a\~nadido el directorio {\tt httpd/} a la lista de ficheros a extraer 
(inicialmente vacia), y extra\'{\i}do dicho directorio a partir del actual. 
Este uso de {\tt restore} proporciona una gran comodidad y facilidad de uso, ya 
que las \'ordenes en modo interactivo son muy sencillas.
\subsection{La orden {\tt tar}}
La utilidad {\tt tar} ({\it Tape Archiver}) es una herramienta de f\'acil 
manejo disponible en todas las versiones de Unix que permite volcar ficheros
individuales o directorios completos en un \'unico fichero; inicialmente fu\'e 
dise\~nada para crear archivos de cinta (esto es, para transferir archivos de 
un disco a una cinta magn\'etica y viceversa), aunque en la actualidad casi 
todas sus versiones pueden utilizarse para copiar a cualquier dipositivo o
fichero, denominado `contenedor'. Su principal desventaja es que, bajo ciertas 
condiciones,
si falla una porci\'on del medio (por ejemplo, una cinta) se puede perder toda
la copia de seguridad; adem\'as, {\tt tar} no es capaz de realizar por s\'{\i}
mismo m\'as que copias de seguridad completas, por lo que hace falta un poco
de programaci\'on {\it shellscripts} para realizar copias progresivas o 
diferenciales.\\
\\En la tabla \ref{tarops} se muestran las opciones de {\tt tar} m\'as 
habituales; algunas de ellas no est\'an disponibles en todas las 
versiones de {\tt tar}, por lo que es recomendable consultar la p\'agina 
del manual de esta orden antes de utilizarla. Si la implementaci\'on de {\tt
tar} que existe en nuestro sistema no se ajusta a nuestras necesidades, siempre
podemos utilizar la versi\'on de {\sc gnu} ({\tt http://www.gnu.org/}), quiz\'as
la m\'as completa hoy en d\'{\i}a.
\begin{table}
\begin{center}
\begin{tabular}{|c|c|}
\hline
Opci\'on & Acci\'on realizada\\
\hline\hline
c & Crea un contenedor\\
\hline
x & Extrae archivos de un contenedor\\
\hline
t & Testea los archivos almacenados en un contenedor\\
\hline
r & A\~nade archivos al final de un contenedor\\
\hline
v & Modo {\it verbose}\\
\hline
f & Especifica el nombre del contenedor\\
\hline
Z & Comprime o descomprime mediante {\tt compress/uncompress}\\
\hline
z & Comprime o descomprime mediante {\tt gzip}\\
\hline
p & Conserva los permisos de los ficheros\\
\hline
\end{tabular}
\caption{Opciones de la orden {\tt tar}}
\label{tarops}
\end{center}
\end{table}
En primer lugar debemos saber c\'omo crear contenedores con los archivos 
deseados; por ejemplo, imaginemos que deseamos volcar todo el directorio
{\tt /export/home/} a la unidad de cinta {\tt /dev/rmt/0}. Esto lo conseguimos
con la siguiente orden:
\begin{quote}
\begin{verbatim}
anita:~# tar cvf /dev/rmt/0 /export/home/
\end{verbatim}
\end{quote}
Como podemos ver, estamos especificando juntas las diferentes opciones
necesarias para hacer la copia de seguridad de los directorios de usuario; la
opci\'on {\tt `v'} no ser\'{\i}a necesaria, pero es \'util para ver un listado
de lo que estamos almacenando en la cinta. En muchas situaciones tambi\'en
resulta \'util comprimir la informaci\'on guardada ({\tt tar} no
comprime, s\'olo empaqueta); esto lo conseguir\'{\i}amos con las opciones
{\tt `cvzf'}.\\
\\Si en lugar de (o aparte de) un \'unico directorio con todos sus ficheros y 
subdirectorios quisi\'eramos especificar m\'ultiples archivos (o directorios),
podemos indic\'arselos uno a uno a {\tt tar} en la l\'{\i}nea de comandos; 
as\'{\i} mismo, podemos indicar un nombre de archivo contenedor en lugar de
un dispositivo. Por ejemplo, la siguiente orden crear\'a el fichero {\tt 
/tmp/backup.tar}, que contendr\'a {\tt /etc/passwd} y {\tt /etc/hosts*}: 
\begin{quote}
\begin{verbatim}
anita:~# tar cvf /tmp/backup.tar /etc/passwd /etc/hosts*
tar: Removing leading `/' from absolute path names in the archive
etc/passwd
etc/hosts
etc/hosts.allow
etc/hosts.deny
etc/hosts.equiv
anita:~# 
\end{verbatim}
\end{quote}
Una vez creado el contenedor podemos testear su contenido con la opci\'on
{\tt `t'} para comprobar la integridad del archivo, y tambi\'en para ver qu\'e
ficheros se encuentran en su interior:
\begin{quote}
\begin{verbatim}
anita:~# tar tvf /tmp/backup.tar
-rw-r--r-- root/other      965 2000-03-11 03:41 etc/passwd
-rw-r--r-- root/other      704 2000-03-14 00:56 etc/hosts
-rw-r--r-- root/other      449 2000-02-17 01:48 etc/hosts.allow
-rw-r--r-- root/other      305 1998-04-18 07:05 etc/hosts.deny
-rw-r--r-- root/other      313 1994-03-16 03:30 etc/hosts.equiv
-rw-r--r-- root/other      345 1999-10-13 03:31 etc/hosts.lpd
anita:~# 
\end{verbatim}
\end{quote}
Si lo que queremos es recuperar ficheros guardados en un contenedor 
utilizaremos las opciones {\tt `xvf'} (o {\tt `xvzf'} si hemos utilizado 
compresi\'on con {\tt gzip} a la hora de crearlo). Podemos indicar el archivo
o archivos que queremos extraer; si no lo hacemos, se extraer\'an todos:
\begin{quote}
\begin{verbatim}
anita:~# tar xvf /tmp/backup.tar etc/passwd
etc/passwd
anita:~# tar xvf /tmp/backup.tar
etc/passwd
etc/hosts
etc/hosts.allow
etc/hosts.deny
etc/hosts.equiv
etc/hosts.lpd
anita:~#
\end{verbatim}
\end{quote}
La restauraci\'on se habr\'a realizado desde el directorio de trabajo, creando
en \'el un subdirectorio {\tt etc} con los ficheros correspondientes en su
interior. Si queremos que los ficheros del contenedor sobreescriban a los que ya
existen en el sistema hemos de desempaquetarlo en el directorio adecuado, en
este caso el ra\'{\i}z.
\subsection{La orden {\tt cpio}}
{\tt cpio} ({\it Copy In/Out}) es una utilidad que permite copiar archivos a 
o desde un contenedor {\it cpio}, que no es m\'as que un fichero que almacena
otros archivos e informaci\'on sobre ellos (permisos, nombres, 
propietario\ldots). Este contenedor puede ser un disco, otro archivo, una
cinta o incluso una tuber\'{\i}a, mientras que los ficheros a copiar pueden
ser archivos normales, pero tambi\'en dispositivos o sistemas de ficheros
completos.\\
\\En la tabla \ref{cpioops} se muestran las opciones de {\tt cpio} m\'as
utilizadas; la sintaxis de esta orden es bastante m\'as confusa que la de {\tt 
tar} debido a la interpretaci\'on de lo que {\tt cpio} entiende por {\it 
`dentro'} y {\it `fuera'}: copiar {\it `fuera'} es generar un contenedor en
salida est\'andar (que con toda probabilidad desearemos redireccionar), 
mientras que copiar {\it `dentro'} es lo contrario, es decir, extraer archivos
de la entrada est\'andar (tambi\'en es seguro que deberemos redireccionarla).\\
\begin{table}
\begin{center}
\begin{tabular}{|c|c|}
\hline
Opci\'on & Acci\'on realizada\\
\hline\hline
o & Copiar `fuera' ({\it out})\\
\hline
i & Copiar `dentro' ({\it in})\\
\hline
m & Conserva fecha y hora de los ficheros\\
\hline
t & Crea tabla de contenidos\\
\hline
A & A\~nade ficheros a un contenedor existente\\
\hline
v & Modo {\it verbose}\\
\hline
\end{tabular}
\caption{Opciones de la orden {\tt cpio}.}
\label{cpioops}
\end{center}
\end{table}
\\Por ejemplo, si deseamos copiar los archivos de {\tt /export/home/} en el 
fichero contenedor\\ {\tt /tmp/backup.cpio}
podemos utilizar la siguiente sintaxis:
\begin{quote}
\begin{verbatim}
anita:~# find /export/home/ |cpio -o > /tmp/backup.cpio
\end{verbatim}
\end{quote}
Como podemos ver, {\tt cpio} lee la entrada est\'andar esperando los nombres
de ficheros a guardar, por lo que es conveniente utilizarlo tras una 
tuber\'{\i}a pas\'andole esos nombres de archivo. Adem\'as, hemos de redirigir
su salida al nombre que queramos asignarle al contenedor, ya que de lo 
contrario se mostrar\'{\i}a el resultado en salida est\'andar (lo que 
evidentemente no es muy utilizado para realizar {\it backups}). Podemos fijarnos
tambi\'en en que estamos usando la orden {\tt `find'} en lugar de un simple
{\tt `ls'}: esto es debido a que {\tt `ls'} mostrar\'{\i}a s\'olo el nombre de
cada fichero (por ejemplo, {\tt `passwd'}) en lugar de su ruta completa ({\tt
`/etc/passwd'}), por lo que {\tt cpio} buscar\'{\i}a dichos ficheros a partir
del directorio actual.\\
\\Una vez creado el fichero contenedor quiz\'as resulte interesante chequear su
contenido, con la opci\'on {\tt `t'}. Por ejemplo, la siguiente orden mostrar\'a
en pantalla el contenido de {\tt /tmp/backup.cpio}:
\begin{quote}
\begin{verbatim}
anita:~# cpio -t < /tmp/backup.cpio
\end{verbatim}
\end{quote}
Igual que para almacenar ficheros en un contenedor hemos de pasarle a {\tt cpio}
la ruta de los mismos, para extraerlos hemos de hacer lo mismo; si no indicamos
lo contrario, {\tt cpio -i} extraer\'a todos los archivos de un contenedor, pero
si s\'olo nos interesan algunos de ellos podemos especificar su nombre de la
siguiente forma:
\begin{quote}
\begin{verbatim}
anita:~# echo "/export/home/toni/hola.tex" |cpio -i </tmp/backup.cpio
\end{verbatim}
\end{quote}
Para conocer m\'as profundamente el funcionamiento de {\tt cpio}, as\'{\i}
como opciones propias de cada implementaci\'on, es indispensable consultar la
p\'agina del manual de esta orden en cada clon de Unix donde vayamos a 
utilizarla.
\subsection{{\it Backups} sobre CD-ROM} 
\label{cdrom}
Como antes hemos dicho, cada vez es m\'as com\'un que se realicen copias de
seguridad sobre discos compactos; en estos casos no se suelen utilizar las
aplicaciones vistas hasta ahora ({\tt tar} o {\tt cpio}), sino que se necesita
un {\it software} dedicado: aqu\'{\i} vamos a comentar las nociones 
m\'as b\'asicas para poder crear {\it backups} sobre este medio.
Para poder grabar una copia de seguridad en un CD-ROM necesitamos en primer
lugar que el n\'ucleo del sistema operativo reconozca nuestra grabadora como 
tal; si se trata
de una IDE, y dependiendo del clon de Unix utilizado, quiz\'as sea necesario
modificar el {\it kernel}, ya que el acceso que los diferentes programas 
realizan al dispositivo se efectua a trav\'es de un interfaz SCSI del n\'ucleo.
Es necesario consultar la documentaci\'on y la lista de compatibilidad {\it
hardware} para cada Unix particular.\\
\\Si asumimos que el reconocimiento del dispositivo es correcto, lo que 
necesitamos a continuaci\'on es {\it software} capaz de grabar un CD-ROM. Por
un lado es necesario un programa para crear im\'agenes ISO, el `molde' de lo
que ser\'a el futuro CD-ROM; el m\'as conocido es sin duda {\tt mkisofs}. 
Adem\'as necesitaremos un programa para realizar lo que es la grabaci\'on en
s\'{\i}, como {\tt cdrecord}. De esta forma lo primero que generaremos es una
imagen de los ficheros a grabar, imagen que a continuaci\'on pasaremos al
CD-ROM; por ejemplo, si queremos hacer un {\it backup} de {\tt /export/home/},
en primer lugar utilizaremos {\tt mkisofs} para crear una imagen con todos
los ficheros y subdirectorios de los usuarios:
\begin{quote}
\begin{verbatim}
anita:~# mkisofs -a -R -l -o /mnt/imagen.iso /export/home/
\end{verbatim}
\end{quote}
Con esta orden hemos creado una imagen ISO denominada {\tt /mnt/imagen.iso} y
que contiene toda la estructura de directorios por debajo de {\tt 
/export/home/}; con las diferentes opciones hemos indicado que se almacenen
todos los ficheros, que se sigan los enlaces simb\'olicos y que se registre
adem\'as informaci\'on sobre los permisos de cada archivo. Una vez que tenemos
esta imagen (que en los Unices con soporte para sistemas de ficheros {\it loop}
podremos montar como si se tratara de una partici\'on, para a\~nadir, borrar,
modificar\ldots ficheros antes de la grabaci\'on) hemos de pasarla a un CD-ROM,
por ejemplo mediante {\tt cdrecord}:
\begin{quote}
\begin{verbatim}
anita:~# cdrecord dev=0,1,0 fs=16m /mnt/imagen.iso
\end{verbatim}
\end{quote}
Con esta orden le hemos indicado al sistema la ubicaci\'on de nuestra grabadora,
as\'{\i} como un {\it buffer} de grabaci\'on de 16MB y tambi\'en la ubicaci\'on
de la imagen ISO.\\
\\Algo muy interesante es la posibilidad de grabar sin necesidad de crear 
primero im\'agenes con los ficheros que queremos meter en un CD-ROM; esto nos
ahorrar\'a tiempo (y sobre todo, espacio en disco) a la hora de realizar 
copias de seguridad, adem\'as de permitir una mayor automatizaci\'on del 
proceso. Para ello, debemos calcular con {\tt mkisofs} el espacio que ocupan los
ficheros a grabar (con la opci\'on {\tt `-print-size'}), y posteriormente
pasarle este valor a {\tt cdrecord}; podemos hacerlo de forma autom\'atica,
por ejemplo tal y como muestra el siguiente programa:
\begin{quote}
\begin{verbatim}
anita:~# cat `which graba-cd`
#!/bin/sh
# Vuelca el directorio pasado como parametro, y todos sus descendientes,
# en un CD-ROM
MKISOFS=/usr/local/bin/mkisofs
CDRECORD=/usr/local/bin/cdrecord
if (test $# -lt 1); then
        echo "Usage: $0 /files"
        exit
fi
size=`$MKISOFS -r -J -l -print-size -f $1 2>&1|tail -1|awk '{print $8}'` 
nice --20 $MKISOFS -r -J -l -f $1 | nice --20 $CDRECORD dev=0,1,0 fs=16m\
 tsize=$size*2048 -eject -
anita:~#
\end{verbatim}
\end{quote}
Como vemos, se asigna el tama\~no de los datos a grabar a la variable {\tt 
`size'}, y despu\'es se pasa este n\'umero a {\tt cdrecord}; de esta forma, 
para 
realizar una copia de seguridad de un directorio como {\tt /export/home/toni/},
no tenemos m\'as que ejecutar el {\it shellscript} pas\'andole el nombre de
este directorio como par\'ametro.
\section{Pol\'{\i}ticas de copias de seguridad}
% Copias completas, incrementales o progresivas, y diferenciales
% Hablar de como recuperar, tiempo empleado y tal...
La forma m\'as elemental de realizar una copia de seguridad consiste simplemente
en volcar los archivos a salvaguardar a un dispositivo de {\it backup}, con
el procedimiento que sea; por ejemplo, si deseamos guardar todo el contenido
del directorio {\tt /export/home/}, podemos empaquetarlo en un archivo, 
comprimirlo y a continuaci\'on almacenarlo en una cinta:
\tt
\begin{quote}
\begin{verbatim}
anita:~# tar cf backup.tar /export/home/ 
anita:~# compress backup.tar
anita:~# dd if=backup.tar.Z of=/dev/rmt/0
\end{verbatim}
\end{quote}
\rm
Si en lugar de una cinta quisi\'eramos utilizar otro disco duro, por ejemplo
montado en {\tt /mnt/}, podemos simplemente copiar los ficheros deseados:
\tt
\begin{quote}
\begin{verbatim}
anita:~# cp -rp /export/home/  /mnt/
\end{verbatim}
\end{quote}
\rm
Esta forma de realizar {\it backups} volcando en el dispositivo de copia los
archivos o directorios deseados se denomina copia de seguridad {\bf completa} o
de nivel 0. Unix utiliza el concepto de {\bf nivel de copia de seguridad} para
distinguir diferentes tipos de {\it backups}: una copia de cierto nivel 
almacena los archivos modificados desde el \'ultimo {\it backup} de nivel 
inferior. As\'{\i}, las copias completas son, por definici\'on, las de nivel 0;
las copias de nivel 1 guardan los archivos modificados desde la \'ultima copia
de nivel 0 (es decir, desde el \'ultimo {\it backup} completo), mientras que
las de nivel 2 guardan los archivos modificados desde la \'ultima copia de
nivel 1, y as\'{\i} sucesivamente (en realidad, el nivel m\'aximo utilizado en
la pr\'actica es el 2).\\
\\Como hemos dicho, las copias completas constituyen la pol\'{\i}tica m\'as 
b\'asica para realizar {\it 
backups}, y como todas las pol\'{\i}ticas tiene ventajas e inconvenientes;
la principal ventaja de las copias completas es su facilidad de realizaci\'on
y, dependiendo del mecanismo utilizado, la facilidad que ofrecen para restaurar
ficheros en algunas situaciones: si nos hemos limitado a copiar una serie de 
directorios a otro disco
y necesitamos restaurar cierto archivo, no tenemos m\'as que montar el disco
de {\it backup} y copiar el fichero solicitado a su ubicaci\'on original.\\
\\Sin embargo, las copias completas presentan graves inconvenientes; uno de 
ellos es la dificultad para restaurar ficheros si utilizamos m\'ultiples 
dispositivos de copia de seguridad (por ejemplo, varias cintas). Otro 
inconveniente, m\'as importante, de las copias de nivel 0 es la cantidad de 
recursos que consumen, tanto en tiempo como en {\it hardware}; para solucionar
el problema de la cantidad de recursos utilizados aparece el concepto de
copia de seguridad incremental. Un {\it backup} {\bf incremental} o {\bf 
progresivo} consiste en
copiar s\'olamente los archivos que han cambiado desde la realizaci\'on de 
otra copia (incremental o total). Por ejemplo, si hace una semana realizamos un
{\it backup} de nivel 0 en nuestro sistema y deseamos una copia incremental
con respecto a \'el, hemos de guardar los ficheros modificados en los \'ultimos
siete d\'{\i}as (copia de nivel 1); podemos localizar estos ficheros con 
la orden {\tt find}:
\tt
\begin{quote}
\begin{verbatim}
anita:~# find /export/home/ -mtime 7 -print
\end{verbatim}
\end{quote}
\rm
Si hace un d\'{\i}a ya realizamos una 
copia incremental y ahora queremos hacer otra copia progresiva con respecto a 
ella, hemos de almacenar \'unicamente los archivos modificados en las \'ultimas 
24 horas (copia de nivel 2); como antes, podemos utilizar {\tt find} para 
localizar los archivos modificados en este intervalo de tiempo:
\tt
\begin{quote}
\begin{verbatim}
anita:~# find /export/home/ -mtime 1 -print
\end{verbatim}
\end{quote}
\rm
Esta pol\'{\i}tica de realizar copias de seguridad sobre la \'ultima progresiva 
se denomina de copia de seguridad {\bf diferencial}.\\
\\La principal ventaja de las copias progresivas es que requieren menos tiempo
para ser realizadas y menos capacidad de almacenamiento que las completas; sin
embargo, como desventajas tenemos que la restauraci\'on de ficheros puede ser
m\'as compleja que con las copias de nivel 0, y tambi\'en que un solo fallo en 
uno de los dispositivos de almacenamiento puede provocar la p\'erdida de gran 
cantidad de archivos; para restaurar completamente un sistema, debemos restaurar
la copia m\'as reciente de {\bf cada} nivel, en orden, comenzando por la de
nivel 0. De esta forma, parece l\'ogico que la estrategia seguida 
sea un t\'ermino medio entre las vistas aqu\'{\i}, una pol\'{\i}tica de copias 
de seguridad que mezcle el enfoque completo y el progresivo: una estrategia
muy habitual, tanto por su simpleza como porque no requiere mucho {\it 
hardware} consiste en realizar peri\'odicamente copias de seguridad de nivel
0, y entre ellas realizar ciertas copias progresivas de nivel 1. Por ejemplo,
imaginemos un departamento que decide realizar cada domingo una copia de 
seguridad completa de sus directorios de usuario y de {\tt /etc/}, y una 
progresiva sobre ella, pero s\'olo de los directorios de usuario, cada d\'{\i}a 
lectivo de la semana. Un {\it shellscript} que realize esta tarea puede ser el 
siguiente:
\tt
\begin{quote}
\begin{verbatim}
#!/bin/sh
DIA=`date +%a`    # Dia de la semana
DIREC="/tmp/backup/"  # Un directorio para hacer el backup

hazback () {
    cd $DIREC
    tar cf backup.tar $FILES
    compress backup.tar
    dd if=backup.tar.Z of=/dev/rmt/0
    rm -f backup.tar.Z
}

if [ ! -d $DIREC ]; 
    then
        # No existe $DIREC
        mkdir -p $DIREC
        chmod 700 $DIREC  # Por seguridad
    else 
        rm -rf $DIREC
        mkdir -p $DIREC
        chmod 700 $DIREC
    fi;
case $DIA in
    "Mon") 
        # Lunes, progresiva
        FILES=`find /export/home/ -mtime 1 -print`
        hazback
        ;;	
    "Tue") 
        # Martes, progresiva
        FILES=`find /export/home/ -mtime 2 -print`
        hazback
        ;;	
    "Wed") 
        # Miercoles, progresiva
        FILES=`find /export/home/ -mtime 3 -print`
        hazback
        ;;	
    "Thu") 
        # Jueves, progresiva
        FILES=`find /export/home/ -mtime 4 -print`
        hazback
        ;;	
    "Fri") 
        # Viernes, progresiva
        FILES=`find /export/home/ -mtime 5 -print`
        hazback
        ;;	
    "Sat")
        # Sabado, descansamos...
        ;;
    "Sun")
        # Domingo, copia completa de /export/home y /etc
        FILES="/export/home/ /etc/"
        hazback
        ;;
esac
\end{verbatim}
\end{quote}
\rm
Este programa determina el d\'{\i}a de la semana y en funci\'on de \'el realiza
-- o no, si es s\'abado -- una copia de los ficheros correspondientes (n\'otese
el uso de las comillas inversas en la orden {\tt find}). Podr\'{\i}amos 
automatizarlo mediante la facilidad {\tt cron} de nuestro sistema para que
se ejecute, por ejemplo, cada d\'{\i}a a las tres del mediod\'{\i}a (una hora
en la que la actividad del sistema no ser\'a muy alta); de esta forma, como
administradores, s\'olo deber\'{\i}amos preocuparnos por cambiar las cintas
cada d\'{\i}a, y dejar una preparada para el fin de semana. Si decidimos
planificarlo para que se ejecute de madrugada, hemos de tener en cuenta que el
{\it backup} de un lunes de madrugada, antes de llegar al trabajo, puede 
sobreescribir el completo, realizado el domingo de madrugada, por lo que 
habr\'{\i}a que modificar el {\it shellscript}; tambi\'en hemos de estar atentos
a situaciones inesperadas, como que no existan archivos a copiar o que nuestro
sistema no disponga del suficiente disco duro para almacenar temporalmente la
copia.\\
\\El medio de almacenamiento tambi\'en es importante a la hora de dise\~nar una
pol\'{\i}tica de copias de seguridad correcta. Si se trata de dispositivos 
baratos, como los CD-ROMs, no suele haber muchos problemas: para cada volcado
(sea del tipo que sea) se utiliza una unidad diferente, unidad que adem\'as no
se suele volver a utilizar a no ser que se necesite recuperar los datos; el
uso de unidades regrabables en este caso es minoritario y poco recomendable, por
lo que no vamos a entrar en \'el. No obstante, algo muy diferente son los medios
de almacenamiento m\'as caros, generalmente las cintas magn\'eticas; al ser
ahora el precio algo a tener m\'as en cuenta, lo habitual es reutilizar 
unidades, sobreescribir las copias de seguridad m\'as antiguas con otras m\'as
actualizadas. Esto puede llegar a convertirse en un grave problema si por
cualquier motivo reutilizamos cintas de las que necesitamos recuperar 
informaci\'on; aparte del desgaste f\'{\i}sico del medio, podemos llegar a 
extremos en los que se pierda toda la informaci\'on guardada: imaginemos, por
ejemplo, que s\'olo utilizamos una cinta de 8mm. para crear {\it backups} del
sistema: aunque habitualmente todo funcione correctamente (se cumple de forma
estricta la pol\'{\i}tica de copias, se verifican, se almacenan en un lugar
seguro\ldots), puede darse el caso de que durante el proceso de copia se 
produzca un incendio en la sala de operaciones, incendio que destruir\'a tanto
nuestro sistema como la cinta donde guardamos su {\it backup}, con lo que 
habremos perdido {\bf toda} nuestra informaci\'on. Aunque este es un ejemplo
quiz\'as algo extremo, podemos pensar en lugares donde se utilicen de forma
incorrecta varios juegos de copias o en situaciones en las que el sistema se
corrompe (no ha de tratarse necesariamente de algo tan poco frecuente como un
incendio, sino que se puede tratar de un simple corte de fluido el\'ectrico que
da\~ne los discos); debemos asegurarnos siempre de que podremos recuperar con
una probabilidad alta la \'ultima copia de seguridad realizada sobre cada 
archivo importante de nuestro sistema, especialmente sobre las bases de datos.

% Linux Installation and Getting Started    -*- TeX -*-
% upgrade.tex
% Copyright (c) 1992, 1993 by Matt Welsh <mdw@sunsite.unc.edu>
%
% This file is freely redistributable, but you must preserve this copyright 
% notice on all copies, and it must be distributed only as part of "Linux 
% Installation and Getting Started". This file's use is covered by the 
% copyright for the entire document, in the file "copyright.tex".
%
% Copyright (c) 1998 by Specialized Systems Consultants Inc. 
% <ligs@ssc.com>
%
% Traducci�n al espa�ol:
% Sebasti�n Gurin, <Cancerbero>, anon@adinet.com.uy
% Traducido el 28/02/01  
% Revisado el 7/7/2002 por Francisco Javier Fernandez <serrador@arrakis.es>
% La revisi�n exige la contrastaci�n intensiva con los originales para resolver ciertos pasajes dudosos o de dif�cil soluci�n.

\section{Actualizando e instalando software nuevo}
\markboth{Administraci�n del sistema}{Actualizando e instalando software nuevo}
\label{sec-sysadm-upgrade}

\index{software!actualizar|(}
\index{software!instalar|(}
Otra de las responsabilidades  del administrador del sistema, es la actualizaci�n e instalaci�n de nuevo software. 

El desarrollo del sistema {\linux} es r�pido. Cada pocas semanas aparecen versiones nuevas del
n�cleo, y los dem�s programas se actualizan casi tan a menudo. Por esto, los usuarios nuevos de {\linux},
sienten la necesidad de actualizar sus sistemas constantemente, para
mantenerse, as�, al r�pido paso de los cambios. Esto es innecesario y
una p�rdida de tiempo: si estuvieras todo el tiempo siguiendo el ritmo
de los cambios que ocurren en el mundo de GNU/Linux, se gastar�a todo el
tiempo actualizando y nada del tiempo usando el sistema. 

Algunas personas consideran que se deber�a actualizar el sistema,
solamente cuando una nueva distribuci�n es mostrada al p�blico; por
ejemplo, cuando se presenta una nueva versi�n de Slackware. Entonces,
muchos usuarios de {\linux}, a la hora de actualizar sus sistemas,
reinstalan todo el software, usando la distribuci�n Slackware m�s nueva. 

La mejor manera de actualizar el sistema, depende del tipo de
distribuci�n que se posea. Debian\tm, S.u.S.E.\tm, Caldera\tm y Red Hat
Linux\tm tienen, todos, gestores inteligentes de paquetes de software,
los cuales permiten realizar las actualizaciones mucho m�s f�cilmente,
instalando paquetes nuevos. Por ejemplo, el compilador de C, {\tt gcc},
viene en un paquete binario, pre-compilado. Cuando se instala,
todos los ficheros de la versi�n antigua se sobreecriben o se eliminan. 

Como casi siempre pasa, actualizar insensatamente para "mantenerse a
la moda", no es importante en absoluto. �Esto no es MS-DOS o Microsoft
Windows!. No existe ninguna raz�n importante, para usar la versi�n m�s
reciente de todo el software. Ahora bien, si se siente que se quieren o 
necesitan caracter�sticas que una versi�n nueva ofrece, entonces hay 
que actualizar. Si no, no actualice. En otras palabras, actualizar s�lo
lo que se deba, cuando se deba. No actualizar s�lo por actualizar. 
Esto consume mucho tiempo y esfuerzo. 

\subsection{Actualizando el n�cleo}
\index{n�cleo!actualizar}

Actualizar el n�cleo es s�lo un asunto de obtener las fuentes del n�cleo y
compilarlas. Esto es generalmente un proceso sin dificultad; 
sin embargo, uno puede tener problemas si trata de actualizar a
un n�cleo en desarrollo, o actualizarlo a una nueva versi�n. 
La versi�n de un n�cleo tiene dos partes: la
versi�n del n�cleo, y el nivel del parche. Cuando esto fue escrito, la
�ltima versi�n estable del n�cleo era la {\tt 2.0.33}. Los n�meros
{\tt 2.0} representan la versi�n del n�cleo, y los n�meros {\tt 33} es
el nivel del parche. Las versiones del n�cleo se�aladas con n�meros
impares, por ejemplo {\tt 2.1} son n�cleos en desarrollo. Mant�ngase
lejos de este tipo de n�cleos, �a menos que quiera vivir peligrosamente!
Como regla general, uno deber�a ser capaz
de actualizar su n�cleo f�cilmente a otro nivel de parche; sin
embargo, actualizar a una nueva versi�n requiere, a su vez, la
actualizaci�n de las utilidades del sistema  que interact�an �ntimamente con el n�cleo. 

\index{n�cleo!fuentes del}
El c�digo fuente del n�cleo Linux puede ser obtenido de cualquiera
de los servidores FTP de {\linux}, (ver la p�gina~\pageref{app-ftp} para una lista de ellos).
En {\tt sunsite.unc.edu}, por ejemplo, las fuentes del n�cleo se encuentran en 
{\tt /pub/Linux/kernel}, organizado
en subdirectorios por n�mero de versi�n. 

El c�digo fuente del n�cleo es publicado en un fichero tar comprimido
con gzip. Por ejemplo, el fichero que contiene el c�digo fuente del
n�cleo 2.0.33 es {\tt linux-2.0.33.tar.gz}

Las fuentes del n�cleo deber�n descomprimirse y desempaquetarse en
el directorio {\tt /usr/src}, creando el directorio {\tt
  /usr/src/linux}. Es una costumbre com�n que, {\tt /usr/src/linux} sea
un enlace blando a otro directorio que contenga el n�mero de versi�n
del n�cleo, tal como {\tt /usr/src/GNU/Linux-2.0.33}. De esta manera, se
podr�n instalar nuevos c�digos fuente y verificar su correcto
funcionamiento, antes de eliminar los fuentes del n�cleo antiguo. Las
ordenes para crear el enlace al directorio donde se aloja el c�digo
fuente del n�cleo son:

\begin{tscreen}
\# cd /usr/src \\
\# mkdir linux-2.0.33 \\
\# rm -r linux \\
\# ln -s linux-2.0.33 linux \\
\# tar xzf linux-2.0.33.tar.gz
\end{tscreen}

Cuando se actualiza a un nuevo nivel de parche de la misma versi�n del
n�cleo, un fichero de nivel de parche puede resultar en un ahorro de 
tiempo en la transferencia de ficheros, puesto que las fuentes del 
n�cleo tienen un tama�o alrededor de los 7 Mb. tras ser comprimidas 
con {\tt gzip}. Para actualizar del
n�cleo 2.0.31 al n�cleo 2.0.33, habr�a que descargar los parches
{\tt patch-2.0.32.gz} y {\tt patch-2.0.33.gz}, los cuales pueden encontrarse 
en el mismo servidor FTP de las fuentes del n�cleo. Tras haber
ubicado los parches en el directorio {\tt /usr/src/}, se deben
aplicar en las fuentes del n�cleo, uno tras otro para actualizar el
c�digo fuente. Una forma de hacer esto ser�a

\begin{tscreen}
\# cd /usr/src \\
\# gzip -cd patch-2.0.32.gz $\mid$ patch -p0 \\
\# gzip -cd patch-2.0.33.gz $\mid$ patch -p0
\end{tscreen}

Despu�s de desempaquetar los fuentes y de aplicar los parches,
necesitar� asegurarse de que existan tres enlaces
simb�licos en {\tt /usr/include}, los cu�les son justo los que
necesita el n�cleo de su distribuci�n. Para crear dichos enlaces,
se podr�n usar las �rdenes
\begin{tscreen}
\# cd /usr/include \\
\# rm -rf asm linux scsi \\
\# ln -s /usr/src/linux/include/asm-i386 asm \\
\# ln -s /usr/src/linux/include/linux linux \\
\# ln -s /usr/src/linux/include/scsi scsi
\end{tscreen}

Despu�s de que haya creado los enlaces, no existe ninguna raz�n para
que deba crearlos nuevamente la pr�xima vez que se instale el siguiente
parche, o una nueva versi�n del n�cleo. (Para m�s informaci�n sobre
enlaces simb�licos: ver secci�n~\ref{sec-manage-links})


\index{n�cleo!compilaci�n}
A fin de compilar el n�cleo, habr� que tener el compilador de C {\tt
  gcc} instalado en su sistema. Para compilar la versi�n 2.0 del
n�cleo, se requiere el {\tt gcc}, versi�n 2.6.3 o m�s reciente. 

Primero cambie de directorio a  {\tt /usr/src/linux}. La orden
{\tt make config} ir� preguntando por un n�mero de opciones de
configuraci�n. �ste es el paso d�nde se selecciona el hardware al que
el n�cleo podr� dar soporte. La equivocaci�n m�s grande a evitar, es
no incluir soporte parar el/los controlador/es del/los disco/s
duro/s. Sin el correcto soporte para el disco duro en el n�cleo, el
sistema ni siquiera se iniciar�. Si en el proceso, se siente inseguro
sobre lo que significa una de las opciones del n�cleo, est� disponible
una corta descripci�n pulsando \key{?} y \key{Enter}

Lo siguiente ser� ejecutar la orden {\tt make dep} para actualizar
todas las dependencias del c�digo fuente. �ste es, tambi�n, un paso
importante. {\tt make clean} eliminar� los ficheros binarios antiguos
del �rbol del n�cleo. 

\index{n�cleo!compilando una imagen comprimida}

La instrucci�n {\tt make zImage} compila el n�cleo y lo escribe en el
fichero {\tt /usr/src/linux/arch/i386/boot/zImage}. Los n�cleos de
Linux en los sistemas Intel, est�n siempre comprimidos. Algunas veces,
el n�cleo que se desea compilar es demasiado grande para ser comprimido
por el sistema de compresi�n que usa {\tt make zImage}. Un n�cleo
excesivamente grande para dicho sistema de compresi�n, retornar�
del proceso de compilaci�n del n�cleo con el siguiente mensaje de
error: {\tt Kernel Image Too Large}. Si esto llegara a pasar, se
debe tratar con la orden {\tt make bzImage}. Esta orden usa un
sistema de compresi�n que soporta los n�cleos grandes. El n�cleo ser�
escrito en {\tt /usr/src/linux/arch/i386/boot/bzImage}.

Una vez que se tenga el n�cleo compilado, se podr� copiar a un disquete
de arranque, (por ejemplo, con la orden ``{\tt cp zImage
  /dev/fd0}''), o se podr� instalar la imagen, y as�, LILO iniciar� el
sistema desde el disco duro. Para m�s informaci�n, ver la p�gina~\pageref{sec-lilo}. 

\subsection{Agregando un controlador de dispositivo al n�cleo}
\label{kernel-ppa-driver}
\index{n�cleo!arraglar controlador de dispositivo}

La p�gina~\pageref{sec-zip-backup} describe c�mo usar una unidad Zip
Iomega, para efectuar copias de seguridad. El soporte para este tipo
de unidades, como para muchos otros dispositivos, no son generalmente
compilados en los n�cleos comunes y corrientes de las distribuciones {\linux}
---la variedad de dispositivos es simplemente demasiado extensa como
para poder respaldarlos a todos en un s�lo n�cleo utilizable. No
obstante, el c�digo fuente para el dispositivo de la unidad Zip en
puerto paralelo, est� incluido como una parte de c�digo fuente del
n�cleo de la distribuci�n. Esta secci�n describe c�mo agregar el
soporte para una unidad de puerto paralelo Iomega Zip, y c�mo hacer
para que conviva con una impresora conectada a otro puerto paralelo.

Para esto, usted deber� tener instalado, y haber construido
exitosamente un n�cleo, como el descrito en la secci�n anterior. 

El poder elegir un dispositivo unidad Zip {\tt ppa}, como una de las
opciones del n�cleo, requiere que se conteste {\tt Y} a las
respuestas apropiadas, durante el proceso {\tt make config}, o sea,
cuando se determina la configuraci�n del n�cleo a construir. En
particular, el dispositivo {\tt ppa}, requiere que se conteste
``{\tt Y}'' a tres opciones:

\begin{tscreen}
SCSI support? [Y/n/m] Y \\
SCSI disk support? [Y/n/m] Y \\
IOMEGA Parallel Port Zip Drive SCSI support? [Y/n/m] Y
\end{tscreen}

Despu�s de haber ejecutado exitosamente {\tt make config}, con todas
las opciones que quiere incluir en su n�cleo, ejecutar {\tt make dep},
{\tt make clean}, y {\tt make zImage}, para construirlo. Adem�s, hay que
decirle al n�cleo, de qu� manera instalar el controlador. Esto se
efect�a a trav�s de una l�nea de ordenes al LILO. Como se ha
descrito en la secci�n~\ref{sec-lilo}, el archivo de configuraci�n
del LILO {\tt /etc/lilo.conf} tiene ``estrofas'', una para cada
sistema operativo que domina y tambi�n directivas para ofrecer al
usuario estas opciones, en el momento de arrancar el sistema. 

Una de las directivas que LILO acepta es ``{\tt append=}'', la cual
permite a�adir informaci�n requerida por varios controladores a la
l�nea de ordenes. En este caso, el controlador de la unidad Iomega
Zip {\tt ppa}, requiere de una interrupci�n y una direcci�n del puerto
de entrada/salida, sin uso. Esto es exactamente an�logo a especificar
dispositivos de impresoras separados, como {\tt LPT1:} y {\tt LPT2:}
en MS-DOS. 

Por ejemplo, si la impresora usa la direcci�n del puerto hexadecimal
(en base 16), {\tt 0x378} (ver el manual de instalaci�n de la tarjeta
del puerto paralelo, si no se sabe cu�l es la direcci�n), y est�
sondeada\NT{``polled'' en el original,}, (esto
es, no requiere de una l�nea IRQ, una configuraci�n com�n de {\linux},
se deber�a colocar la siguiente l�nea, en el archivo {\tt
  /etc/lilo.conf} del sistema:

\begin{tscreen}
append="lp=0x378,0"
\end{tscreen}

Es digno de observar que Linux reconoce autom�ticamente un puerto {\tt
  /dev/lp} al arrancar el sistema, pero al especificar algunas otras
configuraciones para los puertos, las instrucci�nes al inicio del
  sistema, son requeridas. 

El ``{\tt 0}'' que se encuentra despu�s de la direcci�n del puerto, le
dice al n�cleo que {\em no} use una l�nea IRQ (pedido de
interrupci�n), para la impresora. Esto es generalmente aceptable, ya
que las impresoras son mucho m�s lentas que la CPU, y tan as� que un
m�todo m�s lento de acceso a los dispositivos E/S, conocido como {\bf
  sondeo}\footnote{``polling'' en el original, (Nota del T.)}, en el
cual el n�cleo comprueba, peri�dicamente, el estado de la impresora,
todav�a permite al computador supervisar este dispositivo.

Sin embargo, los dispositivos que operan a mayores velocidades, como
las l�neas en serie y los discos, requieren, cada uno, de una l�nea
{\bf IRQ,} o {\bf petici�n de interrupci�n (Interrupt ReQuest)}. Esta,
es una se�al del hardware, enviada por el dispositivo hacia el
microprocesador, cada vez que dicho dispositivo requiere la atenci�n
del procesador; por ejemplo: si el dispositivo tiene datos esperando a
ser despachados por el procesador. El procesador, interrumpe lo que
est� haciendo en ese momento para obedecer al pedido de interrupci�n
del dispositivo. El dispositivo unidad Zip {\bf ppa}, exige una
l�nea de interrupci�n libre, la cual debe corresponder con la de la tarjeta de la
impresora a la cual se conecta la unidad Zip. En el momento en que
esto se escrib�a, el controlador del dispositivo {\bf ppa} para GNU/Linux,
no soportaba ``sucesiones'' de puertos paralelos, y se deb�an emplear
puertos paralelos separados para usar el dispositivo Zip {\bf ppa} y
cada impresora. 

Para saber qu� interrupciones est�n actualmente utilizadas por su
sistema, la orden

\begin{tscreen}
\# cat /proc/interrupt
\end{tscreen}

muestra una lista de dispositivos y las l�neas IRQ que usan. Sin
embargo, tambi�n se deber� tener cuidado de no usar ninguna interrupci�n
de ning�n puerto en serie configurada autom�ticamente; la cual puede
no estar listada en el archivo {\tt /proc/interrupt}. El Proyecto de
Documentaci�n de Linux, serial HOWTO, el cual est� disponible en los
recursos listados en el Ap�ndice~\ref{app-sources-num}, describe
detalladamente, la configuraci�n de los puertos en serie. 

\blackdiamond Uno tambi�n deber�a realizar un chequeo de la
configuraci�n de la interfaz de varias tarjetas, abriendo la carcasa
de su m�quina y verificando visualmente la configuraci�n de los
puentes si es necesario, para asegurarse, as�, de no estar asociando
una l�nea IRQ usada por otro dispositivo. La lucha de m�ltiples
dispositivos por una l�nea de interrupci�n es quiz� el problema m�s
sencillo y com�n que causa que los sistemas GNU/Linux no funcionen. 

Un t�pico archivo {\tt /proc/interrupt} suele ser como
\begin{tscreen}
 0:    6091646   timer \\
 1:      40691   keyboard \\
 2:          0   cascade \\
 4:     284686 + serial \\
13:          1   math error \\
14:     192560 + ide0 \\
\end{tscreen}

Aqu�, la primera columna nos es de inter�s. Estos son los n�meros de
las l�neas IRQ usadas por el sistema. Para el controlador {\tt ppa},
necesitamos escoger una l�nea que {\tt no} est� listada. La l�nea IRQ
7 es, a menudo, una buena elecci�n ya que rara vez es usada en las
configuraciones predeterminadas del sistema. Tambi�n necesitamos
especificar la direcci�n del puerto que usar� el dispositivo {\tt
  ppa}. Esta direcci�n necesita estar configurada f�sicamente con la
interfaz de la tarjeta. A los puertos paralelos de E/S se les deben
asignar direcciones espec�ficas, por lo que usted tendr� que leer la
documentaci�n de la tarjeta de su puerto paralelo. En este ejemplo
usaremos, para el puerto de E/S, la direcci�n {\tt 0x278}, la cual
corresponde al puerto {\tt LPT2:} de la impresora, en MS-DOS. Para
a�adir la l�nea IRQ y la direcci�n del puerto en una l�nea de ordenes
cuando arranca el sistema, necesitamos agregar la siguiente
expresi�n a la ``estrofa'' apropiada del archivo {\tt /etc/lilo.conf}:

\begin{tscreen}
append="lp=0x378,0 ppa=0x278,7"
\end{tscreen}

Estas expresiones son a�adidas a los par�metros de arranque del
n�cleo, cuando se inicia el sistema. Aseguran que cualquier impresora
conectada al sistema no interfiera con el funcionamiento de la unidad
Zip. Por supuesto, si el sistema no tiene ninguna impresora instalada
la directiva ``{\tt lp=}'' puede y deber�a ser omitida. 

Despu�s de que haya instalado el n�cleo, como se describi� en la
secci�n~\ref{sec-lilo}, y antes de reiniciar el sistema, hay que
asegurarse de ejecutar la instrucci�n
\begin{tscreen}
\# /sbin/lilo
\end{tscreen}
para as�, instalar la nueva configuraci�n de LILO en el sector de arranque del disco duro. 

\subsection{Instalando controladores en m�dulos}
\label{ftape-module}

La p�gina~\pageref{sec-tape-backups} describe c�mo realizar copias de
seguridad en un accionador de cinta magn�tica. Linux da soporte a una
gran variedad de accionadores de cinta con interfaces IDE, SCSI y
algunas interfaces del propietario. Otro tipo corriente de
accionadores de cinta son aquellos que se conectan directamente al
controlador de la disquetera. Linux suministra el controlador para la
unidad ftape como un m�dulo. 

Cuando esto se estaba escribiendo, la versi�n m�s reciente de ftape
era la 3.04d. Se puede obtener el controlador en el servidor FTP
{\tt sunsite.unc.edu}, (para m�s informaci�n, ver el
Ap�ndice~\ref{app-ftp}). El archivo ftape se encuentra en el
directorio {\tt /pub/Linux/n�cleo/tapes}. Hay que asegurarse de
procurarse la versi�n m�s reciente, la cual, cuando este documento se estaba
editando, era {\tt ftape-3.04d.tar.gz}.

Despu�s de desempaquetar el archivo ftape en el directorio {\tt
  /usr/src}, al escribir {\tt make install} en el directorio padre de
ftape, se compilar�n el m�dulo del controlador ftape y sus utilidades,
si son necesarias, y luego se instalar�n. Si experimenta problemas de
compatibilidad entre los ficheros de la distribuci�n ejecutable ftape
y su n�cleo o las bibliotecas de su sistema, ejecute las �rdenes {\tt
  make~clean} y {\tt make~install}, y se asegurar�, de que los m�dulos
sean compilados en su sistema. 

\blackdiamond Para usar esta versi�n del controlador ftape, usted
deber� tener el soporte para m�dulos en el n�cleo, como tambi�n
soporte para el demonio {\tt n�cleod}. Sin embargo, {\em no} deber�
incluir el c�digo interno del n�cleo para ftape como una opci�n del
n�cleo, ya que las versi�nes m�s recientes del m�dulo ftape remplazan
completamente este c�digo. 

{\tt make install}, tambi�n instalar� el controlador del dispositivo
en el directorio correcto. En un sistema GNU/Linux est�ndar, los m�dulos
se encuentran en el directorio
\begin{tscreen}
/lib/modules/\cparam{n�cleo-version}
\end{tscreen}
Si la versi�n de tu n�cleo es 2.0.30, los m�dulos de su sistema se
encuentran en el directorio {\tt /lib/modules/2.0.30}. {\tt make
  install} tambi�n asegura que estos m�dulos puedan ser localizados en
cualquier momento, agregando las expresi�nes apropiadas en el archivo
{\tt modules.dep}, que se encuentra en el directorio ra�z de los
ficheros m�dulo, en este caso, {\tt /lib/modules/2.0.30}. La
instalaci�n de ftape, a�ade los siguientes m�dulos a su sistema,
(usando, en este ejemplo, la versi�n 2.0.30 del n�cleo):
\begin{tscreen}
/lib/modules/2.0.30/misc/ftape.o \\
/lib/modules/2.0.30/misc/zft-compressor.o \\
/lib/modules/2.0.30/misc/zftape.o
\end{tscreen}
Tambi�n se necesitan agregar las instrucciones para cargar los m�dulos
al archivo de la configuraci�n de m�dulos de su sistema. En muchos
sistemas, este es el archivo {\tt /etc/conf.modules}. Para cargar
autom�ticamente los m�dulos ftape a pedido, agregue las siguientes
l�neas en el archivo {\tt /etc/conf.modules}:
\begin{tscreen}
alias char-major-27 zftape \\
pre-install ftape /sbin/swapout 5
\end{tscreen}

La primera declaraci�n carga los m�dulos relacionados con ftape cuando
un dispositivo con el n�mero principal 27\NT{``majornumber 27''
  en el original} (el dispositivo ftape), es accedido
por el n�cleo. Debido a que el m�dulo de soporte para zftape, (el cual
provee compresi�n autom�tica para los dispositivos ftape) requiere el
soporte de los dem�s m�dulos ftape, todos ellos son cargados en el
momento en que el n�cleo efect�a la demanda. La segunda l�nea
especifica los par�metros que ser�n dados a los m�dulos al iniciarse
el sistema. En este caso, la utilidad {\tt /sbin/swapout}, la cual
viene incorporada en el paquete de software ftape, asegura que hay
suficiente memoria DMA, para el correcto funcionamiento del
controlador ftape. 

Para tener acceso al dispositivo ftape driver se deber� primero, colocar
una cinta formateada en la unidad. Las instrucci�nes para formatear
cintas y operar correctamente la unidad de cintas son dadas en la secci�n~\ref{sec-tape-backups}.


\subsection{Actualizando las bibliotecas compartidas}\label{sec-upgrade-libs}
\index{bibliotecas!actualizaci�n}

Como se mencion� antes, la mayor parte del software del sistema est�
compilado para que utilice las bibliotecas compartidas, las cuales
contienen subrutinas comunes compartidas entre distintos programas.
Si aparece el mensaje

\begin{tscreen}
Incompatible library version
\end{tscreen}

cuando se intenta ejecutar un programa, entonces necesita actualizar a
la versi�n de las bibliotecas que el programa requiere. Las bibliotecas
son compatible-ascendentes; esto es, un programa compilado para
utilizar una versi�n antigua de las bibliotecas, deber�a ser capaz de
trabajar con la nueva versi�n de las bibliotecas instalada. Sin embargo,
esto no se da en sentido contrario. 

La ultima versi�n de las bibliotecas se puede encontrar en los
servidores FTP de GNU/Linux. En {\tt sunsite.unc.edu}, est�n disponibles
en {\tt /pub/GNU/Linux/GCC}. Los ficheros a descargar deber�an explicar
qu� ficheros se necesita obtener y como instalarlos. Deber�a ser
capaz de coger r�pidamente los ficheros {\tt image-{\em
    versi�n}.tar.gz} e {\tt inc-{\em versi�n}.tar.gz}, donde {\em
  versi�n} es la versi�n de las bibliotecas a instalar, por ejemplo {\tt
  4.4.1}. Estos son ficheros tar, comprimidos con {\tt gzip}. El
fichero {\tt imagen} contiene las im�genes de las bibliotecas a instalar
en {\tt /lib} y {\tt /usr/lib}. El fichero {\tt inc} contiene los
ficheros de inclusi�n, a instalar en {\tt /usr/include}.

El fichero {\tt release-}{\em versi�n}{\tt .tar.gz} deber�a explicar
el procedimiento de instalaci�n en detalle (las instrucci�nes exactas
cambian seg�n la versi�n). Generalmente se necesitar� instalar los
ficheros de bibliotecas {\tt .a} y {\tt .sa} en {\tt /usr/lib}. Estas
son las utilizadas al compilar.

Adem�s, los ficheros imagen de las bibliotecas compartidas {\tt
  lib.so.}{\em versi�n} se instalan en {\tt /lib}. Estas son las
im�genes de las bibliotecas compartidas que son cargadas en tiempo de
ejecuci�n por los programas que las utilizan. Cada biblioteca tiene un
enlace simb�lico utilizando el numero de versi�n
principal \NT{``major version number'' en el original.} de la biblioteca en {\tt /lib}.

La versi�n 4.4.1 de la biblioteca {\tt libc} tiene un n�mero de
versi�n principal {\tt 4}. El archivo que contiene a la
biblioteca es el {\tt libc.so.4.4.1}. Existe un enlace simb�lico con
el nombre {\tt libc.so.4} en {\tt/lib} apuntando a este fichero. Es
por esto que se debe cambiar estos enlaces simb�licos cuando se
actualizan las bibliotecas. Por ejemplo, cuando se actualiza de {\tt
  libc.so.4.4} a {\tt libc.so.4.4.1}, se debe cambiar el enlace simb�lico
de tal modo que apunte a la nueva versi�n. 

\blackdiamond Se deber� cambiar el enlace simb�lico de un solo paso, como
se describir� m�s abajo. Si se borra el enlace simb�lico {\tt libc.so.4},
los programas que dependen de �l (incluyendo utilidades b�sicas como
{\tt ls} y {\tt cat}), dejar�n de funcionar. Por lo tanto, es
recomendable usar la siguiente orden para actualizar el enlace
simb�lico {\tt libc.so.4} y hacer que apunte al archivo {\tt libc.so.4.4.1}:

\begin{tscreen}
\# ln -sf /lib/libc.so.4.4.1 /lib/libc.so.4
\end{tscreen}

Tambi�n se necesitar� cambiar el enlace simb�lico {\tt libm.so.}{\em
  versi�n} de la misma manera. Si se est� actualizando a una
versi�n de biblioteca diferente, substituir apropiadamente los nombres
de arriba. Las notas que vienen con el paquete de la biblioteca,
deber�an explicar los detalles. (Mirar en la
p�gina~\pageref{sec-manage-links} para m�s informaci�n sobre los
enlaces simb�licos.) 

\subsection{Actualizando el {\tt gcc}}
\label{sec-upgrade-gcc}
\index{gcc@{\tt gcc}!actualizaci�n}

El compilador de C y C++ {\tt gcc}, es usado para compilar el software
de su sistema, siendo lo m�s importante, el n�cleo. La �ltima versi�n
del {\tt gcc} se puede obtener en los servidores de GNU/Linux FTP. En {\tt
  sunsite.unc.edu}, se encuentra en el directorio {\tt
  /pub/GNU/Linux/GCC} (junto con las bibliotecas). Deber�a de haber un
{\tt fichero de entrega}\footnote{``release file'' en el
  original. Nota del T. } en la distribuci�n del {\tt gcc}, el cual
explique qu� ficheros necesita obtener y c�mo instalarlos. 

La mayor�a de las distribuci�nes de GNU/Linux tienen versiones para
actualizar el {\tt gcc} que trabajan con su propia gesti�n de paquetes
de software. En general, estos paquetes son mucho m�s f�ciles de
instalar que las distribuci�nes ``gen�ricas''. 

\subsection{Actualizando otro software}

Actualizar otro software suele ser simplemente materia de obtener los
ficheros apropiados e instalarlos. La mayor parte del software de
GNU/Linux se distribuye bajo la forma de ficheros tar comprimidos que
pueden incluir los fuentes, los binarios, o ambos. Si los binarios no
est�n incluidos en ese paquete, puede que sea necesario que usted los
compile. Esto significa que, por lo menos, tenga que teclear {\tt
  make} dentro del directorio donde se encuentran los ficheros fuente. 

\index{software!d�nde encontrar versiones}
Leer el grupo de noticias de Usenet {\tt comp.os.GNU/Linux.announce} en
busca de anuncios de nuevas versi�nes de software, es la manera m�s
simple para enterarse de la aparici�n de nuevo software. Cada vez que
se busque software en un servidor FTP, obtener el fichero de
�ndice {\tt ls-lR} del servidor FTP y utilizar {\tt grep} para
encontrar los ficheros en cuesti�n, es la forma mas simple de
localizar software. Si se dispone de {\tt archie}, este puede servir de
ayuda; o de otra manera \footnote{Si no se tiene {\tt archie}}, es posible
conectarse v�a telnet a un servidor {\tt archie} como puede ser
archie.rutgers.edu, identificarse como ``{\tt archie}'' y utilizar la
orden "help". Tambi�n se puede encontrar otros recursos en Internet,
los cuales son consagrados espec�ficamente para GNU/Linux. Mirar el
Ap�ndice~\ref{app-info} para obtener informaci�n m�s detallada.

%% One handy source of GNU/Linux software is the Slackware distribution disk
%% images. Each disk contains a number of {\tt .tgz} files which are
%% simply gzipped tar files. Instead of downloading the disks, you can
%% download the desired {\tt .tgz} files from the Slackware directories
%% on the FTP site and install them directly. If you run the Slackware
%% distribution, the {\tt setup} command can be used to automatically
%% load and install a complete series of disks. 

%% Again, it's usually not a good idea to upgrade by reinstalling with
%% the newest version of Slackware, or another distribution. If you reinstall 
%% in this way, you will no doubt wreck your current installation, including 
%% user directories and all of your customized configuration. The best way
%% to upgrade software is piecewise; that is, if there is a program that
%% you use often that has a new version, upgrade it. Otherwise, don't bother.
%% Rule of thumb: If it ain't broke, don't fix it. If your current software
%% works, there's no reason to upgrade. 
\index{software!actualizar|)}
\index{software!instalar|)}

%% Traducci�n terminada el 11/02/01 por Sebasti�n Gurin, Cancerbero <anon@adinet.com.uy> 





% Linux Installation and Getting Started    -*- TeX -*-
% misc.tex
% Copyright (c) 1992, 1993 by Matt Welsh <mdw@sunsite.unc.edu>
%
% This file is freely redistributable, but you must preserve this copyright 
% notice on all copies, and it must be distributed only as part of "Linux 
% Installation and Getting Started". This file's use is covered by the 
% copyright for the entire document, in the file "copyright.tex".
%
% Copyright (c) 1998 by Specialized Systems Consultants Inc. 
% <ligs@ssc.com> 
% Traducido al espa�ol por Sebasti�n Gurin, Cancerbero <anon@adinet.com.uy>
%Revisi�n 1 el 7/7/2002 por Francisco Javier Fern�ndez <serrador@arrakis.es>
%Revisi�n 2 el 20 de julio de 2002 por Fco. javier Mart�nez


\section{Tareas diversas}
\markboth{Administraci�n del Sistema}{Tareas diversas}

Se crea o no, existen diversas tareas dom�sticas de verificaci�n 
para el administrador del sistema, que no entran en ninguna categor�a en especial.



\subsection{Ficheros de inicio del sistema}\label{sec-rc}
\index{scripts de inicio}
\index{startup scripts}
\index{boot scripts}
Cuando el sistema arranca, una serie de scripts son ejecutados autom�ticamente por el sistema antes de que cualquier usuario ingrese. He aqu� qu� es lo que sucede. 



\index{/etc/init@{\tt /etc/init}}
\index{init@{\tt init}}
\index{/etc/inittab@{\tt /etc/inittab}}
\index{inittab@{\tt inittab}}
\index{/etc/getty@{\tt /etc/getty}}
\index{getty@{\tt getty}}

Cuando el sistema arranca, el n�cleo inicia el proceso {\tt
  /etc/init}. {\tt Init} es un programa que lee su archivo de
configuraci�n, {\tt /etc/inittab}, y a su vez, inicia otros procesos,
los cuales se encuentran en dicho archivo. Uno de los procesos m�s
importantes, de los iniciados por {\tt inittab} es {\tt /etc/getty},
el cual se ``despierta'' con cada consola virtual. El proceso {\tt
  getty} dispone la consola virtual para ser utilizada, e inicia
el proceso {\tt login} en ella. Esto es lo que le permite al usuario
ingresar en cada consola virtual. Si el fichero {\tt /etc/inittab} no tuviera un
proceso {\tt getty} para cierta consola virtual, entonces no ser�a posible ingresar
en dicha consola virtual. 



\index{/etc/rc@{\tt /etc/rc}}
\index{rc@{\tt rc}}
\index{/etc/rc.local@{\tt /etc/rc.local}}
\index{rc.local@{\tt rc.local}}

Otro proceso ejecutado desde {\tt /etc/inittab} es {\tt /etc/rc}, el
archivo de inicializaci�n principal del sistema. �ste es un simple
fichero de ordenes que ejecuta cualquier orden necesaria al iniciarse
el sistema como, por ejemplo, montar el sistema de archivos (ver
p�gina~\pageref{sec-manage-fs}) o iniciar el espacio de
intercambio\NT{``swap space'' en el original.}. 
En algunos sistemas, {\tt init} ejecuta el archivo {\tt /etc/init.d/rc}. 

El sistema tambi�n puede ejecutar otros scripts de inicializaci�n. Por
ejemplo, {\tt /etc/rc.local}, contiene, usualmente, �rdenes de
inicializaci�n espec�ficas del propio sistema, como puede ser
establecer el nombre del host (ver la siguiente secci�n). {\tt
rc.local} puede ser iniciado tanto desde {\tt /etc/rc} como desde {\tt /etc/inittab}.





\subsection{Estableciendo el nombre del anfitri�n (hostname)}\label{sec-set-host name}
\index{host name!configuraci�n}
\index{anfitri�n!configuraci�n}
\index{host name!{\tt host name}}
En un entorno de red el nombre de la m�quina es utilizado para
identificar un�vocamente una m�quina en particular, mientras que en
una m�quina aut�noma, el nombre del anfitri�n, simplemente da a la
m�quina personalidad y encanto. Es como darle un nombre a una mascota:
siempre puede dirigirse a su perro como "El perro", pero es mucho m�s
interesante ponerle al perro un nombre como Mancha o Duque. 

Asignarle un nombre al sistema se trata simplemente de utilizar la
orden {\tt hostname}. Si se est� en una red, el nombre
debe ser el nombre de anfitri�n\NT{``host name'' en el
original.}  completo de su m�quina, por ejemplo, {\tt
goober.norelco.com}. Si no se esta en una red de ning�n tipo,
entonces se podr� escoger el nombre y dominio que prefiera, como {\tt
loomer.vpizza.com}, {\tt shoop.nowhere.edu}, o {\tt floof.org}.

Cuando se designa el nombre del ordenador, dicho nombre debe aparecer
en el fichero {\tt /etc/hosts}, que asigna una direcci�n IP a cada
ordenador. A�n cuando el ordenador no est� en una red, se debe incluir
el nombre del ordenador en {\tt /etc/hosts}. Si no se pertenece a una
red TCP/IP, y se ha asignado {\tt floof.org} como nombre de tu equipo,
entonces se deber� incluir la siguiente l�nea en {\tt /etc/hosts}:

\begin{tscreen}
127.0.0.1\ \ \ \ \ \ \ floof.org localhost
\end{tscreen}


Esto asignar� el nombre de servidor, {\tt floof.org}, a la direcci�n
de bucle\NT{``loopback address'' en el Original.}
127.0.0.1, (utilizada si no se est� en una red). La interfaz de bucle
est� presente, tanto cuando la m�quina esta conectada a una red, o
cuando no lo est�. El alias {\tt localhost} siempre se asigna a esta direcci�n. 
\NT{Si se usa sendmail, cambiar el nombre de host en una estaci�n sin tarjeta de red 
provoca que sendmail tarde unos 5 minutos en iniciarse debido a un cambio del 
nombre de la direcci�n del bucle local. Tendr� que configurar sendmail a mano
para poder hacer esto, y no es f�cil.}
Si se est� en una red TCP/IP, la direcci�n y nombre de servidor
actuales deber�an encontrarse en {\tt /etc/hosts}. Por ejemplo, si el
nombre de servidor es {\tt goober.norelco.com},, y la direcci�n IP es
128.253.154.32, se deber� agregar la siguiente l�nea en {\tt /etc/hosts}:

\begin{tscreen}
128.253.154.32\ \ \ \ \ \ \ goober.norelco.com
\end{tscreen}

%% Si su nombre de servidor no se encuentra en {\tt /etc/hosts}, entonces no ser� capaz de establecerlo. 
Para establecer el nombre de anfitri�n, se deber� usar la instrucci�n {\tt hostname}. Por ejemplo, la orden
\begin{tscreen}
\# hostname -S goober.norelco.com
\end{tscreen}
establece el nombre de anfitri�n como {\tt goober.norelco.com}. En la
mayor�a de los casos, la orden {\tt hostname} es ejecutado desde uno
de los ficheros de inicio del sistema, como por ejemplo {\tt /etc/rc}
o {\tt /etc/rc.local}. Hay que reescribir estos dos ficheros y cambiar la orden
{\tt hostname} que all� se encuentra para determinar su propio nombre
de anfitri�n. Cuando haya reiniciado el equipo, el sistema usar� el nuevo nombre. 

% Linux Installation and Getting Started    -*- TeX -*-
% emergency.tex
% Copyright (c) 1993 by Matt Welsh and Lars Wirzenius
%
% This file is freely redistributable, but you must preserve this copyright 
% notice on all copies, and it must be distributed only as part of "Linux 
% Installation and Getting Started". This file's use is covered by
% the copyright for the entire document, in the file "copyright.tex".
%
% Copyright (c) 1998 by Specialized Systems Consultants Inc. 
% <ligs@ssc.com>
%
% Este fichero es de distribuci�n libre, pero debe mantenerse esta 
% informaci�n de Copyright en todas las copias, y debe distribuirse solo como
% parte de "Instalaci�n y Primeros Pasos en Linux". El uso de este fichero esta
% cubierto por el Copyright del documento completo, en el fichero "copyright.tex"
% Copyright (c) 1995 por Gerardo Izquierdo para la versi�n al Castellano
% $Log: emergency.tex,v $
% Revision 1.9  2003/07/19 06:55:42  joseluis.ranz
% Correcciones varias.
%
% Revision 1.8  2002/10/12 19:53:23  montuno
% quitando defectos y comandos
%
% Revision 1.7  2002/07/30 16:23:05  pakojavi2000
% Beta 2.2 Formatos de p�rrafo
%
% Revision 1.6  2002/07/21 00:56:46  pakojavi2000
% Beta2.1
%
% Revision 1.5  2002/07/20 17:41:16  pakojavi2000
% beta2
%
% Revision 1.4  2002/07/12 10:38:33  pakojavi2000
% Corregidos errores de compilaci�n
%
% Revision 1.3  2002/07/07 21:03:13  pakojavi2000
%  Errores de deletreo
%
% Revision 1.2  2001/05/17 12:33:17  amolina
% traducci�n de emergency.tex: Ya acabamos sysadm/ :-)
%
% Revision 0.5.0.1  1996/02/10 23:45:12  rcamus
% Primera beta publica
%
%

%
% Versi�n para lipp 2.0 por Alberto Molina. Comentarios a:
%            alberto@nucle.us.es 
%

\section{Qu� hacer en caso de emergencia}

\index{emergencias!recuperaci�n de|(}
\index{desastres!recuperaci�n de|(}
En algunas ocasiones, el administrador de sistemas se encuentra con el 
problema de recuperarse de un desastre completo, como puede ser el 
olvidarse la palabra clave del usuario root, o el enfrentarse 
con sistemas de ficheros da�ados. El mejor 
consejo es, {\em obrar sin p�nico}. Todo el mundo comete errores 
est�pidos, �sta es la mejor forma de aprender sobre 
administraci�n de sistemas: la forma dif�cil.

Linux no es una versi�n inestable de UNIX. De hecho, he tenido menos problemas
con ``cuelgues'' de sistemas Linux que con versiones comerciales de UNIX en 
muchas plataformas. Linux tambi�n se beneficia de un fuerte complemento de 
asistentes que pueden ayudar a salir del agujero.

El primer paso al investigar cualquier problema es intentar arreglarlo 
uno mismo. Hay que echar un vistazo y ver c�mo funcionan las cosas. Demasiadas 
veces, un administrador de sistemas pone un mensaje desesperado 
rogando ayuda antes de investigar el problema. Muchas de las veces, 
arreglar problemas por uno mismo es realmente muy 
f�cil. Este es el camino que debe seguir para convertirse en un gur�.

Hay pocos casos en los que sea necesario reinstalar el sistema desde cero.
Muchos nuevos usuarios borran accidentalmente alg�n fichero esencial del 
sistema, e inmediatamente acuden a los discos de instalaci�n. Esta no 
es una buena idea. Antes de tomar medidas dr�sticas como esa, 
investigar el problema y preguntar a otros ayudar� a solucionar las 
cosas. En pr�cticamente todos los casos, podr� recuperar el sistema 
desde un disquete de mantenimiento.

\subsection{Recuperaci�n utilizando un disquete de mantenimiento}
\label{sec-maint-diskette}
\index{desastres!recuperaci�n de!con disquete de mantenimiento}
\index{emergencias!recuperaci�n de!con disquete de mantenimiento}
\index{arrancando!de un disquete de mantenimiento}
\index{disquete de mantenimiento}
\index{disquete de arranque}
\index{disquette!arranque/ra�z}
\index{disquette!de mantenimiento}
Una herramienta indispensable para el administrador de sistemas es el 
llamado ``disco arranque/ra�z'' (``boot/root disk'') ---un disquete 
desde el que se puede arrancar un sistema GNU/Linux completo, independiente 
del disco duro. Los discos de arranque/ra�z son realmente muy
simples, se crea un sistema de ficheros ra�z en el disquete, se ponen todas 
las utilidades necesarias en �l y se instala LILO y un n�cleo 
arrancable en el disquete. Otra t�cnica es usar un disquete para el 
n�cleo y otro para el sistema de ficheros ra�z. En cualquier caso, 
el resultado es el mismo: Ejecutar un sistema Linux completamente desde 
disquete.

El ejemplo m�s claro de un disco de arranque/ra�z son los discos de 
arranque Slackware\footnote{V�ase la 
Secci�n~\ref{sec-getting-internet} para la informaci�n sobre c�mo 
obtener �sta desde Internet. Para este procedimiento, no se necesita 
obtener la versi�n completa de Slackware, s�lo los disquetes de 
arranque y ra�z.}. Estos disquetes contienen un n�cleo capaz de iniciar y 
un sistema de ficheros ra�z, todo en disquete. Est�n dise�ados 
para usarse en la instalaci�n de la distribuci�n Slackware, pero 
vienen muy bien cuando hay que hacer mantenimiento del sistema.

El disco de arranque/ra�z de H.J Lu, disponible en 
{\tt /pub/Linux/GCC/rootdisk} en {\tt sunsite.unc.edu}, es otro ejemplo de 
este tipo de discos de mantenimiento. O, si se es ambicioso, se puede crear
uno su propio disco. En muchos casos, sin embargo, la utilizaci�n de un 
disco de arranque/ra�z prefabricado es mucho m�s simple y 
probablemente ser� m�s completo.

La utilizaci�n de un disco de arranque/ra�z es muy simple. Tan 
s�lo arranque
el sistema con el disco, y haga login como {\tt root} (normalmente sin 
clave). Para poder acceder a los ficheros del disco duro, se necesitar� 
montar el sistema de ficheros a mano. Por ejemplo, la orden
\begin{tscreen}
\# {\em mount -t ext2 /dev/hda2 /mnt}
\end{tscreen}
montar� un sistema de ficheros ext2fs existente en {\tt /dev/hda2} bajo
{\tt /mnt}. Recuerde que {\tt /} es ahora el propio disco de arranque/ra�z;
se necesitar� montar los sistemas de ficheros de su disco duro bajo alg�n
directorio para poder acceder a los ficheros. Por lo tanto, el fichero {\tt 
/etc/passwd} de su disco duro es ahora {\tt /mnt/etc/passwd} si se 
mont� el sistema de ficheros ra�z bajo {\tt /mnt}.

\subsection{Arreglando la clave de root}
\index{contrase�a!arreglando la de root}
\index{root!arreglando la password de}
Si se olvida de la clave de root, no hay problema. S�lo hay que
arrancar del disco de arranque/ra�z, montar su sistema de ficheros ra�z
en {\tt /mnt}, y eliminar el campo de la clave de {\tt /root} en
{\tt /mnt/etc/passwd}, como por ejemplo:
\begin{tscreen}
root::0:0:root:/:/bin/sh
\end{tscreen}
Ahora {\tt root} no tiene clave; al reiniciar desde el disco duro 
deber�a ser capaz de hacer login como {\tt root} y poner la clave
que desee utilizando {\tt passwd}.

�No quiso aprender a utilizar {\tt vi}? En el disco de 
arranque/ra�z probablemente no estar�n disponibles otros editores como 
pueda ser Emacs, pero {\tt vi} deber�a estarlo.

\subsection{Arreglando sistemas de ficheros corrompidos}
\index{sistemas de ficheros!arreglando corrompidos}
\index{e2fsck@{\tt e2fsck}}
\index{fsck@{\tt fsck}}
Si se corrompiese de alguna forma el sistema de ficheros, se puede ejecutar
{\tt e2fsck} o la forma apropiadad de {\tt fsck} para el tipo de
sistema de ficheros (vease la
p�gina~\pageref{sec-checking-file-system}). En muchos casos, es m�s
seguro corregir cualquier dato da�ado en el sistema de ficheros del
disco duro desde un disquete.

\index{super bloque!definici�n}
\index{super bloque!corrompido, arreglo}
Una causa com�n de da�o en un sistema de ficheros es la
corrupci�n del super bloque. 
El {\bf super bloque\/} es la ``cabecera'' del sistema de ficheros que 
contiene informaci�n acerca del estado del sistema de ficheros, tama�o, 
bloques libres, y dem�s. Si se corrompe el super bloque (por ejemplo,
escribiendo accidentalmente datos directamente a la partici�n del sistema 
de ficheros), el sistema no puede reconocer nada del sistema de ficheros.
Cualquier intento de montar el sistema de ficheros fallar� y {\tt e2fsck} no
ser� capaz de arreglar el problema.

Afortunadamente, el tipo de sistema de ficheros {\em ext2fs} salva copias del
super bloque en los l� mites de ``grupos de bloques'' en el disco 
---normalmente cada 8K bloques. Para poder decirle al {\tt e2fsck} que 
utilice una copia del super bloque, se puede utilizar una orden tal que
\begin{tscreen}
\# {\em e2fsck -b 8193 \cparam{partici�n}}
\end{tscreen} 
donde \cparam{partici�n} es la partici�n en la que reside el sistema de 
ficheros. La opci�n {\tt -b 8193} le dice al {\tt e2fsck} que utilice la copia
del super bloque almacenada en el bloque 8193 del sistema de ficheros.

\subsection{Recuperando ficheros perdidos}
\index{ficheros!recuperaci�n}
Si accidentalmente se borran ficheros importantes del sistema no 
hay forma de recuperarlos. Sin embargo, se pueden copiar los ficheros
relevantes desde el disquete al disco duro. Por ejemplo, si se hubiese borrado
{\tt /bin/login} de su sistema (que le permite registrarse en el sistema), simplemente
arranque del disquete de arranque/ra�z, monte el sistema de ficheros ra�z
en {\tt /mnt}, y use la orden
\begin{tscreen}
\# {\em cp -a /bin/login /mnt/bin/login}
\end{tscreen}
La opci�n {\tt -a} le dice a {\tt cp} que conserve los permisos en los
ficheros que se est�n copiando.

Por supuesto, si los ficheros que se borraron no fuesen ficheros esenciales 
del sistema que tengan contrapartidas en el disquete de arranque/ra�z, se
habr� acabado la suerte. Si se hicieron copias de seguridad, siempre se
podr� recuperar de ellas.

\subsection{Arreglando bibliotecas corrompidas}
\index{bibliotecas!arreglando corrompidas}
Si accidentalmente se llegasen a corromper las bibliotecas de enlaces 
simb�licos en {\tt /lib}, es m�s que seguro que instrucciones que dependan de 
estas bibliotecas no vuelvan a funcionar (V�ase la 
Secci�n~\ref{sec-upgrade-libs}). La soluci�n m�s simple es arrancar 
del disquete de arranque/ra�z, montar el sistema de ficheros ra�z 
y arreglar las bibliotecas en {\tt /mnt/lib}.

En la p�gina~\pageref{sec-upgrade-libs} se describe c�mo instalar este
tipo de bibliotecas y sus enlaces simb�licos.

\index{emergencias!recuperaci�n de|)}
\index{desastres!recuperaci�n de|)}



