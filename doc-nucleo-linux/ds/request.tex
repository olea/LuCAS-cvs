\subsection*{request}

Las estructuras de datos \ds{request} se utilizan para realizar
solicitudes a los dispositivos de bloques del sistema.  Estas
solicitudes son siempre para leer o escribir bloques de datos hacia o
desde el cach� del b�fer.
\SeeModule{include/linux/\\blkdev.h}
\begin{tscreen}\begin{verbatim}
struct request {
    volatile int rq_status;    
#define RQ_INACTIVE            (-1)
#define RQ_ACTIVE              1
#define RQ_SCSI_BUSY           0xffff
#define RQ_SCSI_DONE           0xfffe
#define RQ_SCSI_DISCONNECTING  0xffe0

    kdev_t rq_dev;
    int cmd;        /* READ o WRITE */
    int errors;
    unsigned long sector;
    unsigned long nr_sectors;
    unsigned long current_nr_sectors;
    char * buffer;
    struct semaphore * sem;
    struct buffer_head * bh;
    struct buffer_head * bhtail;
    struct request * next;
};
\end{verbatim}\end{tscreen}

