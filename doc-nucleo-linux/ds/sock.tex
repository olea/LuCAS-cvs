\subsection*{sock}

Cada estructura de datos \ds{sock} mantiene informaci�n espec�fica de
protocolo con referencia a un socket BSD.  Por ejemplo, para un socket
INET (Internet Address Domain) esta estructura de datos deber�a tener
toda la informaci�n espec�fica de TCP/IP y UDP/IP.
\SeeModule{include/linux/\\net.h}
\begin{tscreen}\begin{verbatim}
struct sock 
{
    /* This must be first. */
    struct sock             *sklist_next;
    struct sock             *sklist_prev;

    struct options          *opt;
    atomic_t                wmem_alloc;
    atomic_t                rmem_alloc;
    unsigned long           allocation;       /* modo de reserva */
    __u32                   write_seq;
    __u32                   sent_seq;
    __u32                   acked_seq;
    __u32                   copied_seq;
    __u32                   rcv_ack_seq;
    unsigned short          rcv_ack_cnt;      /* cuenta del mismo ack */
    __u32                   window_seq;
    __u32                   fin_seq;
    __u32                   urg_seq;
    __u32                   urg_data;
    __u32                   syn_seq;
    int                     users;            /* cuenta de usuario */
  /*
   *    Not all are volatile, but some are, so we
   *     might as well say they all are.
   */
    volatile char           dead,
                            urginline,
                            intr,
                            blog,
                            done,
                            reuse,
                            keepopen,
                            linger,
                            delay_acks,
                            destroy,
                            ack_timed,
                            no_check,
                            zapped,
                            broadcast,
                            nonagle,
                            bsdism;
    unsigned long           lingertime;
    int                     proc;

    struct sock             *next;
    struct sock             **pprev;
    struct sock             *bind_next;
    struct sock             **bind_pprev;
    struct sock             *pair;
    int                     hashent;
    struct sock             *prev;
    struct sk_buff          *volatile send_head;
    struct sk_buff          *volatile send_next;
    struct sk_buff          *volatile send_tail;
    struct sk_buff_head     back_log;
    struct sk_buff          *partial;
    struct timer_list       partial_timer;
    long                    retransmits;
    struct sk_buff_head     write_queue,
                            receive_queue;
    struct proto            *prot;
    struct wait_queue       **sleep;
    __u32                   daddr;
    __u32                   saddr;            /* Sending source */
    __u32                   rcv_saddr;        /* Bound address */
    unsigned short          max_unacked;
    unsigned short          window;
    __u32                   lastwin_seq;      /* sequence number when we last 
                                                 updated the window we offer */
    __u32                   high_seq;         /* sequence number when we did 
                                                 current fast retransmit */
    volatile unsigned long  ato;              /* ack timeout */
    volatile unsigned long  lrcvtime;         /* jiffies at last data rcv */
    volatile unsigned long  idletime;         /* jiffies at last rcv */
    unsigned int            bytes_rcv;
/*
 *    mss is min(mtu, max_window) 
 */
    unsigned short          mtu;              /* mss negotiated in the syn's */
    volatile unsigned short mss;              /* current eff. mss - can change */
    volatile unsigned short user_mss;         /* mss requested by user in ioctl */
    volatile unsigned short max_window;
    unsigned long           window_clamp;
    unsigned int            ssthresh;
    unsigned short          num;
    volatile unsigned short cong_window;
    volatile unsigned short cong_count;
    volatile unsigned short packets_out;
    volatile unsigned short shutdown;
    volatile unsigned long  rtt;
    volatile unsigned long  mdev;
    volatile unsigned long  rto;
 
    volatile unsigned short backoff;
    int                     err, err_soft;    /* Soft holds errors that don't
                                                 cause failure but are the cause
                                                 of a persistent failure not
                                                 just 'timed out' */
    unsigned char           protocol;
    volatile unsigned char  state;
    unsigned char           ack_backlog;
    unsigned char           max_ack_backlog;
    unsigned char           priority;
    unsigned char           debug;
    int                     rcvbuf;
    int                     sndbuf;
    unsigned short          type;
    unsigned char           localroute;       /* Route locally only */
/*
 *    This is where all the private (optional) areas that don't
 *    overlap will eventually live. 
 */
    union
    {
          struct unix_opt   af_unix;
#if defined(CONFIG_ATALK) || defined(CONFIG_ATALK_MODULE)
        struct atalk_sock   af_at;
#endif
#if defined(CONFIG_IPX) || defined(CONFIG_IPX_MODULE)
        struct ipx_opt      af_ipx;
#endif
#ifdef CONFIG_INET
        struct inet_packet_opt  af_packet;
#ifdef CONFIG_NUTCP        
        struct tcp_opt      af_tcp;
#endif        
#endif
    } protinfo;          
/* 
 *    IP 'private area'
 */
    int                     ip_ttl;           /* TTL setting */
    int                     ip_tos;           /* TOS */
    struct tcphdr           dummy_th;
    struct timer_list       keepalive_timer;  /* TCP keepalive hack */
    struct timer_list       retransmit_timer; /* TCP retransmit timer */
    struct timer_list       delack_timer;     /* TCP delayed ack timer */
    int                     ip_xmit_timeout;  /* Why the timeout is running */
    struct rtable           *ip_route_cache;  /* Cached output route */
    unsigned char           ip_hdrincl;       /* Include headers ? */
#ifdef CONFIG_IP_MULTICAST  
    int                     ip_mc_ttl;        /* Multicasting TTL */
    int                     ip_mc_loop;       /* Loopback */
    char                    ip_mc_name[MAX_ADDR_LEN]; /* Multicast device name */
    struct ip_mc_socklist   *ip_mc_list;      /* Group array */
#endif  

/*
 *    This part is used for the timeout functions (timer.c). 
 */
    int                      timeout;         /* What are we waiting for? */
    struct timer_list        timer;           /* This is the TIME_WAIT/receive 
                                               * timer when we are doing IP
                                               */
    struct timeval           stamp;
 /*
  *    Identd 
  */
    struct socket            *socket;
  /*
   *    Callbacks 
   */
    void                     (*state_change)(struct sock *sk);
    void                     (*data_ready)(struct sock *sk,int bytes);
    void                     (*write_space)(struct sock *sk);
    void                     (*error_report)(struct sock *sk);
  
};
\end{verbatim}\end{tscreen}

