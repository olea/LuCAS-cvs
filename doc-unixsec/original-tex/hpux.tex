\chapter{HP-UX}
\section{Introducci\'on}
HP-UX es la versi\'on de Unix desarrollada y mantenida por Hewlett--Packard 
desde 1983, ejecutable t\'{\i}\-pi\-ca\-men\-te sobre procesadores HP PA RISC y 
en sus \'ultima versiones sobre Intel Itanium (arquitectura Intel de 64 bits); 
a pesar de estar basada ampliamente en {\it System V} incorpora importantes 
caracter\'{\i}sticas BSD. En la actualidad la \'ultima versi\'on del operativo 
es la 11.11, aunque existen numerosas instalaciones de sistemas m\'as antiguos, 
especialmente HP-UX 10.x o incluso 9.x.\\
\\HP-UX es, como la mayor parte de Unices comerciales, un entorno de trabajo
flexible, potente y estable, que soporta un abanico de aplicaciones que van
desde simples editores de texto a complicados programas de dise\~no gr\'afico o
c\'alculo cient\'{\i}fico, pasando por sistemas de control industrial que
incluyen planificaciones de tiempo real.\\
\\Durante los \'ultimos a\~nos Hewlett--Packard, como muchos otros fabricantes,
parece haberse interesado bastante por la seguridad en general, y en concreto
por los sistemas de protecci\'on para sus sistemas; prueba de ello es la gama
de productos relacionados con este campo desarrollados para HP-UX, como el
sistema de detecci\'on de intrusos {\sc ids/9000} para HP-UX 11.x corriendo 
sobre m\'aquinas HP-9000 o la utilidad {\it Security Patch Check}, similar al 
{\it PatchDiag} de Sun Microsystems. Tambi\'en es importante destacar las 
grandes mejoras en cuanto a seguridad del sistema se refiere entre HP-UX 9.x, 
HP-UX 10.x y muy especialmente HP-UX 11.x.\\
\\En este cap\'{\i}tulo, como hemos hecho antes con Solaris, Linux y AIX,
hablaremos de aspectos de la seguridad particulares de HP-UX, teniendo siempre
presente que el resto de cap\'{\i}tulos del documento tambi\'en son aplicables
a este operativo. Para conocer m\'as HP-UX podemos consultar \cite{kn:reh00} o
\cite{kn:pon01}, y si nos interesa especialmente su seguridad una obra 
obligatoria es \cite{kn:won01}. Aparte de documentaci\'on impresa, a trav\'es 
de Internet podemos acceder a numerosa documentaci\'on t\'ecnica de HP-UX, en 
la URL {\tt http://docs.hp.com/}; por supuesto, tambi\'en son b\'asicas para 
una correcta administraci\'on las p\'aginas de manual que se instalan con el 
producto.
\section{Seguridad f\'{\i}sica en PA--RISC}
Los sistemas de las series HP9000 incluyen un {\it firmware} muy similar a la 
{\it OpenBoot PROM} de las estaciones y servidores {\sc sparc} que se denomina 
{\sc pdc} ({\it Processor Dependent Code}) y que implementa las funcionalidades
dependientes del procesador; su funci\'on b\'asica es, en el arranque de la 
m\'aquina, inicializar el procesador y comprobar que su estado es correcto
mediante unas pruebas llamadas {\sc post} ({\it Power On Self Test}). Si todo es
satisfactorio el PDC carga el {\sc isl} ({\it Initial System Loader}) o {\sc 
ipl} ({\it Initial Program Loader}) y le transfiere el control para que este 
pueda arrancar el sistema operativo.\\
\\Como en cualquier dispositivo {\it hardware} existen formas de interrumpir el 
proceso de arranque habitual para modificar su comportamiento; en concreto, en 
la familia HP9000 suele existir un intervalo de 10 
segundos antes de cargar el {\sc isl} en el que sin m\'as que pulsar la tecla 
{\sc esc}\footnote{Dependiendo de la serie concreta es posible que la 
pulsaci\'on de cualquier tecla interrumpa el proceso; esto sucede, por ejemplo,
en los modelos de las series 800.} se puede acceder al men\'u de arranque del 
equipo y, desde este, definir dispositivos de arranque alternativos o 
interactuar con el {\sc isl}, por ejemplo para pasar un par\'ametro al {\it 
kernel} de HP-UX antes de cargarlo; decimos `suele existir' porque el intervalo
durante el cual se puede interrumpir el autoarranque se respeta s\'olo si las
variables {\tt autoboot} y {\tt autosearch} del {\sc isl} est\'an activadas, ya
que en caso contrario el sistema autom\'aticamente accede al men\'u de arranque
y no se inicia hasta que un operador no interactua con el mismo.\\
\\Si al interrumpir el proceso de arranque elegimos interactuar con el {\sc 
isl} llegamos a un {\it prompt} sencillo en el que podemos comenzar a introducir
\'ordenes desde el cargador {\tt hpux}, como {\tt `hpux -v'}, que muestra la
versi\'on del {\sc isl} o {\tt `hpux -iS'}, que inicia el operativo en modo
monousuario:
\begin{quote}
\begin{verbatim}
Hard booted.

ISL Revision A.00.09  March 27, 1990

ISL> hpux -v
Secondary Loader 9000/700
Revision 1.1
@(#) Revision 1.1 Wed Dec 10 17:24:28 PST 1986

ISL>
\end{verbatim}
\end{quote}
Antes de acceder a este {\it prompt} podemos activar o resetar una variable 
denominada {\tt secure}, que indica el modo de arranque seguro del equipo; si
est\'a activada no se podr\'a interactuar con el arranque del sistema, lo cual
implica una protecci\'on relativa: un atacante situado delante del equipo lo
tendr\'a m\'as dif\'{\i}cil para arrancar el sistema desde un dispositivo que
no sea el est\'andar, o para iniciarlo en modo monousuario. El hecho de que esta
protecci\'on sea relativa es l\'ogico: si no fuera reversible, nunca nadie m\'as
podr\'{\i}a modificar la secuencia de arranque del equipo; si aunque el modo de 
arranque seguro est\'e activado queremos o necesitamos interrumpir el arranque,
no tenemos m\'as que desconectar todos los dispositivos arrancables del equipo
(discos, red, unidad de CD-ROM\ldots), de forma que el arranque falle y se
acceda al men\'u para poder resetear la variable {\tt secure}.
\section{Usuarios y accesos al sistema}
Como sucede en Linux, el acceso {\tt root} remoto a una m\'aquina HP-UX se puede
y debe limitar desde el archivo {\tt /etc/securetty}, donde se listan las
terminales desde las que el superusuario del sistema puede acceder al mismo; si
queremos que el {\tt root} s\'olo pueda conectar desde la consola f\'{\i}sica
de la m\'aquina este archivo debe ser creado de la forma siguiente:
\begin{quote}
\begin{verbatim}
marta:/# echo console > /etc/securetty 
marta:/# chown root:sys /etc/securetty 
marta:/# chmod 644 /etc/securetty 
marta:/# 
\end{verbatim}
\end{quote}
De nuevo, al igual que suced\'{\i}a en Linux, esto no evita que se pueda 
ejecutar {\tt `su'} desde cualquier sesi\'on para 
conseguir privilegios de administrador, ni deshabilita el acceso remoto 
v\'{\i}a {\sc ssh} o {\it X Window}. En el primer caso debemos vetar el acceso
desde el archivo de configuraci\'on de {\tt sshd}, mientras que en el segundo
podemos hacerlo de la forma siguiente:
\begin{quote}
\begin{verbatim}
marta:/# cp -p /usr/dt/config/Xstartup /etc/dt/config/Xstartup 
marta:/# print "[[ ${USER} = root ]] && exit 1" >>/etc/dt/config/Xstartup 
marta:/# 
\end{verbatim}
\end{quote}
Muchos de los par\'ametros que controlan el acceso de los usuarios a un sistema
HP-UX se definen en el archivo {\tt /etc/default/security}, que ha de tener 
permiso de escritura para {\tt root} y de lectura para todos los usuarios; si 
este fichero no existe, el administrador puede -- y debe -- crearlo. Esta
{\it feature} del operativo fue introducida de forma no documentada en HP-UX
11.0, y se incluye tambi\'en en HP-UX 11i, ya documentada en la p\'agina de 
manual de {\tt `security'}. Un ejemplo de este archivo puede ser el siguiente:
\begin{quote}
\begin{verbatim}
marta:/# cat /etc/default/security
ABORT_LOGIN_ON_MISSING_HOMEDIR=1
MIN_PASSWORD_LENGTH=8
NOLOGIN=1
NUMBER_OF_LOGINS_ALLOWED=3
PASSWORD_HISTORY_DEPTH=2
SU_ROOT_GROUP=admins
#SU_DEFAULT_PATH=
marta:/# 
\end{verbatim}
\end{quote}
Seguramente los par\'ametros m\'as importantes que podemos encontrar en este
archivo son los referentes a las claves de usuario; tenemos en primer lugar 
{\sc min$\_$password$\_$length}, que como su nombre indica
define la longitud m\'{\i}nima de las contrase\~nas en el sistema; mientras que
en un HP-UX normal no afecta al {\tt root}, en Trusted HP-UX s\'{\i} que lo 
hace. Puede adoptar cualquier valor entre 6 (el m\'{\i}nimo por defecto) y 8 en
los entornos normales, y entre 6 y 80 en Trusted HP-UX. El otro par\'ametro
referente a las contrase\~nas es {\sc password$\_$history$\_$depth}, que marca
los {\it passwords} antiguos contra los que se compara uno nuevo: si su valor es
N, cuando un usuario cambia su clave no podr\'a utilizar ninguna de sus N
anteriores contrase\~nas. El valor de esta variable puede ser cualquier n\'umero
entre 1 (valor por defecto) y 10, el valor m\'aximo; si su valor es 2, evitamos
que un usuario alterne constantemente dos contrase\~nas en el sistema.\\
\\Otro grupo de directivas es el formado por aquellas que afectan a la entrada
de usuarios en el sistema; tenemos en primer lugar {\sc 
abort$\_$login$\_$on$\_$missing$\_$homedir}, que define c\'omo ha de 
comportarse el operativo cuando un usuario se autentica correctamente pero su 
directorio {\it \$HOME} no existe. Si su valor es 0, el acceso se autorizar\'a y
el usuario entrar\'a directamente al directorio {\tt `/'}; si es 1, el {\it 
login} ser\'a rechazado a pesar de que la autenticaci\'on haya sido correcta.La 
segunda directiva de este grupo es {\sc nologin}: si su valor es 1 y el fichero
{\tt /etc/nologin} existe, cuando un usuario diferente del {\tt root} se 
autentique para entrar al sistema se le mostrar\'a el contenido del archivo y
se le cerrar\'a la conexi\'on (lo habitual en todos los Unices); por contra, si
el valor de esta variable es 0, el archivo {\tt /etc/nologin} simplemente ser\'a
ignorado. Finalmente, la directiva {\sc number$\_$of$\_$logins$\_$allowed} 
delimita el n\'umero m\'aximo de conexiones simult\'aneas que los usuarios 
diferentes de {\tt `root'} pueden poseer en el sistema; si su valor es 0, este
n\'umero es ilimitado.\\
\\Vamos a ver por \'ultimo un par de directivas que afectan a la ejecuci\'on 
de la orden {\tt `su'} en el sistema. En primer lugar tenemos {\sc 
su$\_$root$\_$group}, que indica qu\'e grupo de usuarios puede ejecutar la 
orden para convertirse -- tecleando la contrase\~na, evidentemente -- en 
administrador del sistema. Si este par\'ametro no est\'a definido, cualquier
usuario que conozca la clave puede convertirse en {\tt root}, mientras que si
lo est\'a s\'olo los usuarios pertenecientes al grupo indicado pueden hacerlo.\\
\\Aparte del anterior par\'ametro, en HP-UX 11i se introdujo una nueva 
directiva en el fichero {\tt /etc/default/security}; se
trata de {\sc su$\_$default$\_$path}, que marca el valor de la variable {\it
\$PATH} cuando alguien cambia su identificador de usuario mediante {\tt `su'} 
sin la opci\'on {\tt `-'} (esto es, sin emular un {\it login} real).\\
\\Acabamos de ver que en el archivo {\tt /etc/default/security} se pueden
configurar diferentes aspectos relativos a las pol\'{\i}ticas de contrase\~nas a
seguir en un sistema HP-UX; no obstante, algunos de los puntos m\'as importantes
de cualquier pol\'{\i}tica est\'an, como ocurre en Solaris, integrados dentro
de la propia orden {\tt passwd}: entre ellos algunos tan decisivos como la
longitud m\'{\i}nima de una contrase\~na. Un esquema de este tipo resulta algo
pobre actualmente, y como ya dijimos, cualquier Unix moderno deber\'{\i}a 
incluir `de serie' la posibilidad de ofrecer una granularidad m\'as adecuada en
todo lo respectivo a las claves de los usuarios. Sea como sea, el esquema 
seguido en HP-UX es muy similar al de Solaris en cuanto a los requisitos 
m\'{\i}nimos para un {\it password}: al menos seis caracteres, dos de los cuales
han de ser letras y uno num\'erico o especial, diferencias con la contrase\~na
anterior en al menos tres caracteres -- considerando equivalentes a may\'usculas
y min\'usculas para este prop\'osito --, clave diferente del nombre de usuario
y cualquier rotaci\'on del mismo, etc.\\
\\Ya para finalizar este punto, y relacionado tambi\'en con la gesti\'on de 
usuarios en HP-UX (aunque no expl\'{\i}citamente con el acceso de los mismos al 
sistema), es 
necesario hablar brevemente de los privilegios de grupo; este mecanismo, 
introducido en HP-UX 9.0, permite asignar a un grupo ciertos privilegios de
administraci\'on, distribuyendo en cierta forma el `poder' del superusuario y
rompiendo la aproximaci\'on al reparto de privilegios cl\'asico de Unix (todo o
nada). En la tabla \ref{privgrp} se muestran los privilegios de grupo
soportados en HP-UX junto a la versi\'on del operativo en la que fueron 
introducidos (\cite{kn:hpfaq}).\\
\begin{table}
\begin{center}
\begin{tabular}{|c|c|c|}
\hline
Versi\'on & Privilegio & Descripci\'on\\
\hline
\hline
9 & {\sc rtprio} & Especificaci\'on de prioridades de tiempo real\\
\hline
9 & {\sc mlock} & Utilizaci\'on de {\tt plock()}\\
\hline
9 & {\sc chown} & {\it System V} {\tt chown}\\
\hline
9 & {\sc lockrdonly} & Utilizaci\'on de {\tt lockf()}\\
\hline
9 & {\sc setrugid} & Utilizaci\'on de {\tt setuid()} y {\tt setgid()}\\
\hline
10 & {\sc mpctl} & Utilizaci\'on de {\tt mpctl()}\\
\hline
10 & {\sc rtsched} & Utilizaci\'on de {\tt sched$\_$setparam()}\\
\hline
10 & {\sc serialize} & Utilizaci\'on de {\tt serialize()}\\
\hline
11 & {\sc spuctl} & Utilizaci\'on de {\tt spuctl()}\\
\hline
11i & {\sc fssthread} & Utilizaci\'on de {\tt fss()}\\
\hline
11i & {\sc pset} & Utilizaci\'on de {\tt pset$\_\ast$()}\\
\hline
\end{tabular}
\caption{Privilegios de grupo en HP-UX}
\label{privgrp}
\end{center}
\end{table}
\\Para asignar privilegios a un determinado grupo se utiliza la orden {\tt
setprivgrp} desde l\'{\i}nea de comandos, y para que las modificaciones sean
permanentes en el arranque de la m\'aquina se lee el archivo {\tt 
/etc/privgroup} (si existe) desde {\tt /etc/rc} en HP-UX 9.x o de {\tt
/etc/init.d/set$\_$prvgrp} en versiones superiores; en este fichero se indican 
(uno por l\'{\i}nea) los privilegios a otorgar o eliminar a cada grupo:
\begin{quote}
\begin{verbatim}
marta:/# cat /etc/privgroup
-n CHOWN
admins CHOWN
admins SETRUGID
marta:/# 
\end{verbatim}
\end{quote}
En el anterior ejemplo se limita de forma global el permiso para ejecutar {\tt
chown} a todos los usuarios, y a continuaci\'on se habilita ese mismo permiso, 
junto a la capacidad para utilizar {\tt setuid()} y {\tt setgid()}, a los 
miembros del grupo {\tt admins}. Al menos la primera entrada deber\'{\i}a 
encontrarse siempre en HP-UX, ya que por defecto el operativo presenta un 
comportamiento de {\tt chown} basado en {\it System V}, lo que permite que un 
usuario pueda cambiar la propiedad de sus archivos asign\'andolos al resto de 
usuarios -- incluido el {\tt root} --, lo que claramente puede llegar a suponer 
un problema de seguridad; es mucho m\'as recomendable una aproximaci\'on basada
en {\sc bsd}, que limita la ejecuci\'on de {\tt chown} al {\tt root}.\\
\\Como en otros sistemas Unix, cuando en HP-UX un usuario quiere cambiar de
grupo puede ejecutar la orden {\tt newgrp}; sin embargo, se introduce una 
caracter\'{\i}stica adicional: en el fichero {\tt /etc/logingroup}, de formato
similar a {\tt /etc/group} (de hecho puede ser un enlace a este por cuestiones
de simplicidad), se define la lista inicial de grupos a los que un usuario 
pertenece, es decir, aquellos sobre los cuales no tiene que ejecutar {\tt 
`newgrp'} para convertirse en miembro de los mismos. Si este archivo no existe
o est\'a vac\'{\i}o, el usuario ha de ejecutar {\tt `newgrp'} siempre que
quiera acceder a un grupo secundario, y evidentemente conocer la clave de grupo
(como en cualquier Unix, si no est\'a definido un {\it password} para el grupo,
ning\'un usuario puede pertenecer a \'el si no es como grupo primario); en
cualquier caso, la relaci\'on de grupos secudarios para un usuario ha de
definirse en {\tt /etc/logingroup}, ya que si s\'olo lo est\'a en {\tt 
/etc/group} se ignorar\'a y el usuario deber\'a ejecutar {\tt `newgrp'} para 
acceder a los grupos secundarios no definidos.\\
\\La orden {\tt `groups'} nos indicar\'a a qu\'e grupos pertenece de forma
directa -- sin tener que teclear ninguna contrase\~na -- un cierto usuario, 
bas\'andose para ello en la informaci\'on almacenada en {\tt /etc/passwd},
{\tt /etc/group} y {\tt /etc/logingroup}:
\begin{quote}
\begin{verbatim}
marta:/# groups root
adm bin daemon lmadmin lp mail other root sys users
marta:/# 
\end{verbatim}
\end{quote}
\section{El sistema de parcheado}
En los sistemas HP-UX la gesti\'on de {\it software}, tanto paquetes como 
parches, se puede llevar a cabo de una forma sencilla mediante la familia de 
comandos {\tt sw$\ast$}: {\tt swinstall}, {\tt swlist}, {\tt swremove}\ldots; a
este sistema de gesti\'on se le denomina SD-UX ({\it Software Distributor 
HP-UX}), y a pesar de estar basado en una tecnolog\'{\i}a distribuida 
cliente--servidor permite \'unicamente la gesti\'on en la m\'aquina local. Para
obtener m\'as informaci\'on sobre el funcionamiento de SD-UX es recomendable
consultar \cite{kn:hp96}.\\
\\Igual que suced\'{\i}a en AIX, HP-UX posee una terminolog\'{\i}a propia para
referirse a los diferentes objetos del {\it Software Distributor}. El objeto
m\'as peque\~no manejado por SD-UX es el {\bf fileset}, que no es m\'as que un
subconjunto de los diferentes ficheros que forman un {\bf producto}, que a su
vez es un conjunto de {\it filesets} estructurado de cierta forma, en el que se
incluyen {\it scripts} de control para facilitar su manejo como un \'unico 
objeto. HP-UX suele manejar sus parches a nivel de
producto: un parche es un producto que puede estar formado por diferentes {\it
filesets}, aunque lo m\'as normal es que lo est\'e por uno solo; si no fuera 
as\'{\i}, y descarg\'aramos un parche formado por varios {\it filesets}, es 
recomendable instalar todos como un \'unico producto: aunque SD-UX permite 
el manejo de parches a nivel de {\it filesets} individuales, aplicar s\'olo
algunos de ellos puede provocar que la vulnerabilidad que el parche ha de 
solucionar permanezca en el sistema (\cite{kn:hp00b}.\\
\\El siguiente nivel de {\it software} es el {\bf bundle}, un objeto manejado
por SD-UX que agrupa diferentes productos y {\it filesets} relacionados entre
s\'{\i}, consiguiendo de esta forma una gesti\'on de {\it software} m\'as 
c\'omoda: por ejemplo, podemos utilizar {\it bundles} de parches para 
actualizar nuestro sistema, sin necesidad de descargar e instalar cada uno de 
esos parches de forma individual. Por \'ultimo, en SD-UX se introduce el 
concepto de {\bf depot}, un almac\'en de {\it software} en forma de directorio,
cinta, CD-ROM o fichero \'unico, que permite la instalaci\'on local o remota de
sus contenidos: por ejemplo, el directorio {\tt /var/adm/sw/products/} puede
ser considerado un {\it depot}, que a su vez contiene productos en forma de 
subdirectorios, que a su vez contienen {\it filesets} tambi\'en en forma de
subdirectorio, como veremos en el pr\'oximo ejemplo; mediante la orden {\tt 
swcopy} podemos crear nuestros propios {\it depots}, que nos facilitar\'an la 
tarea de gestionar el {\it software} de diferentes sistemas.\\
\\Todo el {\it software} instalado en una m\'aquina queda registrado en 
un cat\'alogo denominado {\sc ipd} ({\it Installed Products Database}), que se 
encuentra en el directorio {\tt /var/adm/sw/products/}. Por ejemplo, para 
obtener un listado de los {\it filesets} instalados en la m\'aquina 
relacionados con el producto {\it accounting}, junto a la versi\'on de los 
mismos podemos utilizar la orden {\tt swlist} o simplemente consultar la {\sc 
ipd} con comandos como {\tt ls} o {\tt cat} (por supuesto, la primera forma es 
la recomendable):
\begin{quote}
\begin{verbatim}
marta:/# swlist -l fileset Accounting
# Initializing...
# Contacting target "marta"...
#
# Target:  marta:/
#

# Accounting                                    B.10.20        Accounting     
  Accounting.ACCOUNTNG                          B.10.20                       
  Accounting.ACCT-ENG-A-MAN                     B.10.20                       
marta:/# cd /var/adm/sw/products/Accounting
marta:/var/adm/sw/products/Accounting/# ls
ACCOUNTNG       ACCT-ENG-A-MAN  pfiles
marta:/var/adm/sw/products/Accounting/# grep ^revision ACC*/INDEX
ACCOUNTNG/INDEX:revision B.10.20
ACCT-ENG-A-MAN/INDEX:revision B.10.20
marta:/var/adm/sw/products/Accounting/# 
\end{verbatim}
\end{quote}
Como hemos dicho, tambi\'en mediante la orden {\tt swlist} podemos obtener un 
listado de los parches aplicados a nuestro sistema HP-UX a nivel de 
aplicaci\'on; de nuevo, podr\'{\i}amos conseguir esta informaci\'on accediendo
directamente a {\tt /var/adm/sw/products/}:
\begin{quote}
\begin{verbatim}
marta:/# swlist -l fileset -a patch_state |grep PH|grep -v ^\#|head -3
  PHCO_10124.PHCO_10124                 
  PHCO_10175.PHCO_10175                 
  PHCO_10272.PHCO_10272                 
marta:/#
\end{verbatim}
\end{quote}
Para obtener los parches aplicados al n\'ucleo del sistema operativo podemos
utilizar la orden {\tt `what'} sobre el {\it kernel} de HP-UX:
\begin{quote}
\begin{verbatim}
marta:/# what /stand/vmunix|grep PH |head -2
         PATCH_10.20: tty_pty.o  1.13.112.5  98/01/13  PHNE_13800
         PATCH_10.20: mux2.o  1.8.112.5  98/01/13  PHNE_13800
marta:/#
\end{verbatim}
\end{quote}
Todos los parches oficiales de HP-UX (tanto a nivel de n\'ucleo como de 
aplicaci\'on) se identifican por cuatro letras que definen el tipo de parche, 
seguidas de un n\'umero y 
separados ambos por un subrayado: por ejemplo, como acabamos de ver, 
el nombre de un parche puede ser {\sc phne$\_$13800}. Actualmente existen 
cuatro tipos de parches definidos por Hewlett--Packard: comandos y 
librer\'{\i}as ({\sc phco}), {\it kernel} ({\sc phkl}), red ({\sc phne}) y 
subsistemas ({\sc phss}); todos los parches comienzan con la denominaci\'on 
com\'un {\sc ph} ({\it patch HP-UX}), y el n\'umero que los identifica es 
\'unico, independientemente del tipo de parche.\\
\\Hewlett--Packard distribuye cada cuatro meses sus parches oficiales en CDROMs 
o cintas, denominados {\it Support Plus}; tambi\'en a trav\'es de Internet 
podemos descargar actualizaciones y parches en la direcci\'on {\tt 
http://www.itrc.hp.com/}, que nos indicar\'a la URL adecuada a la que debemos
dirigirnos en funci\'on de nuestra ubicaci\'on geogr\'afica (Europa, 
Am\'erica\ldots). A diferencia de lo que sucede con otros fabricantes, en el 
caso de Hewlett--Packard es necesario estar registrado para acceder a sus 
parches, pero este registro es completamente gratuito.\\
\\Cuando descargamos un parche para HP-UX obtenemos un \'unico fichero que no 
es m\'as que un {\it shellscript}, y para instalarlo evidentemente lo primero
que tenemos que hacer es ejecutar dicho archivo; esto generar\'a un par de 
ficheros con
el nombre del parche correspondiente y extensiones {\tt .text} y {\tt .depot}
respectivamente\footnote{Recordemos que en Unix el concepto de `extensi\'on' de 
un archivo no exite; realmente, obtendremos dos archivos con nombres 
finalizados en {\tt .text} y {\tt .depot}.}: el primero de ellos contiene 
informaci\'on sobre el parche (nombre, plataformas, instrucciones de 
instalaci\'on\ldots), mientras que el segundo es un archivo con formato TAR que
podremos instalar mediante la orden {\tt swinstall}. Si por ejemplo queremos
instalar el parche {\sc phne$\_$12492} ejecutaremos una orden similar a la 
siguiente:
\begin{quote}
\begin{verbatim}
marta:/# swinstall -x autoreboot=true -x match_target=true \
-s ./PHNE_12492.depot 2>&1 >/dev/null
marta:/#
\end{verbatim}
\end{quote}
La opci\'on que m\'as nos interesa de {\tt swinstall} es sin duda {\tt `-s'},
que especifica la ubicaci\'on del archivo {\tt .depot} en el sistema de 
ficheros; la opci\'on {\tt `-x autoreboot=true'} no indica obligatoriamente un
reinicio del sistema, sino que simplemente lo permite: cada parche de HP-UX 
`sabe' si para instalarse correctamente es necesario reiniciar el operativo 
(nosotros lo podemos saber simplemente leyendo el fichero {\tt .text} que hemos
generado anteriormente), y si el {\it reboot} es necesario con esta opci\'on 
indicamos que se lleve a cabo autom\'aticamente, algo \'util en instalaciones
autom\'aticas pero que puede ser cr\'{\i}tico en algunas ocasiones.\\
\\Una vez instalado el parche correspondiente es recomendable ejecutar {\tt 
swverify}; esta orden verifica que la informaci\'on registrada en la {\sc ipd}
de HP-UX (dependencias, integridad de archivos\ldots) se corresponde con la que 
se encuentra en los ficheros y directorios del sistema. Podemos verificar todos
los registros de {\it software} (productos y parches) mediante el comod\'{\i}n 
{\tt `$\ast$'}, o bien \'unicamente el parche que acabamos de instalar; aparte
de mostrarse en la consola o terminal desde la que se ejecute, el resultado de 
la ejecuci\'on de esta orden se guardar\'a en el fichero de {\it log} 
correspondiente, dentro del directorio {\tt /var/adm/sw/}:
\begin{quote}
\begin{verbatim}
marta:/# swverify PHNE_12492

=======  12/11/01 03:23:34 MET  BEGIN swverify SESSION
         (non-interactive)

       * Session started for user "root@marta".

       * Beginning Selection
       * Target connection succeeded for "marta:/".
       * Software selections:
             PHNE_12492.PHNE_12492,l=/,r=B.10.00.00.AA,\
             a=HP-UX_B.10.20_700/800,v=HP:/
       * Selection succeeded.


       * Beginning Analysis
       * Session selections have been saved in the file
         "/.sw/sessions/swverify.last".
       * The analysis phase succeeded for "marta:/".
       * Verification succeeded.


NOTE:    More information may be found in the agent logfile (location
         is marta:/var/adm/sw/swagent.log).

=======  12/11/01 03:23:40 MET  END swverify SESSION (non-interactive)

marta:/#
\end{verbatim}
\end{quote}
Antes de finalizar este punto es necesario recordar que siempre, antes de
cualquier actualizaci\'on del sistema o de la aplicaci\'on de cualquier tipo de
parche, es {\bf imprescindible} hacer una copia de seguridad completa de todos
los sistemas de ficheros; esto es algo que se recomienda en {\bf todos} los
procedimientos de instalaci\'on de parches oficiales de Hewlett--Packard, y que
nos puede ayudar a devolver al sistema a un estado operativo ante cualquier
m\'{\i}nimo problema durante una actualizaci\'on.
\section{Extensiones de la seguridad}
\subsection{Product Description Files}
HP-UX (tanto en su versi\'on {\it Trusted} como en la habitual) posee un 
interesante mecanismo que nos permite verificar si ciertos ficheros han 
sido o no modificados: se trata de PDF (aqu\'{\i} no significa {\it Portable
Document Format} sino {\it Product Description File}), que no es m\'as que un
repositorio de informaci\'on acerca de todos y cada uno de los archivos que un
determinado {\it fileset} instala en una m\'aquina. Aunque el uso de los PDFs 
est\'a desfasado desde la versi\'on 10.20 del operativo (ha sido 
sustituido por el {\it Software Distributor}), la informaci\'on que contienen es
bastante interesante desde el punto de vista de la seguridad, ya que incluye
caracter\'{\i}sticas como el grupo, el propietario, los permisos, el tama\~no o
un {\it checksum} de cada archivo.\\
\\En el directorio {\tt /system/} de un sistema HP-UX encontramos un 
subdirectorio por cada {\it fileset} instalado en la m\'aquina; dentro de cada 
uno de estos subdirectorios existe -- o ha de existir -- un fichero denominado 
{\tt pdf}, que contiene justamente la informaci\'on de la que estamos hablando:
\begin{quote}
\begin{verbatim}
marta:/system/UX-CORE# pwd
/system/UX-CORE
marta:/system/UX-CORE# ls -l pdf
-r--r--r--   1 bin      bin        36199 Jan 10  1995 pdf
marta:/system/UX-CORE# head pdf
% Product Description File
% UX-CORE fileset, Release 9.05
/bin/alias:bin:bin:-r-xr-xr-x:160:1:70.2:4285122165:
/bin/basename:bin:bin:-r-xr-xr-x:16384:1:70.5:3822935702:
/bin/bg:bin:bin:-r-xr-xr-x:151:1:70.2:735613650:
/bin/cat:bin:bin:-r-xr-xr-x:16384:1:70.2:1679699346:
/bin/cd:bin:bin:-r-xr-xr-x:151:1:70.2:525751009:
/bin/chgrp:bin:bin:-r-xr-xr-x:20480:1:70.2:203825783:
/bin/chmod:bin:bin:-r-xr-xr-x:24576:1:70.6:1826477440:
/bin/chown:bin:bin:-r-xr-xr-x:20480:1:70.2:1796648176:
marta:/system/UX-CORE# 
\end{verbatim}
\end{quote}
Como vemos, cada l\'{\i}nea de este fichero (excepto las que comienzan con el
car\'acter {\tt `\%'}) es la entrada correspondiente a un cierto archivo que
el {\it fileset} instala en la m\'aquina; su formato es el siguiente:
\begin{quote}
-- {\tt pathname}: Ruta absoluta del fichero.\\
-- {\tt owner}: Propietario (nombre o UID) del archivo.\\
-- {\tt group}: Grupo (nombre o GID) al que pertenece.\\
-- {\tt mode}: Tipo y permisos (en formato {\tt `ls -l'}).\\
-- {\tt size}: Tama\~no del fichero en {\it bytes}.\\
-- {\tt links}: N\'umero total de enlaces en el {\it filesystem}.\\
-- {\tt version}: Versi\'on del archivo.\\
-- {\tt checksum}: Comprobaci\'on de errores seg\'un el algoritmo IEEE 802.3 
CRC.\\
-- {\tt linked$\_$to}: Nombres adicionales del fichero (enlaces duros o 
simb\'olicos).
\end{quote}
Para comprobar que los archivos de un determinado {\it fileset} no han sufrido
modificaciones que afecten a alguno de los campos anteriores podemos 
programar un sencillo {\it shellscript} que utilizando la salida de \'ordenes 
como {\tt `ls -l'}, {\tt `what'}, {\tt `ident'} o {\tt `cksum'} compruebe el
resultado de las mismas con la informaci\'on almacenada en el PDF 
correspondiente:
\begin{quote}
\begin{verbatim}
marta:/# grep ^/bin/cat /system/UX-CORE/pdf
/bin/cat:bin:bin:-r-xr-xr-x:16384:1:70.2:1679699346:
marta:/# ls -l /bin/cat
-r-xr-xr-x   1 bin      bin        16384 Jan 10  1995 /bin/cat
marta:/# what /bin/cat
/bin/cat:
         $Revision: 1.1 $
marta:/# cksum /bin/cat
1679699346 16384 /bin/cat 
marta:/# 
\end{verbatim}
\end{quote}
De la misma forma, podemos ejecutar la orden {\tt `pdfck'},  que nos informar\'a
de cualquier cambio detectado en los ficheros de un {\it fileset}:
\begin{quote}
\begin{verbatim}
marta:/# pdfck /system/UX-CORE/pdf
/etc/eisa/HWP0C70.CFG: deleted
/etc/eisa/HWP0C80.CFG: deleted
/etc/eisa/HWP1850.CFG: deleted
/etc/eisa/HWP2051.CFG: deleted
/etc/eisa/HWPC000.CFG: deleted
/etc/eisa/HWPC010.CFG: deleted
/etc/eisa/HWPC051.CFG: deleted
/etc/newconfig/905RelNotes/hpuxsystem: deleted
/etc/newconfig/inittab: size(848 -> 952), checksum(214913737 -> 2198533832)
/usr/lib/nls/POSIX: owner(bin -> root), group(bin -> sys)
marta:/# 
\end{verbatim}
\end{quote}
La aproximaci\'on a los verificadores de integridad que ofrece HP-UX es
interesante, pero tiene tambi\'en importantes carencias; en primer lugar, los
algoritmos de CRC est\'an pensados para realizar una comprobaci\'on de errores
pura, especialmente en el tr\'ansito de datos a trav\'es de una red (de hecho,
el utilizado por los PDFs es el mismo que el definido por el est\'andar IEEE
802.3 para redes {\it Ethernet}), no para detectar cambios en el contenido de
un archivo. As\'{\i}, es factible -- y no dif\'{\i}cil -- que un atacante
consiga modificar sustancialmente un fichero sin afectar a su CRC global, con
lo cual este hecho no ser\'{\i}a detectado; si realmente queremos verificar que
dos archivos son iguales hemos de recurrir a funciones {\it hash} como {\sc
md5}. Por si esto fuera poco, {\tt pdfck} es incapaz de detectar la presencia
de nuevos ficheros en un directorio, con lo que algo tan simple como la 
aparici\'on de un archivo {\it setuidado} en el sistema no ser\'{\i}a notificado
por esta utilidad; incluso ciertos cambios sobre los archivos de los cuales 
s\'{\i} tiene constancia pasan desapercibidos para ella, como la modificaci\'on 
de listas de control de acceso en los ficheros:
\begin{quote}
\begin{verbatim}
marta:/# lsacl /bin/ps
(bin.%,r-x)(%.sys,r-x)(%.%,r-x) /bin/ps
marta:/# chacl "toni.users+rwx" /bin/ps
marta:/# lsacl /bin/ps
(toni.users,rwx)(bin.%,r-x)(%.sys,r-x)(%.%,r-x) /bin/ps
marta:/# 
\end{verbatim}
\end{quote}
Como vemos, la orden anterior est\'a otorgando a un usuario un control total
sobre el archivo {\tt /bin/ps}, con todo lo que esto implica; evidentemente,
si esto lo ha realizado un atacante, estamos ante una violaci\'on muy grave de 
nuestra seguridad. Sin embargo, {\tt pdfck} ejecutado sobre el PDF 
correspondiente no reportar\'{\i}a ninguna anomal\'{\i}a, con lo que la 
intrusi\'o pasar\'{\i}a completamente desapercibida para el administrador.\\
\\A la vista de estos problemas, parece evidente que si necesitamos un 
verificador de integridad `de verdad' no podemos limitarnos a utilizar las
facilidades proporcionadas por los PDFs de HP-UX; no obstante, ya que este
mecanismo viene `de serie' con el sistema, no est\'a de m\'as usarlo, pero sin
basarnos ciegamente en sus resultados. Siempre hemos de tener en cuenta que la
informaci\'on de cada fichero registrada en los archivos {\tt 
/system/$\ast$/pdf}
se almacena igual que se est\'a almacenando el propio archivo, en un cierto 
directorio de la m\'aquina donde evidentemente el administrador puede escribir 
sin poblemas. Por tanto, a nadie se le escapa que si un atacante consigue el 
privilegio suficiente para modificar una herramienta de base (por ejemplo, {\tt 
/bin/ls}) y troyanizarla, no tendr\'a muchos problemas en modificar tambi\'en
el archivo donde se ha guardado la informaci\'on asociada a la misma, de forma 
que limit\'andonos a comparar ambas no nos daremos cuenta de la 
intrusi\'on. As\'{\i}, es una buena idea guardar, nada m\'as instalar el 
sistema, una copia del directorio {\tt /system/}, donde est\'an los PDFs, en una
unidad de s\'olo lectura; cuando lleguemos al tema dedicado a la detecci\'on de
intrusos hablaremos con m\'as detalle de los verificadores de integridad y la
importancia de esa primera copia, sobre un sistema limpio, de toda la 
informaci\'on que posteriormente vamos a verificar.
\subsection{\tt inetd.sec(4)}
Desde hace m\'as de diez a\~nos, HP-UX incorpora en todas sus {\it releases} 
un mecanismo de control de acceso a los servicios que el sistema ofrece a 
trav\'es de {\tt inetd}; se trata de un esquema muy similar al ofrecido por {\it
TCP Wrappers}, esto es, basado en la direcci\'on IP de la m\'aquina o red 
solicitante del servicio, y procesado {\bf antes} de ejecutar el demonio que
va a servir la petici\'on correspondiente. De esta forma, se establece un nivel
de seguridad adicional al ofrecido por cada uno de estos demonios.\\
\\Evidentemente, este esquema basa su configuraci\'on en un determinado archivo;
el equivalente ahora a los ficheros {\tt /etc/hosts.allow} y {\tt 
/etc/hosts.deny} de {\it TCP Wrappers} es {\tt /usr/adm/inetd.sec} (si no 
existe, el sistema funciona de la forma habitual, sin ning\'un nivel de 
protecci\'on adicional). El formato de cada l\'{\i}nea del mismo es muy simple:
\begin{center}
{\it servicio allow$\mid$deny direccion}
\end{center}
La directiva {\it `servicio'} indica el nombre del servicio a proteger tal y 
como aparece en el archivo {\tt /etc/services} (esto es una diferencia con 
respecto a {\it TCP
Wrappers}, que se gu\'{\i}a por el nombre concreto del demonio y no por el del
servicio); {\it `allow'} o {\it `deny'} (opcionales, si no se indica este campo
se asume un {\it `allow'}) definen si vamos a permitir o denegar 
el acceso, respectivamente, y por \'ultimo en el campo {\it `direccion'} podemos
indicar (tambi\'en es un campo opcional) la direcci\'on IP de uno o m\'as {\it 
hosts} o redes, as\'{\i} como los
nombres de los mismos. Si para un mismo servicio existe m\'as de una entrada,
s\'olo se aplica la \'ultima de ellas; aunque a primera vista esto nos pueda 
parecer una limitaci\'on importante, no lo es: si definimos una entrada {\it
`allow'}, a todos los sistemas no contemplados en ella se les negar\'a el 
acceso, si definimos una {\it `deny'} se le permitir\'a a todo el mundo excepto 
a los expl\'{\i}citamente indicados, y si para un servicio no existe una entrada
en el fichero, no se establece ning\'un tipo de control.\\
\\Imaginemos que el contenido de {\tt /usr/adm/inetd.sec} es el siguiente:
\begin{quote}
\begin{verbatim}
marta:/# grep -v ^\# /usr/adm/inetd.sec
finger allow localhost 158.42.* 
ssh allow localhost 158.42.* 192.168.0.*
pop allow
marta:/#
\end{verbatim}
\end{quote}
En este caso lo que estamos haciendo es permitir el acceso al servicio {\tt 
finger} a todas las m\'aquinas cuya direcci\'on comience por {\tt 158.42.},
as\'{\i} como a la m\'aquina local, y
negarlo al resto, permitir acceso a {\tt ssh} adem\'as de a estas a las que 
tengan una direcci\'on {\tt 192.169.0.}, y dejar que cualquier sistema (no 
definimos el tercer campo) pueda consultar nuestro servicio {\sc pop3}
(la entrada mostrada ser\'{\i}a equivalente a un indicar \'unicamente el
nombre del servicio, o a no definir una entrada para el mismo). As\'{\i},
en el momento que un sistema trate de acceder a un servicio para el que no 
tiene autorizaci\'on obtendr\'a una respuesta similar a la ofrecida por {\it
TCP Wrappers} (esto es, el cierre de la conexi\'on):
\begin{quote}
\begin{verbatim}
luisa:~# telnet 158.42.2.1 79  
Trying 158.42.2.1...
Connected to 158.42.2.1.
Escape character is '^]'.
Connection closed by foreign host.
luisa:~#
\end{verbatim}
\end{quote}
Como vemos, se trata de un mecanismo sencillo que, aunque no ofrezca el mismo
nivel de protecci\'on que un buen cortafuegos (nos podemos fijar que la 
conexi\'on se establece y luego se corta, en lugar de denegarse directamente
como har\'{\i}a un {\it firewall}), y aunque no sea tan parametrizable -- ni
conocido -- como {\it TCP Wrappers}, puede ahorrar m\'as de un problema a los
administradores de una m\'aquina HP-UX; y ya que existe, no cuesta nada 
utilizarlo junto a cualquier otra medida de seguridad adicional que se nos
ocurra (recordemos que en el mundo de la seguridad hay muy pocos mecanismos 
excluyentes, casi todos son complementarios). Para obtener m\'as informaci\'on
acerca de este, podemos consultar la p\'agina de manual de {\tt inetd.sec(4)}.
\section{El subsistema de red}
Igual que al hablar de Solaris o AIX hemos hecho referencia a \'ordenes como
como {\tt ndd} o {\tt no}, en HP-UX es obligatorio comentar la orden {\tt 
nettune}, que permite examinar y modificar diferentes par\'ametros del 
subsistema de red del operativo en HP-UX 10.x (en versiones anteriores era
necesario utilizar comandos como {\tt adb}, y en HP-UX 11 se introduce la orden
{\tt ndd}, como veremos m\'as adelante, muy similar a la existente en Solaris); 
por ejemplo, una consulta t\'{\i}pica puede ser la siguiente:
\begin{quote}
\begin{verbatim}
marta:/# /usr/contrib/bin/nettune -l
arp_killcomplete = 1200 default = 1200 min = 60 max = 3600 units = seconds
arp_killincomplete = 600 default = 600 min = 30 max = 3600 units = seconds
arp_unicast = 300 default = 300 min = 60 max = 3600 units = seconds
arp_rebroadcast = 60 default = 60 min = 30 max = 3600 units = seconds
icmp_mask_agent = 0 default = 0 min = 0 max = 1
ip_check_subnet_addr = 1 default = 1 min = 0 max = 1
ip_defaultttl = 255 default = 255 min = 0 max = 255 units = hops
ip_forwarding = 1 default = 1 min = 0 max = 1
ip_intrqmax = 50 default = 50 min = 10 max = 1000 units = entries
pmtu_defaulttime = 20 default = 20 min = 10 max = 32768
tcp_localsubnets = 1 default = 1 min = 0 max = 1
tcp_receive = 32768 default = 32768 min = 256 max = 262144 units = bytes
tcp_send = 32768 default = 32768 min = 256 max = 262144 units = bytes
tcp_defaultttl = 64 default = 64 min = 0 max = 255 units = hops
tcp_keepstart = 7200 default = 7200 min = 5 max = 12000 units = seconds
tcp_keepfreq = 75 default = 75 min = 5 max = 2000 units = seconds
tcp_keepstop = 600 default = 600 min = 10 max = 4000 units = seconds
tcp_maxretrans = 12 default = 12 min = 4 max = 12
tcp_urgent_data_ptr = 0 default = 0 min = 0 max = 1
udp_cksum = 1 default = 1 min = 0 max = 1
udp_defaultttl = 64 default = 64 min = 0 max = 255 units = hops
udp_newbcastenable = 1 default = 1 min = 0 max = 1
udp_pmtu = 0 default = 0 min = 0 max = 1
tcp_pmtu = 1 default = 1 min = 0 max = 1
tcp_random_seq = 0 default = 0 min = 0 max = 2
so_qlimit_max = 4096 default = 4096 min = 1 max = 8192
sb_max = 262144 default = 262144 min = 10240 max = 4294967295
hp_syn_protect = 0 default = 0 min = 0 max = 1
so_qlimit_min = 500 default = 500 min = 0 max = 8192
high_port_enable = 0 default = 0 min = 0 max = 1
high_port_max = 65535 default = 65535 min = 49153 max = 65535
ip_forward_directed_broadcasts = 1 default = 1 min = 0 max = 1
marta:/#
\end{verbatim}
\end{quote}
Podemos ver que simplemente por el nombre de estos par\'ametros el valor de 
algunos de ellos parece importante (y lo es) para la seguridad del sistema; 
este es el caso de {\tt ip$\_$forwarding} o {\tt tcp$\_$random$\_$seq}, por 
poner unos ejemplos. Podremos modificar el valor de todos aquellos par\'ametros
que nos interese tambi\'en mediante la orden {\tt nettune}:
\begin{quote}
\begin{verbatim}
marta:/# /usr/contrib/bin/nettune -l ip_forwarding 
ip_forwarding = 1 default = 1 min = 0 max = 1
marta:/# /usr/contrib/bin/nettune -s ip_forwarding 0
marta:/# /usr/contrib/bin/nettune -l ip_forwarding
ip_forwarding = 0 default = 1 min = 0 max = 1
marta:/#
\end{verbatim}
\end{quote}
Quiz\'as son dos los par\'ametros de los que m\'as nos interesa estar pendientes
para reforzar nuestra seguridad; el primero de ellos lo acabamos de ver, y es
{\tt ip$\_$forwarding}. Como su nombre indica, esta directiva indica si la 
m\'aquina ha de reenviar paquetes (si su valor es 1, el establecido por defecto)
o si no ha de hacerlo (valor 0); como ya sabemos, lo m\'as probable es que no 
nos interese este tipo de comportamiento en nuestro sistema HP-UX, por lo que
debemos establecerle un valor {\tt `0'} tal y como hemos visto en el ejemplo
anterior.\\
\\El segundo par\'ametro al que debemos estar atentos para incrementar la 
robustez de un sistema HP-UX es {\tt tcp$\_$random$\_$seq}, que es equivalente
al {\sc tcp$\_$strong$\_$iss} de Solaris: si su valor es 0 (por defecto es 
as\'{\i}), la generaci\'on de n\'umeros iniciales de secuencia {\sc tcp} es
bastante d\'ebil, si es 1 es algo m\'as robusta, y si es 2 (el valor 
recomendado) se adapta al esquema definido en \cite{kn:rfc1498}, que como ya 
sabemos es m\'as robusto que los anteriores.\\
\\Aparte de los dos anteriores, existe otro par\'ametro configurable v\'{\i}a 
{\tt nettune} que es interesante para nuestra seguridad: 
{\tt hp$\_$syn$\_$protect}, introducido en HP-UX 10.x, y que protege a una 
m\'aquina de ataques {\it SYN Flood} si su valor es `1' (por defecto est\'a a 
0, desactivado), algo con un objetivo similar a las {\it SYN Cookies} del 
n\'ucleo de Linux:
\begin{quote}
\begin{verbatim}
marta:/# /usr/contrib/bin/nettune -l hp_syn_protect
hp_syn_protect = 0 default = 0 min = 0 max = 1
marta:/# /usr/contrib/bin/nettune -s hp_syn_protect 1
marta:/# /usr/contrib/bin/nettune -l hp_syn_protect
hp_syn_protect = 0 default = 0 min = 0 max = 1
marta:/#
\end{verbatim}
\end{quote}
No todos los par\'ametros importantes para la seguridad del subsistema de red
de HP-UX son accesibles a trav\'es de {\tt nettune}; un buen ejemplo es {\tt
ip$\_$block$\_$source$\_$routed}, que como su nombre indica bloquea las tramas
{\it source routed} que llegan a los interfaces de red cuando su valor es 
verdadero (`1'), enviando ante la recepci\'on de una de ellas un paquete {\sc
icmp} de destino inalcanzable hacia el origen de la misma (\cite{kn:ste98b}). 
Otro ejemplo interesante es {\tt lanc$\_$outbound$\_$promisc$\_$flag}, que
permite a las aplicaciones que utilizan el modo promiscuo de un interfaz 
capturar tanto los paquetes {\it inbound} (los `vistos' por el sistema) como 
los {\it outbound} (los transmitidos por el propio sistema); por defecto el
valor de este par\'ametro es `0', lo que provoca que aplicaciones como {\tt
tcpdump} puedan no funcionar correctamente al ver s\'olo el tr\'afico no 
generado por el propio {\it host}. Para asignarle el valor {\it true} a ambos
par\'ametros no podemos utilizar {\tt nettune}, sino que tenemos que escribir
directamente sobre el n\'ucleo en ejecuci\'on:
\begin{quote}
\begin{verbatim}
marta:/# echo 'ip_block_source_routed/W1'|adb -w /stand/vmunix /dev/kmem
marta:/# echo 'lanc_outbound_promisc_flag/W1'|adb -w /stand/vmunix /dev/kmem
marta:/#
\end{verbatim}
\end{quote}
Como hemos dicho al principio de este punto, en HP-UX 11 se introduce un comando
{\tt ndd}, similar al que existe en Solaris, que facilita enormemente el ajuste 
de par\'ametros de la seguridad del subsistema de red. Para obtener un listado
de cada par\'ametro configurable a trav\'es de este interfaz podemos ejecutar
{\tt `ndd -h'}, y para hacernos una idea de cuales de estos par\'ametros son
los m\'as importantes para nuestra seguridad una excelente referencia es 
\cite{kn:ste00}; en cualquier caso, el nombre de los mismos, as\'{\i} como la
sintaxis de la orden, es muy similar a la que existe en Solaris.\\
\\Como siempre, nos va a interesar deshabilitar diferentes tipos de {\it 
forwarding} en nuestro sistema: el {\it IP Forwarding}, el reenv\'{\i}o de 
paquetes con opciones de {\it source routing}, y los {\it broadcasts}; para 
conseguirlo podemos ejecutar {\tt `ndd -set'}:
\begin{quote}
\begin{verbatim}
marta11:/# ndd -set /dev/ip ip_forwarding 0
marta11:/# ndd -set /dev/ip ip_forward_src_routed 0
marta11:/# ndd -set /dev/ip ip_forward_directed_broadcasts 0
marta11:/# 
\end{verbatim}
\end{quote}
Como ya sabemos, el protocolo {\sc icmp} puede ser fuente de diferentes 
problemas de seguridad en el sistema, por lo que en ocasiones conviene modificar
algunos de sus par\'ametros; es importante no responder a {\it broadcasts} de
tramas {\sc icmp$\_$echo$\_$request}, {\sc icmp$\_$address$\_$mask$\_$request}
o {\sc icmp$\_$timestamp$\_$request}, as\'{\i} como tampoco hacerlo a 
peticiones {\sc icmp$\_$timestamp$\_$request} dirigidas directamente a la 
m\'aquina (no en {\it broadcast}). En este orden, los par\'ametros del interfaz
{\tt ndd} son los siguientes:
\begin{quote}
\begin{verbatim}
marta11:/# ndd -set /dev/ip ip_respond_to_echo_broadcast 0
marta11:/# ndd -set /dev/ip ip_respond_to_address_mask_broadcast 0
marta11:/# ndd -set /dev/ip ip_respond_to_timestamp_broadcast 0
marta11:/# ndd -set /dev/ip ip_respond_to_timestamp 0
marta11:/#
\end{verbatim}
\end{quote}
El envio de tramas {\sc icmp$\_$redirect} e {\sc icmp$\_$source$\_$quench} se
puede evitar tambi\'en mediante {\tt ndd}, as\'{\i} como la activaci\'on de la
defensa contra el {\it SYN flood} que proporciona HP-UX:
\begin{quote}
\begin{verbatim}
marta11:/# ndd -set /dev/ip ip_send_redirects 0
marta11:/# ndd -set /dev/ip ip_send_source_quench 0
marta11:/# ndd -set /dev/tcp tcp_syn_rcvd_max 500
marta11:/#
\end{verbatim}
\end{quote}
Al igual que suced\'{\i}a en Solaris (o en AIX con la orden {\tt no}), los 
cambios efectuados por {\tt ndd} tienen efecto s\'olo mientras no se reinicia 
el sistema, por lo que si queremos hacerlos permanentes hemos de ejecutarlos 
autom\'aticamente en el arranque de la m\'aquina. HP-UX ejecuta en uno de sus
{\it scripts} de arranque la orden {\tt `ndd -c'}, que inicializa los valores 
por defecto de cada par\'ametro, para lo que lee el archivo {\tt 
/etc/rc.config.d/nddconf}. En este fichero (de texto), podemos definir las 
entradas correspondientes a los valores de cada par\'ametro que nos interesen,
de forma que en cada reinicio del sistema se asignen autom\'aticamente; 
{\bf no} se trata de un simple {\it shellscript}, sino de un fichero de 
configuraci\'on con tres entradas por par\'ametro a configurar, que definen el
componente sobre el que se aplica ({\tt tcp}, {\tt ip}, {\tt arp}\ldots), el
nombre del par\'ametro, y el valor que queremos darle. Es conveniente, tras
modificar el fichero, que comprobemos que efectivamente todo ha funcionado como
hab\'{\i}amos definido tras un reinicio del sistema (es decir, que cada uno de 
los par\'ametros tiene el valor que nosotros queremos), ya que, como se cita en 
\cite{kn:ste00}, existen algunos problemas relacionados con esta forma de 
trabajar; si no fuera as\'{\i}, en la misma obra se explica una sencilla 
modificaci\'on del sistema que har\'a que todo funcione correctamente.
\section{El n\'ucleo de HP-UX}
Generalmente se recomienda utilizar la herramienta SAM ({\it System 
Administration Manager}) en los sistemas HP-UX, que adem\'as de las tareas 
cl\'asicas de administraci\'on permite modificar par\'ametros de un n\'ucleo, 
reconstruirlo e instalarlo en el sistema de una forma sencilla, guardando una
copia del {\it kernel} actual en {\tt /SYSBACKUP} (algo muy \'util, ya que 
recordemos que un n\'ucleo mal configurado puede hacer que la m\'aquina no 
arranque). Por tanto, desde SAM hemos de entrar en el men\'u {\tt `Kernel 
Configuration'} y desde ah\'{\i} elegir los par\'ametros que deseamos modificar
para construir el nuevo {\it kernel}; como en el caso de Solaris, podemos fijar 
el par\'ametro {\tt maxusers} (tambi\'en con un significado similar al que esta 
variable posee en Solaris) y tambi\'en el n\'umero m\'aximo de procesos por 
usuario (par\'ametro {\tt maxuprc}).\\
\\Si deseamos modificar y reconstruir el nuevo n\'ucleo a mano, el proceso 
difiere entre HP-UX 9.x, HP-UX 10.x y HP-UX 11.x. Los pasos en cada caso son 
los siguientes:
\begin{itemize}
\item {\bf HP-UX 9.x}
\begin{quote}
\begin{verbatim}
# cd /etc/conf
# cp dfile dfile.old
# vi dfile
# config dfile
# make -f config.mk
# mv /hp-ux /hp-ux.old
# mv /etc/conf/hp-ux /hp-ux
# cd / ; shutdown -ry 0
\end{verbatim}
\end{quote}
\item {\bf HP-UX 10.x}
\begin{quote}
\begin{verbatim}
# cd /stand/build
# /usr/lbin/sysadm/system_prep -s system
# vi system
# mk_kernel -s system
# mv /stand/system /stand/system.prev
# mv /stand/build/system /stand/system
# mv /stand/vmunix /stand/vmunix.prev
# mv /stand/build/vmunix_test /stand/vmunix
# cd / ; shutdown -ry 0
\end{verbatim}
\end{quote}
\item {\bf HP-UX 11.x}
\begin{quote}
\begin{verbatim}
# cd /stand/build 
# /usr/lbin/sysadm/system_prep -s system 
# vi system 
# /usr/sbin/mk_kernel -s /stand/build/system 
# mv /stand/system /stand/system.prev 
# mv /stand/vmunix /stand/vmunix.prev 
# mv system .. 
# /usr/sbin/kmupdate 
# cd / ; shutdown -ry 0
\end{verbatim}
\end{quote}
\end{itemize}
Al editar los ficheros {\tt /etc/conf/dfile} (HP-UX 9.x) o {\tt 
/stand/build/system} (HP-UX 10.x y 11.x) hemos de especificar los par\'ametros 
comentados anteriormente, de la forma
\begin{quote}
\begin{verbatim}
maxuprc     60
maxusers    100
\end{verbatim}
\end{quote}
Otros par\'ametros a tener en cuenta relacionados con la gesti\'on de procesos
son {\tt nproc} (n\'umero m\'aximo de procesos en el sistema), {\tt nkthread}
(n\'umero m\'aximo de hilos simult\'aneos en el n\'ucleo) o {\tt 
max$\_$thread$\_$proc} (n\'umero m\'aximo de hilos en un proceso).\\ 
\\Igual que en Solaris -- y en cualquier Unix en general -- tambi\'en nos puede
interesar limitar algunos par\'ametros relacionados con el sistema de ficheros,
de cara a evitar posibles consumos excesivos de recursos que puedan comprometer
nuestra seguridad. Por ejemplo, {\tt maxfiles} indica un l\'{\i}mite {\it soft}
a los ficheros abiertos por un proceso y {\tt maxfiles$\_$lim} un l\'{\i}mite
{\it hard} (que obviamente ha de ser mayor que el anterior); {\tt nfile} 
indica el n\'umero m\'aximo de ficheros abiertos en el sistema y {\tt ninode}
el n\'umero de inodos (se recomienda que ambos coincidan). Por \'ultimo, {\tt 
nflocks} indica el n\'umero m\'aximo de ficheros abiertos y bloqueados en la 
m\'aquina.\\
\\En el n\'ucleo de HP-UX 11i se ha introducido un nuevo par\'ametro que puede
resultar muy interesante para incrementar nuestra seguridad; se trata de {\tt 
executable$\_$stack}, que permite evitar que los programas ejecuten c\'odigo de
su pila, previniendo as\'{\i} desbordamientos (\cite{kn:hp00}). Por motivos de 
compatibilidad
su valor es 1 (esto es, est\'a habilitado y los programas pueden ejecutar 
c\'odigo de su pila), pero desde {\sc sam} podemos cambiar este valor a 0; si
lo hacemos as\'{\i} y un programa concreto necesita que su pila sea ejecutable,
podemos darle ese privilegio mediante la orden {\tt chatr}:
\begin{quote}
\begin{verbatim}
marta:/# chatr +es enable /usr/local/bin/check
marta:/# 
\end{verbatim}
\end{quote}
