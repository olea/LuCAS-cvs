\chapter{Normativa}
\section{Nuevo C\'odigo Penal}\footnotetext{Delitos relacionados con nuevas
tecnolog\'{\i}as.}
\begin{center}
{\LARGE T\'ITULO X} 
\end{center}
{\large Delitos contra la intimidad, el derecho a la propia imagen y la 
inviolabilidad del domicilio} 
\begin{center} 
{\Large CAP\'ITULO I}\\ 
Del descubrimiento y revelaci\'on de secretos 
\end{center} 
{\large {\bf Art\'{\i}culo 197}} 

{\bf 1.} El que para descubrir los secretos o vulnerar la intimidad de otro, 
sin su consentimiento, se apodere de sus papeles, cartas, mensajes de correo 
electr\'onico o cualesquiera otros documentos o efectos personales o intercepte 
sus telecomunicaciones o utilice artificios t\'ecnicos de escucha, 
transmisi\'on, grabaci\'on o reproducci\'on del sonido o de la imagen, o de 
cualquier otra se\~nal de comunicaci\'on, ser\'a castigado con las penas de 
prision de uno a cuatro a\~nos y multa de doce a veinticuatro meses. 
 
{\bf 2.} Las mismas penas se impondr\'an al que, sin estar autorizado, se 
apodere, utilice o modifique, en perjuicio de tercero, datos reservados de 
car\'acter personal o familiar de otro que se hallen registrados en ficheros o 
soportes inform\'aticos, electr\'onicos o telem\'aticos, o en cualquier otro 
tipo de archivo o registro p\'ublico o privado. Iguales penas se impondr\'an a 
quien, sin estar autorizado, acceda por cualquier medio a los mismos y a quien
los altere o utilice en perjuicio del titular de los datos o de un tercero.  
 
{\bf 3.} Se impondr\'a la pena de prisi\'on de dos a cinco a\~nos si se 
difunden, revelan o ceden a terceros los datos o hechos descubiertos o las 
im\'agenes captadas a que se refieren los n\'umeros anteriores. Ser\'a 
castigado con las penas de prisi\'on de uno a tres a\~nos y multa de doce a 
veinticuatro meses, el que, con conocimiento de su origen il\'{\i}cito y sin 
haber tomado parte en su descubrimiento, realizare la conducta descrita en el 
p\'arrafo anterior. 
 
{\bf 4.} Si los hechos descritos en los apartados 1 y 2 de este art\'{\i}culo 
se realizan por las personas encargadas o responsables de los ficheros, 
soportes inform\'aticos, electr\'onicos o telem\'aticos, archivos o registros, 
se impondr\'a la pena de prisi\'on de tres a cinco a\~nos, y si se difunden, 
ceden o revelan los datos reservados, se impondr\'a la pena en su mitad 
superior. 
 
{\bf 5.} Igualmente, cuando los hechos descritos en los apartados anteriores 
afecten a datos de car\'acter personal que revelen la ideolog\'{\i}a, 
religi\'on, creencias, salud, origen racial o vida sexual, o la v\'{\i}ctima 
fuere un menor de edad o un incapaz, se impondr\'an las penas previstas en su 
mitad superior. 
 
{\bf 6.} Si los hechos se realizan con fines lucrativos, se impondr\'an las 
penas respectivamente previstas en los apartados 1 al 4 de este art\'{\i}culo 
en su mitad superior. Si adem\'as afectan a datos de los mencionados en el 
apartado 5, la pena a imponer ser\'a la de prisi\'on de cuatro a siete a\~nos.
\vspace{0.3cm}\\
{\large {\bf Art\'{\i}culo 198}} 
 
La autoridad o funcionario p\'ublico que, fuera de los casos permitidos por 
la Ley, sin mediar causa legal por delito, y prevali\'endose de su cargo, 
realizare cualquiera de las conductas descritas en el art\'{\i}culo anterior, 
ser\'a castigado con las penas respectivamente previstas en el mismo, en su 
mitad superior y, adem\'as, con la de inhabilitaci\'on absoluta por tiempo de 
seis a doce a\~nos. 
\vspace{0.3cm}\\ 
{\large {\bf Art\'{\i}culo 199}}
 
{\bf 1.} El que revelare secretos ajenos, de los que tenga conocimiento por 
raz\'on de su oficio o sus relaciones laborales, ser\'a castigado con la pena 
de prisi\'on de uno a tres a\~nos y multa de seis a doce meses. 
 
{\bf 2.} El profesional que, con incumplimiento de su obligaci\'on de sigilo o 
reserva, divulgue los secretos de otra persona, ser\'a castigado con la pena de 
prisi\'on de uno a cuatro a\~nos, multa de doce a veinticuatro meses e 
inhabilitaci\'on especial para dicha profesi\'on por tiempo de dos a seis 
a\~nos. 
\vspace{0.3cm}\\ 
{\large {\bf Art\'{\i}culo 200}}
 
Lo dispuesto en este cap\'{\i}tulo ser\'a aplicable al que descubriere, 
revelare o cediere datos reservados de personas jur\'{\i}dicas, sin el 
consentimiento de sus representantes, salvo lo dispuesto en otros preceptos de 
este C\'odigo. 
\vspace{0.3cm}\\ 
{\large {\bf Art\'{\i}culo 201}}
 
{\bf 1.} Para proceder por los delitos previstos en este cap\'{\i}tulo ser\'a 
necesaria denuncia de la persona agraviada o de su representante legal. Cuando 
aquella sea menor de edad, incapaz o una persona desvalida, tambien podr\'a 
denunciar el Ministerio Fiscal. 
 
{\bf 2.} No ser\'a precisa la denuncia exigida en el apartado anterior para 
proceder por los hechos descritos en el art\'{\i}culo 198 de este C\'odigo, ni 
cuando la comisi\'on del delito afecte a los intereses generales o a una 
pluralidad de personas. 
 
{\bf 3.} El perd\'on del ofendido o de su representante legal, en su caso, 
extingue la acci\'on penal o la pena impuesta, sin perjuicio de lo dispuesto en 
el segundo p\'arrafo del numero 4$^{o}$ del art\'{\i}culo 130. 
\vspace{0.3cm}\\ 
{\large {\bf Art\'{\i}culo 248}}

{\bf 1.} Cometen estafa los que, con \'animo de lucro, utilizaren enga\~no 
bastante para producir error en otro, induci\'endolo a realizar un acto de 
disposici\'on en perjuicio propio o ajeno. 
 
{\bf 2.} Tambi\'en se consideran reos de estafa los que, con \'animo de lucro, 
y vali\'endose de alguna manipulaci\'on inform\'atica o artificio semejante 
consigan la transferencia no consentida de cualquier activo patrimonial en 
perjuicio de tercero. 
\vspace{0.3cm}\\ 
{\large {\bf Art\'{\i}culo 263}}
 
El que causare da\~nos en propiedad ajena no comprendidos en otros T\'{\i}tulos 
de este C\'odigo, ser\'a castigado con la pena de multa de seis a veinticuatro 
meses, atendidas la condici\'on econ\'omica de la v\'{\i}ctima y la 
cuant\'{\i}a del da\~no, si \'este excediera de cincuenta mil pesetas. 
\vspace{0.3cm}\\ 
{\large {\bf Art\'{\i}culo 264}}
 
{\bf 1.} Ser\'a castigado con la pena de prisi\'on de uno a tres a\~nos y multa 
de doce a veinticuatro meses el que causare da\~nos expresados en el 
art\'{\i}culo anterior, si concurriera alguno de los supuestos siguientes: 
\begin{enumerate} 
\item Que se realicen para impedir el libre ejercicio de la autoridad 
o en venganza de sus determinaciones, bien se cometiere el delito contra 
funcionarios publicos, bien contra particulares que, como testigos o de 
cualquier otra manera, hayan contribuido o pueden contribuir a la ejecucion 
o aplicacion de las Leyes o disposiciones generales. 
\item Que se cause por cualquier medio infecci\'on o contagio de ganado. 
\item Que se empleen sustancias venenosas o corrosivas. 
\item Que afecten a bienes de dominio o uso p\'ublico o comunal. 
\item Que arruinen al perjudicado o se le coloque en grave situaci\'on 
econ\'omica. 
\end{enumerate} 

{\bf 2.} La misma pena se impondr\'a al que por cualquier medio destruya, 
altere, inutilice o de cualquier otro modo da\~ne los datos, programas o 
documentos electr\'onicos ajenos contenidos en redes, soportes o sistemas 
inform\'aticos. 
 
\begin{center} 
{\Large CAP\'ITULO XI}\\
De los delitos relativos a la propiedad intelectual e industrial, al mercado y 
a los consumidores 
\end{center} 
 
\begin{center} 
	Secci\'on 1$^{a}$.- DE LOS DELITOS RELATIVOS A LA PROPIEDAD 
INTELECTUAL. 
\end{center} 
{\large {\bf Art\'{\i}culo 270}}
 
Ser\'a castigado con la pena de prisi\'on de seis meses a dos a\~nos o de 
multa de seis a veinticuatro meses quien, con \'animo de lucro y en perjuicio 
de tercero, reproduzca, plagie, distribuya o comunique p\'ublicamente, en todo 
o en parte, una obra literaria, art\'{\i}stica o cient\'{\i}fica, o su 
transformaci\'on, interpretaci\'on o ejecuci\'on art\'{\i}stica fijada en 
cualquier tipo de soporte o comunicada a trav\'es de cualquier medio, sin la 
autorizaci\'on de los titulares de los correspondientes derechos de propiedad 
intelectual o de sus cesionarios.\\ 

La misma pena se impondr\'a a quien intencionadamente importe, exporte 
o almacene ejemplares de dichas obras o producciones o ejecuciones sin la 
referida autorizaci\'on.\\

Ser\'a castigada tambi\'en con la misma pena la fabricaci\'on, puesta en 
circulaci\'on y tenencia de cualquier medio espec\'{\i}ficamente destinada a 
facilitar la supresi\'on no autorizada o la neutralizaci\'on de cualquier 
dispositivo t\'ecnico que se haya utilizado para proteger programas de 
ordenador. 
\vspace{0.3cm}\\
{\large {\bf Art\'{\i}culo 278}} 
 
{\bf 1.} El que, para descubrir un secreto de empresa se apoderare por 
cualquier medio de datos, documentos escritos o electr\'onicos, soportes 
inform\'aticos u otros objetos que se refieran al mismo, o empleare alguno de 
los medios o instrumentos se\~nalados en el apartado 1 del art\'{\i}culo 197, 
ser\'a castigado con la pena de prisi\'on de dos a cuatro a\~nos y multa de 
doce a veinticuatro meses. 
 
{\bf 2.} Se impondr\'a la pena de prisi\'on de tres a cinco a\~nos y multa de 
doce a veinticuatro meses si se difundieren, revelaren o cedieren a terceros 
los secretos descubiertos. 
 
{\bf 3.} Lo dispuesto en el presente art\'{\i}culo se entender\'a sin perjuicio 
de las penas que pudieran corresponder por el apoderamiento o destrucci\'on de 
los soportes inform\'aticos. 
 
\begin{center}
{\Large CAP\'ITULO III}\\
Disposici\'on general 
\end{center}
{\large {\bf Art\'{\i}culo 400}}
 
La fabricaci\'on o tenencia de \'utiles, materiales, instrumentos, sustancias, 
m\'aquinas, programas de ordenador o aparatos, espec\'{\i}ficamente 
destinados a la comisi\'on de los delitos descritos en los cap\'{\i}tulos 
anteriores, se castigar\'an con la pena se\~nalada en cada paso para los 
autores. 
\vspace{0.3cm}\\
{\large {\bf Art\'{\i}culo 536}}
 
La autoridad, funcionario p\'ublico o agente de estos que, mediando 
causa por delito, interceptare las telecomunicaciones o utilizare artificios 
t\'ecnicos de escuchas, transmisi\'on, grabaci\'on o reproducci\'on del sonido, 
de la imagen o de cualquier otra se\~nal de comunicaci\'on, con violaci\'on de 
las garant\'{\i}as constitucionales o legales, incurrir\'a en la pena de 
inhabilitaci\'on especial para empleo o cargo p\'ublico de dos a seis a\~nos.\\
 
Si divulgare o revelare la informaci\'on obtenida, se impondr\'an las 
penas de inhabilitaci\'on especial, en su mitad superior y, adem\'as la de 
multa de seis a dieciocho meses.
\cleardoublepage
\section{Reglamento de Seguridad de la LORTAD}\footnotetext{Reglamento de 
Medidas 
de Seguridad de los ficheros automatizados que contengan datos de car\'acter 
personal, aprobado por el Real Decreto 994/1999, de 11 de junio.}
\begin{center}
{\LARGE CAP\'ITULO I}\\ {\large Disposiciones Generales}
\end{center}
\vspace{0.3cm}
{\large {\bf Art\'{\i}culo 1.} \'Ambito de aplicaci\'on y fines.}

El presente Reglamento tiene por objeto establecer las medidas de \'{\i}ndole 
t\'ecnica y organizativas necesarias para garantizar la seguridad que deben 
reunir los ficheros automatizados, los centros de tratamiento, locales, equipos,
sistemas, programas y las personas que intervengan en el tratamiento 
automatizado de los datos de car\'acter personal sujetos al r\'egimen de la 
Ley Org\'anica 5/1992, de 29 de octubre, de Regulaci\'on del Tratamiento 
Automatizado de los Datos de car\'acter personal.
\vspace{0.3cm}\\
{\large {\bf Art\'{\i}culo 2.} Definiciones.}

A efectos de este Reglamento, se entender\'a por:
\begin{itemize}
\item[1.-]{\bf Sistema de informaci\'on:} Conjunto de ficheros automatizados, 
programas, soportes y equipos empleados para el almacenamiento y tratamiento de 
datos de car\'acter personal.
\item[2.-]{\bf Usuario:} Sujeto o proceso autorizado para acceder a datos o 
recursos.
\item[3.-]{\bf Recurso:} Cualquier parte componente de un sistema de 
informaci\'on.
\item[4.-]{\bf Accesos autorizados:} Autorizaciones concedidas a un usuario 
para la utilizaci\'on de los diversos recursos.
\item[5.-]{\bf Identificaci\'on:} Procedimiento de reconocimiento de la 
identidad de un usuario.
\item[6.-]{\bf Autenticaci\'on:} Procedimiento de comprobaci\'on de la 
identidad de un usuario.
\item[7.-]{\bf Control de acceso:} Mecanismo que en funci\'on a la 
identificaci\'on ya autenticada permite acceder a datos o recursos.
\item[8.-]{\bf Contrase\~na:} Informaci\'on confidencial, frecuentemente 
constituida por una cadena de caracteres, que puede ser usada en la 
autenticaci\'on de un usuario.
\item[9.-]{\bf Incidencia:} Cualquier anomal\'{\i}a que afecte o pudiera 
afectar a la seguridad de los datos.
\item[10.-]{\bf Soporte:} Objeto f\'{\i}sico susceptible de ser tratado en un 
sistema de informaci\'on sobre el cual se pueden grabar o recuperar datos.
\item[11.-]{\bf Responsable de seguridad:} Persona o personas de la 
organizaci\'on a las que el responsable del fichero ha asignado formalmente la 
funci\'on de coordinar y controlar las medidas de seguridad aplicables.
\item[12.-]{\bf Copia de respaldo:} Copia de los datos de un fichero 
automatizado en un soporte que posibilite su recuperaci\'on.
\end{itemize}
\vspace{0.3cm}
{\large {\bf Art\'{\i}culo 3.} Niveles de seguridad.}

{\bf 1.} Las medidas de seguridad exigibles se clasifican en tres niveles: 
b\'asico, medio y alto.\\

{\bf 2.} Dichos niveles se establecen atendiendo a la naturaleza de la 
informaci\'on tratada, en relaci\'on con la mayor o menor necesidad de 
garantizar la confidencialidad y la integridad de la informaci\'on.
\vspace{0.3cm}\\
{\large {\bf Art\'{\i}culo 4.} Aplicaci\'on de los niveles de seguridad.}

{\bf 1.} Todos los ficheros que contengan datos de car\'acter personal 
deber\'an adoptar las medidas de seguridad calificadas como de nivel 
b\'asico.\\

{\bf 2.} Los ficheros que contengan datos relativos a la comisi\'on de 
infracciones administrativas o penales, Hacienda P\'ublica, servicios 
financieros y aquellos ficheros cuyo funcionamiento se rija por el art\'{\i}culo
28 de la Ley Org\'anica 5/1992, deber\'an reunir, adem\'as de las medidas de 
nivel b\'asico, las calificadas como de nivel medio.\\

{\bf 3.} Los ficheros que contengan datos de ideolog\'{\i}a, religi\'on, 
creencias, origen racial, salud o vida sexual, as\'{\i} como los que contengan 
datos recabados para fines policiales sin consentimiento de las personas 
afectadas deber\'an reunir, adem\'as de las medidas de nivel b\'asico y medio, 
las calificadas como de nivel alto.\\

{\bf 4.} Cuando los ficheros contengan un conjunto de datos de car\'acter 
personal suficientes que permitan obtener una evaluaci\'on de la personalidad 
del individuo deber\'an garantizar las medidas de nivel medio establecidas en 
los art\'{\i}culos 17, 18, 19 y 20.\\

{\bf 5.} Cada uno de los niveles descritos anteriormente tienen la condici\'on 
de m\'{\i}nimos exigibles, sin perjuicio de las disposiciones legales o 
reglamentarias espec\'{\i}ficas vigentes.
\vspace{0.3cm}\\
{\large {\bf Art\'{\i}culo 5.} Acceso a datos a trav\'es de redes de 
comunicaciones.}

Las medidas de seguridad exigibles a los accesos a datos de car\'acter personal 
a trav\'es de redes de comunicaciones deber\'an garantizar un nivel de 
seguridad equivalente al correspondiente a los accesos en modo local.
\vspace{0.3cm}\\
{\large {\bf Art\'{\i}culo 6.} R\'egimen de trabajo fuera de los locales de 
ubicaci\'on del fichero.}

La ejecuci\'on de tratamiento de datos de car\'acter personal fuera de los 
locales de la ubicaci\'on del fichero deber\'a ser autorizada expresamente por 
el responsable del fichero y, en todo caso, deber\'a garantizarse el nivel de 
seguridad correspondiente al tipo de fichero tratado.
\vspace{0.3cm}\\
{\large {\bf Art\'{\i}culo 7.} Ficheros temporales.}

{\bf 1.} Los ficheros temporales deber\'an cumplir el nivel de seguridad que 
les corresponda con arreglo a los criterios establecidos en el presente 
Reglamento.\\

{\bf 2.} Todo fichero temporal ser\'a borrado una vez que haya dejado de ser 
necesario para los fines que motivaron su creaci\'on.
\begin{center}
{\LARGE CAP\'ITULO II}\\ {\large Medidas de seguridad de nivel b\'asico}
\end{center}
\vspace{0.3cm}
{\large {\bf Art\'{\i}culo 8.} Documento de seguridad.}

{\bf 1.} El responsable del fichero elaborar\'a e implantar\'a la normativa de 
seguridad, mediante un documento de obligado cumplimiento para el personal con 
acceso a los datos automatizados de car\'acter personal y a los sistemas de 
informaci\'on.\\

{\bf 2.} El documento deber\'a contener, como m\'{\i}nimo, los siguientes 
aspectos:
\begin{itemize}
\item[(a)] \'Ambito de aplicaci\'on del documento con especificaci\'on 
detallada de los recursos protegidos.
\item[(b)] Medidas, normas, procedimientos, reglas y est\'andares encaminados a 
garantizar el nivel de seguridad exigido en este Reglamento.
\item[(c)] Funciones y obligaciones del personal.
\item[(d)] Estructura de los ficheros con datos de car\'acter personal y 
descripci\'on de los sistemas de informaci\'on que los tratan.
\item[(e)] Procedimiento de notificaci\'on, gesti\'on y respuesta ante las 
incidencias.
\item[(f)] Los procedimientos de realizaci\'on de copias de respaldo y de 
recuperaci\'on de los datos.
\end{itemize}

{\bf 3.} El documento deber\'a mantenerse en todo momento actualizado y 
deber\'a ser revisado siempre que se produzcan cambios en el sistema de 
informaci\'on o en la organizaci\'on.\\

{\bf 4} El contenido del documento deber\'a adecuarse, en todo momento, a las 
disposiciones vigentes en materia de seguridad de los datos de car\'acter 
personal.
\vspace{0.3cm}\\
{\large {\bf Art\'{\i}culo 9.} Funciones y obligaciones del personal.}

{\bf 1.} Las funciones y obligaciones de cada una de las personas con acceso a 
los datos de car\'acter personal y a los sistemas de informaci\'on estar\'an 
claramente definidas y documentadas, de acuerdo con lo previsto en el 
art\'{\i}culo 8.2.(c).\\

{\bf 2.} El responsable del fichero adoptar\'a las medidas necesarias para que 
el personal conozca las normas de seguridad que afecten al desarrollo de sus 
funciones as\'{\i} como las consecuencias en que pudiera incurrir en caso de 
incumplimiento.
\vspace{0.3cm}\\
{\large {\bf Art\'{\i}culo 10.} Registro de incidencias.}

El procedimiento de notificaci\'on y gesti\'on de incidencias contendr\'a 
necesariamente un registro en el que se haga constar el tipo de incidencia, el 
momento en que se ha producido, la persona que realiza la notificaci\'on, a 
qui\'en se le comunica y los efectos que se hubieran derivado de la misma.
\vspace{0.3cm}\\
{\large {\bf Art\'{\i}culo 11.} Identificaci\'on y autenticaci\'on.}

{\bf 1.} El responsable del fichero se encargar\'a de que exista una relaci\'on 
actualizada de usuarios que tengan acceso al sistema de informaci\'on y de 
establecer procedimientos de identificaci\'on y autenticaci\'on para dicho 
acceso.\\

{\bf 2.} Cuando el mecanismo de autenticaci\'on se base en la existencia de 
contrase\~nas existir\'a un procedimiento de asignaci\'on, distribuci\'on y 
almacenamiento que garantice su confidencialidad e integridad.\\

{\bf 3.} Las contrase\~nas se cambiar\'an con la periodicidad que se determine 
en el documento de seguridad y mientras est\'en vigentes se almacenar\'an de 
forma ininteligible.
\vspace{0.3cm}\\
{\large {\bf Art\'{\i}culo 12.} Control de acceso.}

{\bf 1.} Los usuarios tendr\'an acceso autorizado \'unicamente a aquellos datos 
y recursos que precisen para el desarrollo de sus funciones.\\

{\bf 2.} El responsable del fichero establecer\'a mecanismos para evitar que un 
usuario pueda acceder a datos o recursos con derechos distintos de los 
autorizados.\\

{\bf 3.} La relaci\'on de usuarios a la que se refiere el art\'{\i}culo 11.1 de 
este Reglamento contendr\'a los derechos de acceso autorizados para cada uno de 
ellos.\\

{\bf 4.} Exclusivamente el personal autorizado para ello en el documento de 
seguridad podr\'a conceder, alterar o anular el acceso autorizado sobre los 
datos y recursos, conforme a los criterios establecidos por el responsable del 
fichero.
\vspace{0.3cm}\\
{\large {\bf Art\'{\i}culo 13.} Gesti\'on de soportes.}

{\bf 1.} Los soportes inform\'aticos que contengan datos de car\'acter personal 
deber\'an permitir identificar el tipo de informaci\'on que contienen, ser 
inventariados y almacenarse en un lugar con acceso restringido al personal 
autorizado para ello en el documento de seguridad.\\

{\bf 2.} La salida de soportes inform\'aticos que contengan datos de car\'acter 
personal, fuera de los locales en que est\'e ubicado el fichero, \'unicamente 
podr\'a ser autorizada por el responsable del fichero.
\vspace{0.3cm}\\
{\large {\bf Art\'{\i}culo 14.} Copias de respaldo y recuperaci\'on}

{\bf 1.} El responsable de fichero se encargar\'a de verificar la definici\'on 
y correcta aplicaci\'on de los procedimientos de realizaci\'on de copias de 
respaldo y de recuperaci\'on de los datos.\\

{\bf 2.} Los procedimientos establecidos para la realizaci\'on de copias de 
respaldo y para la recuperaci\'on de los datos deber\'a garantizar su 
reconstrucci\'on en el estado en que se encontraban al tiempo de producirse la 
p\'erdida o destrucci\'on.

{\bf 3.} Deber\'an realizarse copias de respaldo, al menos semanalmente, salvo 
que en dicho per\'{\i}odo no se hubiera producido ninguna actualizaci\'on de 
los datos.
\begin{center}
{\LARGE CAP\'ITULO III}\\ {\large Medidas de seguridad de nivel medio}
\end{center}
\vspace{0.3cm}
{\large {\bf Art\'{\i}culo 15.} Documento de seguridad.}

El documento de seguridad deber\'a contener, adem\'as de lo dispuesto en el 
articulo 8 del presente Reglamento, la identificaci\'on del responsable o 
responsables de seguridad, los controles peri\'odicos que se deban realizar 
para verificar el cumplimiento de lo dispuesto en el propio documento y las 
medidas que sea necesario adoptar cuando un soporte vaya a ser desechado o 
reutilizado.
\vspace{0.3cm}\\
{\large {\bf Art\'{\i}culo 16.} Responsable de seguridad.}

El responsable del fichero designar\'a uno o varios responsables de seguridad 
encargados de coordinar y controlar las medidas definidas en el documento de 
seguridad. En ning\'un caso esta designaci\'on supone una delegaci\'on de la 
responsabilidad que corresponde al responsable del fichero de acuerdo con este 
Reglamento.
\vspace{0.3cm}\\
{\large {\bf Art\'{\i}culo 17.} Auditor\'{\i}a.}

{\bf 1.} Los sistemas de informaci\'on e instalaciones de tratamiento de datos 
se someter\'an a una auditor\'{\i}a interna o externa, que verifique el 
cumplimiento del presente Reglamento, de los procedimientos e instrucciones 
vigentes en materia de seguridad de datos, al menos, cada dos a\~nos.\\

{\bf 2.} El informe de auditor\'{\i}a deber\'a dictaminar sobre la adecuaci\'on 
de las medidas y controles al presente Reglamento, identificar sus deficiencias 
y proponer las medidas correctoras o complementarias necesarias. Deber\'a, 
igualmente, incluir los datos, hechos y observaciones en que se basen los 
dict\'amenes alcanzados y recomendaciones propuestas.\\

{\bf 3.} Los informes de auditor\'{\i}a ser\'an analizados por el responsable 
de seguridad competente, que elevar\'a las conclusiones al responsable del 
fichero para que adopte las medidas correctoras adecuadas y quedar\'an a 
disposici\'on de la Agencia de Protecci\'on de Datos.
\vspace{0.3cm}\\
{\large {\bf Art\'{\i}culo 18.} Identificaci\'on y autenticaci\'on.}

{\bf 1.} El responsable del fichero establecer\'a un mecanismo que permita la 
identificaci\'on de forma inequ\'{\i}voca y personalizado de todo aquel usuario 
que intente acceder al sistema de informaci\'on y la verificaci\'on de que 
est\'a autorizado.\\

{\bf 2.} Se limitar\'a la posibilidad de intentar reiteradamente el acceso no 
autorizado al sistema de informaci\'on.
\vspace{0.3cm}\\
{\large {\bf Art\'{\i}culo 19.} Control de acceso f\'{\i}sico.}

Exclusivamente el personal autorizado en el documento de seguridad podr\'a 
tener acceso a los locales donde se encuentren los sistemas de informaci\'on 
con datos de car\'acter personal.
\vspace{0.3cm}\\
{\large {\bf Art\'{\i}culo 20.} Gesti\'on de soportes.}

{\bf 1.} Deber\'a establecerse un sistema de registro de entrada de soportes 
inform\'aticos que permita, directa o indirectamente, conocer el tipo de 
soporte, la fecha y hora, el emisor, el n\'umero de soportes, el tipo de 
informaci\'on que contienen, la forma de env\'{\i}o y la persona responsable 
de la recepci\'on que deber\'a estar debidamente autorizada.\\

{\bf 2.} Igualmente, se dispondr\'a de un sistema de registro de salida de 
soportes inform\'aticos que permita, directa o indirectamente, conocer el tipo 
de soporte, la fecha v hora, el destinatario, el n\'umero de soportes, el tipo 
de informaci\'on que contienen, la forma de env\'{\i}o y la persona responsable 
de la entrega que deber\'a estar debidamente autorizada.\\

{\bf 3.} Cuando un soporte vaya a ser desechado o reutilizado, se adoptar\'an 
las medidas necesarias para impedir cualquier recuperaci\'on posterior de la 
informaci\'on almacenada en \'el, previamente a que se proceda a su baja en el 
inventario.\\

{\bf 4.} Cuando los soportes vayan a salir fuera de los locales en que 
encuentren ubicados los ficheros como consecuencia de operaciones de 
mantenimiento, se adoptar\'an las medidas necesarias para impedir cualquier 
recuperaci\'on indebida de la informaci\'on almacenada en ellos.
\vspace{0.3cm}\\
{\large {\bf Art\'{\i}culo 21.} Registro de incidencias.}

{\bf 1.} En el registro regulado en el art\'{\i}culo 10 deber\'an consignarse,
adem\'as los procedimientos realizados de recuperaci\'on de los datos, 
indicando la persona que ejecut\'o el proceso, los datos restaurados y, en su
caso, qu\'e datos ha sido necesario grabar manualmente en el proceso de 
recuperaci\'on.\\

{\bf 2.} Ser\'a necesaria la autorizaci\'on por escrito del responsable del 
fichero para la ejecuci\'on de los procedimientos de recuperaci\'on de los 
datos.
\vspace{0.3cm}\\
{\large {\bf Art\'{\i}culo 22.} Pruebas con datos reales.}

Las pruebas anteriores a la implantaci\'on o modificaci\'on de los sistemas de 
informaci\'on que traten ficheros con datos de car\'acter personal no se 
realizar\'an con datos reales, salvo que se asegure el nivel de seguridad 
correspondiente al tipo de fichero tratado.
\begin{center}
{\LARGE CAP\'ITULO IV}\\ {\large Medidas de seguridad de nivel alto}
\end{center}
\vspace{0.3cm}
{\large {\bf Art\'{\i}culo 23.} Distribuci\'on de soportes.}

La distribuci\'on de los soportes que contengan datos de car\'acter personal se 
realizar\'a cifrando dichos datos o bien utilizando cualquier otro mecanismo 
que garantice que dicha informaci\'on no sea inteligible ni manipulada durante 
su transporte.
\vspace{0.3cm}\\
{\large {\bf Art\'{\i}culo 24.} Registro de accesos.}

{\bf 1.} De cada acceso se guardar\'an, como m\'{\i}nimo, la identificaci\'on 
del usuario, la fecha y hora en que se realiz\'o, el fichero accedido, el tipo 
de acceso y si ha sido autorizado o denegado.\\

{\bf 2.} En el caso de que el acceso haya sido autorizado, ser\'a preciso 
guardar la informaci\'on que permita identificar el registro accedido.\\

{\bf 3.} Los mecanismos que permiten el registro de los datos detallados en los 
p\'arrafos anteriores estar\'an bajo el control directo del responsable de 
seguridad sin que se deba permitir, en ning\'un caso, la desactivaci\'on de los 
mismos.\\

{\bf 4.} El periodo m\'{\i}nimo de conservaci\'on de los datos registrados 
ser\'a de dos a\~nos.\\

{\bf 5.} El responsable de seguridad competente se encargar\'a de revisar 
peri\'odicamente la informaci\'on de control registrada y elaborar\'a un 
informe de las revisiones realizadas y los problemas detectados al menos una 
vez al mes.
\vspace{0.3cm}\\
{\large {\bf Art\'{\i}culo 25.} Copias de respaldo y recuperaci\'on.}

Deber\'a conservarse una copia de respaldo y de los procedimientos de 
recuperaci\'on de los datos en un lugar diferente de aqu\'el en que se 
encuentren los equipos inform\'aticos que los tratan cumpliendo en todo caso,
las medidas de seguridad exigidas en este Reglamento.
\vspace{0.3cm}\\
{\large {\bf Art\'{\i}culo 26.} Telecomunicaciones.}

La transmisi\'on de datos de car\'acter personal a trav\'es de redes de 
telecomunicaciones se realizar\'a cifrando dichos datos o bien utilizando 
cualquier otro mecanismo que garantice que la informaci\'on no sea inteligible 
ni manipulada por terceros.
\begin{center}
{\LARGE CAP\'ITULO V}\\ {\large Infracciones y sanciones}
\end{center}
\vspace{0.3cm}
{\large {\bf Art\'{\i}culo 27.} Infracciones y sanciones.}

{\bf 1.} El incumplimiento de las medidas de seguridad descritas en el presente 
Reglamento ser\'a sancionado de acuerdo con lo establecido en los 
art\'{\i}culos 43 y 44 de la Ley Org\'anica 5/1992, cuando se trate de ficheros 
de titularidad privada.\\
El procedimiento a seguir para la imposici\'on de la sanci\'on a la que se 
refiere el p\'arrafo anterior ser\'a el establecido en el Real Decreto 
1332/1994, de 20 de junio, por el que se desarrollan determinados aspectos de 
la Ley Org\'anica 5/1992, de 29 de octubre, de Regulaci\'on del Tratamiento 
Automatizado de los Datos de Car\'acter Personal.\\

{\bf 2.} Cuando se trate de ficheros de los que sean responsables las 
Administraciones P\'ublicas se estar\'a, en cuanto al procedimiento y a las 
sanciones, a lo dispuesto en el art\'{\i}culo 45 de la Ley Org\'anica 5/1992.
\vspace{0.3cm}\\
{\large {\bf Art\'{\i}culo 28.} Responsables.}

Los responsables del fichero, sujetos al r\'egimen sancionador de la Ley 
Org\'anica 5/1992, deber\'an adoptar las medidas de \'{\i}ndole t\'ecnica y
organizativas necesarias que garanticen la seguridad de los datos de car\'acter 
personal en los t\'erminos establecidos en el presente Reglamento.
\begin{center}
{\LARGE CAP\'ITULO VI}\\ {\large Competencias del Director de la Agencia de
Protecci\'on de Datos}
\end{center}
\vspace{0.3cm}
{\large {\bf Art\'{\i}culo 29.} Competencias del Director de la Agencia de 
Protecci\'on de Datos.}

El Director de la Agencia de Protecci\'on de Datos podr\'a, de conformidad con 
lo establecido en el art\'{\i}culo 36 de la Ley Org\'anica 5/1992:
\begin{itemize}
\item[1.-] Dictar, en su caso y sin perjuicio de las competencias de otros 
\'organos, las instrucciones precisas para adecuar los tratamientos 
automatizados a los principios de la Ley Org\'anica 5/1992.
\item[2.-] Ordenar la cesaci\'on de los tratamientos de datos de car\'acter 
personal y la cancelaci\'on de los ficheros cuando no se cumplan las medidas de 
seguridad previstas en el presente Reglamento.
\end{itemize}
\vspace{0.3cm}
{\large {\bf Disposici\'on transitoria \'unica.} Plazos de implantaci\'on de las
medidas.}

En el caso de sistemas de informaci\'on que se encuentren en funcionamiento a 
la entrada en vigor del presente Reglamento, las medidas de seguridad de nivel 
b\'asico previstas en el presente Reglamento deber\'an implantarse en el plazo 
de seis meses desde su entrada en vigor, las de nivel medio en el plazo de un 
a\~no y las de nivel alto en el plazo de dos a\~nos.\\
\\Cuando los sistemas de informaci\'on que se encuentren en funcionamiento no 
permitan tecnol\'ogicamente la implantaci\'on de alguna de las medidas de 
seguridad previstas en el presente Reglamento, la adecuaci\'on de dichos 
sistemas y la implantaci\'on de las medidas de seguridad deber\'an realizarse 
en el plazo m\'aximo de tres a\~nos a contar desde la entrada en vigor del 
presente Reglamento.
\cleardoublepage
\section{Ley Org\'anica de Protecci\'on de Datos}\footnotetext{Ley Org\'anica 
15/1999, de 13 de diciembre, de Protecci\'on de Datos de Car\'acter Personal.}
\begin{center}
{\LARGE T\'ITULO I}\\
{\large Disposiciones generales}
\end{center}
{\large {\bf Art\'{\i}culo 1.} Objeto.}

La presente Ley Org\'anica tiene por objeto garantizar y proteger, en lo que 
concierne al tratamiento de los datos personales, las libertades p\'ublicas y 
los derechos fundamentales de las personas f\'{\i}sicas, y especialmente de su 
honor e intimidad personal y familiar.
\vspace{0.3cm}\\
{\large {\bf Art\'{\i}culo 2.} \'Ambito de aplicaci\'on.}

{\bf 1.} La presente Ley Org\'anica ser\'a de aplicaci\'on a los datos de 
car\'acter personal registrados en soporte f\'{\i}sico que los haga 
susceptibles de tratamiento, y a toda modalidad de uso posterior de estos datos 
por los sectores p\'ublico y privado.\\
Se regir\'a por la presente Ley Org\'anica todo tratamiento de datos de 
car\'acter personal:
~\par
\begin{itemize}
\item[(a)] Cuando el tratamiento sea efectuado en territorio espa\~nol 
 en el marco de las actividades de un establecimiento del responsable del 
 tratamiento.
\item[(b)] Cuando al responsable del tratamiento no establecido en territorio 
espa\~nol, le sea de aplicaci\'on la legislaci\'on espa\~nola en aplicaci\'on 
de normas de Derecho Internacional p\'ublico.
\item[(c)] Cuando el responsable del tratamiento no est\'e establecido en 
territorio de la Uni\'on Europea y utilice en el tratamiento de datos medios 
situados en territorio espa\~nol, salvo que tales medios se utilicen 
\'unicamente con fines de tr\'ansito.
\end{itemize}

{\bf 2.} El r\'egimen de protecci\'on de los datos de car\'acter personal que 
se establece en la presente Ley Org\'anica no ser\'a de aplicaci\'on:
\begin{itemize}
\item [(a)] A los ficheros mantenidos por personas f\'{\i}sicas en el ejercicio 
de actividades exclusivamente personales o dom\'esticas.
\item [(b)] A los ficheros sometidos a la normativa sobre protecci\'on de 
 materias clasificadas.
\item [(c)] A los ficheros establecidos para la investigaci\'on del terrorismo 
y de formas graves de delincuencia organizada. No obstante, en estos supuestos 
el responsable del fichero comunicar\'a previamente la existencia del mismo,
sus caracter\'{\i}sticas generales y su finalidad a la Agencia de Protecci\'on 
de Datos.
\end{itemize}

{\bf 3.} Se regir\'an por sus disposiciones espec\'{\i}ficas, y por lo 
especialmente previsto, en su caso, por esta Ley Org\'anica los siguientes 
tratamientos de datos personales:
\begin{itemize}
\item [(a)] Los ficheros regulados por la legislaci\'on de r\'egimen electoral.
\item [(b)] Los que sirvan a fines exclusivamente estad\'{\i}sticos, y est\'en 
amparados por la legislaci\'on estatal o auton\'omica sobre la funci\'on 
estad\'{\i}stica p\'ublica.
\item [(c)] Los que tengan por objeto el almacenamiento de los datos contenidos 
en los informes personales de calificaci\'on a que se refiere la legislaci\'on 
del R\'egimen del personal de las Fuerzas Armadas.
\item [(d)] Los derivados del Registro Civil y del Registro Central de penados 
y rebeldes.
\item [(e)] Los procedentes de im\'agenes y sonidos obtenidos mediante la 
utilizaci\'on de videoc\'amaras por las Fuerzas y Cuerpos de Seguridad,
de conformidad con la legislaci\'on sobre la materia.
\end{itemize}
\vspace{0.3cm}
{\large {\bf Art\'{\i}culo 3.} Definiciones.}

A los efectos de la presente Ley Org\'anica se entender\'a por:
\begin{itemize}
\item [(a)] Datos de car\'acter personal: Cualquier informaci\'on concerniente 
a personas f\'{\i}sicas identificadas o identificables.
\item [(b)] Fichero: Todo conjunto organizado de datos de car\'acter personal,
cualquiera que fuere la forma o modalidad de su creaci\'on, almacenamiento,
organizaci\'on y acceso.
\item [(c)] Tratamiento de datos: Operaciones y procedimientos t\'ecnicos de 
car\'acter automatizado o no, que permitan la recogida, grabaci\'on, 
conservaci\'on, elaboraci\'on, modificaci\'on, bloqueo y cancelaci\'on, as\'{\i}
como las cesiones de datos que resulten de comunicaciones, consultas, 
interconexiones y transferencias.
\item [(d)] Responsable del fichero o tratamiento: Persona f\'{\i}sica o 
jur\'{\i}dica, de naturaleza p\'ublica o privada, u \'organo administrativo,
que decida sobre la finalidad, contenido y uso del tratamiento.
\item [(e)] Afectado o interesado: Persona f\'{\i}sica titular de los datos 
que sean objeto del tratamiento a que se refiere el apartado c) del presente 
art\'{\i}culo.
\item [(f)] Procedimiento de disociaci\'on: Todo tratamiento de datos 
personales de modo que la informaci\'on que se obtenga no pueda asociarse a 
persona identificada o identificable.
\item [(g)] Encargado del tratamiento: La persona f\'{\i}sica o jur\'{\i}dica,
autoridad p\'ublica, servicio o cualquier otro organismo que, solo o 
conjuntamente con otros, trate datos personales por cuenta del responsable del 
tratamiento.
\item [(h)] Consentimiento del interesado: Toda manifestaci\'on de voluntad, 
libre, inequ\'{\i}voca, espec\'{\i}fica e informada, mediante la que el 
interesado consienta el tratamiento de datos personales que le conciernen.
\item [(i)] Cesi\'on o comunicaci\'on de datos: Toda revelaci\'on de datos 
realizada a una persona distinta del interesado.
\item [(j)] Fuentes accesibles al p\'ublico: Aquellos ficheros cuya consulta 
puede ser realizada por cualquier persona, no impedida por una norma limitativa,
o sin m\'as exigencia que, en su caso, el abono de una contraprestaci\'on. 
Tienen la consideraci\'on de fuentes de acceso p\'ublico, exclusivamente, el 
censo promocional, los repertorios telef\'onicos en los t\'erminos previstos 
por su normativa espec\'{\i}fica y las listas de personas pertenecientes a 
grupos de profesionales que contengan \'unicamente los datos de nombre, 
t\'{\i}tulo, profesi\'on, actividad, grado acad\'emico, direcci\'on e 
indicaci\'on de su pertenencia al grupo. Asimismo, tienen el car\'acter de 
fuentes de acceso p\'ublico, los Diarios y Boletines oficiales y los medios de 
comunicaci\'on.
\end{itemize}
\begin{center}
{\LARGE T\'ITULO II}\\
{\large Principios de la protecci\'on de datos}
\end{center} 
{\large {\bf Art\'{\i}culo 4.} Calidad de los datos.}

{\bf 1.} Los datos de car\'acter personal s\'olo se podr\'an recoger para su 
tratamiento, as\'{\i} como someterlos a dicho tratamiento, cuando sean 
adecuados, pertinentes y no excesivos en relaci\'on con el \'ambito y las 
finalidades determinadas, expl\'{\i}citas y leg\'{\i}timas para las que se 
hayan obtenido.\\

{\bf 2.} Los datos de car\'acter personal objeto de tratamiento no podr\'an 
usarse para finalidades incompatibles con aquellas para las que los datos 
hubieran sido recogidos. No se considerar\'a incompatible el tratamiento 
posterior de \'estos con fines hist\'oricos, estad\'{\i}sticos o 
cient\'{\i}ficos.\\

{\bf 3.} Los datos de car\'acter personal ser\'an exactos y puestos al d\'{\i}a 
de forma que respondan con veracidad a la situaci\'on actual del afectado.\\

{\bf 4.} Si los datos de car\'acter personal registrados resultaran ser 
inexactos, en todo o en parte, o incompletos, ser\'an cancelados y sustituidos 
de oficio por los correspondientes datos rectificados o completados, sin 
perjuicio de las facultades que a los afectados reconoce el art\'{\i}culo 16.\\

{\bf 5.} Los datos de car\'acter personal ser\'an cancelados cuando hayan 
dejado de ser necesarios o pertinentes para la finalidad para la cual hubieran 
sido recabados o registrados.\\
No ser\'an conservados en forma que permita la identificaci\'on del interesado 
durante un per\'{\i}odo superior al necesario para los fines en base a los 
cuales hubieran sido recabados o registrados.\\
Reglamentariamente se determinar\'a el procedimiento por el que, por 
excepci\'on, atendidos los valores hist\'oricos, estad\'{\i}sticos o 
cient\'{\i}ficos de acuerdo con la legislaci\'on espec\'{\i}fica, se decida el 
mantenimiento \'{\i}ntegro de determinados datos.\\

{\bf 6.} Los datos de car\'acter personal ser\'an almacenados de forma que 
permitan el ejercicio del derecho de acceso, salvo que sean legalmente 
cancelados.\\

{\bf 7.} Se proh\'{\i}be la recogida de datos por medios fraudulentos, 
desleales o il\'{\i}citos.
\vspace{0.3cm}\\
{\large {\bf Art\'{\i}culo 5.} Derecho de informaci\'on en la recogida de 
datos.}

{\bf 1.} Los interesados a los que se soliciten datos personales deber\'an ser 
previamente informados de modo expreso, preciso e inequ\'{\i}voco:
\begin{itemize}
\item [(a)] De la existencia de un fichero o tratamiento de datos de car\'acter 
personal, de la finalidad de la recogida de \'estos y de los destinatarios de 
la informaci\'on.
\item [(b)] Del car\'acter obligatorio o facultativo de su respuesta a las 
preguntas que les sean planteadas.
\item [(c)] De las consecuencias de la obtenci\'on de los datos o de la 
negativa a suministrarlos.
\item [(d)] De la posibilidad de ejercitar los derechos de acceso, 
rectificaci\'on, cancelaci\'on y oposici\'on.
\item [(e)] De la identidad y direcci\'on del responsable del tratamiento o,
en su caso, de su representante.
\end{itemize}
Cuando el responsable del tratamiento no est\'e establecido en el territorio de 
la Uni\'on Europea y utilice en el tratamiento de datos medios situados en 
territorio espa\~nol, deber\'a designar, salvo que tales medios se utilicen con 
fines de tr\'ansito, un representante en Espa\~na, sin perjuicio de las 
acciones que pudieran emprenderse contra el propio responsable del 
tratamiento.\\

{\bf 2.} Cuando se utilicen cuestionarios u otros impresos para la recogida,
figurar\'an en los mismos, en forma claramente legible, las advertencias a que 
se refiere el apartado anterior.\\

{\bf 3.} No ser\'a necesaria la informaci\'on a que se refieren las letras b),
c) y d) del apartado 1 si el contenido de ella se deduce claramente de la 
naturaleza de los datos personales que se solicitan o de las circunstancias en 
que se recaban.\\

{\bf 4.} Cuando los datos de car\'acter personal no hayan sido recabados del 
interesado, \'este deber\'a ser informado de forma expresa, precisa e 
inequ\'{\i}voca, por el responsable del fichero o su representante, dentro de 
los tres meses siguientes al momento del registro de los datos, salvo que ya 
hubiera sido informado con anterioridad, del contenido del tratamiento, de la 
procedencia de los datos, as\'{\i} como de lo previsto en las letras a), d) y 
e) del apartado 1 del presente art\'{\i}culo.\\

{\bf 5.} No ser\'a de aplicaci\'on lo dispuesto en el apartado anterior cuando 
expresamente una Ley lo prevea, cuando el tratamiento tenga fines hist\'oricos,
estad\'{\i}sticos o cient\'{\i}ficos, o cuando la informaci\'on al interesado 
resulte imposible o exija esfuerzos desproporcionados, a criterio de la Agencia 
de Protecci\'on de Datos o del organismo auton\'omico equivalente, en 
consideraci\'on al n\'umero de interesados, a la antig\"uedad de los datos y a 
las posibles medidas compensatorias.\\
Asimismo, tampoco regir\'a lo dispuesto en el apartado anterior cuando los 
datos procedan de fuentes accesibles al p\'ublico y se destinen a la actividad 
de publicidad o prospecci\'on comercial, en cuyo caso, en cada comunicaci\'on 
que se dirija al interesado se le informar\'a del origen de los datos y de la 
identidad del responsable del tratamiento as\'{\i} como de los derechos que le 
asisten.
\vspace{0.3cm}\\
{\large {\bf Art\'{\i}culo 6.} Consentimiento del afectado.}

{\bf 1.} El tratamiento de los datos de car\'acter personal requerir\'a el 
consentimiento inequ\'{\i}voco del afectado, salvo que la Ley disponga otra 
cosa.\\

{\bf 2.} No ser\'a preciso el consentimiento cuando los datos de car\'acter 
personal se recojan para el ejercicio de las funciones propias de las 
Administraciones P\'ublicas en el \'ambito de sus competencias; cuando se 
refieran a las partes de un contrato o precontrato de una relaci\'on negocial,
laboral o administrativa y sean necesarios para su mantenimiento o 
cumplimiento; cuando el tratamiento de los datos tenga por finalidad proteger 
un inter\'es vital del interesado en los t\'erminos del art\'{\i}culo 7, 
apartado 6 de la presente Ley, o cuando los datos figuren en fuentes accesibles 
al p\'ublico y su tratamiento sea necesario para la satisfacci\'on del 
inter\'es leg\'{\i}timo perseguido por el responsable del fichero o por el del 
tercero a quien se comuniquen los datos, siempre que no se vulneren los 
derechos y libertades fundamentales del interesado.\\

{\bf 3.} El consentimiento a que se refiere el art\'{\i}culo podr\'a ser 
revocado cuando exista causa justificada para ello y no se le atribuyan efectos 
retroactivos.\\

{\bf 4.} En los casos en los que no sea necesario el consentimiento del 
afectado para el tratamiento de los datos de car\'acter personal, y siempre que 
una Ley no disponga lo contrario, \'este podr\'a oponerse a su tratamiento 
cuando existan motivos fundados y leg\'{\i}timos relativos a una concreta 
situaci\'on personal. En tal supuesto, el responsable del fichero excluir\'a 
del tratamiento los datos relativos al afectado.
\vspace{0.3cm}\\
{\large {\bf Art\'{\i}culo 7.} Datos especialmente protegidos.}

{\bf 1.} De acuerdo con lo establecido en el apartado 2 del art\'{\i}culo 16 de 
la Constituci\'on, nadie podr\'a ser obligado a declarar sobre su 
ideolog\'{\i}a, religi\'on o creencias.\\
Cuando en relaci\'on con estos datos se proceda a recabar el consentimiento a 
que se refiere el apartado siguiente, se advertir\'a al interesado acerca de su 
derecho a no prestarlo.\\

{\bf 2.} S\'olo con el consentimiento expreso y por escrito del afectado 
podr\'an ser objeto de tratamiento los datos de car\'acter personal que revelen 
la ideolog\'{\i}a, afiliaci\'on sindical, religi\'on y creencias. Se 
except\'uan los ficheros mantenidos por los partidos pol\'{\i}ticos,
sindicatos, iglesias, confesiones o comunidades religiosas y asociaciones,
fundaciones y otras entidades sin \'animo de lucro, cuya finalidad sea 
pol\'{\i}tica, filos\'ofica, religiosa o sindical, en cuanto a los datos 
relativos a sus asociados o miembros, sin perjuicio de que la cesi\'on de 
dichos datos precisar\'a siempre el previo consentimiento del afectado.\\

{\bf 3.} Los datos de car\'acter personal que hagan referencia al origen racial,
a la salud y a la vida sexual s\'olo podr\'an ser recabados, tratados y cedidos 
cuando, por razones de inter\'es general, as\'{\i} lo disponga una Ley o el 
afectado consienta expresamente.\\

{\bf 4.} Quedan prohibidos los ficheros creados con la finalidad exclusiva de 
almacenar datos de car\'acter personal que revelen la ideolog\'{\i}a, 
afiliaci\'on sindical, religi\'on, creencias, origen racial o \'etnico, o vida 
sexual.\\

{\bf 5.} Los datos de car\'acter personal relativos a la comisi\'on de 
infracciones penales o administrativas s\'olo podr\'an ser incluidos en 
ficheros de las Administraciones P\'ublicas competentes en los supuestos 
previstos en las respectivas normas reguladoras.\\

{\bf 6.} No obstante lo dispuesto en los apartados anteriores podr\'an ser 
objeto de tratamiento los datos de car\'acter personal a que se refieren los 
apartados 2 y 3 de este art\'{\i}culo, cuando dicho tratamiento resulte 
necesario para la prevenci\'on o para el diagn\'ostico m\'edicos, la 
prestaci\'on de asistencia sanitaria o tratamientos m\'edicos o la gesti\'on de 
servicios sanitarios, siempre que dicho tratamiento de datos se realice por un 
profesional sanitario sujeto al secreto profesional o por otra persona sujeta 
asimismo a una obligaci\'on equivalente de secreto.\\
Tambi\'en podr\'an ser objeto de tratamiento los datos a que se refiere el 
p\'arrafo anterior cuando el tratamiento sea necesario para salvaguardar el 
inter\'es vital del afectado o de otra persona, en el supuesto de que el 
afectado est\'e f\'{\i}sica o jur\'{\i}dicamente incapacitado para dar su 
consentimiento.
\vspace{0.3cm}\\
{\large {\bf Art\'{\i}culo 8.} Datos relativos a la salud.}

Sin perjuicio de lo que se dispone en el art\'{\i}culo 11 respecto de la 
cesi\'on, las instituciones y los centros sanitarios p\'ublicos y privados y 
los profesionales correspondientes podr\'an proceder al tratamiento de los 
datos de car\'acter personal relativos a la salud de las personas que a ellos 
acudan o hayan de ser tratados en los mismos, de acuerdo con lo dispuesto en la 
legislaci\'on estatal o auton\'omica sobre sanidad.
\vspace{0.3cm}\\
{\large {\bf Art\'{\i}culo 9.} Seguridad de los datos.}

{\bf 1.} El responsable del fichero, y, en su caso, el encargado del 
tratamiento, deber\'an adoptar las medidas de \'{\i}ndole t\'ecnica y 
organizativas necesarias que garanticen la seguridad de los datos de car\'acter 
personal y eviten su alteraci\'on, p\'erdida, tratamiento o acceso no
autorizado, habida cuenta del estado de la tecnolog\'{\i}a, la naturaleza de 
los datos almacenados y los riesgos a que est\'an expuestos, ya provengan de la 
acci\'on humana o del medio f\'{\i}sico o natural.\\

{\bf 2.} No se registrar\'an datos de car\'acter personal en ficheros que no 
re\'unan las condiciones que se determinen por v\'{\i}a reglamentaria con 
respecto a su integridad y seguridad y a las de los centros de tratamiento,
locales, equipos, sistemas y programas.\\

{\bf 3.} Reglamentariamente se establecer\'an los requisitos y condiciones que 
deban reunir los ficheros y las personas que intervengan en el tratamiento de 
los datos a que se refiere el art\'{\i}culo 7 de esta Ley.
\vspace{0.3cm}\\
{\large {\bf Art\'{\i}culo 10.} Deber de secreto.}

El responsable del fichero y quienes intervengan en cualquier fase del 
tratamiento de los datos de car\'acter personal est\'an obligados al secreto 
profesional respecto de los mismos y al deber de guardarlos, obligaciones que 
subsistir\'an aun despu\'es de finalizar sus relaciones con el titular del 
fichero o, en su caso, con el responsable del mismo.\\
\vspace{0.3cm}\\
{\large {\bf Art\'{\i}culo 11.} Comunicaci\'on de datos.}

{\bf 1.} Los datos de car\'acter personal objeto del tratamiento s\'olo 
podr\'an ser comunicados a un tercero para el cumplimiento de fines 
directamente relacionados con las funciones leg\'{\i}timas del cedente y del 
cesionario con el previo consentimiento del interesado.\\

{\bf 2.} El consentimiento exigido en el apartado anterior no ser\'a preciso:
\begin{itemize}
\item [(a)] Cuando la cesi\'on est\'a autorizada en una Ley.
\item [(b)] Cuando se trate de datos recogidos de fuentes accesibles al 
p\'ublico.
\item [(c)] Cuando el tratamiento responda a la libre y leg\'{\i}tima 
aceptaci\'on de una relaci\'on jur\'{\i}dica cuyo desarrollo, cumplimiento y 
control implique necesariamente la conexi\'on de dicho tratamiento con ficheros 
de terceros. En este caso la comunicaci\'on s\'olo ser\'a leg\'{\i}tima en 
cuanto se limite a la finalidad que la justifique.
\item [(d)] Cuando la comunicaci\'on que deba efectuarse tenga por destinatario 
al Defensor del Pueblo, el Ministerio Fiscal o los Jueces o Tribunales o el 
Tribunal de Cuentas, en el ejercicio de las funciones que tiene atribuidas. 
Tampoco ser\'a preciso el consentimiento cuando la comunicaci\'on tenga como 
destinatario a instituciones auton\'omicas con funciones an\'alogas al Defensor 
del Pueblo o al Tribunal de Cuentas.
\item [(e)] Cuando la cesi\'on se produzca entre Administraciones P\'ublicas y 
tenga por objeto el tratamiento posterior de los datos con fines hist\'oricos,
estad\'{\i}sticos o cient\'{\i}ficos.
\item [(f)] Cuando la cesi\'on de datos de car\'acter personal relativos a la 
salud sea necesaria para solucionar una urgencia que requiera acceder a un 
fichero o para realizar los estudios epidemiol\'ogicos en los t\'erminos 
establecidos en la legislaci\'on sobre sanidad estatal o auton\'omica.
\end{itemize}

{\bf 3.} Ser\'a nulo el consentimiento para la comunicaci\'on de los datos de 
car\'acter personal a un tercero cuando la informaci\'on que se facilite al 
interesado no le permita conocer la finalidad a que destinar\'an los datos cuya 
comunicaci\'on se autoriza o el tipo de actividad de aqu\'el a quien se 
pretenden comunicar.\\

{\bf 4.} El consentimiento para la comunicaci\'on de los datos de car\'acter 
personal tiene tambi\'en un car\'acter de revocable.\\

{\bf 5.} Aqu\'el a quien se comuniquen los datos de car\'acter personal se 
obliga, por el solo hecho de la comunicaci\'on, a la observancia de las 
disposiciones de la presente Ley.\\

{\bf 6.} Si la comunicaci\'on se efect\'ua previo procedimiento de 
disociaci\'on, no ser\'a aplicable lo establecido en los apartados anteriores.
\vspace{0.3cm}\\
{\large {\bf Art\'{\i}culo 12.} Acceso a los datos por cuenta de terceros.}

{\bf 1.} No se considerar\'a comunicaci\'on de datos el acceso de un tercero a 
los datos cuando dicho acceso sea necesario para la prestaci\'on de un servicio 
al responsable del tratamiento.\\

{\bf 2.} La realizaci\'on de tratamientos por cuenta de terceros deber\'a estar 
regulada en un contrato que deber\'a constar por escrito o en alguna otra forma 
que permita acreditar su celebraci\'on y contenido, estableci\'endose 
expresamente que el encargado del tratamiento \'unicamente tratar\'a los datos 
conforme a las instrucciones del responsable del tratamiento, que no los 
aplicar\'a o utilizar\'a con fin distinto al que figure en dicho contrato, ni 
los comunicar\'a, ni siquiera para su conservaci\'on, a otras personas.\\
En el contrato se estipular\'an, asimismo, las medidas de seguridad a que se 
refiere el art\'{\i}culo 9 de esta Ley que el encargado del tratamiento est\'a 
obligado a implementar.\\

{\bf 3.} Una vez cumplida la prestaci\'on contractual, los datos de car\'acter 
personal deber\'an ser destruidos o devueltos al responsable del tratamiento,
al igual que cualquier soporte o documentos en que conste alg\'un dato de 
car\'acter personal objeto del tratamiento.\\

{\bf 4.} En el caso de que el encargado del tratamiento destine los datos a 
otra finalidad, los comunique o los utilice incumpliendo las estipulaciones del 
contrato, ser\'a considerado, tambi\'en, responsable del tratamiento, 
respondiendo de las infracciones en que hubiera incurrido personalmente.\\
\begin{center}
{\LARGE T\'ITULO III}\\
{\large Derechos de las personas}
\end{center}
\vspace{0.3cm}
{\large {\bf Art\'{\i}culo 13.} Impugnaci\'on de valoraciones.}

{\bf 1.} Los ciudadanos tienen derecho a no verse sometidos a una decisi\'on 
con efectos jur\'{\i}dicos, sobre ellos o que les afecte de manera 
significativa, que se base \'unicamente en un tratamiento de datos destinados a 
evaluar determinados aspectos de su personalidad.\\

{\bf 2.} El afectado podr\'a impugnar los actos administrativos o decisiones 
privadas que impliquen una valoraci\'on de su comportamiento, cuyo \'unico 
fundamento sea un tratamiento de datos de car\'acter personal que ofrezca una 
definici\'on de sus caracter\'{\i}sticas o personalidad.\\

{\bf 3.} En este caso, el afectado tendr\'a derecho a obtener informaci\'on del 
responsable del fichero sobre los criterios de valoraci\'on y el programa 
utilizados en el tratamiento que sirvi\'o para adoptar la decisi\'on en que 
consisti\'o el acto.\\

{\bf 4.} La valoraci\'on sobre el comportamiento de los ciudadanos basada en un 
tratamiento de datos, \'unicamente podr\'a tener valor probatorio a petici\'on 
del afectado.
\vspace{0.3cm}\\
{\large {\bf Art\'{\i}culo 14.} Derecho de Consulta al Registro General de 
Protecci\'on de Datos.}

Cualquier persona podr\'a conocer, recabando a tal fin la informaci\'on 
oportuna del Registro General de Protecci\'on de Datos, la existencia de 
tratamientos de datos de car\'acter personal, sus finalidades y la identidad 
del responsable del tratamiento. El Registro General ser\'a de consulta 
p\'ublica y gratuita.
\vspace{0.3cm}\\
{\large {\bf Art\'{\i}culo 15.} Derecho de acceso.}

{\bf 1.} El interesado tendr\'a derecho a solicitar y obtener gratuitamente 
informaci\'on de sus datos de car\'acter personal sometidos a tratamiento, el 
origen de dichos datos as\'{\i} como las comunicaciones realizadas o que se 
prev\'en hacer de los mismos.\\

{\bf 2.} La informaci\'on podr\'a obtenerse mediante la mera consulta de los 
datos por medio de su visualizaci\'on, o la indicaci\'on de los datos que son 
objeto de tratamiento mediante escrito, copia, telecopia o fotocopia, 
certificada o no, en forma legible e inteligible, sin utilizar claves o 
c\'odigos que requieran el uso de dispositivos mec\'anicos espec\'{\i}ficos.\\

{\bf 3.} El derecho de acceso a que se refiere este art\'{\i}culo s\'olo 
podr\'a ser ejercitado a intervalos no inferiores a doce meses, salvo que el 
interesado acredite un inter\'es leg\'{\i}timo al efecto, en cuyo caso podr\'a 
ejercitarlo antes.
\vspace{0.3cm}\\
{\large {\bf Art\'{\i}culo 16.} Derecho de rectificaci\'on y cancelaci\'on.}

{\bf 1.} El responsable del tratamiento tendr\'a la obligaci\'on de hacer 
efectivo el derecho de rectificaci\'on o cancelaci\'on del interesado en el 
plazo de diez d\'{\i}as.\\

{\bf 2.} Ser\'an rectificados o cancelados, en su caso, los datos de car\'acter 
personal cuyo tratamiento no se ajuste a lo dispuesto en la presente Ley y, 
en particular, cuando tales datos resulten inexactos o incompletos.\\

{\bf 3.} La cancelaci\'on dar\'a lugar al bloqueo de los datos, conserv\'andose 
\'unicamente a disposici\'on de las Administraciones P\'ublicas, Jueces y 
Tribunales, para la atenci\'on de las posibles responsabilidades nacidas del 
tratamiento, durante el plazo de prescripci\'on de \'estas.\\
Cumplido el citado plazo deber\'a procederse a la supresi\'on.\\

{\bf 4.} Si los datos rectificados o cancelados hubieran sido comunicados 
previamente, el responsable del tratamiento deber\'a notificar la 
rectificaci\'on o cancelaci\'on efectuada a quien se hayan comunicado, en el 
caso de que se mantenga el tratamiento por este \'ultimo, que deber\'a 
tambi\'en proceder a la cancelaci\'on.\\

{\bf 5.} Los datos de car\'acter personal deber\'an ser conservados durante los 
plazos previstos en las disposiciones aplicables o, en su caso, en las 
relaciones contractuales entre la persona o entidad responsable del tratamiento 
y el interesado.
\vspace{0.3cm}\\
{\large {\bf Art\'{\i}culo 17.} Procedimiento de oposici\'on, acceso, 
rectificaci\'on o cancelaci\'on.}

{\bf 1.} Los procedimientos para ejercitar el derecho de oposici\'on, acceso,
as\'{\i} como los de rectificaci\'on y cancelaci\'on ser\'an establecidos 
reglamentariamente.\\

{\bf 2.} No se exigir\'a contraprestaci\'on alguna por el ejercicio de los 
derechos de oposici\'on, acceso, rectificaci\'on o cancelaci\'on.
\vspace{0.3cm}\\
{\large {\bf Art\'{\i}culo 18.} Tutela de los derechos.}

{\bf 1.} Las actuaciones contrarias a lo dispuesto en la presente Ley pueden 
ser objeto de reclamaci\'on por los interesados ante la Agencia de Protecci\'on 
de Datos, en la forma que reglamentariamente se determine.\\

{\bf 2.} El interesado al que se deniegue, total o parcialmente, el ejercicio 
de los derechos de oposici\'on, acceso, rectificaci\'on o cancelaci\'on, 
podr\'a ponerlo en conocimiento de la Agencia de Protecci\'on de Datos o, en su 
caso, del Organismo competente de cada Comunidad Aut\'onoma, que deber\'a 
asegurarse de la procedencia o improcedencia de la denegaci\'on.\\

{\bf 3.} El plazo m\'aximo en que debe dictarse la resoluci\'on expresa de 
tutela de derechos ser\'a de seis meses.\\

{\bf 4.} Contra las resoluciones de la Agencia de Protecci\'on de Datos 
proceder\'a recurso contencioso--administrativo.
\vspace{0.3cm}\\
{\large {\bf Art\'{\i}culo 19.} Derecho a indemnizaci\'on.}

{\bf 1.} Los interesados que, como consecuencia del incumplimiento de lo 
dispuesto en la presente Ley por el responsable o el encargado del tratamiento,
sufran da\~no o lesi\'on en sus bienes o derechos tendr\'an derecho a ser 
indemnizados.\\

{\bf 2.} Cuando se trate de ficheros de titularidad p\'ublica, la 
responsabilidad se exigir\'a de acuerdo con la legislaci\'on reguladora del 
r\'egimen de responsabilidad de las Administraciones P\'ublicas.\\

{\bf 3.} En el caso de los ficheros de titularidad privada, la acci\'on se 
ejercitar\'a ante los \'organos de la jurisdicci\'on ordinaria.
\begin{center}
{\LARGE T\'ITULO IV}\\
{\large Disposiciones sectoriales}
\end{center}
\vspace{0.3cm}
{\large CAP\'ITULO PRIMERO. Ficheros de titularidad p\'ublica}
\vspace{0.3cm}\\
{\large {\bf Art\'{\i}culo 20.} Creaci\'on, modificaci\'on o supresi\'on.}

{\bf 1.} La creaci\'on, modificaci\'on o supresi\'on de los ficheros de las 
Administraciones P\'ublicas s\'olo podr\'an hacerse por medio de disposici\'on 
general publicada en el `Bolet\'{\i}n Oficial del Estado' o diario oficial 
correspondiente.\\

{\bf 2.} Las disposiciones de creaci\'on o de modificaci\'on de ficheros 
deber\'an indicar:
\begin{itemize}
\item [(a)] La finalidad del fichero y los usos previstos para el mismo.
\item [(b)] Las personas o colectivos sobre los que se pretenda obtener datos 
de car\'acter personal o que resulten obligados a suministrarlos.
\item [(c)] El procedimiento de recogida de los datos de car\'acter personal.
\item [(d)] La estructura b\'asica del fichero y la descripci\'on de los tipos 
de datos de car\'acter personal incluidos en el mismo.
\item [(e)] Las cesiones de datos de car\'acter personal y, en su caso, las 
transferencias de datos que se prevean a pa\'{\i}ses terceros.
\item [(f)] Los \'organos de las Administraciones responsables del fichero.
\item [(g)] Los servicios o unidades ante los que pudiesen ejercitarse los 
derechos de acceso, rectificaci\'on, cancelaci\'on y oposici\'on.
\item [(h)] Las medidas de seguridad con indicaci\'on del nivel b\'asico, medio 
o alto exigible.
\end{itemize}

{\bf 3.} En las disposiciones que se dicten para la supresi\'on de los ficheros 
se establecer\'a el destino de los mismos o, en su caso, las previsiones que se 
adopten para su destrucci\'on.
\vspace{0.3cm}\\
{\large {\bf Art\'{\i}culo 21.} Comunicaci\'on de datos entre Administraciones 
P\'ublicas.}

{\bf 1.} Los datos de car\'acter personal recogidos o elaborados por las 
Administraciones P\'ublicas para el desempe\~no de sus atribuciones no ser\'an 
comunicados a otras Administraciones P\'ublicas para el ejercicio de 
competencias diferentes o de competencias que versen sobre materias distintas,
salvo cuando la comunicaci\'on hubiere sido prevista por las disposiciones de 
creaci\'on del fichero o por disposici\'on de superior rango que regule su uso,
o cuando la comunicaci\'on tenga por objeto el tratamiento posterior de los 
datos con fines hist\'oricos, estad\'{\i}sticos o cient\'{\i}ficos.\\

{\bf 2.} Podr\'an, en todo caso, ser objeto de comunicaci\'on los datos de 
car\'acter personal que una Administraci\'on P\'ublica obtenga o elabore con 
destino a otra.\\

{\bf 3.} No obstante lo establecido en el art\'{\i}culo 11.2 (b), la 
comunicaci\'on de datos recogidos de fuentes accesibles al p\'ublico no 
podr\'a efectuarse a ficheros de titularidad privada, sino con el 
consentimiento del interesado o cuando una Ley prevea otra cosa.\\

{\bf 4.} En los supuestos previstos en los apartados 1 y 2 del presente 
art\'{\i}culo no ser\'a necesario el consentimiento del afectado a que se 
refiere el art\'{\i}culo 11 de la presente Ley.
\vspace{0.3cm}\\
{\large {\bf Art\'{\i}culo 22.} Ficheros de las Fuerzas y Cuerpos de Seguridad.}

{\bf 1.} Los ficheros creados por las Fuerzas y Cuerpos de Seguridad que 
contengan datos de car\'acter personal que, por haberse recogido para fines 
administrativos, deban ser objeto de registro permanente, estar\'an sujetos al 
r\'egimen general de la presente Ley.\\

{\bf 2.} La recogida y tratamiento para fines policiales de datos de car\'acter 
personal por las Fuerzas y Cuerpos de Seguridad sin consentimiento de las 
personas afectadas est\'an limitados a aquellos supuestos y categor\'{\i}as de 
datos que resulten necesarios para la prevenci\'on de un peligro real para la 
seguridad p\'ublica o para la represi\'on de infracciones penales, debiendo ser 
almacenados en ficheros espec\'{\i}ficos establecidos al efecto, que deber\'an 
clasificarse por categor\'{\i}as en funci\'on de su grado de fiabilidad.\\

{\bf 3.} La recogida y tratamiento por las Fuerzas y Cuerpos de Seguridad de 
los datos a que hacen referencia los apartados 2 y 3 del art\'{\i}culo 7, 
podr\'an realizarse exclusivamente en los supuestos en que sea absolutamente 
necesario para los fines de una investigaci\'on concreta, sin perjuicio del 
control de legalidad de la actuaci\'on administrativa o de la obligaci\'on de 
resolver las pretensiones formuladas en su caso por los interesados que 
corresponden a los \'organos jurisdiccionales.\\

{\bf 4.} Los datos personales registrados con fines policiales se cancelar\'an 
cuando no sean necesarios para las averiguaciones que motivaron su 
almacenamiento.\\
A estos efectos, se considerar\'a especialmente la edad del afectado y el 
car\'acter de los datos almacenados, la necesidad de mantener los datos hasta 
la conclusi\'on de una investigaci\'on o procedimiento concreto, la 
resoluci\'on judicial firme, en especial la absolutoria, el indulto, la 
rehabilitaci\'on y la prescripci\'on de responsabilidad.
\vspace{0.3cm}\\
{\large {\bf Art\'{\i}culo 23.} Excepciones a los derechos de acceso, 
rectificaci\'on y cancelaci\'on.}

{\bf 1.} Los responsables de los ficheros que contengan los datos a que se 
refieren los apartados 2, 3 y 4 del art\'{\i}culo anterior podr\'an denegar el 
acceso, la rectificaci\'on o cancelaci\'on en funci\'on de los peligros que 
pudieran derivarse para la defensa del Estado o la seguridad p\'ublica, la 
protecci\'on de los derechos y libertades de terceros o las necesidades de las 
investigaciones que se est\'en realizando.\\

{\bf 2.} Los responsables de los ficheros de la Hacienda P\'ublica podr\'an,
igualmente, denegar el ejercicio de los derechos a que se refiere el apartado 
anterior cuando el mismo obstaculice las actuaciones administrativas tendentes 
a asegurar el cumplimiento de las obligaciones tributarias y, en todo caso, 
cuando el afectado est\'e siendo objeto de actuaciones inspectoras.\\

{\bf 3.} El afectado al que se deniegue, total o parcialmente, el ejercicio de 
los derechos mencionados en los apartados anteriores podr\'a ponerlo en 
conocimiento del Director de la Agencia de Protecci\'on de Datos o del 
Organismo competente de cada Comunidad Aut\'onoma en el caso de ficheros 
mantenidos por Cuerpos de Polic\'{\i}a propios de \'estas, o por las 
Administraciones Tributarias Auton\'omicas, quienes deber\'an asegurarse de la 
procedencia o improcedencia de la denegaci\'on.
\vspace{0.3cm}\\
{\large {\bf Art\'{\i}culo 24.} Otras excepciones a los derechos de los 
afectados.}

{\bf 1.} Lo dispuesto en los apartados 1 y 2 del art\'{\i}culo 5 no ser\'a 
aplicable a la recogida de datos cuando la informaci\'on al afectado impida o 
dificulte gravemente el cumplimiento de las funciones de control y 
verificaci\'on de las Administraciones P\'ublicas o cuando afecte a la Defensa 
Nacional, a la seguridad p\'ublica o a la persecuci\'on de infracciones penales 
o administrativas.\\

{\bf 2.} Lo dispuesto en el art\'{\i}culo 15 y en el apartado 1 del 
art\'{\i}culo 16 no ser\'a de aplicaci\'on si, ponderados los intereses en 
presencia, resultase que los derechos que dichos preceptos conceden al afectado 
hubieran de ceder ante razones de inter\'es p\'ublico o ante intereses de 
terceros m\'as dignos de protecci\'on. Si el \'organo administrativo 
responsable del fichero invocase lo dispuesto en este apartado, dictar\'a 
resoluci\'on motivada e instruir\'a al afectado del derecho que le asiste a 
poner la negativa en conocimiento del Director de la Agencia de Protecci\'on de 
Datos o, en su caso, del \'organo equivalente de las Comunidades Aut\'onomas.
\vspace{0.3cm}\\
{\large CAP\'ITULO SEGUNDO. Ficheros de titularidad privada}
\vspace{0.3cm}\\
{\large {\bf Art\'{\i}culo 25.} Creaci\'on.}

Podr\'an crearse ficheros de titularidad privada que contengan datos de 
car\'acter personal cuando resulte necesario para el logro de la actividad u 
objeto leg\'{\i}timos de la persona, empresa o entidad titular y se respeten 
las garant\'{\i}as que esta Ley establece para la protecci\'on de las personas.
\vspace{0.3cm}\\
{\large {\bf Art\'{\i}culo 26.} Notificaci\'on e inscripci\'on registral.}

{\bf 1.} Toda persona o entidad que proceda a la creaci\'on de ficheros de 
datos de car\'acter personal lo notificar\'a previamente a la Agencia de 
Protecci\'on de Datos.\\

{\bf 2.} Por v\'{\i}a reglamentaria se proceder\'a a la regulaci\'on detallada 
de los distintos extremos que debe contener la notificaci\'on, entre los cuales 
figurar\'an necesariamente el responsable del fichero, la finalidad del mismo,
su ubicaci\'on, el tipo de datos de car\'acter personal que contiene, las 
medidas de seguridad, con indicaci\'on del nivel b\'asico, medio o alto 
exigible y las cesiones de datos de car\'acter personal que se prevean realizar 
y, en su caso, las transferencias de datos que se prevean a pa\'{\i}ses 
terceros.\\

{\bf 3.} Deber\'an comunicarse a la Agencia de Protecci\'on de Datos los 
cambios que se produzcan en la finalidad del fichero automatizado, en su 
responsable y en la direcci\'on de su ubicaci\'on.\\

{\bf 4.} El Registro General de Protecci\'on de Datos inscribir\'a el fichero 
si la notificaci\'on se ajusta a los requisitos exigibles.\\
En caso contrario podr\'a pedir que se completen los datos que falten o se 
proceda a su subsanaci\'on.\\

{\bf 5.} Transcurrido un mes desde la presentaci\'on de la solicitud de 
inscripci\'on sin que la Agencia de Protecci\'on de Datos hubiera resuelto 
sobre la misma, se entender\'a inscrito el fichero automatizado a todos los 
efectos.
\vspace{0.3cm}\\
{\large {\bf Art\'{\i}culo 27.} Comunicaci\'on de la cesi\'on de datos.}

{\bf 1.} El responsable del fichero, en el momento en que se efect\'ue la 
primera cesi\'on de datos, deber\'a informar de ello a los afectados, indicando,
asimismo, la finalidad del fichero, la naturaleza de los datos que han sido 
cedidos y el nombre y direcci\'on del cesionario.\\

{\bf 2.} La obligaci\'on establecida en el apartado anterior no existir\'a en 
el supuesto previsto en los apartados 2, letras (c), (d), (e) y 6 del 
art\'{\i}culo 11, ni cuando la cesi\'on venga impuesta por Ley.
\vspace{0.3cm}\\
{\large {\bf Art\'{\i}culo 28.} Datos incluidos en las fuentes de acceso 
p\'ublico.}

{\bf 1.} Los datos personales que figuren en el censo promocional o las listas 
de personas pertenecientes a grupos de profesionales a que se refiere el 
art\'{\i}culo 3 (j) de esta Ley deber\'an limitarse a los que sean 
estrictamente necesarios para cumplir la finalidad a que se destina cada 
listado. La inclusi\'on de datos adicionales por las entidades responsables del 
mantenimiento de dichas fuentes requerir\'a el consentimiento del interesado, 
que podr\'a ser revocado en cualquier momento.\\

{\bf 2.} Los interesados tendr\'an derecho a que la entidad responsable del 
mantenimiento de los listados de los Colegios profesionales indique 
gratuitamente que sus datos personales no pueden utilizarse para fines de 
publicidad o prospecci\'on comercial.\\
Los interesados tendr\'an derecho a exigir gratuitamente la exclusi\'on de la 
totalidad de sus datos personales que consten en el censo promocional por las 
entidades encargadas del mantenimiento de dichas fuentes.\\
La atenci\'on a la solicitud de exclusi\'on de la informaci\'on innecesaria o 
de inclusi\'on de la objeci\'on al uso de los datos para fines de publicidad o 
venta a distancia deber\'a realizarse en el plazo de diez d\'{\i}as respecto de 
las informaciones que se realicen mediante consulta o comunicaci\'on 
telem\'atica y en la siguiente edici\'on del listado cualquiera que sea el 
soporte en que se edite.\\

{\bf 3.} Las fuentes de acceso p\'ublico que se editen en forma de libro o 
alg\'un otro soporte f\'{\i}sico, perder\'an el car\'acter de fuente accesible 
con la nueva edici\'on que se publique.\\
En el caso de que se obtenga telem\'aticamente una copia de la lista en formato 
electr\'onico, \'esta perder\'a el car\'acter de fuente de acceso p\'ublico en 
el plazo de un a\~no, contado desde el momento de su obtenci\'on.\\

{\bf 4.} Los datos que figuren en las gu\'{\i}as de servicios de 
telecomunicaciones disponibles al p\'ublico se regir\'an por su normativa 
espec\'{\i}fica.
\vspace{0.3cm}\\
{\large {\bf Art\'{\i}culo 29.} Prestaci\'on de servicios de informaci\'on 
sobre solvencia patrimonial y cr\'edito.}

{\bf 1.} Quienes se dediquen a la prestaci\'on de servicios de informaci\'on 
sobre la solvencia patrimonial y el cr\'edito s\'olo podr\'an tratar datos de 
car\'acter personal obtenidos de los registros y las fuentes accesibles al 
p\'ublico establecidos al efecto o procedentes de informaciones facilitadas por 
el interesado o con su consentimiento.\\

{\bf 2.} Podr\'an tratarse tambi\'en datos de car\'acter personal relativos al 
cumplimiento o incumplimiento de obligaciones dinerarias facilitados por el 
creedor o por quien act\'ue por su cuenta o inter\'es. En estos casos se 
notificar\'a a los interesados respecto de los que hayan registrado datos de 
car\'acter personal en ficheros, en el plazo de treinta d\'{\i}as desde dicho 
registro, una referencia de los que hubiesen sido incluidos y se les 
informar\'a de su derecho a recabar informaci\'on de la totalidad de ellos, en 
los t\'erminos establecidos por la presente Ley.\\

{\bf 3.} En los supuestos a que se refieren los dos apartados anteriores cuando 
el interesado lo solicite, el responsable del tratamiento le comunicar\'a los 
datos, as\'{\i} como las evaluaciones y apreciaciones que sobre el mismo hayan 
sido comunicadas durante los \'ultimos seis meses y el nombre y direcci\'on de 
la persona o entidad a quien se hayan revelado los datos.\\

{\bf 4.} S\'olo se podr\'an registrar y ceder los datos de car\'acter personal 
que sean determinantes para enjuiciar la solvencia econ\'omica de los 
interesados y que no se refieran, cuando sean adversos, a m\'as de seis a\~nos,
siempre que respondan con veracidad a la situaci\'on actual de aquellos.
\vspace{0.3cm}\\
{\large {\bf Art\'{\i}culo 30.} Tratamientos con fines de publicidad y de 
prospecci\'on comercial.}

{\bf 1.} Quienes se dediquen a la recopilaci\'on de direcciones, reparto de 
documentos, publicidad, venta a distancia, prospecci\'on comercial y otras 
actividades an\'alogas, utilizar\'an nombres y direcciones u otros datos de 
car\'acter personal cuando los mismos figuren en fuentes accesibles al 
p\'ublico o cuando hayan sido facilitados por los propios interesados u 
obtenidos con su consentimiento.\\

{\bf 2.} Cuando los datos procedan de fuentes accesibles al p\'ublico, de 
conformidad con lo establecido en el p\'arrafo segundo del art\'{\i}culo 5.5 de 
esta Ley, en cada comunicaci\'on que se dirija al interesado se informar\'a del 
origen de los datos y de la identidad del responsable del tratamiento, as\'{\i} 
como de los derechos que le asisten.\\

{\bf 3.} En el ejercicio del derecho de acceso los interesados tendr\'an 
derecho a conocer el origen de sus datos de car\'acter personal, as\'{\i} como 
del resto de informaci\'on a que se refiere el art\'{\i}culo 15.\\

{\bf 4.} Los interesados tendr\'an derecho a oponerse, previa petici\'on y sin 
gastos, al tratamiento de los datos que les conciernan, en cuyo caso ser\'an 
dados de baja del tratamiento, cancel\'andose las informaciones que sobre ellos 
figuren en aqu\'el, a su simple solicitud.
\vspace{0.3cm}\\
{\large {\bf Art\'{\i}culo 31.} Censo Promocional.}

{\bf 1.} Quienes pretendan realizar permanente o espor\'adicamente la actividad 
de recopilaci\'on de direcciones, reparto de documentos, publicidad, venta a 
distancia, prospecci\'on comercial u otras actividades an\'alogas, podr\'an 
solicitar del Instituto Nacional de Estad\'{\i}stica o de los \'organos 
equivalentes de las Comunidades Aut\'onomas una copia del censo promocional,
formado con los datos de nombre, apellidos y domicilio que constan en el censo 
electoral.\\

{\bf 2.} El uso de cada lista de censo promocional tendr\'a un plazo de 
vigencia de un a\~no. Transcurrido el plazo citado, la lista perder\'a su 
car\'acter de fuente de acceso p\'ublico.\\

{\bf 3.} Los procedimientos mediante los que los interesados podr\'an solicitar 
no aparecer en el censo promocional se regular\'an reglamentariamente. Entre 
estos procedimientos, que ser\'an gratuitos para los interesados, se incluir\'a 
el documento de empadronamiento.\\
Trimestralmente se editar\'a una lista actualizada del censo promocional,
excluyendo los nombres y domicilios de los queas\'{\i} lo hayan solicitado.\\

{\bf 4.} Se podr\'a exigir una contraprestaci\'on por la facilitaci\'on de la 
citada lista en soporte inform\'atico.
\vspace{0.3cm}\\
{\large {\bf Art\'{\i}culo 32.} C\'odigos tipo.}

{\bf 1.} Mediante acuerdos sectoriales, convenios administrativos o decisiones 
de empresa, los responsables de tratamientos de titularidad p\'ublica y privada 
as\'{\i} como las organizaciones en que se agrupen, podr\'an formular c\'odigos 
tipo que establezcan las condiciones de organizaci\'on, r\'egimen de 
funcionamiento, procedimientos aplicables, normas de seguridad del entorno, 
programas o equipos, obligaciones de los implicados en el tratamiento y uso de 
la informaci\'on personal, as\'{\i} como las garant\'{\i}as, en su \'ambito, 
para el ejercicio de los derechos de las personas con pleno respeto a los 
principios y disposiciones de la presente Ley y sus normas de desarrollo.\\

{\bf 2.} Los citados c\'odigos podr\'an contener o no reglas operacionales 
detalladas de cada sistema particular y est\'andares t\'ecnicos de 
aplicaci\'on.\\
En el supuesto de que tales reglas o est\'andares no se incorporen directamente 
al c\'odigo, las instrucciones u \'ordenes que los establecieran deber\'an 
respetar los principios fijados en aqu\'el.\\

{\bf 3.} Los c\'odigos tipo tendr\'an el car\'acter de c\'odigos 
deontol\'ogicos o de buena pr\'actica profesional, debiendo ser depositados o 
inscritos en el Registro General de Protecci\'on de Datos y, cuando corresponda,
en los creados a estos efectos por las Comunidades Aut\'onomas, de acuerdo con 
el art\'{\i}culo 41. El Registro General de Protecci\'on de Datos podr\'a 
denegar la inscripci\'on cuando considere que no se ajusta a las disposiciones 
legales y reglamentarias sobre la materia, debiendo, en este caso, el Director 
de la Agencia de Protecci\'on de Datos requerir a los solicitantes para que 
efect\'uen las correcciones oportunas.
\begin{center}
{\LARGE T\'ITULO V}\\
{\large Movimiento internacional de datos}
\end{center}
\vspace{0.3cm}
{\large {\bf Art\'{\i}culo 33.} Norma general.}

{\bf 1.} No podr\'an realizarse transferencias temporales ni definitivas de 
datos de car\'acter personal que hayan sido objeto de tratamiento o hayan sido 
recogidos para someterlos a dicho tratamiento con destino a pa\'{\i}ses que no 
proporcionen un nivel de protecci\'on equiparable al que presta la presente 
Ley, salvo que, adem\'as de haberse observado lo dispuesto en \'esta, se 
obtenga autorizaci\'on previa del Director de la Agencia de Protecci\'on de 
Datos, que s\'olo podr\'a otorgarla si se obtienen garant\'{\i}as adecuadas.\\

{\bf 2.} El car\'acter adecuado del nivel de protecci\'on que ofrece el 
pa\'{\i}s de destino se evaluar\'a por la Agencia de Protecci\'on de Datos 
atendiendo a todas las circunstancias que concurran en la transferencia o 
categor\'{\i}a de transferencia de datos. En particular, se tomar\'a en 
consideraci\'on la naturaleza de los datos de finalidad y la duraci\'on del 
tratamiento o de los tratamientos previstos, el pa\'{\i}s de origen y el 
pa\'{\i}s de destino final, las normas de Derecho, generales o sectoriales, 
vigentes en el pa\'{\i}s tercero de que se trate, el contenido de los informes 
de la Comisi\'on de la Uni\'on Europea, as\'{\i} como las normas profesionales 
y las medidas de seguridad en vigor en dichos pa\'{\i}ses.
\vspace{0.3cm}\\
{\large {\bf Art\'{\i}culo 34.} Excepciones.}

Lo dispuesto en el art\'{\i}culo anterior no ser\'a de aplicaci\'on:
\begin{itemize}
\item[(a)] Cuando la transferencia internacional de datos de car\'acter 
personal resulte de la aplicaci\'on de tratados o convenios en los que sea 
parte Espa\~na.
\item [(b)] Cuando la transferencia se haga a efectos de prestar o solicitar 
auxilio judicial internacional.
\item [(c)] Cuando la transferencia sea necesaria para la prevenci\'on o para 
el diagn\'ostico m\'edicos, la prestaci\'on de asistencia sanitaria o 
tratamiento m\'edicos o la gesti\'on de servicios sanitarios.
\item [(d)] Cuando se refiera a transferencias dinerarias conforme a su 
legislaci\'on espec\'{\i}fica.
\item [(e)] Cuando el afectado haya dado su consentimiento inequ\'{\i}voco a la 
transferencia prevista.
\item [(f)] Cuando la transferencia sea necesaria para la ejecuci\'on de un 
contrato entre el afectado y el responsable del fichero o para la adopci\'on de 
medidas precontractuales adoptadas a petici\'on del afectado.
\item [(g)] Cuando la transferencia sea necesaria para la celebraci\'on o 
ejecuci\'on de un contrato celebrado o por celebrar, en inter\'es del afectado,
por el responsable del fichero y un tercero.
\item [(h)] Cuando la transferencia sea necesaria o legalmente exigida para la 
salvaguarda de un inter\'es p\'ublico. Tendr\'a esta consideraci\'on la 
transferencia solicitada por una Administraci\'on fiscal o aduanera para el 
cumplimiento de sus competencias.
\item [(i)] Cuando la transferencia sea precisa para el reconocimiento, 
ejercicio o defensa de un derecho en un proceso judicial.
\item [(j)] Cuando la transferencia se efect\'ue, a petici\'on de persona con 
inter\'es leg\'{\i}timo, desde un Registro P\'ublico y aqu\'ella sea acorde con 
la finalidad del mismo.
\item [(k)] Cuando la transferencia tenga como destino un Estado miembro de la 
Uni\'on Europea, o un Estado respecto del cual la Comisi\'on de las Comunidades 
Europeas, en el ejercicio de sus competencias, haya declarado que garantiza un 
nivel de protecci\'on adecuado.
\end{itemize}
\begin{center}
{\LARGE T\'ITULO VI}\\
{\large Agencia de Protecci\'on de Datos}
\end{center}
\vspace{0.3cm}
{\large {\bf Art\'{\i}culo 35.} Naturaleza y r\'egimen jur\'{\i}dico.}

{\bf 1.} La Agencia de Protecci\'on de Datos es un Ente de Derecho p\'ublico,
con personalidad jur\'{\i}dica propia y plena capacidad p\'ublica y privada, 
que act\'ua con plena independencia de las Administraciones P\'ublicas en 
el ejercicio de sus funciones. Se regir\'a por lo dispuesto en la presente Ley 
y en un Estatuto propio, que ser\'a aprobado por el Gobierno.\\

{\bf 2.} En el ejercicio de sus funciones p\'ublicas, y en defecto de lo que 
disponga la presente Ley y sus disposiciones de desarrollo, la Agencia de 
Protecci\'on de Datos actuar\'a de conformidad con la Ley 30/1992, de 26 de 
noviembre, de R\'egimen Jur\'{\i}dico de las Administraciones P\'ublicas y del 
Procedimiento Administrativo Com\'un. En sus adquisiciones patrimoniales y 
contrataci\'on estar\'a sujeta al Derecho privado.\\

{\bf 3.} Los puestos de trabajo de los \'organos y servicios que integren la 
Agencia de Protecci\'on de Datos ser\'an desempe\~nados por funcionarios de las 
Administraciones P\'ublicas y por personal contratado al efecto, seg\'un la 
naturaleza de las funciones asignadas a cada puesto de trabajo. Este personal 
est\'a obligado a guardar secreto de los datos de car\'acter personal de que 
conozca en el desarrollo de su funci\'on.\\

{\bf 4.} La Agencia de Protecci\'on de Datos contar\'a, para el cumplimiento de 
sus fines, con los siguientes bienes y medios econ\'omicos:
\begin{itemize}
\item [(a)] Las asignaciones que se establezcan anualmente con cargo a los 
Presupuestos Generales del Estado.
\item [(b)] Los bienes y valores que constituyan su patrimonio, as\'{\i} como 
los productos y rentas del mismo.
\item [(c)] Cualesquiera otros que legalmente puedan serle atribuidos.
\end{itemize}

{\bf 5.} La Agencia de Protecci\'on de Datos elaborar\'a y aprobar\'a con 
car\'acter anual el correspondiente anteproyecto de presupuesto y lo remitir\'a 
al Gobierno para que sea integrado, con la debida independencia, en los 
Presupuestos Generales del Estado.
\vspace{0.3cm}\\
{\large {\bf Art\'{\i}culo 36.} El Director.}

{\bf 1.} El Director de la Agencia de Protecci\'on de Datos dirige la Agencia y 
ostenta su representaci\'on. Ser\'a nombrado, de entre quienes componen el 
Consejo Consultivo, mediante Real Decreto, por un per\'{\i}odo de cuatro 
a\~nos.\\

{\bf 2.} Ejercer\'a sus funciones con plena independencia y objetividad, y no 
estar\'a sujeto a instrucci\'on alguna en el desempe\~no de aqu\'ellas.\\
En todo caso, el Director deber\'a o\'{\i}r al Consejo Consultivo en aquellas 
propuestas que \'este le realice en el ejercicio de sus funciones.\\

{\bf 3.} El Director de la Agencia de Protecci\'on de Datos s\'olo cesar\'a 
antes de la expiraci\'on del per\'{\i}odo a que se refiere el apartado 1 a 
petici\'on propia o por separaci\'on acordada por el Gobierno, previa 
instrucci\'on de expediente, en el que necesariamente ser\'an o\'{\i}dos los 
restantes miembros del Consejo Consultivo, por incumplimiento grave de sus 
obligaciones, incapacidad sobrevenida para el ejercicio de su funci\'on, 
incompatibilidad o condena por delito doloso.\\

{\bf 4.} El Director de la Agencia de Protecci\'on de Datos tendr\'a la 
consideraci\'on de alto cargo y quedar\'a en la situaci\'on de servicios 
especiales si con anterioridad estuviera desempe\~nando una funci\'on 
p\'ublica. En el supuesto de que sea nombrado para el cargo alg\'un miembro de 
la carrera judicial o fiscal, pasar\'a asimismo a la situaci\'on administrativa 
de servicios especiales.
\vspace{0.3cm}\\
{\large {\bf Art\'{\i}culo 36.} Funciones.}

Son funciones de la Agencia de Protecci\'on de Datos:
\begin{itemize}
\item [(a)] Velar por el cumplimiento de la legislaci\'on sobre protecci\'on de 
datos y controlar su aplicaci\'on, en especial en lo relativo a los derechos de 
informaci\'on, acceso, rectificaci\'on, oposici\'on y cancelaci\'on de datos.
\item [(b)] Emitir las autorizaciones previstas en la Ley o en sus disposiciones
reglamentarias.
\item [(c)] Dictar, en su caso y sin perjuicio de las competencias de otros 
\'organos, las instrucciones precisas para adecuar los tratamientos a los 
principios de la presente Ley.
\item [(d)] Atender las peticiones y reclamaciones formuladas por las personas 
afectadas.
\item [(e)] Proporcionar informaci\'on a las personas acerca de sus derechos en 
materia de tratamiento de los datos de car\'acter personal.
\item [(f)] Requerir a los responsables y los encargados de los tratamientos, 
previa audiencia de \'estos, la adopci\'on de las medidas necesarias para la 
adecuaci\'on del tratamiento de datos a las disposiciones de esta Ley y, en su 
caso, ordenar la cesaci\'on de los tratamientos y la cancelaci\'on de los 
ficheros, cuando no se ajuste a sus disposiciones.
\item [(g)] Ejercer la potestad sancionadora en los t\'erminos previstos por el 
T\'{\i}tulo VII de la presente Ley.
\item [(h)] Informar, con car\'acter preceptivo, los proyectos de disposiciones 
generales que desarrollen esta Ley.
\item [(i)] Recabar de los responsables de los ficheros cuanta ayuda e 
informaci\'on estime necesaria para el desempe\~no de sus funciones.
\item [(j)] Velar por la publicidad de la existencia de los ficheros de datos 
con car\'acter personal, a cuyo efecto publicar\'a peri\'odicamente una 
relaci\'on de dichos ficheros con la informaci\'on adicional que el Director de 
la Agencia determine.
\item [(k)] Redactar una memoria anual y remitirla al Ministerio de Justicia.
\item [(l)] Ejercer el control y adoptar las autorizaciones que procedan en 
relaci\'on con los movimientos internacionales de datos, as\'{\i} como 
desempe\~nar las funciones de cooperaci\'on internacional en materia de 
protecci\'on de datos personales.
\item [(m)] Velar por el cumplimiento de las disposiciones que la Ley de la 
Funci\'on Estad\'{\i}stica P\'ublica establece respecto a la recogida de datos 
estad\'{\i}sticos y al secreto estad\'{\i}stico, as\'{\i} como dictar las 
instrucciones precisas, dictaminar sobre las condiciones de seguridad de los 
ficheros constituidos con fines exclusivamente estad\'{\i}sticos y ejercer la 
potestad a la que se refiere el art\'{\i}culo 46.
\item [(n)] Cuantas otras le sean atribuidas por normas legales o 
reglamentarias.
\end{itemize}
\vspace{0.3cm}
{\large {\bf Art\'{\i}culo 38.} Consejo Consultivo.}

El Director de la Agencia de Protecci\'on de Datos estar\'a asesorado por un 
Consejo Consultivo compuesto por los siguientes miembros:
\begin{itemize}
\item Un Diputado, propuesto por el Congreso de los Diputados.
\item Un Senador, propuesto por el Senado.
\item Un representante de la Administraci\'on Central, designado por el 
Gobierno.
\item Un representante de la Administraci\'on Local, propuesto por la 
Federaci\'on Espa\~nola de Municipios y Provincias.
\item Un miembro de la Real Academia de la Historia, propuesto por la misma.
\item Un experto en la materia, propuesto por el Consejo Superior de 
Universidades.
\item Un representante de los usuarios y consumidores, seleccionado del modo 
que se prevea reglamentariamente.
\item Un representante de cada Comunidad Aut\'onoma que haya creado una agencia 
de protecci\'on de datos en su \'ambito territorial, propuesto de acuerdo con 
el procedimiento que establezca la respectiva Comunidad Aut\'onoma.
\item Un representante del sector de ficheros privados, para cuya propuesta se 
seguir\'a el procedimiento que se regule reglamentariamente.
\end{itemize}
El funcionamiento del Consejo Consultivo se regir\'a por las normas 
reglamentarias que al efecto se establezcan.
\vspace{0.3cm}\\
{\large {\bf Art\'{\i}culo 39.} El Registro General de Protecci\'on de Datos.}

{\bf 1.} El Registro General de Protecci\'on de Datos es un \'organo integrado 
en la Agencia de Protecci\'on de Datos.\\

{\bf 2.} Ser\'an objeto de inscripci\'on en el Registro General de Protecci\'on 
de Datos:
\begin{itemize}
\item [(a)] Los ficheros de que sean titulares las Administraciones P\'ublicas.
\item [(b)] Los ficheros de titularidad privada.
\item [(c)] Las autorizaciones a que se refiere la presente Ley.
\item [(d)] Los c\'odigos tipo a que se refiere el art\'{\i}culo 32 de la 
presente Ley.
\item [(e)] Los datos relativos a los ficheros que sean necesarios para el 
ejercicio de los derechos de informaci\'on, acceso, rectificaci\'on, 
cancelaci\'on y oposici\'on.
\end{itemize}

{\bf 3.} Por v\'{\i}a reglamentaria se regular\'a el procedimiento de 
inscripci\'on de los ficheros, tanto de titularidad p\'ublica como de 
titularidad privada, en el Registro General de Protecci\'on de Datos, el 
contenido de la inscripci\'on, su modificaci\'on, cancelaci\'on, reclamaciones 
y recursos contra las resoluciones correspondientes y dem\'as extremos 
pertinentes.
\vspace{0.3cm}\\
{\large {\bf Art\'{\i}culo 40.} Potestad de inspecci\'on.}

{\bf 1.} Las autoridades de control podr\'an inspeccionar los ficheros a que 
hace referencia la presente Ley, recabando cuantas informaciones precisen para 
el cumplimiento de sus cometidos.\\
A tal efecto, podr\'an solicitar la exhibici\'on o el env\'{\i}o de documentos 
y datos y examinarlos en el lugar en que se encuentren depositados, as\'{\i} 
como inspeccionar los equipos f\'{\i}sicos y l\'ogicos utilizados para el 
tratamiento de los datos, accediendo a los locales donde se hallen instalados.\\

{\bf 2.} Los funcionarios que ejerzan la inspecci\'on a que se refiere el 
apartado anterior tendr\'an la consideraci\'on de autoridad p\'ublica en el 
desempe\~no de sus cometidos.\\
Estar\'an obligados a guardar secreto sobre las informaciones que conozcan en 
el ejercicio de las mencionadas funciones, incluso despu\'es de haber cesado en 
las mismas.
\vspace{0.3cm}\\
{\large {\bf Art\'{\i}culo 41.} \'Organos correspondientes de las Comunidades 
Aut\'onomas.}

{\bf 1.} Las funciones de la Agencia de Protecci\'on de Datos reguladas en el 
art\'{\i}culo 37, a excepci\'on de las mencionadas en los apartados (j), (k) y 
(l), y en los apartados (f) y (g) en lo que se refiere a las transferencias 
internacionales de datos, as\'{\i} como en los art\'{\i}culos 46 y 49, en 
relaci\'on con sus espec\'{\i}ficas competencias ser\'an ejercidas, cuando 
afecten a ficheros de datos de car\'acter personal creados o gestionados por 
las Comunidades Aut\'onomas y por la Administraci\'on local de su \'ambito 
territorial, por los \'organos correspondientes de cada Comunidad, que 
tendr\'an la consideraci\'on de autoridades de control, a los que 
garantizar\'an plena independencia y objetividad en el ejercicio de su 
cometido.\\

{\bf 2.} Las Comunidades Aut\'onomas podr\'an crear y mantener sus propios 
registros de ficheros para el ejercicio de las competencias que se les reconoce 
sobre los mismos.\\

{\bf 3.} El Director de la Agencia de Protecci\'on de Datos podr\'a convocar 
regularmente a los \'organos correspondientes de las Comunidades Aut\'onomas a 
efectos de cooperaci\'on institucional y coordinaci\'on de criterios o 
procedimientos de actuaci\'on. El Director de la Agencia de Protecci\'on de 
Datos y los \'organos correspondientes de las Comunidades Aut\'onomas podr\'an 
solicitarse mutuamente la informaci\'on necesaria para el cumplimiento de sus 
funciones.
\vspace{0.3cm}\\
{\large {\bf Art\'{\i}culo 42.} Ficheros de las Comunidades Aut\'onomas en 
materia de su exclusiva competencia.}

{\bf 1.} Cuando el Director de la Agencia de Protecci\'on de Datos constate que 
el mantenimiento o uso de un determinado fichero de las Comunidades Aut\'onomas 
contraviene alg\'un precepto de esta Ley en materia de su exclusiva competencia 
podr\'a requerir a la Administraci\'on correspondiente que se adopten las 
medidas correctoras que determine en el plazo que expresamente se fije en el 
requerimiento.\\

{\bf 2.} Si la Administraci\'on P\'ublica correspondiente no cumpliera el 
requerimiento formulado, el Director de la Agencia de Protecci\'on de Datos 
podr\'a impugnar la resoluci\'on adoptada por aquella Administraci\'on.
\begin{center}
{\LARGE T\'ITULO VII}\\ {\large Infracciones y sanciones} \end{center}
\vspace{0.3cm} {\large {\bf Art\'{\i}culo 43.} Responsables.}

{\bf 1.} Los responsables de los ficheros y los encargados de los
tratamientos estar\'an sujetos al r\'egimen sancionador establecido en
la presente Ley.\\

{\bf 2.} Cuando se trate de ficheros de los que sean responsables las
Administraciones P\'ublicas se estar\'a, en cuanto al procedimiento
y a las sanciones, a lo dispuesto en el art\'{\i}culo 46, apartado 2.
\vspace{0.3cm}\\ {\large {\bf Art\'{\i}culo 44.} Tipos de infracciones.}

{\bf 1.} Las infracciones se calificar\'an como leves, graves o muy
graves.\\

{\bf 2.} Son infracciones leves: \begin{itemize} \item[(a)] No atender,
por motivos formales, la solicitud del interesado de rectificaci\'on
o cancelaci\'on de los datos personales objeto de tratamiento cuando
legalmente proceda.  \item [(b)] No proporcionar la informaci\'on que
solicite la Agencia de Protecci\'on de Datos en el ejercicio de las
competencias que tiene legalmente atribuidas, en relaci\'on con aspectos
no sustantivos de la protecci\'on de datos.  \item [(c)] No solicitar
la inscripci\'on del fichero de datos de car\'acter personal en el
Registro General de Protecci\'on de Datos, cuando no sea constitutivo
de infracci\'on grave.  \item [(d)] Proceder a la recogida de datos
de car\'acter personal de los propios afectados sin proporcionarles
la informaci\'on que se\~nala el art\'{\i}culo 5 de la presente Ley.
\item [(e)] Incumplir el deber de secreto establecido en el art\'{\i}culo
10 de esta Ley, salvo que constituya infracci\'on grave.  \end{itemize}

{\bf 3.} Son infracciones graves: \begin{itemize} \item [(a)] Proceder a
la creaci\'on de ficheros de titularidad p\'ublica o iniciar la recogida
de datos de car\'acter personal para los mismos, sin autorizaci\'on
de disposici\'on general, publicada en el `Bolet\'{\i}n Oficial del
Estado' o diario oficial correspondiente.  \item [(b)] Proceder a la
creaci\'on de ficheros de titularidad privada o iniciar la recogida de
datos de car\'acter personal para los mismos con finalidades distintas
de las que constituyen el objeto leg\'{\i}timo de la empresa o entidad.
\item [(c)] Proceder a la recogida de datos de car\'acter personal sin
recabar el consentimiento expreso de las personas afectadas, en los casos
en que \'este sea exigible.  \item [(d)] Tratar los datos de car\'acter
personal o usarlos posteriormente con conculcaci\'on de los principios y
garant\'{\i}as establecidos en la presente Ley o con incumplimiento de los
preceptos de protecci\'on que impongan las disposiciones reglamentarias
de desarrollo, cuando no constituya infracci\'on muy grave.  \item [(e)]
El impedimento o la obstaculizaci\'on del ejercicio de los derechos de
acceso y oposici\'on y la negativa a facilitar la informaci\'on que sea
solicitada.  \item [(f)] Mantener datos de car\'acter personal inexactos
o no efectuar las rectificaciones o cancelaciones de los mismos que
legalmente procedan cuando resulten afectados los derechos de las personas
que la presente Ley ampara.  \item [(g)] La vulneraci\'on del deber de
guardar secreto sobre los datos de car\'acter personal incorporados a
ficheros que contengan datos relativos a la comisi\'on de infracciones
administrativas o penales, Hacienda P\'ublica, servicios financieros,
prestaci\'on de servicios de solvencia patrimonial y cr\'edito,
as\'{\i} como aquellos otros ficheros que contengan un conjunto de
datos de car\'acter personal suficientes para obtener una evaluaci\'on
de la personalidad del individuo.  \item [(h)] Mantener los ficheros,
locales, programas o equipos que contengan datos de car\'acter personal
sin las debidas condiciones de seguridad que por v\'{\i}a reglamentaria
se determinen.  \item [(i)] No remitir a la Agencia de Protecci\'on de
Datos las notificaciones previstas en esta Ley o en sus disposiciones
de desarrollo, as\'{\i} como no proporcionar en plazo a la misma cuantos
documentos e informaciones deba recibir o sean requeridos por aqu\'el a
tales efectos.  \item [(j)] La obstrucci\'on al ejercicio de la funci\'on
inspectora.  \item [(k)] No inscribir el fichero de datos de car\'acter
personal en el Registro General de Protecci\'on de Datos, cuando haya
sido requerido para ello por el Director de la Agencia de Protecci\'on de
Datos.  \item [(l)] Incumplir el deber de informaci\'on que se establece
en los art\'{\i}culos 5, 28 y 29 de esta Ley, cuando los datos hayan
sido recabados de persona distinta del afectado.  \end{itemize}

{\bf 4.} Son infracciones muy graves: \begin{itemize} \item [(a)]
La recogida de datos en forma enga\~nosa y fraudulenta.  \item [(b)]
La comunicaci\'on o cesi\'on de los datos de car\'acter personal, fuera
de los casos en que est\'en permitidas.  \item [(c)] Recabar y tratar
los datos de car\'acter personal a los que se refiere el apartado 2 del
art\'{\i}culo 7 cuando no medie el consentimiento expreso del afectado;
recabar y tratar los datos referidos en el apartado 3 del art\'{\i}culo
7 cuando no lo disponga una Ley o el afectado no haya consentido
expresamente, o violentar la prohibici\'on contenida en el apartado 4 del
art\'{\i}culo 7.  \item [(d)] No cesar en el uso ileg\'{\i}timo de los
tratamientos de datos de car\'acter personal cuando sea requerido para
ello por el Director de la Agencia de Protecci\'on de Datos o por las
personas titulares del derecho de acceso.  \item [(e)] La transferencia
temporal o definitiva de datos de car\'acter personal que hayan sido
objeto de tratamiento o hayan sido recogidos para someterlos a dicho
tratamiento, con destino a pa\'{\i}ses que no proporcionen un nivel de
protecci\'on equiparable sin autorizaci\'on del Director de la Agencia
de Protecci\'on de Datos.  \item [(f)] Tratar los datos de car\'acter
personal de forma ileg\'{\i}tima o con menosprecio de los principios
y garant\'{\i}as que les sean de aplicaci\'on, cuando con ello se
impida o se atente contra el ejercicio de los derechos fundamentales.
\item [(g)] La vulneraci\'on del deber de guardar secreto sobre los
datos de car\'acter personal a que hacen referencia los apartados 2
y 3 del art\'{\i}culo 7, as\'{\i} como los que hayan sido recabados
para fines policiales sin consentimiento de las personas afectadas.
\item [(h)] No atender, u obstaculizar de forma sistem\'atica el
ejercicio de los derechos de acceso, rectificaci\'on, cancelaci\'on
u oposici\'on.  \item [(i)] No atender de forma sistem\'atica el deber
legal de notificaci\'on de la inclusi\'on de datos de car\'acter personal
en un fichero.  \end{itemize} \vspace{0.3cm} {\large {\bf Art\'{\i}culo
45.} Tipo de sanciones.}

{\bf 1.} Las infracciones leves ser\'an sancionadas con multa de 100.000
a 10.000.000 de pesetas.\\

{\bf 2.} Las infracciones graves ser\'an sancionadas con multa de
10.000.000 a 50.000.000 de pesetas.\\

{\bf 3.} Las infracciones muy graves ser\'an sancionadas con multa de
50.000.000 a 100.000.000 de pesetas.\\

{\bf 4.} La cuant\'{\i}a de las sanciones se graduar\'a atendiendo
a la naturaleza de los derechos personales afectados, al volumen de
los tratamientos efectuados, a los beneficios obtenidos, al grado de
intencionalidad, a la reincidencia, a los da\~nos y perjuicios causados
a las personas interesadas y a terceras personas, y a cualquier
otra circunstancia que sea relevante para determinar el grado de
antijuridicidad y de culpabilidad presentes en la concreta actuaci\'on
infractora.\\

{\bf 5.} Si, en raz\'on de las circunstancias concurrentes, se apreciara
una cualificada disminuci\'on de la culpabilidad del imputado o de la
antijuridicidad del hecho, el \'organo sancionador establecer\'a la
cuant\'{\i}a de la sanci\'on aplicando la escala relativa a la clase de
infracciones que preceda inmediatamente en gravedad a aquella en que se
integra la considerada en el caso de que se trate.\\

{\bf 6.} En ning\'un caso podr\'a imponerse una sanci\'on m\'as grave que
la fijada en la Ley para la clase de infracci\'on en la que se integre
la que se pretenda sancionar.\\

{\bf 7.} El Gobierno actualizar\'a peri\'odicamente la cuant\'{\i}a
de las sanciones de acuerdo con las variaciones que experimenten los
\'{\i}ndices de precios.  \vspace{0.3cm}\\ {\large {\bf Art\'{\i}culo 46.}
Infracciones de las Administraciones P\'ublicas.}

{\bf 1.} Cuando las infracciones a que se refiere el art\'{\i}culo
44 fuesen cometidas en ficheros de los que sean responsables las
Administraciones P\'ublicas, el Director de la Agencia de Protecci\'on de
Datos dictar\'a una resoluci\'on estableciendo las medidas que procede
adoptar para que cesen o se corrijan los efectos de la infracci\'on.\\
Esta resoluci\'on se notificar\'a al responsable del fichero, al \'organo
del que dependa jer\'arquicamente y a los afectados si los hubiera.\\

{\bf 2.} El Director de la Agencia podr\'a proponer tambi\'en la
iniciaci\'on de actuaciones disciplinarias, si procedieran. El
procedimiento y las sanciones a aplicar ser\'an las establecidas en
la legislaci\'on sobre r\'egimen disciplinario de las Administraciones
P\'ublicas.\\

{\bf 3.} Se deber\'an comunicar a la Agencia las resoluciones que
recaigan en relaci\'on con las medidas y actuaciones a que se refieren
los apartados anteriores.\\

{\bf 4.} El Director de la Agencia comunicar\'a al Defensor del Pueblo
las actuaciones que efect\'ue y las resoluciones que dicte al amparo
de los apartados anteriores.
\vspace{0.3cm}\\
{\large {\bf Art\'{\i}culo 47.} Prescripci\'on.}

{\bf 1.} Las infracciones muy graves prescribir\'an a los tres a\~nos,
las graves a los dos a\~nos y las leves al a\~no.\\

{\bf 2.} El plazo de prescripci\'on comenzar\'a a contarse desde el
d\'{\i}a en que la infracci\'on se hubiera cometido.\\

{\bf 3.} Interrumpir\'a la prescripci\'on la iniciaci\'on,
con conocimiento del interesado, del procedimiento sancionador,
reanud\'andose el plazo de prescripci\'on si el expediente sancionador
estuviere paralizado durante m\'as de seis meses por causas no imputables
al presunto infractor.\\

{\bf 4.} Las sanciones impuestas por faltas muy graves prescribir\'an
a los tres a\~nos, las impuestas por faltas graves a los dos a\~nos y
las impuestas por faltas leves al a\~no.\\

{\bf 5.} El plazo de prescripci\'on de las sanciones comenzar\'a a
contarse desde el d\'{\i}a siguiente a aquel en que adquiera firmeza la
resoluci\'on por la que se impone la sanci\'on.\\

{\bf 6.} La prescripci\'on se interrumpir\'a por la iniciaci\'on,
con conocimiento del interesado, del procedimiento de ejecuci\'on,
volviendo a transcurrir el plazo si el mismo est\'a paralizado durante
m\'as de seis meses por causa no imputable al infractor.
\vspace{0.3cm}\\
{\large {\bf Art\'{\i}culo 48.} Procedimiento sancionador.}

{\bf 1.} Por v\'{\i}a reglamentaria se establecer\'a el procedimiento
a seguir para la determinaci\'on de las infracciones y la imposici\'on
de las sanciones a que hace referencia el presente T\'{\i}tulo.

{\bf 2.} Las resoluciones de la Agencia de Protecci\'on de Datos u
\'organo correspondiente de la Comunidad Aut\'onoma agotan la v\'{\i}a
administrativa.
\vspace{0.3cm}\\
{\large {\bf Art\'{\i}culo 49.} Potestad de inmovilizaci\'on de ficheros.}

En los supuestos, constitutivos de infracci\'on muy grave, de
utilizaci\'on o cesi\'on il\'{\i}cita de los datos de car\'acter
personal en que se impida gravemente o se atente de igual modo contra el
ejercicio de los derechos de los ciudadanos y el libre desarrollo de la
personalidad que la Constituci\'on y las leyes garantizan, el Director
de la Agencia de Protecci\'on de Datos podr\'a, adem\'as de ejercer la
potestad sancionadora, requerir a los responsables de ficheros de datos
de car\'acter personal, tanto de titularidad p\'ublica como privada, la
cesaci\'on en la utilizaci\'on o cesi\'on il\'{\i}cita de los datos.\\
Si el requerimiento fuera desatendido, la Agencia de Protecci\'on de
Datos podr\'a, mediante resoluci\'on motivada, inmovilizar tales ficheros
a los solos
efectos de restaurar los derechos de las personas afectadas.
\begin{center}
{\LARGE DISPOSICIONES ADICIONALES}
\end{center}
\vspace{0.3cm} {\large {\bf Primera.} Ficheros preexistentes.}

Los ficheros y tratamientos automatizados, inscritos o no en el Registro 
General de Protecci\'on de Datos deber\'an adecuarse a la presente Ley 
Org\'anica dentro del plazo de tres a\~nos, a contar desde su entrada en vigor. 
En dicho plazo, los ficheros de titularidad privada deber\'an ser comunicados a 
la Agencia de Protecci\'on de Datos y las Administraciones P\'ublicas, 
responsables de ficheros de titularidad p\'ublica, deber\'an aprobar la 
pertinente disposici\'on de regulaci\'on del fichero o adaptar la existente.\\

En el supuesto de ficheros y tratamientos no automatizados, su adecuaci\'on a 
la presente Ley Org\'anica y la obligaci\'on prevista en el p\'arrafo anterior 
deber\'a cumplimentarse en el plazo de doce a\~nos a contar desde el 24 de 
octubre de 1995, sin perjuicio del ejercicio de los derechos de acceso, 
rectificaci\'on y cancelaci\'on por parte de los afectados.
\vspace{0.3cm}\\
{\large {\bf Segunda.} Ficheros y Registro de Poblaci\'on de las 
Administraciones P\'ublicas.}

{\bf 1.} La Administraci\'on General del Estado y las Administraciones de las 
Comunidades Aut\'onomas podr\'an solicitar al Instituto Nacional de 
Estad\'{\i}stica, sin consentimiento del interesado, una copia actualizada del 
fichero formado con los datos del nombre, apellidos, domicilio, sexo y fecha de 
nacimiento que constan en los padrones municipales de habitantes y en el censo 
electoral correspondientes a los territorios donde ejerzan sus competencias, 
para la creaci\'on de ficheros o registros de poblaci\'on.\\

{\bf 2.} Los ficheros o registros de poblaci\'on tendr\'an como finalidad la 
comunicaci\'on de los distintos \'organos de cada administraci\'on p\'ublica 
con los interesados residentes en los respectivos territorios, respecto a las 
relaciones jur\'{\i}dico administrativas derivadas de las competencias 
respectivas de las Administraciones P\'ublicas.
\vspace{0.3cm}\\
{\large {\bf Tercera.} Tratamiento de los expedientes de las derogadas Leyes de 
Vagos y Maleantes y de Peligrosidad y Rehabilitaci\'on Social.}

Los expedientes espec\'{\i}ficamente instruidos al amparo de las derogadas 
Leyes de Vagos y Maleantes, y de Peligrosidad y Rehabilitaci\'on Social, que 
contengan datos de cualquier \'{\i}ndole susceptibles de afectar a la seguridad,
al honor, a la intimidad o a la imagen de las personas, no podr\'an ser 
consultados sin que medie consentimiento expreso de los afectados, o hayan 
transcurrido 50 a\~nos desde la fecha de aqu\'ellos.\\
En este \'ultimo supuesto, la Administraci\'on General del Estado, salvo que 
haya constancia expresa del fallecimiento de los afectados, pondr\'a a 
disposici\'on del solicitante la documentaci\'on, suprimiendo de la misma los 
datos aludidos en el p\'arrafo anterior, mediante la utilizaci\'on de los 
procedimientos t\'ecnicos pertinentes en cada caso.
\vspace{0.3cm}\\
{\large {\bf Cuarta.} Modificaci\'on del art\'{\i}culo 112.4 de la Ley General 
Tributaria.}

El apartado cuarto del art\'{\i}culo 112 de la Ley General Tributaria pasa a 
tener la siguiente redacci\'on:\\
{\it `4. La cesi\'on de aquellos datos de car\'acter personal, objeto de 
tratamiento que se debe efectuar a la Administraci\'on tributaria conforme a lo 
dispuesto en el art\'{\i}culo 111, en los apartados anteriores de este 
art\'{\i}culo o en otra norma de rango legal, no requerir\'a el consentimiento 
del afectado. En este \'ambito tampoco ser\'a de aplicaci\'on lo que respecto a 
las Administraciones P\'ublicas establece el apartado 1 del art\'{\i}culo 21 de 
la Ley Org\'anica de Protecci\'on de Datos de car\'acter personal'.}
\vspace{0.3cm}\\
{\large {\bf Quinta.} Competencias del Defensor del Pueblo y \'organos 
auton\'omicos semejantes.}

Lo dispuesto en la presente Ley Org\'anica se entiende sin perjuicio de las 
competencias del Defensor del Pueblo y de los \'organos an\'alogos de las 
Comunidades Aut\'onomas.
\vspace{0.3cm}\\
{\large {\bf Sexto.} Modificaci\'on del art\'{\i}culo 24.3 de la Ley de 
Ordenaci\'on y Supervisi\'on de los Seguros Privados.}

Se modifica el art\'{\i}culo 24.3, p\'arrafo 2$^{o}$ de la Ley 30/1995, de 8 de 
noviembre, de Ordenaci\'on y Supervisi\'on de los Seguros Privados con la 
siguiente redacci\'on:\\
{\it `Las entidades aseguradoras podr\'an establecer ficheros comunes que 
contengan datos de car\'acter personal para la liquidaci\'on de siniestros y la 
colaboraci\'on estad\'{\i}stico actuarial con la finalidad de permitir la 
tarificaci\'on y selecci\'on de riesgos y la elaboraci\'on de estudios de 
t\'ecnica aseguradora. La cesi\'on de datos a los citados ficheros no 
requerir\'a el consentimiento previo del afectado, pero s\'{\i} la 
comunicaci\'on al mismo de la posible cesi\'on de sus datos personales a 
ficheros comunes para los fines se\~nalados con expresa indicaci\'on del 
responsable para que se puedan ejercitar los derechos de acceso, 
rectificaci\'on y cancelaci\'on previstos en la Ley.\\
Tambi\'en podr\'an establecerse ficheros comunes cuya finalidad sea prevenir el 
fraude en el seguro sin que sea necesario el consentimiento del afectado. No 
obstante, ser\'a necesaria en estos casos la comunicaci\'on al afectado, en la 
primera introducci\'on de sus datos, de qui\'en sea el responsable del fichero 
y de las formas de ejercicio de los derechos de acceso, rectificaci\'on y 
cancelaci\'on.\\
En todo caso, los datos relativos a la salud s\'olo podr\'an ser objeto de 
tratamiento con el consentimiento expreso del afectado.'}
\begin{center}
{\LARGE DISPOSICIONES TRANSITORIAS}
\end{center}
\vspace{0.3cm}
{\large {\bf Primera.} Tratamientos creados por Convenios Internacionales.}

La Agencia de Protecci\'on de Datos ser\'a el organismo competente para la 
protecci\'on de las personas f\'{\i}sicas en lo que respecta al tratamiento de 
datos de car\'acter personal respecto de los tratamientos establecidos en 
cualquier Convenio Internacional del que sea parte Espa\~na que atribuya a una 
autoridad nacional de control esta competencia, mientras no se cree una 
autoridad diferente para este cometido en desarrollo del Convenio.
\vspace{0.3cm}\\
{\large {\bf Segunda.} Utilizaci\'on del Censo Promocional.}

Reglamentariamente se desarrollar\'an los procedimientos de formaci\'on del 
Censo Promocional, de oposici\'on a aparecer en el mismo, de puesta a 
disposici\'on de sus solicitantes, y de control de las listas difundidas. El 
Reglamento establecer\'a los plazos para la puesta en operaci\'on del Censo 
Promocional.
\vspace{0.3cm}\\
{\large {\bf Tercera.} Subsistencia de normas preexistentes.}

Hasta tanto se lleven a efecto las previsiones de la Disposici\'on Final 
Primera de esta Ley, continuar\'an en vigor, con su propio rango, las normas 
reglamentarias existentes y, en especial, los Reales Decretos 428/1993, de 26 
de marzo, 1332/1994, de 20 de junio y 994/1999, de 11 de junio, en cuanto no se 
opongan a la presente Ley.
\begin{center}
{\LARGE DISPOSICI\'ON DEROGATORIA}
\end{center}
\vspace{0.3cm} 
{\large {\bf \'Unica.}}

Queda derogada la Ley Org\'anica 5/1992, de 29 de octubre, de regulaci\'on del 
tratamiento automatizado de los datos de car\'acter personal.
\begin{center}
{\LARGE DISPOSICIONES FINALES}
\end{center}
\vspace{0.3cm}
{\large {\bf Primera.} Habilitaci\'on para el desarrollo reglamentario.}

El Gobierno aprobar\'a, o modificar\'a, las disposiciones reglamentarias 
necesarias para la aplicaci\'on y desarrollo de la presente Ley.
\vspace{0.3cm}\\
{\large {\bf Segunda.} Preceptos con car\'acter de Ley Ordinaria.}

Los t\'{\i}tulos IV, VI excepto el \'ultimo inciso del p\'arrafo 4 del 
art\'{\i}culo 36 y VII de la presente Ley, la Disposici\'on Adicional Cuarta, 
la Disposici\'on Transitoria Primera y la Final Primera, tienen el car\'acter 
de Ley Ordinaria.
\vspace{0.3cm}\\
{\large {\bf Tercera.} Entrada en vigor.}

La presente Ley entrar\'a en vigor en el plazo de un mes, contado desde su 
publicaci\'on en el Bolet\'{\i}n Oficial del Estado.
