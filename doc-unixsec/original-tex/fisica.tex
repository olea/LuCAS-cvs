\chapter{Seguridad f\'{\i}sica de los sistemas}
\section{Introducci\'on}
Seg\'un \cite{kn:glo}, la seguridad f\'{\i}sica de los sistemas inform\'aticos
consiste en {\it la aplicaci\'on de barreras f\'{\i}sicas y procedimientos de
control como medidas de prevenci\'on y contramedidas contra las amenazas a los
recursos y la informaci\'on confidencial}. M\'as claramente, y particularizando
para el caso de equipos Unix y sus centros de operaci\'on, por `seguridad 
f\'{\i}sica' podemos entender todas aquellas mecanismos -- generalmente de
prevenci\'on y detecci\'on -- destinados a proteger f\'{\i}sicamente cualquier
recurso del sistema; estos recursos son desde un simple teclado hasta una 
cinta de {\it backup} con toda la informaci\'on que hay en el sistema, pasando 
por la propia {\sc cpu} de la m\'aquina.\\
\\Desgraciadamente, la seguridad f\'{\i}sica es un aspecto olvidado con 
demasiada frecuencia a la hora de hablar de seguridad inform\'atica en general;
en muchas organizaciones se suelen tomar medidas para prevenir o detectar 
accesos no autorizados o negaciones de servicio, pero rara vez para prevenir
la acci\'on de un atacante que intenta acceder f\'{\i}sicamente a la sala de 
operaciones o al lugar donde se depositan las impresiones del sistema. Esto 
motiva que en determinadas situaciones un atacante se decline por aprovechar
vulnerabilidades f\'{\i}sicas en lugar de l\'ogicas, ya que posiblemente le
sea m\'as f\'acil robar una cinta con una imagen completa del sistema que 
intentar acceder a \'el mediante fallos en el {\it software}. Hemos de ser 
conscientes de que la seguridad f\'{\i}sica es demasiado importante como para
ignorarla: un ladr\'on que roba un ordenador para venderlo, un incendio o un
pirata que accede sin problemas a la sala de operaciones nos pueden hacer mucho
m\'as da\~no que un intruso que intenta conectar remotamente con una m\'aquina
no autorizada; no importa que utilicemos los m\'as avanzados medios de cifrado
para conectar a nuestros servidores, ni que hayamos definido una pol\'{\i}tica
de {\it firewalling} muy restrictiva: si no tenemos en cuenta factores 
f\'{\i}sicos, estos esfuerzos para proteger nuestra informaci\'on no van a 
servir de nada. Adem\'as, en el caso de organismos con requerimientos de 
seguridad medios, unas medidas de seguridad f\'{\i}sicas ejercen un efecto 
disuasorio sobre la mayor\'{\i}a de piratas: como casi todos los atacantes de 
los equipos de estos entornos son casuales (esto
es, no tienen inter\'es espec\'{\i}fico sobre {\it nuestros} equipos, sino sobre
{\it cualquier} equipo), si notan a trav\'es de medidas f\'{\i}sicas que 
nuestra organizaci\'on est\'a preocupada por la seguridad probablemente 
abandonar\'an el ataque para lanzarlo contra otra red menos protegida.\\
\\Aunque como ya dijimos en la introducci\'on este proyecto no puede 
centrarse en el dise\~no de edificios resistentes a un terremoto o en la 
instalaci\'on de alarmas electr\'onicas, s\'{\i} que se van a intentar comentar
ciertas medidas de prevenci\'on y detecci\'on que se han de tener en cuenta a
la hora de definir mecanismos y pol\'{\i}ticas para la seguridad de nuestros
equipos. Pero hemos de recordar que cada sitio es diferente, y por tanto 
tambi\'en lo son sus necesidades de seguridad; de esta forma, no se pueden dar 
recomendaciones espec\'{\i}ficas sino pautas generales a tener en cuenta, que
pueden variar desde el simple sentido com\'un (como es el cerrar con llave la
sala de operaciones cuando salimos de ella) hasta medidas mucho m\'as 
complejas, como la prevenci\'on de radiaciones electromagn\'eticas de los
equipos o la utilizaci\'on de {\it degaussers}. En entornos habituales
suele ser suficiente con un poco de sentido com\'un para conseguir una 
m\'{\i}nima seguridad f\'{\i}sica; de cualquier forma, en cada instituci\'on se
ha de analizar el valor de lo que se quiere proteger y la probabilidad de las 
amenazas potenciales, para en funci\'on de los resultados obtenidos dise\~nar
un plan de seguridad adecuado. Por ejemplo, en una empresa ubicada en 
Valencia quiz\'as parezca absurdo hablar de la prevenci\'on ante terremotos
(por ser esta un \'area de bajo riesgo), pero no suceder\'a lo mismo en una
universidad situada en una zona s\'{\i}smicamente activa; de la misma forma, en
entornos de I+D es absurdo hablar de la prevenci\'on ante un ataque nuclear,
pero en sistemas militares esta amenaza se ha de tener en cuenta\footnote{Al 
menos en teor\'{\i}a, ya que nadie sabe con certeza lo que sucede en 
organismos de defensa, excepto ellos mismos.}.
\section{Protecci\'on del {\it hardware}}
El {\it hardware} es frecuentemente el elemento m\'as caro de todo sistema
inform\'atico\footnote{Como dijimos, el m\'as caro, pero no el m\'as 
dif\'{\i}cil de recuperar.}. Por tanto, las medidas encaminadas a asegurar 
su integridad son una parte importante de la seguridad f\'{\i}sica de cualquier
organizaci\'on, especialmente en las dedicadas a I+D: universidades, centros
de investigaci\'on, institutos tecnol\'ogicos\ldots suelen poseer entre sus
equipos m\'aquinas muy caras, desde servidores con una gran potencia de 
c\'alculo hasta {\it routers} de \'ultima tecnolog\'{\i}a, pasando por modernos
sistemas de transmisi\'on de datos como la fibra \'optica.\\
\\Son muchas las amenazas al {\it hardware} de una instalaci\'on inform\'atica;
aqu\'{\i} se van a presentar algunas de ellas, sus posibles efectos y algunas
soluciones, si no para evitar los problemas s\'{\i} al menos para minimizar sus
efectos.
\subsection{Acceso f\'{\i}sico}
La posibilidad de acceder f\'{\i}sicamente a una m\'aquina Unix -- en general,
a cualquier sistema operativo -- hace in\'utiles casi todas las medidas de 
seguridad que hayamos aplicado sobre ella: hemos de pensar que si un atacante 
puede llegar con total libertad hasta una estaci\'on puede por ejemplo abrir la 
CPU y llevarse un disco duro; sin necesidad de privilegios en el sistema, sin
importar la robustez de nuestros cortafuegos, sin nisiquiera una clave de 
usuario, el atacante podr\'a seguramente modificar la informaci\'on almacenada, 
destruirla o simplemente leerla. Incluso sin llegar al extremo de desmontar
la m\'aquina, que quiz\'as resulte algo exagerado en entornos cl\'asicos
donde hay cierta vigilancia, como un laboratorio o una sala de inform\'atica,
la persona que accede al equipo puede pararlo o arrancar una versi\'on 
diferente del sistema operativo sin llamar mucho la atenci\'on. Si por ejemplo
alguien accede a un laboratorio con m\'aquinas Linux, seguramente le resultar\'a
f\'acil utilizar un disco de arranque, montar los discos duros de la m\'aquina
y extraer de ellos la informaci\'on deseada; incluso es posible que utilice un
{\it ramdisk} con ciertas utilidades que constituyan una amenaza para otros 
equipos, como {\it nukes} o {\it sniffers}.\\
\\Visto esto, parece claro que cierta seguridad f\'{\i}sica es necesaria para
garantizar la seguridad global de la red y los sistemas conectados a ella; 
evidentemente el nivel de seguridad f\'{\i}sica depende completamente del 
entorno donde se ubiquen los puntos a proteger (no es necesario hablar s\'olo
de equipos Unix, sino de cualquier elemento f\'{\i}sico que se pueda utilizar
para amenazar la seguridad, como una toma de red apartada en cualquier rinc\'on
de un edificio de nuestra organizaci\'on). Mientras que parte de los equipos 
estar\'an bien protegidos,
por ejemplo los servidores de un departamento o las m\'aquinas de los despachos,
otros muchos estar\'an en lugares de acceso semip\'ublico, como laboratorios
de pr\'acticas; es justamente sobre estos \'ultimos sobre los que debemos 
extremar las precauciones, ya que lo m\'as f\'acil y discreto para un atacante
es acceder a uno de estos equipos y, en segundos, lanzar un ataque completo 
sobre la red.
\subsubsection{Prevenci\'on}
>C\'omo prevenir un acceso f\'{\i}sico no autorizado a un determinado punto? 
Hay soluciones para todos los gustos, y tambi\'en de todos los precios: desde
analizadores de retina hasta videoc\'amaras, pasando por tarjetas inteligentes
o control de las llaves que abren determinada puerta. Todos los modelos de
autenticaci\'on de usuarios (cap\'{\i}tulo \ref{auth}) son aplicables, aparte
de para controlar el acceso l\'ogico a los sistemas, para controlar el acceso
f\'{\i}sico; de todos ellos, quiz\'as los m\'as adecuados a la seguridad
f\'{\i}sica sean los biom\'etricos y los basados en algo pose\'{\i}do; aunque
como comentaremos m\'as tarde suelen resultar algo caros para utilizarlos 
masivamente en entornos de seguridad media.\\
\\Pero no hay que irse a sistemas tan complejos para prevenir accesos 
f\'{\i}sicos no autorizados; normas tan elementales como cerrar las puertas 
con llave al salir de un laboratorio o un despacho o bloquear las tomas de
red que no se suelan utilizar y que est\'en situadas en lugares apartados son
en ocasiones m\'as que suficientes para prevenir ataques. Tambi\'en basta el
sentido com\'un para darse cuenta de que el cableado de red es un elemento 
importante para la seguridad, por lo que es recomendable apartarlo del acceso
directo; por desgracia, en muchas organizaciones podemos ver excelentes
ejemplos de lo que {\bf no} hay que hacer en este sentido: cualquiera que 
pasee por entornos m\'as o menos amplios (el campus de una universidad, por
ejemplo) seguramente podr\'a ver -- o pinchar, o cortar\ldots -- cables 
descolgados al alcance de todo el mundo, especialmente durante el verano, 
\'epoca que se suele aprovechar para hacer obras.\\ 
\\Todos hemos visto pel\'{\i}culas en las que se mostraba un estricto control
de acceso a instalaciones militares mediante tarjetas inteligentes, analizadores
de retina o verificadores de la geometr\'{\i}a de la mano; aunque algunos de
estos m\'etodos a\'un suenen a ciencia ficci\'on y sean demasiado caros para
la mayor parte de entornos (recordemos que si el sistema de protecci\'on es
m\'as caro que lo que se quiere proteger tenemos un grave error en nuestros
planes de seguridad), otros se pueden aplicar, y se aplican, en muchas 
organizaciones. Concretamente, el uso de lectores de tarjetas para poder acceder
a ciertas dependencias es algo muy a la orden del d\'{\i}a; la idea es 
sencilla: alguien pasa una tarjeta por el lector, que conecta con un sistema 
-- por ejemplo un ordenador -- en el que existe una base de datos con 
informaci\'on de los usuarios y los recintos a los que se le permite el acceso.
Si la tarjeta pertenece a un usuario capacitado para abrir la puerta, \'esta se
abre, y en caso contrario se registra el intento y se niega el acceso. Aunque 
este m\'etodo quiz\'as resulte algo caro para extenderlo a
todos y cada uno de los puntos a proteger en una organizaci\'on, no ser\'{\i}a 
tan descabellado instalar peque\~nos lectores de c\'odigos de barras conectados
a una m\'aquina Linux en las puertas de muchas \'areas, especialmente en las
que se maneja informaci\'on m\'as o menos sensible. Estos lectores podr\'{\i}an 
leer una tarjeta que todos los miembros de la organizaci\'on poseer\'{\i}an, 
conectar con la base de datos de usuarios, y autorizar o denegar la apertura de 
la puerta. Se 
tratar\'{\i}a de un sistema sencillo de implementar, no muy caro, y que cubre
de sobra las necesidades de seguridad en la mayor\'{\i}a de entornos:
incluso se podr\'{\i}a abaratar si en lugar de utilizar un mecanismo para abrir
y cerrar puertas el sistema se limitara a informar al administrador del \'area 
o a un guardia de seguridad mediante un mensaje en pantalla o una luz 
encendida: de esta forma
los \'unicos gastos ser\'{\i}an los correspondientes a los lectores de c\'odigos
de barras, ya que como equipo con la base de datos se puede utilizar una 
m\'aquina vieja o un servidor de prop\'osito general.
\subsubsection{Detecci\'on}
Cuando la prevenci\'on es dif\'{\i}cil por cualquier motivo (t\'ecnico, 
econ\'omico, humano\ldots) es deseable que un potencial ataque sea detectado
cuanto antes, para minimizar as\'{\i} sus efectos. Aunque en la detecci\'on de
problemas, generalmente accesos f\'{\i}sicos no autorizados, intervienen medios
t\'ecnicos, como c\'amaras de vigilancia de circuito cerrado o alarmas, en 
entornos m\'as normales el esfuerzo en detectar estas amenazas se ha de centrar 
en las personas que
utilizan los sistemas y en las que sin utilizarlos est\'an relacionadas de 
cierta forma con ellos; sucede lo mismo que con la seguridad l\'ogica: se ha
de ver toda la protecci\'on como una cadena que falla si falla su eslab\'on
m\'as d\'ebil.\\
\\Es importante concienciar a todos de su papel en la pol\'{\i}tica de seguridad
del entorno; si por ejemplo un usuario autorizado detecta presencia de alguien 
de quien sospecha que no tiene autorizaci\'on para estar en una determinada
estancia debe avisar inmediatamente al administrador o al responsable de los
equipos, que a su vez puede avisar al servicio de seguridad si es necesario. No
obstante, utilizar este servicio debe ser s\'olamente un \'ultimo recurso: 
generalmente en la mayor\'{\i}a de entornos no estamos tratando con 
terroristas, sino por fortuna con elementos mucho menos peligrosos. Si 
cada vez que se sospecha de alguien se avisa al servicio de seguridad esto 
puede repercutir en el ambiente de trabajo de los usuarios autorizados 
estableciendo cierta presi\'on que no es en absoluto recomendable; un simple
{\it `>puedo ayudarte en algo?'} suele ser m\'as efectivo que un guardia 
solicitando una identificaci\'on formal. Esto es especialmente recomendable en
lugares de acceso restringido, como laboratorios de investigaci\'on o centros
de c\'alculo, donde los usuarios habituales suelen conocerse entre ellos y es
f\'acil detectar personas ajenas al entorno.
\subsection{Desastres naturales}
En el anterior punto hemos hecho referencia a accesos f\'{\i}sicos no 
autorizados a zonas o a elementos que pueden comprometer la seguridad de los
equipos o de toda la red; sin embargo, no son estas las \'unicas amenazas
relacionadas con la seguridad f\'{\i}sica. Un problema que no suele ser tan
habitual, pero que en caso de producirse puede acarrear grav\'{\i}simas
consecuencias, es el derivado de los desastres naturales y su (falta de) 
prevenci\'on.
\subsubsection{Terremotos}
Los terremotos son el desastre natural menos probable en la mayor\'{\i}a de 
organismos ubicados en Espa\~na,
simplemente por su localizaci\'on geogr\'afica: no nos encontramos en una zona
donde se suelan producir temblores de intensidad considerable; incluso en zonas
del sur de Espa\~na, como Almer\'{\i}a, donde la probabilidad de un temblor es
m\'as elevada, los terremotos no suelen alcanzan la magnitud necesaria para 
causar da\~nos en los equipos. Por tanto, no se suelen tomar medidas serias 
contra los movimientos s\'{\i}smicos, ya que la probabilidad de que sucedan
es tan baja que no merece la pena invertir dinero para minimizar sus efectos.\\
\\De cualquier forma, aunque algunas medidas contra terremotos son excesivamente
caras para la mayor parte de organizaciones en Espa\~na (evidentemente 
ser\'{\i}an igual
de caras en zonas como Los \'Angeles, pero all\'{\i} el coste estar\'{\i}a 
justificado por la alta probabilidad de que se produzcan movimientos de magnitud
considerable), no cuesta nada tomar ciertas medidas de prevenci\'on; por 
ejemplo, es muy recomendable no situar nunca equipos delicados en superficies
muy elevadas (aunque tampoco es bueno situarlos a ras de suelo, como veremos 
al hablar de inundaciones). Si lo hacemos, un peque\~no temblor puede tirar
desde una altura considerable un complejo {\it hardware}, lo que con toda 
probabilidad lo inutilizar\'a; puede incluso ser conveniente (y barato) utilizar
fijaciones para los elementos m\'as cr\'{\i}ticos, como las CPUs, los monitores
o los {\it routers}. De la misma forma, tampoco es recomendable situar
objetos pesados en superficies altas cercanas a los equipos, ya que si lo que
cae son esos objetos tambi\'en da\~nar\'an el {\it hardware}.\\
\\Para evitar males mayores ante un terremoto, tambi\'en es muy importante 
no situar equipos cerca de las ventanas: si se produce un temblor pueden
caer por ellas, y en ese caso la p\'erdida de datos o {\it hardware} pierde
importancia frente a los posibles accidentes -- incluso mortales -- que puede
causar una pieza voluminosa a las personas a las que les cae encima. Adem\'as,
situando los equipos alejados de las ventanas estamos dificultando las acciones
de un potencial ladr\'on que se descuelgue por la fachada hasta las ventanas, ya
que si el equipo estuviera cerca no tendr\'{\i}a m\'as que alargar el brazo para
llev\'arselo.\\
\\Quiz\'as hablar de terremotos en un trabajo dedicado a sistemas `normales'
especialmente cen\-tr\'an\-do\-nos en lugares con escasa actividad s\'{\i}smica
-- 
como es Espa\~na y m\'as concretamente la Comunidad Valenciana -- pueda resultar
incluso gracioso, o cuanto menos exagerado. No obstante, no debemos entender
por terremotos \'unicamente a los grandes desastres que derrumban edificios y
destrozan v\'{\i}as de comunicaci\'on; quiz\'as ser\'{\i}a mas apropiado hablar
incluso de {\bf vibraciones}, desde las m\'as grandes (los terremotos) hasta 
las m\'as peque\~nas (un simple motor cercano a los equipos). Las vibraciones,
incluso las m\'as imperceptibles, pueden da\~nar seriamente cualquier elemento
electr\'onico de nuestras m\'aquinas, especialmente si se trata de vibraciones
cont\'{\i}nuas: los primeros efectos pueden ser problemas con los cabezales de
los discos duros o con los circuitos integrados que se da\~nan en las placas.
Para hacer frente a peque\~nas vibraciones podemos utilizar plataformas de goma
donde situar a los equipos, de forma que la plataforma absorba la mayor parte
de los movimientos; incluso sin llegar a esto, una regla com\'un es evitar 
que entren en contacto equipos que poseen una electr\'onica delicada con {\it
hardware} m\'as mec\'anico, como las impresoras: estos dispositivos no paran de
generar vibraciones cuando est\'an en funcionamiento, por lo que situar una
peque\~na impresora encima de la CPU de una m\'aquina es una idea nefasta. Como
dicen algunos expertos en seguridad (\cite{kn:spa96}), el espacio en la sala de 
operaciones es un problema sin importancia comparado con las consecuencias de
fallos en un disco duro o en la placa base de un ordenador.
\subsubsection{Tormentas el\'ectricas}
Las tormentas con aparato el\'ectrico, especialmente frecuentes en verano 
(cuando mucho personal se encuentra de vacaciones, lo que las hace m\'as
peligrosas) generan subidas s\'ubitas de tensi\'on infinitamente superiores a
las que pueda generar un problema en la red el\'ectrica, como veremos a 
continuaci\'on. Si cae un rayo sobre la estructura met\'alica del edificio donde
est\'an situados nuestros equipos es casi seguro que podemos ir pensando en 
comprar otros nuevos; sin llegar a ser tan dram\'aticos, la ca\'{\i}da de un
rayo en un lugar cercano puede inducir un campo magn\'etico lo suficientemente
intenso como para destruir {\it hardware} incluso protegido contra voltajes
elevados.\\
\\Sin embargo, las tormentas poseen un lado positivo: son predecibles con m\'as
o menos exactitud, lo que permite a un administrador parar sus m\'aquinas y
desconectarlas de la l\'{\i}nea el\'ectrica\footnote{Al contrario de lo que
mucha gente piensa, no basta s\'olo con apagar un sistema para que se encuentre
a salvo.}. Entonces, >cu\'al es el problema? Aparte de las propias tormentas,
el problema son los responsables de los equipos: la ca\'{\i}da de un rayo es 
algo poco probable -- pero no imposible -- en una gran ciudad donde existen
artilugios destinados justamente a atraer rayos de una forma controlada; tanto
es as\'{\i} que mucha gente ni siquiera ha visto caer cerca un rayo, por lo que 
directamente tiende a asumir que eso no le va a suceder nunca, y menos a sus
equipos. Por tanto, muy pocos administradores se molestan en parar m\'aquinas
y desconectarlas ante una tormenta; si el fen\'omeno sucede durante las horas
de trabajo y la tormenta es fuerte, quiz\'as s\'{\i} que lo hace, pero si
sucede un s\'abado por la noche nadie va a ir a la sala de operaciones a 
proteger a los equipos, y nadie antes se habr\'a tomado la molestia de 
protegerlos por una simple previsi\'on meteorol\'ogica. Si a esto a\~nadimos lo
que antes hemos comentado, que las tormentas se producen con m\'as frecuencia
en pleno verano, cuando casi toda la plantilla est\'a de vacaciones y s\'olo
hay un par de personas de guardia, tenemos el caldo de cultivo ideal para que
una amenaza que {\it a priori} no es muy grave se convierta en el final de 
algunos de nuestros equipos. Conclusi\'on: todos hemos de tomar m\'as en serio
a la Naturaleza cuando nos avisa con un par de truenos\ldots\\
\\Otra medida de protecci\'on contra las tormentas el\'ectricas hace referencia
a la ubicaci\'on de los medios magn\'eticos, especialmente las copias de 
seguridad; aunque hablaremos con m\'as detalle de la protecci\'on de los {\it
backups} en el punto \ref{backups}, de momento podemos adelantar que se han
de almacenar lo m\'as alejados posible de la estructura met\'alica de los 
edificios. Un rayo en el propio edificio, o en un lugar cercano, puede inducir
un campo electromagn\'etico lo suficientemente grande como para borrar de golpe
todas nuestras cintas o discos, lo que a\~nade a los problemas por da\~nos en
el {\it hardware} la p\'erdida de toda la informaci\'on de nuestros sistemas.
\subsubsection{Inundaciones y humedad}
Cierto grado de humedad es necesario para un correcto funcionamiento de nuestras
m\'aquinas: en ambientes extremadamente secos el nivel de electricidad 
est\'atica es elevado, lo que, como veremos m\'as tarde, puede transformar un
peque\~no contacto entre una persona y un circuito, o entre diferentes 
componentes de una m\'aquina, en un da\~no irreparable al {\it hardware} y a
la informaci\'on. No obstante, niveles de humedad elevados son perjudiciales
para los equipos porque pueden producir condensaci\'on en los circuitos 
integrados, lo que origina cortocircuitos que evidentemente tienen efectos 
negativos sobre cualquier elemento electr\'onico de una m\'aquina.\\
\\Controlar el nivel de humedad en los entornos habituales es algo
innecesario, ya que por norma nadie ubica estaciones en los lugares m\'as
h\'umedos o que presenten situaciones extremas; no obstante, ciertos equipos
son especialmente sensibles a la humedad, por lo que es conveniente consultar
los manuales de todos aquellos de los que tengamos dudas. Quiz\'as sea 
necesario utilizar alarmas que se activan al detectar condiciones de muy poca
o demasiada humedad, especialmente en sistemas de alta disponibilidad o de 
altas prestaciones, donde un fallo en un componente puede ser crucial.\\
\\Cuando ya no se habla de una humedad m\'as o menos elevada sino de completas
inundaciones, los problemas generados son mucho mayores. Casi cualquier medio
(una m\'aquina, una cinta, un {\it router}\ldots) que entre en contacto con
el agua queda autom\'aticamente inutilizado, bien por el propio l\'{\i}quido
o bien por los cortocircuitos que genera en los sistemas electr\'onicos.\\
\\Evidentemente, contra las inundaciones las medidas m\'as efectivas son las de
prevenci\'on (frente a las de detecci\'on); podemos utilizar detectores de 
agua en los suelos o falsos suelos de las salas de operaciones, y apagar 
autom\'aticamente los sistemas en caso de que se activen. Tras apagar los 
sistemas podemos tener tambi\'en instalado un sistema autom\'atico que corte
la corriente: algo muy com\'un es intentar sacar los equipos -- previamente 
apagados o no -- de una sala que se est\'a empezando a inundar; esto, que a
primera vista parece lo l\'ogico, es el mayor error que se puede cometer si no
hemos desconectado completamente el sistema el\'ectrico, ya que la mezcla de
corriente y agua puede causar incluso la muerte a quien intente salvar equipos.
Por muy caro que sea el {\it hardware} o por muy valiosa que sea la 
informaci\'on a proteger, nunca ser\'an magnitudes comparables a lo que supone
la p\'erdida de vidas humanas. Otro error com\'un relacionado con los detectores
de agua es situar a los mismos a un nivel superior que a los propios equipos a
salvaguardar (<incluso en el techo, junto a los detectores de humo!); 
evidentemente, cuando en estos casos el agua llega al detector poco se puede 
hacer ya por las m\'aquinas o la informaci\'on que contienen.\\
\\Medidas de protecci\'on menos sofisticadas pueden ser la instalaci\'on de
un falso suelo por encima del suelo real, o simplemente tener la
precauci\'on de situar a los equipos con una cierta elevaci\'on respecto al
suelo, pero sin llegar a situarlos muy altos por los problemas que ya hemos 
comentado al hablar de terremotos y vibraciones.
\subsection{Desastres del entorno}
\subsubsection{Electricidad}
Quiz\'as los problemas derivados del entorno de trabajo m\'as frecuentes son
los relacionados con el sistema el\'ectrico que alimenta nuestros equipos; 
cortocircuitos, picos de tensi\'on, cortes de flujo\ldots a diario amenazan
la integridad tanto de nuestro {\it hardware} como de los datos que almacena
o que circulan por \'el.\\
\\El problema menos com\'un en las instalaciones modernas son las subidas 
de tensi\'on, conocidas como `picos' porque generalmente duran muy poco: durante
unas fracciones de segundo el voltaje que recibe un equipo sube hasta sobrepasar
el l\'{\i}mite aceptable que dicho equipo soporta. Lo normal es 
que estos picos apenas afecten al {\it hardware} o a los datos gracias a que en
la mayor\'{\i}a de equipos hay instalados fusibles, elementos que se funden ante
una subida de tensi\'on y dejan de conducir la corriente, provocando que la 
m\'aquina permanezca apagada. Disponga o no de fusibles el equipo a proteger
(lo normal es que s\'{\i} los tenga) una medida efectiva y barata es utilizar
tomas de tierra para asegurar a\'un m\'as la integridad; estos mecanismos 
evitan los problemas de sobretensi\'on desviando el exceso de corriente hacia
el suelo de una sala o edificio, o simplemente hacia cualquier lugar con voltaje
nulo. Una toma de tierra sencilla puede consistir en un buen conductor conectado
a los chasis de los equipos a proteger y a una barra maciza, tambi\'en 
conductora, que se introduce lo m\'as posible en el suelo; el coste de la 
instalaci\'on es peque\~no, especialmente si lo comparamos con las p\'erdidas
que supondr\'{\i}a un incendio que afecte a todos o a una parte de nuestros
equipos.\\
\\Incluso teniendo un sistema protegido con los m\'etodos anteriores, si la
subida de tensi\'on dura demasiado, o si es demasiado r\'apida, podemos sufrir
da\~nos en los equipos; existen acondicionadores de tensi\'on comerciales que
protegen de los picos hasta en los casos m\'as extremos, y que tambi\'en se
utilizan como filtros para ruido el\'ectrico. Aunque en la mayor\'{\i}a de
situaciones no es necesario su uso, si nuestra organizaci\'on tiene problemas
por el voltaje excesivo quiz\'as sea conveniente instalar alguno de estos
aparatos.\\
\\Un problema que los estabilizadores de tensi\'on o las tomas de tierra no
pueden solucionar es justamente el contrario a las subidas de tensi\'on: las
bajadas, situaciones en las que la corriente desciende por debajo del voltaje
necesario para un correcto funcionamiento del sistema, pero sin llegar a ser
lo suficientemente bajo para que la m\'aquina se apague (\cite{kn:san90}). En
estas situaciones la m\'aquina se va a comportar de forma extra\~na e 
incorrecta, por ejemplo no aceptando algunas instrucciones, no completando 
escrituras en disco o memoria, etc. Es una situaci\'on similar a la de una 
bombilla que pierde intensidad moment\'aneamente por falta de corriente, pero
trasladada a un sistema que en ese peque\~no intervalo ejecuta miles o millones
de instrucciones y transferencias de datos.\\
\\Otro problema, much\'{\i}simo m\'as habituales que los anteriores en redes
el\'ectricas modernas, son los cortes
en el fluido el\'ectrico que llega a nuestros equipos. Aunque un simple corte
de corriente no suele afectar al {\it hardware}, lo m\'as peligroso (y que
sucede en muchas ocasiones) son las idas y venidas r\'apidas de la corriente;
en esta situaci\'on, aparte de perder datos, nuestras m\'aquinas pueden sufrir
da\~nos.\\
\\La forma m\'as efectiva de proteger nuestros equipos contra estos problemas
de la corriente el\'ectrica es utilizar una SAI (Servicio de Alimentaci\'on
Ininterrumpido) conectada al elemento que queremos proteger. Estos dispositivos
mantienen un flujo de corriente correcto y estable de corriente, protegiendo 
as\'{\i} los equipos de subidas, cortes y bajadas de tensi\'on; tienen capacidad
para seguir alimentando las m\'aquinas incluso en caso de que no reciban 
electricidad (evidentemente no las alimentan de forma indefinida, sino durante
un cierto tiempo -- el necesario para detener el sistema de forma ordenada). 
Por tanto, en caso de fallo de la corriente el SAI informar\'a a la m\'aquina
Unix, que a trav\'es de un programa como {\tt /sbin/powerd} recibe la 
informaci\'on y decide cuanto tiempo de corriente le queda para poder pararse
correctamente; si de nuevo vuelve el flujo la SAI vuelve a informar de este
evento y el sistema desprograma su parada. As\'{\i} de simple: por poco m\'as
de diez mil pesetas podemos obtener una SAI peque\~na, m\'as que suficiente para
muchos servidores, que nos va a librar de la mayor\'{\i}a de los problemas
relacionados con la red el\'ectrica.\\
\\Un \'ultimo problema contra el que ni siquiera las SAIs nos protegen es la 
corriente est\'atica, un fen\'omeno extra\~no del que la mayor\'{\i}a de gente
piensa que no afecta a los equipos, s\'olo a otras personas. Nada m\'as lejos
de la realidad: simplemente tocar con la mano la parte met\'alica de teclado o 
un conductor de una placa puede destruir un equipo completamente. Se trata de
corriente de muy poca intensidad pero un alt\'{\i}simo voltaje, por lo que 
aunque la persona no sufra ning\'un da\~no -- s\'olo un peque\~no calambrazo --
el ordenador sufre una descarga que puede ser suficiente para destrozar todos
sus componentes, desde el disco duro hasta la memoria RAM. Contra el problema
de la corriente est\'atica existen muchas y muy baratas soluciones: {\it spray}
antiest\'atico, ionizadores antiest\'aticos\ldots No obstante en la mayor\'{\i}a
de situaciones s\'olo hace falta un poco de sentido com\'un del usuario para
evitar accidentes: no tocar directamente ninguna parte met\'alica, protegerse
si debe hacer operaciones con el {\it hardware}, no mantener el entorno 
excesivamente seco\ldots
\subsubsection{Ruido el\'ectrico}
Dentro del apartado anterior podr\'{\i}amos haber hablado del ruido el\'ectrico
como un problema m\'as relacionado con la electricidad; sin embargo este 
problema no es una incidencia directa de la corriente en nuestros equipos, sino
una incidencia relacionada con la corriente de otras m\'aquinas que pueden 
afectar al funcionamiento de la nuestra. El ruido el\'ectrico suele ser 
generado por motores o por maquinaria pesada, pero tambi\'en puede serlo por
otros ordenadores o por multitud de aparatos, especialmente muchos de los 
instalados en los laboratorios de
organizaciones de I+D, y se transmite a trav\'es del espacio o de l\'{\i}neas
el\'ectricas cercanas a nuestra instalaci\'on.\\
\\Para prevenir los problemas que el ruido el\'ectrico puede causar en nuestros
equipos lo m\'as barato es intentar no situar {\it hardware} cercano a la
maquinaria que puede causar dicho ruido; si no tenemos m\'as remedio que 
hacerlo, podemos instalar filtros en las l\'{\i}neas de alimentaci\'on que 
llegan hasta los ordenadores. Tambi\'en es recomendable mantener alejados de
los equipos dispositivos emisores de ondas, como tel\'efonos m\'oviles, 
transmisores de radio o {\it walkie-talkies}; estos elementos puede incluso
da\~nar permanentemente a nuestro {\it hardware} si tienen la suficiente 
potencia de transmisi\'on, o influir directamente en elementos que pueden 
da\~narlo como detectores de incendios o cierto tipo de alarmas.
\subsubsection{Incendios y humo}
Una causa casi siempre relacionada con la electricidad son los incendios, y
con ellos el humo; aunque la causa de un fuego puede ser un desastre natural,
lo habitual en muchos entornos es que el mayor peligro de incendio provenga de
problemas el\'ectricos por la sobrecarga de la red debido al gran n\'umero de
aparatos conectados al tendido. Un simple cortocircuito o un equipo que se
calienta demasiado pueden convertirse en la causa directa de un incendio en
el edificio, o al menos en la planta, donde se encuentran invertidos millones
de pesetas en equipamiento.\\
\\Un m\'etodo efectivo contra los incendios son los extintores situados en el
techo, que se activan autom\'aticamente al detectar humo o calor. Algunos de
ellos, los m\'as antiguos, utilizaban agua para apagar las llamas, lo que 
provocaba que el {\it hardware} no llegara a sufrir los efectos del fuego si
los extintores se activaban correctamente, pero que quedara destrozado por el
agua expulsada. Visto este problema, a mitad de los ochenta se comenzaron a
utilizar extintores de hal\'on; este compuesto no conduce electricidad ni 
deja residuos, por lo que resulta ideal para no da\~nar los equipos. Sin 
embargo, tambi\'en el hal\'on presentaba problemas: por un lado, resulta 
excesivamente contaminante para la atm\'osfera, y por otro puede axfisiar a las
personas a la vez que acaba con el fuego. Por eso se han sustituido los 
extintores de hal\'on (aunque se siguen utilizando mucho hoy en d\'{\i}a) por
extintores de di\'oxido de carbono, menos contaminante y menos perjudicial. De
cualquier forma, al igual que el hal\'on el di\'oxido de carbono no es 
precisamente sano para los humanos, por lo que antes de activar el extintor es
conveniente que todo el mundo abandone la sala; si se trata de sistemas de 
activaci\'on autom\'atica suelen avisar antes de expulsar su compuesto mediante
un pitido.\\
\\Aparte del fuego y el calor generado, en un incendio existe un tercer 
elemento perjudicial para los equipos: el humo, un potente abrasivo que ataca
especialmente los discos magn\'eticos y \'opticos. Quiz\'as ante un incendio el
da\~no provocado por el humo sea insignificante en comparaci\'on con el causado
por el fuego y el calor, pero hemos de recordar que
puede existir humo sin necesidad de que haya un fuego: por ejemplo, en salas
de operaciones donde se fuma. Aunque muchos no apliquemos esta regla y fumemos
demasiado -- siempre es demasiado -- delante de nuestros equipos, ser\'{\i}a
conveniente no permitir esto; aparte de la suciedad generada que se deposita
en todas las partes de un ordenador, desde el teclado hasta el monitor, 
generalmente todos tenemos el cenicero cerca de los equipos, por lo que el
humo afecta directamente a todos los componentes; incluso al ser algo m\'as
habitual que un incendio, se puede considerar m\'as perjudicial -- para los
equipos y las personas -- el humo del tabaco que el de un fuego.\\
\\En muchos manuales de seguridad se insta a los usuarios, administradores, o
al personal en general a intentar controlar el fuego y salvar el equipamiento;
esto tiene, como casi todo, sus pros y sus contras. Evidentemente, algo
l\'ogico cuando estamos ante un incendio de peque\~nas dimensiones es intentar
utilizar un extintor para apagarlo, de forma que lo que podr\'{\i}a haber sido
una cat\'astrofe sea un simple susto o un peque\~no accidente. Sin embargo, 
cuando las dimensiones de las llamas son considerables lo \'ultimo que debemos
hacer es intentar controlar el fuego nosotros mismos, arriesgando vidas para
salvar {\it hardware}; como suced\'{\i}a en el caso de inundaciones, no 
importa el precio de nuestros equipos o el valor de nuestra informaci\'on:
nunca ser\'an tan importantes como una vida humana. Lo m\'as recomendable en
estos casos es evacuar el lugar del incendio y dejar su control en manos de
personal especializado.
\subsubsection{Temperaturas extremas}
No hace falta ser un genio para comprender que las temperaturas extremas, ya
sea un calor excesivo o un frio intenso, perjudican gravemente a todos los
equipos. Es recomendable que los equipos operen entre 10 y 32 grados Celsius
(\cite{kn:spa96}), aunque peque\~nas variaciones en este rango tampoco han de
influir en la mayor\'{\i}a de sistemas.\\
\\Para controlar la temperatura ambiente en el entorno de operaciones nada
mejor que un acondicionador de aire, aparato que tambi\'en influir\'a 
positivamente en el rendimiento de los usuarios (las personas tambi\'en tenemos
rangos de temperaturas dentro de los cuales trabajamos m\'as c\'omodamente).
Otra condici\'on b\'asica para el correcto funcionamiento de cualquier equipo 
que \'este se encuentre correctamente ventilado, sin elementos que obstruyan los
ventiladores de la CPU. La organizaci\'on f\'{\i}sica del computador tambi\'en
es decisiva para evitar sobrecalentamientos: si los discos duros, elementos
que pueden alcanzar temperaturas considerables, se encuentran excesivamente 
cerca de la memoria RAM, es muy probable que los m\'odulos acaben 
quem\'andose.
\section{Protecci\'on de los datos} 
La seguridad f\'{\i}sica tambi\'en implica una protecci\'on a la informaci\'on
de nuestro sistema, tanto a la que est\'a almacenada en \'el como a la que
se transmite entre diferentes equipos. Aunque los apartados comentados en la
anterior secci\'on son aplicables a la protecci\'on f\'{\i}sica de los datos
(ya que no olvidemos que si protegemos el {\it hardware} tambi\'en protegemos
la informaci\'on que se almacena o se transmite por \'el), hay ciertos aspectos
a tener en cuenta a la hora de dise\~nar una pol\'{\i}tica de seguridad 
f\'{\i}sica que afectan principalmente, aparte de a los elementos f\'{\i}sicos,
a los datos de nuestra organizaci\'on; existen ataques cuyo objetivo no es
destruir el medio f\'{\i}sico de nuestro sistema, sino simplemente conseguir la
informaci\'on almacenada en dicho medio.
\subsection{{\it Eavesdropping}}
\label{eavesdropping}
La interceptaci\'on o {\it eavesdropping}, tambi\'en conocida por {\it passive
wiretapping} (\cite{kn:cesid}) es un proceso mediante el cual un agente capta
informaci\'on -- en claro o cifrada -- que no le iba dirigida; esta
captaci\'on puede realizarse por much\'{\i}simos medios (por ejemplo, 
capturando las radiaciones electromagn\'eticas, como veremos luego). Aunque es
en principio un ataque completamente pasivo, lo m\'as
peligroso del {\it eavesdropping} es que es muy dif\'{\i}cil de detectar
mientras que se produce, de forma que un atacante puede capturar informaci\'on
privilegiada y claves para acceder a m\'as informaci\'on sin que nadie se de
cuenta hasta que dicho atacante utiliza la informaci\'on capturada, 
convirtiendo el ataque en activo.\\
\\Un medio de interceptaci\'on bastante habitual es el {\it sniffing}, 
consistente en capturar tramas que circulan por la red mediante un programa 
ejecut\'andose en una m\'aquina conectada a ella o bien mediante un dispositivo
que se engancha directamente el cableado\footnote{En este caso tambi\'en se 
suele llamar a esta actividad {\it wiretapping}.}. Estos dispositivos, 
denominados {\it 
sniffers} de alta impedancia, se conectan en paralelo con el cable de forma que
la impedancia total del cable y el aparato es similar a la del cable solo, lo
que hace dif\'{\i}cil su detecci\'on. Contra estos ataques existen diversas
soluciones; la m\'as barata a nivel f\'{\i}sico es no permitir la existencia de 
segmentos de red de f\'acil acceso, lugares id\'oneos para que un atacante 
conecte uno de estos aparatos y capture todo nuestro tr\'afico. No obstante
esto resulta dif\'{\i}cil en redes ya instaladas, donde no podemos modificar
su arquitectura; en estos existe una soluci\'on generalmente gratuita pero que
no tiene mucho que ver con el nivel f\'{\i}sico: el uso de aplicaciones de
cifrado para realizar las comunicaciones o el almacenamiento de la informaci\'on
(hablaremos m\'as adelante de algunas de ellas). Tampoco debemos descuidar las
tomas de red libres, donde un intruso con un portatil puede conectarse para
capturar tr\'afico; es recomendable analizar regularmente nuestra red para
verificar que todas las m\'aquinas activas est\'an autorizadas.\\
\\Como soluciones igualmente efectivas contra la interceptaci\'on a nivel
f\'{\i}sico podemos citar
el uso de dispositivos de cifra (no simples programas, sino {\it hardware}),
generalmente {\it chips} que implementan algoritmos como {\sc des}; esta 
soluci\'on es muy poco utilizada en entornos de I+D, ya que es much\'{\i}simo 
m\'as cara que utilizar implementaciones {\it software} de tales algoritmos y
en muchas ocasiones la \'unica diferencia entre los programas y los dispositivos
de cifra es la velocidad. Tambi\'en se puede utilizar, como soluci\'on m\'as
cara, el cableado en vac\'{\i}o para evitar la interceptaci\'on de datos que
viajan por la red: la idea es situar los cables en tubos donde artificialmente
se crea el vac\'{\i}o o se inyecta aire a presi\'on; si un atacante intenta
`pinchar' el cable para interceptar los datos, rompe el vac\'{\i}o o el nivel
de presi\'on y el ataque es detectado inmediatamente. Como decimos, esta 
soluci\'on es enormemente cara y s\'olamente se aplica en redes de 
per\'{\i}metro reducido para entornos de alta seguridad.\\
\\Antes de finalizar este punto debemos recordar un peligro que
muchas veces se ignora: el de la interceptaci\'on de datos emitidos en forma de
sonido o simple ruido en nuestro entorno de operaciones. Imaginemos una 
situaci\'on en la que los responsables de la seguridad de nuestra organizaci\'on
se reunen para discutir nuevos mecanismos de protecci\'on; todo lo que en esa
reuni\'on se diga puede ser capturado por multitud de m\'etodos, algunos de los
cuales son tan simples que ni siquiera se contemplan en los planes de seguridad.
Por ejemplo, una simple tarjeta de sonido instalada en un PC situado en la sala
de reuniones puede transmitir a un atacante todo lo que se diga en esa 
reuni\'on; mucho m\'as simple y sencillo: un tel\'efono mal 
colgado -- intencionada o inintencionadamente -- tambi\'en puede transmitir 
informaci\'on muy \'util para un potencial enemigo. Para evitar estos problemas
existen numerosos m\'etodos: por ejemplo, en el caso de los tel\'efonos fijos
suele ser suficiente un poco de atenci\'on y sentido com\'un, ya que basta con
comprobar que est\'an bien colgados\ldots o incluso desconectados de la red
telef\'onica. El caso de los m\'oviles suele ser algo m\'as complejo de 
controlar, ya que su peque\~no tama\~no permite camuflarlos f\'acilmente; no 
obstante, podemos instalar en la sala de reuniones un sistema de aislamiento 
para bloquear el uso de estos tel\'efonos: se trata de sistemas que ya se 
utilizan en ciertos entornos (por ejemplo en conciertos musicales) para evitar 
las molestias de un m\'ovil sonando, y que trabajan bloqueando cualquier 
transmisi\'on en los rangos de frecuencias en los que trabajan los diferentes
operadores telef\'onicos. Otra medida preventiva (ya no para voz, sino para 
prevenir la fuga de datos v\'{\i}a el ruido ambiente) muy \'util -- y no muy 
cara -- puede ser sustituir todos los tel\'efonos fijos de disco por 
tel\'efonos de teclado, ya que el ruido de un disco al girar puede permitir a 
un pirata deducir el n\'umero de tel\'efono marcado desde ese aparato.
\subsection{{\it Backups}}
\label{backups}
En este apartado no vamos a hablar de las normas para establecer una 
pol\'{\i}tica de realizaci\'on de copias de seguridad correcta, ni tampoco de
los mecanismos necesarios para implementarla o las precauciones que hay que 
tomar para que todo funcione correctamente; el tema que vamos a tratar en este
apartado es la protecci\'on f\'{\i}sica de la informaci\'on almacenada en {\it 
backups}, esto es, de la protecci\'on de los diferentes medios donde residen
nuestras copias de seguridad. Hemos de tener siempre presente que si las copias 
contienen toda nuestra informaci\'on tenemos que protegerlas igual que 
protegemos nuestros sistemas.\\
\\Un error muy habitual es almacenar los dispositivos de {\it backup} en lugares
muy cercanos a la sala de operaciones, cuando no en la misma sala; esto, que
en principio puede parecer correcto (y c\'omodo si necesitamos restaurar unos
archivos) puede convertirse en un problema: imaginemos simplemente que se
produce un incendio de grandes dimensiones y todo el edificio queda reducido
a cenizas. En este caso extremo tendremos que unir al problema de perder todos
nuestros equipos -- que seguramente cubrir\'a el seguro, por lo que no se puede
considerar una cat\'astrofe -- el perder tambi\'en todos nuestros datos, tanto
los almacenados en los discos como los guardados en {\it backups} (esto
evidentemente no hay seguro que lo cubra). Como podemos ver, resulta 
recomendable guardar las copias de seguridad en una zona alejada de la sala de
operaciones, aunque en este caso descentralizemos la seguridad y tengamos que
proteger el lugar donde almacenamos los {\it backups} igual que protegemos la
propia sala o los equipos situados en ella, algo que en ocasiones puede
resultar caro.\\
\\Tambi\'en suele ser com\'un etiquetar las cintas donde hacemos copias de 
seguridad con abundante informaci\'on sobre su contenido (sistemas de ficheros
almacenados, d\'{\i}a y hora de la realizaci\'on, sistema al que 
corresponde\ldots); esto tiene una parte positiva y una negativa. Por un lado,
recuperar un fichero es r\'apido: s\'olo tenemos que ir leyendo las etiquetas
hasta encontrar la cinta adecuada. Sin embargo, si nos paramos a pensar, igual
que para un administrador es f\'acil encontrar el {\it backup} deseado tambi\'en
lo es para un intruso que consiga acceso a las cintas, por lo que si el acceso
a las mismas no est\'a bien restringido un atacante lo tiene f\'acil para 
sustraer una cinta con toda nuestra informaci\'on; no necesita saltarse nuestro
cortafuegos, conseguir una clave del sistema o chantajear a un operador: 
nosotros mismos le estamos poniendo en bandeja toda nuestros datos. No obstante,
ahora nos debemos plantear la duda habitual: {\it si no etiqueto las copias de 
seguridad,
>c\'omo puedo elegir la que debo restaurar en un momento dado?} Evidentemente,
se necesita cierta informaci\'on en cada cinta para poder clasificarlas, pero
esa informaci\'on {\bf nunca} debe ser algo que le facilite la tarea a un 
atacante; por ejemplo, se puede dise\~nar cierta codificaci\'on que s\'olo 
conozcan las personas responsables de las copias de seguridad, de forma que 
cada cinta vaya convenientemente etiquetada, pero sin conocer el c\'odigo sea
dif\'{\i}cil imaginar su contenido. Aunque en un caso extremo el atacante puede
llevarse todos nuestros {\it backups} para analizarlos uno a uno, siempre es
m\'as dif\'{\i}cil disimular una carretilla llena de cintas de 8mm que una
peque\~na unidad guardada en un bolsillo. Y si a\'un pensamos que alguien puede
sustraer todas las copias, simplemente tenemos que realizar {\it backups} 
cifrados\ldots y controlar m\'as el acceso al lugar donde las guardamos.
\subsection{Otros elementos}
En muchas ocasiones los responsables de seguridad de los sistemas tienen muy
presente que la informaci\'on a proteger se encuentra en los equipos, en las
copias de seguridad o circulando por la red (y por lo tanto toman medidas para
salvaguardar estos medios), pero olvidan que esa informaci\'on tambi\'en puede
encontrarse en lugares menos obvios, como listados de impresora, facturas
telef\'onicas o la propia documentaci\'on de una m\'aquina.\\
\\Imaginemos una situaci\'on muy t\'{\i}pica en los sistemas Unix: un usuario,
desde su terminal o el equipo de su despacho, imprime en el servidor un 
documento de cien p\'aginas, documento que ya de entrada ning\'un operador 
comprueba -- y quiz\'as no pueda comprobar, ya que se puede comprometer la
privacidad del usuario -- pero que puede contener, disimuladamente, una copia
de nuestro fichero de contrase\~nas. Cuando la impresi\'on finaliza, el
administrador lleva el documento fuera de la sala de operaciones, pone como 
portada una hoja con los datos del usuario en la m\'aquina ({\it login} 
perfectamente visible, nombre del fichero, hora en que se lanz\'o\ldots) y lo
deja, junto a los documentos que otros usuarios han imprimido -- y con los que
se ha seguido la misma pol\'{\i}tica -- en una estanter\'{\i}a perdida en un
pasillo, lugar al que cualquier persona puede acceder con total libertad y
llevarse la impresi\'on, leerla o simplemente curiosear las portadas de todos
los documentos. As\'{\i}, de repente, a nadie se le escapan bastante problemas
de seguridad derivados de esta pol\'{\i}tica: sin entrar en lo que un usuario
pueda imprimir -- que repetimos, quiz\'as no sea legal, o al menos \'etico,
curiosear --, cualquiera puede robar una copia de un proyecto o un 
examen\footnote{Evidentemente, si alguien imprime un examen de esta forma, no
tenemos un problema con nuestra pol\'{\i}tica sino con nuestros usuarios.}, 
obtener informaci\'on sobre nuestros sistemas de ficheros y las horas a las
que los usuarios suelen trabajar, o simplemente descubrir, simplemente pasando
por delante de la estanter\'{\i}a, diez o veinte nombres v\'alidos de usuario
en nuestras m\'aquinas; todas estas informaciones pueden ser de gran utilidad
para un atacante, que por si fuera poco no tiene que hacer nada para 
obtenerlas, simplemente darse un paseo por el lugar donde depositamos las
impresiones. Esto, que a muchos les puede parecer una exageraci\'on, no es ni
m\'as ni menos la pol\'{\i}tica que se sigue en muchas organizaciones hoy en
d\'{\i}a, e incluso en centros de proceso de datos, donde {\it a priori} ha de
haber una mayor concienciaci\'on por la seguridad inform\'atica.\\
\\Evidentemente, hay que tomar medidas contra estos problemas. En primer lugar,
las impresoras, {\it plotters}, faxes, teletipos, o cualquier dispositivo por
el que pueda salir informaci\'on de nuestro sistema ha de estar situado en un
lugar de acceso restringido; tambi\'en es conveniente que sea de acceso 
restringido el lugar donde los usuarios recogen los documentos que lanzan a 
estos dispositivos. Ser\'{\i}a conveniente que un usuario que recoge una copia
se acredite como alguien autorizado a hacerlo, aunque quiz\'as esto puede ser
imposible, o al menos muy dif\'{\i}cil, en grandes sistemas (imaginemos que 
en una m\'aquina con cinco mil usuarios obligamos a todo aqu\'el que va a 
recoger una impresi\'on a identificarse y comprobamos que la identificaci\'on
es correcta antes de darle su documento\ldots con toda seguridad 
necesitar\'{\i}amos una persona encargada exclusivamente de este trabajo), 
siempre es conveniente demostrar cierto grado de inter\'es por el destino de
lo que sale por nuestra impresora: sin llegar a realizar un control f\'erreo,
si un atacante sabe que el acceso a los documentos est\'a m\'{\i}nimamente
controlado se lo pensar\'a dos veces antes de intentar conseguir algo que otro
usuario ha imprimido.\\
\\Elementos que tambi\'en pueden ser aprovechados por un atacante para 
comprometer nuestra seguridad son todos aquellos que revelen informaci\'on de
nuestros sistemas o del personal que los utiliza, como ciertos manuales 
(proporcionan versiones de los sistemas operativos utilizados), facturas de 
tel\'efono del centro (pueden indicar los n\'umeros de nuestros m\'odems) o
agendas de operadores (revelan los tel\'efonos de varios usuarios, algo muy
provechoso para alguien que intente efectuar ingenier\'{\i}a social contra
ellos). Aunque es conveniente no destruir ni dejar a la vista de todo el mundo
esta informaci\'on, si queremos eliminarla no podemos limitarnos a arrojar
documentos a la papelera: en el cap\'{\i}tulo siguiente hablaremos del 
basureo, algo que aunque parezca sacado de pel\'{\i}culas de esp\'{\i}as 
realmente se utiliza contra todo tipo de entornos. Es
recomendable utilizar una trituradora de papel, dispositivo que dificulta
much\'{\i}simo la reconstrucci\'on y lectura de un documento destruido; por
poco dinero podemos conseguir uno de estos aparatos, que suele ser suficiente
para acabar con cantidades moderadas de papel.
\section{Radiaciones electromagn\'eticas}
Dentro del apartado \ref{eavesdropping} 
pod\'{\i}amos haber hablado del acceso no autorizado a los datos a trav\'es de
las radiaciones que el {\it hardware} emite; sin embargo, este es un tema que ha
cobrado especial importancia (especialmente en organismos militares) a 
ra\'{\i}z del programa {\sc tempest}, un t\'ermino ({\it Transient 
ElectroMagnetic Pulse Emanation STandard}) que identifica una serie de 
est\'andares del gobierno estadounidense para limitar las radiaciones 
el\'ectricas y electromagn\'eticas del equipamiento electr\'onico, desde 
estaciones de trabajo hasta cables de red, pasando por terminales, {\it 
mainframes}, ratones\ldots\\
\\La idea es sencilla: la corriente que circula por un conductor provoca un
campo electromagn\'etico alrededor del conductor, campo que var\'{\i}a de la
misma forma que lo hace la intensidad de la corriente. Si situamos otro 
conductor en ese campo, sobre \'el se induce una se\~nal que tambi\'en 
var\'{\i}a proporcionalmente a la intensidad de la corriente inicial; de esta
forma, cualquier dispositivo electr\'onico (no s\'olo el inform\'atico) emite 
cont\'{\i}nuamente radiaciones a trav\'es del aire o de conductores, 
radiaciones que con el equipo adecuado se pueden captar y reproducir 
remotamente con la consiguiente amenaza a la seguridad que esto implica. 
Conscientes de este problema -- obviamente las emisiones de una batidora no son
peligrosas para la seguridad, pero s\'{\i} que lo pueden ser las de un 
dipositivo de cifrado o las de un teclado desde el que se env\'{\i}en mensajes
confidenciales -- en la d\'ecada de los 50 el gobierno de Estados Unidos 
introdujo una serie de est\'andares para reducir estas radiaciones en los 
equipos destinados a almacenar, procesar o transmitir informaci\'on que 
pudiera comprometer la seguridad nacional. De esta forma, el {\it hardware} 
certificado {\sc tempest} se suele usar con la informaci\'on clasificada y 
confidencial de algunos sistemas gubernamentales para asegurar que el {\it 
eavesdropping} electromagn\'etico no va a afectar a privacidad de los datos.\\
\\Casi medio siglo despu\'es de las primeras investigaciones sobre emanaciones
de este tipo, casi todos los paises desarrollados y organizaciones militares
internacionales tienen programas similares a {\sc tempest} con el mismo fin:
proteger informaci\'on confidencial. Para los gobiernos, esto es algo reservado 
a informaciones militares, nunca a organizaciones `normales' y mucho menos a 
particulares (la NRO, {\it National Reconnaissance Office}, elimin\'o en 1992
los est\'andares {\sc tempest} para dispositivos de uso dom\'estico); sin 
embargo, y como ejemplo -- algo extremo quiz\'as -- de hasta que punto 
un potencial atacante puede llegar a comprometer la informaci\'on que circula 
por una red o que se lee en un monitor, vamos a dar aqu\'{\i} unas nociones 
generales sobre el problema de las radiaciones electromagn\'eticas.\\
\\Existen numerosos tipos de se\~nales electromagn\'eticas; sin duda las m\'as 
peligrosas son las de video y las de transmisi\'on serie, ya que por sus 
caracter\'{\i}sticas no es dif\'{\i}cil interceptarlas con el equipamiento 
adecuado (\cite{kn:eck85} y \cite{kn:smu90}). Otras se\~nales que {\it a 
priori} tambi\'en son f\'aciles de captar, como las de enlaces por 
radiofrecuencia o las de redes basadas en infrarrojos, no presentan tantos 
problemas ya que desde un principio los dise\~nadores fueron conscientes de la 
facilidad de captaci\'on y las amenazas a la seguridad que una captura implica;
esta inseguridad tan palpable provoc\'o la r\'apida aparici\'on de mecanismos 
implementados para dificultar el trabajo de un atacante, como el salto en 
frecuencias o el espectro disperso (\cite{kn:koh95}), o simplemente el uso de 
protocolos cifrados. Este tipo de emisiones quedan fuera del alcance de {\sc 
tempest}, pero son cubiertas por otro est\'andar denominado {\sc nonstop}, 
tambi\'en del Departamento de Defensa estadounidense.\\
\\Sin embargo, nadie suele tomar precauciones contra la radiaci\'on que emite 
su monitor, su impresora o el cable de su m\'odem. Y son justamente las
radiaciones de este {\it hardware} desprotegido las m\'as
preocupantes en ciertos entornos, ya que lo \'unico que un atacante necesita
para recuperarlas es el equipo adecuado. Dicho equipo puede variar desde 
esquemas extremadamente simples y baratos -- pero efectivos -- 
(\cite{kn:hig88}) hasta complejos sistemas que en teor\'{\i}a utilizan los 
servicios de inteligencia de algunos pa\'{\i}ses. La empresa {\it 
Consumertronics} ({\tt www.tsc-global.com}) fabrica y vende diversos 
dispositivos de monitorizaci\'on, entre ellos el basado en \cite{kn:eck85}, que
se puede considerar uno de los pioneros en el mundo civil.\\
\\Pero, >c\'omo podemos protegernos contra el {\it eavesdropping} de las
radiaciones electromagn\'eticas de nuestro {\it hardware}? Existe un amplio 
abanico de soluciones, desde simples medidas de prevenci\'on hasta complejos 
-- y caros -- sistemas para apantallar los equipos. La soluci\'on m\'as barata
y simple que podemos aplicar es la {\bf distancia}: las se\~nales que se 
transmiten por el espacio son atenuadas conforme aumenta la separaci\'on de la
fuente, por lo que si definimos un per\'{\i}metro f\'{\i}sico de seguridad 
lo suficientemente grande alrededor de una m\'aquina, ser\'a dif\'{\i}cil para 
un atacante interceptar desde lejos nuestras emisiones. No obstante, esto no
es aplicable a las se\~nales inducidas a trav\'es de conductores, que aunque
tambi\'en se atenuan por la resistencia e inductancia del cableado, la p\'erdida
no es la suficiente para considerar seguro el sistema.\\ 
\\Otra soluci\'on consiste en la {\bf confusi\'on}: cuantas m\'as se\~nales
existan en el mismo medio, m\'as dif\'{\i}cil ser\'a para un atacante filtrar
la que est\'a buscando; aunque esta medida no hace imposible la 
interceptaci\'on, s\'{\i} que la dificulta enormemente. Esto se puede conseguir
simplemente manteniendo diversas piezas emisoras (monitores, terminales, 
cables\ldots) cercanos entre s\'{\i} y emitiendo cada una de ellas informaci\'on
diferente (si todas emiten la misma, facilitamos el ataque ya que aumentamos
la intensidad de la se\~nal inducida). Tambi\'en existe {\it hardware} 
dise\~nado expl\'{\i}citamente para crear ruido electromagn\'etico, generalmente
a trav\'es de se\~nales de radio que enmascaran las radiaciones emitidas por
el equipo a proteger; dependiendo de las frecuencias utilizadas, quiz\'as el
uso de tales dispositivos pueda ser ilegal: en todos los paises el
espectro electromagn\'etico est\'a dividido en bandas, cada una de las cuales
se asigna a un determinado uso, y en muchas de ellas se necesita una licencia
especial para poder transmitir. En Espa\~na estas licencias son otorgadas por
la Secretar\'{\i}a General de Comunicaciones, dependiente del Ministerio de
Fomento.\\
\\Por \'ultimo, la soluci\'on m\'as efectiva, y m\'as cara, consiste en el
uso de dispositivos certificados que aseguran m\'{\i}nima emisi\'on, as\'{\i}
como de instalaciones que apantallan las radiaciones. En el {\it hardware} hay
dos aproximaciones principales para prevenir las emisiones: una es la 
utilizaci\'on de circuitos especiales que apenas emiten radiaci\'on (denominados
de fuente eliminada, {\it source suppressed}), y la otra es la contenci\'on de
las radiaciones, por ejemplo aumentando la atenuaci\'on; generalmente ambas
aproximaciones se aplican conjuntamente (\cite{kn:swi92}). En cuanto a las
instalaciones utilizadas para prevenir el {\it eavesdropping}, la idea general
es aplicar la contenci\'on no s\'olo a ciertos dispositivos, sino a un 
edificio o a una sala completa. Quiz\'as la soluci\'on m\'as utilizada son las 
jaulas de Faraday sobre lugares donde se trabaja con informaci\'on sensible; se
trata de separar el espacio en dos zonas electromagn\'eticamente aisladas
(por ejemplo, una sala y el resto del espacio) de forma que fuera de una zona
no se puedan captar las emisiones que se producen en su interior. Para 
implementar esta soluci\'on se utilizan materiales especiales, como algunas 
clases de cristal, o simplemente un recubrimiento conductor conectado a 
tierra.\\
\\Antes de finalizar este punto quiz\'as es recomendable volver a insistir en
que todos
los problemas y soluciones derivados de las radiaciones electromagn\'eticas no 
son aplicables a los entornos o empresas normales, sino que est\'an pensados 
para lugares donde se trabaja con informaci\'on altamente confidencial, como 
ciertas empresas u organismos militares o de inteligencia. Aqu\'{\i} simplemente
se han presentado como una introducci\'on para mostrar hasta donde puede llegar
la preocupaci\'on por la seguridad en esos lugares. La radiaci\'on 
electromagn\'etica {\bf no} es un riesgo importante en la mayor\'{\i}a de 
organizaciones ya que suele tratarse de un ataque costoso en tiempo y dinero,
de forma que un atacante suele tener muchas otras puertas para intentar 
comprometer el sistema de una forma m\'as f\'acil.
