\chapter{Algunos servicios y protocolos}
\section{Introducci\'on}
En este cap\'{\i}tulo vamos a hablar de la seguridad (e inseguridad) de algunos
de los protocolos, servicios y programas que los implementan en los entornos
Unix. No vamos a entrar en detalles sobre el funcionamiento de cada uno de 
ellos, ya que ese ser\'{\i}a un trabajo que exceder\'{\i}a los objetivos de este
proyecto; para m\'as referencias se puede consultar \cite{kn:ste90} (detalles
de la implementaci\'on interna de algunos servicios) o \cite{kn:ste94}.\\
\\Podemos ver los diferentes servicios que un sistema Unix ofrece como 
potenciales puertas de entrada al mismo, o al menos como fuentes de ataques
que ni siquiera tienen por qu\'e proporcionar acceso a la m\'aquina -- como las
negaciones de servicio --. De esta forma, si cada servicio ofrecido es un 
posible problema para nuestra seguridad, parece claro que lo ideal ser\'{\i}a
no ofrecer ninguno, poseer una m\'aquina completamente aislada del resto; 
evidentemente, esto no suele ser posible hoy en d\'{\i}a en la mayor parte de
los sistemas\footnote{No obstante, algunos equipos que no necesitan estar 
conectados entre s\'{\i}, lo est\'an; cada administrador deber\'{\i}a 
preguntarse si realmente necesita sus m\'aquinas conectadas a la red.}. Por 
tanto, ya que es necesaria la conectividad entre equipos, hemos de ofrecer {\bf 
los m\'{\i}nimos servicios necesarios} para que todo funcione correctamente;
esto choca frontalmente con las pol\'{\i}ticas de la mayor\'{\i}a de 
fabricantes de sistemas Unix, que por defecto mantienen la mayor\'{\i}a de
servicios abiertos al instalar un equipo nuevo: es responsabilidad del 
administrador preocuparse de cerrar los que no sean estrictamente necesarios.\\
\\T\'{\i}picos ejemplos de servicios que suele ser necesario ofrecer son {\it
telnet} o {\it ftp}; en estos casos no se puede aplicar el esquema todo o nada
que vimos al estudiar el sistema de red de Unix, donde o bien ofrec\'{\i}amos
un servicio o lo deneg\'abamos completamente: es necesaria una correcta 
configuraci\'on para
que s\'olo sea posible acceder a ellos desde ciertas m\'aquinas, como veremos
al hablar de {\it TCP Wrappers}. Tambi\'en es una buena idea sustituir estos
servicios por equivalentes cifrados, como la familia de aplicaciones {\sc ssh},
y concienciar a los usuarios para que utilicen estos equivalentes: hemos de
recordar siempre -- y recordar a los usuarios -- que cualquier conexi\'on en 
texto claro entre dos sistemas puede ser f\'acilmente capturada por cualquier
persona situada en una m\'aquina intermedia, con lo simplemente utilizando
{\tt telnet} estamos poniendo en juego la seguridad de sistemas y redes
completas.\\
\\Aparte de puertas de entrada, los servicios ofrecidos tambi\'en son muy
susceptibles de ataques de negaci\'on de servicio ({\it DoS}), por ejemplo por
demasiadas conexiones abiertas simult\'aneamente en una misma m\'aquina; incluso
es posible que uno de estos ataques contra cierto servicio inutilice 
completamente a {\tt inetd}, de forma que todos los ofrecidos desde \'el quedan
bloqueados hasta que el demonio se reinicia. Este problema incluso puede ser
muy grave: imaginemos que -- por cualquier motivo -- {\tt inetd} deja de 
responder peticiones; si esto sucede es posible que ni siquiera podamos acceder
a la m\'aquina remotamente para solucionar el problema (por ejemplo {\tt 
telnet} o incluso {\sc ssh} si lo servimos deste {\tt inetd} dejar\'{\i}an de 
funcionar). Para evitar este problema, muchos administradores planifican una
tarea que se ejecute cada pocos minutos mediante {\tt cron}, y que simplemente
env\'{\i}e la se\~nal {\sc sighup} a {\tt inetd}, por ejemplo a\~nadiendo esta
entrada a su fichero {\it crontab}\footnote{Es recomendable consultar la
sintaxis de estos ficheros en el clon de Unix en que trabajemos, ya que puede
variar entre diferentes Unices.}:
\begin{quote}
\begin{verbatim}
* * * * *           killall -HUP inetd 
\end{verbatim}
\end{quote}
Si en nuestro clon de Unix no disponemos de una \'orden para enviar se\~nales
a los procesos en funci\'on de su nombre (como {\tt pkill} en Solaris o {\tt
killall} en Linux o IRIX) podemos utilizar un poco de programaci\'on {\it
shellscript} para conseguirlo:
\begin{quote}
\begin{verbatim}
* * * * *     kill -HUP `ps -auxw|grep inetd|grep -v grep|awk '{print $2}'`
\end{verbatim}
\end{quote}
\section{Servicios b\'asicos de red}
Dentro de este apartado vamos a comentar brevemente la funci\'on de algunos
servicios de Unix y sus potenciales problemas de seguridad. Los aqu\'{\i} 
expuestos son servicios que habitualmente han de estar {\bf cerrados}, por lo
que no implican excesivos problemas de seguridad conocidos. As\'{\i}, 
no vamos a entrar en muchos detalles con ellos; en puntos siguientes 
hablaremos con m\'as extensi\'on de otros servicios que suelen estar ofrecidos
en todas las m\'aquinas, como {\it ftp}, {\it telnet} o {\sc smtp}, y que 
en su mayor\'{\i}a presentan mayores problemas de seguridad.
\subsection{systat}
El servicio {\it systat} se asocia al puerto 11 de una m\'aquina Unix, de forma
que al recibir una petici\'on mediante {\sc tcp} el demonio {\tt inetd} 
ofrece una imagen de la tabla de procesos del sistema, por ejemplo ejecutando
una orden como {\tt ps -auwwx} en Linux o {\tt ps -ef} en Solaris; en algunos 
Unices se ofrece la salida de \'ordenes
como {\tt who} o {\tt w} en lugar de la tabla de procesos: es f\'acil 
configurar lo que cada administrador desee mostrar simplemente modificando
la l\'{\i}nea correspondiente de {\tt /etc/inetd.conf}:
\begin{quote}
\begin{verbatim}
anita:~# grep systat /etc/inetd.conf
systat stream  tcp     nowait  root    /usr/bin/ps             ps -ef
anita:~# 
\end{verbatim}
\end{quote}
Bien se ofrezca la tabla de procesos o bien otro tipo de informaci\'on sobre
el sistema, este servicio es habitual encontarlo {\bf deshabilitado}, ya que
cualquier dato sobre nuestro sistema (especialmente procesos, nombres de 
usuario, m\'aquinas desde las que conectan\ldots) puede ser aprovechado por un
pirata para atacar el equipo. Si por motivos de comodidad a la hora de 
administrar varios {\it hosts} dentro de una red local necesitamos tener 
abierto {\it systat}, debemos restringir las direcciones desde las que se puede 
acceder al servicio mediante {\it TCP Wrappers}.
\subsection{daytime}
El servicio {\it daytime}, asociado al puerto 13, tanto {\sc tcp} como {\sc
udp}, es un servicio interno de {\tt inetd} (esto es, no hay un programa 
externo que lo sirva, el propio {\tt inetd} se encarga de ello); al recibir
una conex\'on a este puerto, el sistema mostrar\'a la fecha y la hora, en un
formato muy similar al resultado de la orden {\tt date}:
\begin{quote}
\begin{verbatim}
anita:~# telnet rosita daytime
Trying 192.168.0.1...
Connected to rosita.
Escape character is '^]'.
Thu Apr 20 05:02:33 2000
Connection closed by foreign host.
anita:~#
\end{verbatim}
\end{quote}
Aunque a primera vista este servicio no represente un peligro para la integridad
de nuestro sistema, siempre hemos de recordar una norma de seguridad 
fundamental: s\'olo hay que ofrecer los servicios estrictamente necesarios para 
el correcto funcionamiento de nuestras m\'aquinas. Como {\it daytime} no es un
servicio b\'asico, suele ser recomendable cerrarlo; adem\'as, la informaci\'on
que proporciona, aunque escasa, puede ser suficiente para un atacante: le 
estamos indicando el estado del reloj de nuestro sistema, lo que por ejemplo
le da una idea de la ubicaci\'on geogr\'afica del equipo.\\
\\Un servicio parecido en muchos aspectos a {\it daytime} es {\it time} (puerto
37, {\sc tcp} y {\sc udp}); tambi\'en indica la fecha y hora del equipo, pero
esta vez en un formato que no es inteligible para las personas:
\begin{quote}
\begin{verbatim}
anita:~# telnet rosita time
Trying 192.168.0.1...
Connected to rosita.
Escape character is '^]'.
['^Connection closed by foreign host.
anita:~#
\end{verbatim}
\end{quote}
Este servicio suele ser m\'as \'util que el anterior: aunque una persona no 
entienda la informaci\'on mostrada por {\it time}, s\'{\i} que lo hace una
m\'aquina Unix. De esta forma, se utiliza {\it time} en un servidor para que
las estaciones cliente puedan sincronizar sus relojes con \'el con \'ordenes
como {\tt netdate} o {\tt rdate}:
\begin{quote}
\begin{verbatim}
luisa:~# date
Thu Apr 20 02:19:15 CEST 2000
luisa:~# rdate rosita
[rosita] Thu Apr 20 05:10:49 2000
luisa:~# date
Thu Apr 20 02:20:02 CEST 2000
luisa:~# rdate -s rosita
luisa:~# date
Thu Apr 20 05:11:59 2000
luisa:~#
\end{verbatim}
\end{quote}
Los problemas de {\it time} son en principio los mismos que los de {\it 
daytime}; aunque tambi\'en es recomendable mantener este servicio cerrado, es
m\'as f\'acil imaginar situaciones en las que un administrador desee ofrecer
{\it time} en varias m\'aquinas que imaginar la necesidad de ofrecer {\it
daytime}.
\subsection{netstat}
De la misma forma que {\it systat} ofrec\'{\i}a informaci\'on sobre el estado
de nuestro sistema, {\it netstat} la ofrece sobre el estado de nuestra red. 
Este servicio, asociado al puerto 15 con protocolo {\sc tcp}, ejecuta una orden
como {\tt netstat} (con argumentos que dependen del clon de Unix utilizado)
para mostar principalmente las conexiones activas en la m\'aquina; por ejemplo,
si en Linux invocamos a {\tt netstat} desde {\tt /etc/inetd.conf} con la 
opci\'on {\tt `-A inet'}, al recibir una conexi\'on se mostrar\'a algo parecido 
a lo siguiente:
\begin{quote}
\begin{verbatim}
anita:~# telnet rosita netstat
Trying 192.168.0.1...
Connected to rosita.
Escape character is '^]'.
Active Internet connections (w/o servers)
Proto Recv-Q Send-Q Local Address      Foreign Address   State      
tcp        0      0 rosita:netstat     anita:4990        ESTABLISHED 
Connection closed by foreign host.
anita:~#
\end{verbatim}
\end{quote}
Como suced\'{\i}a con {\it systat}, es recomendable {\bf deshabilitar} este
servicio comentando la l\'{\i}nea correspondiente de {\tt /etc/inetd.conf}, o
en todo caso restringir el acceso al mismo a m\'aquinas de nuestra red local,
mediante {\it TCP Wrappers}. La informaci\'on sobre el estado del sistema de
red -- o al menos de parte del mismo -- puede ser muy \'util para un atacante,
ya que por ejemplo le est\'a mostrando nombres de {\it hosts} y adem\'as le
permite hacerse una idea del tr\'afico que soporta la m\'aquina, de los 
servicios que ofrece, de los h\'abitos de conexi\'on de los usuarios\ldots
\subsection{chargen}
{\it chargen} (puerto 19, {\sc tcp} y {\sc udp}) es un generador de caracteres
servido internamente por {\tt inetd}, que se utiliza sobre todo para comprobar 
el estado de las conexiones en la red; cuando alguien accede a este servicio
simplemente ve en su terminal una secuencia de caracteres ASCII que se repite
indefinidamente.\\
\\Los posibles problemas de seguridad relacionados con {\it chargen} suelen ser
negaciones de servicio, tanto para la parte cliente como para la servidora. Sin
duda el ejemplo m\'as famoso de utilizaci\'on de {\it chargen} es una de las
an\'ecdotas del experto en seguridad Tsutomu Shimomura (el principal 
contribuidor en la captura de Kevin Mitnick, el pirata m\'as famoso de los 
noventa): cuando conectaba a un servidor de {\it ftp} an\'onimo, Shimomura se 
di\'o cuenta de que la m\'aquina lanzaba un {\it finger} contra el cliente que 
realizaba la conexi\'on. Esto no le gust\'o, y decidi\'o comprobar si ese
sistema utilizaba el {\it finger} habitual; para ello modific\'o el fichero 
{\tt /etc/inetd.conf} de su sistema de forma que las peticiones {\it finger}
se redireccionaran al generador de caracteres {\it chargen}. Conect\'o al 
servidor de nuevo, y al hacer \'este otro {\it finger}, la m\'aquina de 
Shimomura se dedic\'o a enviar {\it megas} y {\it megas} de caracteres ({\it
chargen} no finaliza hasta que el cliente corta la conexi\'on); en unas pocas
horas el sistema remoto qued\'o inoperativo, y a la ma\~nana siguiente ese {\it 
finger}
autom\'atico hab\'{\i}a sido eliminado de la configuraci\'on del servidor.
Ese servidor no habr\'{\i}a sufrido una ca\'{\i}da si hubiera utilizado {\tt 
safe$\_$finger}, un programa de Wietse 
Venema que se distribuye junto a {\it TCP Wrappers} y que limita la potencial
cantidad de informaci\'on que {\it finger} puede recibir.
\subsection{tftp}
{\it tftp} ({\it Trivial File Transfer Protocol}) es un protocolo de 
transferencia de ficheros asociado al puerto 69 y basado en {\sc udp} que no 
proporciona {\bf ninguna seguridad}. Por tanto en la mayor\'{\i}a de sistemas
es obligatorio que este servicio est\'e desactivado; su uso principal es el
arranque de estaciones {\it diskless} o de {\it routers} a trav\'es de la red, 
ya que la simpleza
del protocolo permite implementarlo en un {\it chip}, y s\'olo en ese caso 
nos veremos obligados a ofrecer el servicio. Si es este el caso, los ficheros
que deseemos que sean p\'ublicos se han de situar en un determinado directorio 
(dependiendo del clon de Unix, {\tt /tftpboot/}, {\tt /etc/tftpboot/}, {\tt
/usr/local/boot/}\ldots) o utilizar otros nombres de directorio como argumentos 
del demonio en {\tt /etc/inetd.conf}, algo 
no recomendable. Por ejemplo, si en {\tt /tftpboot/} guardamos una copia de
la imagen del {\it kernel}, los clientes podr\'an acceder a ella mediante la
orden {\tt tftp}:
\begin{quote}
\begin{verbatim}
luisa:~# tftp rosita
tftp> get vmlinuz
Received 531845 bytes in 3.4 seconds
tftp> quit
luisa:~# 
\end{verbatim}
\end{quote}
Podemos ver que en ning\'un momento se solicita un nombre de usuario o una
clave, lo que nos da una idea de los graves problemas de seguridad que el 
ofrecer este servicio puede implicarnos. Hasta hace unos a\~nos, era normal
que los fabricantes de sistemas Unix vendieran sus productos con {\it tftp}
abierto y sin configurar, con lo que un pirata lo ten\'{\i}a muy f\'acil para
conseguir cualquier fichero de contrase\~nas:
\begin{quote}
\begin{verbatim}
luisa:~# tftp victima
tftp> get /etc/passwd /tmp/salida
Received 1845 bytes in 0.6 seconds
tftp> quit
luisa:~#
\end{verbatim}
\end{quote}

\subsection{finger}
T\'{\i}picamente el servicio {\it finger} (puerto 79, {\sc tcp}) ha sido una de
las principales fuentes de problemas de Unix. Este protocolo proporciona
informaci\'on -- demasiado detallada -- de los usuarios de una m\'aquina, 
est\'en o no conectados en el momento de acceder al servicio; para hacerlo, se
utiliza la aplicaci\'on {\tt finger} desde un cliente, d\'andole como argumento
un nombre de m\'aquina precedido del s\'{\i}mbolo {\tt `@'} y, opcionalmente,
de un nombre de usuario ({\tt finger} sobre el sistema local no utiliza el
servicio de red, por lo que no lo vamos a comentar aqu\'{\i}). En el primer
caso, {\tt finger} nos dar\'a datos generales de los usuarios conectados en 
ese momento a la m\'aquina, y en el segundo nos informar\'a con m\'as detalle
del usuario especificado como par\'ametro, est\'e o no conectado:
\begin{quote}
\begin{verbatim}
anita:~# finger @rosita
[rosita]
Login     Name               Tty   Idle  Login Time   Office     Office Phone
toni      Toni at ROSITA    */0      28  Apr 20 04:43 (anita)
root      El Spiritu Santo    1      12  Apr 11 02:10 
anita:~# finger toni@rosita
[rosita]
Login: toni                             Name: Toni at ROSITA
Directory: /home/toni                   Shell: /bin/bash
On since Thu Apr 20 04:43 (CEST) on pts/0 from anita
   30 minutes 28 seconds idle
     (messages off)
No mail.
No Plan.
anita:~#
\end{verbatim}
\end{quote}
Como podemos ver, {\it finger} est\'a proporcionando mucha informaci\'on que
podr\'{\i}a ser de utilidad para un atacante: nombres de usuario, h\'abitos de
conexi\'on, cuentas inactivas\ldots incluso algunas organizaciones rellenan
exhaustivamente el campo {\it gecos} del fichero de contrase\~nas, con datos
como n\'umeros de habitaci\'on de los usuarios o incluso su tel\'efono. Est\'a
claro que esto es f\'acilmente aprovechable por un pirata para practicar
ingenier\'{\i}a social contra nuestros usuarios -- o contra el propio 
administrador --. Es {\bf b\'asico} para la integridad de nuestras m\'aquinas
{\bf deshabilitar} este servicio, restringir su acceso a unos cuantos equipos
de la red local mediante {\it TCP Wrappers} o utilizar versiones del demonio
{\tt fingerd} como {\tt ph} ({\it Phone Book}), que permiten especificar la
informaci\'on que se muestra al acceder al servicio desde cada m\'aquina.
\subsection{POP}
El servicio {\sc pop} ({\it Post Office Protocol}, puertos 109 y 110 en {\sc 
tcp}) se
utiliza para que los usuarios puedan acceder a su correo sin necesidad de 
montar sistemas de ficheros compartidos mediante {\sc nfs}: los clientes
utilizan {\sc smtp} para enviar correo y {\sc pop} para recogerlo del servidor,
de forma que el procesamiento se realice en la m\'aquina del usuario. Se trata 
de un 
servicio que podr\'{\i}amos considerar peligroso, por lo que -- como el resto,
pero este especialmente -- debemos {\bf deshabilitarlo} a no ser que sea 
estrictamente necesario ofrecerlo; en ese caso debemos restringir al m\'aximo 
los lugares desde los que se puede acceder, mediante {\it TCP Wrappers}.\\
\\En algunos sistemas se utiliza {\sc pop} simplemente para evitar otorgar 
cuentas completas a los usuarios: si s\'olo van a utilizar la m\'aquina 
para leer su correo, >por qu\'e ofrecerles un {\it shell} `completo', con
acceso a todo el sistema? Realmente esto es cierto (ser\'{\i}a un error 
permitir ejecutar ciertas \'ordenes a aquellos que s\'olo utilizar\'an el equipo
para gestionar su correo), pero en muchas ocasiones esta soluci\'on no es del
todo conveniente: aparte de los peligros que implica un servicio adicional, que
de otra forma no utilizar\'{\i}amos -- en algunos demonios de {\sc pop} han
surgido {\it bugs} que incluso otorgaban un privilegio de {\it root} remoto 
sin necesidad de ninguna clave --, estamos generando un tr\'ansito peligroso
de contrase\~nas a trav\'es de la red. {\sc pop} ofrece tres modelos distintos
de autenticaci\'on: uno basado en {\it Kerberos}, apenas utilizado, otro basado
en un protocolo desaf\'{\i}o--respuesta ({\sc apop}, que tampoco se suele
utilizar), y otro basado en un simple nombre de usuario con su {\it password} 
correspondiente. Este \'ultimo, el m\'as usado en todo tipo de entornos, es
un excelente objetivo para un pirata con un {\it sniffer}: los usuarios suelen
configurar sus clientes para que chequeen el buz\'on de correo cada pocos
minutos, con lo que a intervalos muy cortos env\'{\i}an su clave a un puerto 
conocido de una m\'aquina conocida; al realizar toda esta comunicaci\'on en 
texto claro, un atacante no tiene m\'as que interceptar la sesi\'on {\sc pop} 
para averiguar nombres de usuario y claves (aparte de poder leer el correo que 
baja del servidor al cliente). Si lo que deseamos es que nuestros usuarios no
disfruten de una cuenta completa simplemente para gestionar su correo, podemos
sustituir su {\it shell} en {\tt /etc/passwd} por el nombre de dicho lector:
\begin{quote}
\begin{verbatim}
ircd:x:1001:100:Gestion IRC,,,:/home/ircd:/usr/bin/pine
\end{verbatim}
\end{quote}
En este caso hemos de tomar una precauci\'on adicional: la mayor\'{\i}a de 
programas de correo ({\tt elm}, {\tt pine}\ldots) permiten escapes al {\it
shell}, procedimientos que tarde o temprano ejecutan con \'exito un
int\'erprete de \'ordenes; por ejemplo, con {\tt elm} no tenemos m\'as que
iniciar {\tt vi} para escribir un mensaje y en el editor ejecutar {\tt 
:!/bin/sh} para ejecutar este int\'erprete. Para evitar estos escapes o bien
podemos modificar el c\'odigo del gestor de correo -- algo no muy habitual -- o
utilizar ya versiones modificadas disponibles a trav\'es de Internet.

\subsection{auth}
Se llama {\it socket} a la combinaci\'on de una direcci\'on de m\'aquina y un
puerto; esta entidad identifica un proceso \'unico en la red (\cite{kn:bre95}).
Un par de {\it sockets}, uno en la m\'aquina receptora y otro en la emisora
definen una conexi\'on en protocolos como {\sc tcp}; esta conexi\'on tambi\'en
ser\'a \'unica en la red en un instante dado. Como vemos, no entra en juego
ning\'un nombre de usuario: en {\sc tcp/ip} se establecen canales de 
comunicaci\'on entre m\'aquinas, no entre personas; no obstante, en muchas 
ocasiones nos puede interesar conocer el nombre de usuario bajo el que cierta
conexi\'on se inicia. Por ejemplo, de esta forma podr\'{\i}amos ofrecer o 
denegar un servicio en funci\'on del usuario que lo solicita, aparte de la 
m\'aquina desde donde viene la petici\'on.\\
\\El protocolo {\it auth} (puerto 113, {\sc tcp}) viene a solucionar este 
problema con un esquema muy simple: cuando un servidor necesita determinar el
usuario que ha iniciado una conexi\'on contacta con el demonio {\tt identd} y
le env\'{\i}a los datos necesarios para distinguir dicha conexi\'on (los
componentes de los dos {\it sockets} que intervienen) de las
dem\'as. De esta forma, el demonio identifica al usuario en cuesti\'on y
devuelve al servidor informaci\'on sobre dicho usuario, generalmente su {\it 
login}. Por ejemplo, si utilizamos {\it TCP Wrappers} -- un programa servidor
que utiliza este mecanismo para determinar nombres de usuario siempre que sea
posible --, se registar\'a el {\it login} del usuario remoto que solicita un
servicio en nuestra m\'aquina si el sistema remoto tiene habilitado {\it auth}:
\begin{quote}
\begin{verbatim}
luisa:~# tail -2 ~adm/syslog
Apr 24 04:16:19 luisa wu.ftpd[1306]: connect from rosita
Apr 24 04:16:21 luisa ftpd[1306]: ANONYMOUS FTP LOGIN FROM \
           rosita [192.168.0.1], toni@
luisa:~#
\end{verbatim}
\end{quote}
No obstante, si el sistema desde el que esa persona conecta no tiene habilitado
dicho servicio, el nombre de usuario no se va a poder conseguir:
\begin{quote}
\begin{verbatim}
luisa:~# tail -2 ~adm/syslog
Apr 24 04:19:37 luisa wu.ftpd[1331]: connect from root@anita
Apr 24 04:19:39 luisa ftpd[1331]: ANONYMOUS FTP LOGIN FROM \
           root @ anita [192.168.0.3], toni@
luisa:~#
\end{verbatim}
\end{quote}
El servicio {\it auth} {\bf no} se debe utilizar nunca con prop\'ositos de 
autenticaci\'on robusta, ya que dependemos no de nuestros sistemas, sino de la
honestidad de la m\'aquina remota; un atacante con el suficiente nivel de
privilegio en esta puede enviarnos cualquier nombre de usuario que desee. 
Incluso en ciertas situaciones, si {\it ident} no est\'a habilitado ni siquiera
hacen falta privilegios para devolver un nombre falso: cualquier usuario puede
hacerlo. En cambio, s\'{\i} que es \'util para detectar peque\~nas violaciones
de seguridad, por lo que quiz\'as interese
habilitar el servicio en nuestras m\'aquinas (aunque limitemos su uso mediante
{\it TCP Wrappers}.
\subsection{NNTP}
El servicio {\sc nntp} ({\it Network News Transfer Protocol}, puerto 119 {\sc
tcp}) se utiliza para intercambiar mensajes de grupos de noticias entre 
servidores de {\it news}. Los diferentes demonios encargados de esta tarea
(como {\tt in.nntpd} o {\tt innd}) suelen discriminar conexiones en funci\'on 
de la direcci\'on o el nombre de la m\'aquina cliente; por ejemplo, el primero
utiliza el fichero {\tt nntp$\_$access} para decidir si ofrece el servicio de
{\it news} a un determinado {\it host}, y si es as\'{\i} concretar de que forma
puede acceder a \'el (s\'olo lectura, s\'olo ciertos grupos\ldots). De esta 
forma, los servidores {\sc nntp} son muy vulnerables a cualquier ataque que 
permita falsear la identidad de la m\'aquina origen, como el {\it IP 
Spoofing}.\\
\\Los problemas relacionados con las {\it news} no suelen ser excesivamente
graves desde un punto de vista estrictamente t\'ecnico, pero en ocasiones 
s\'{\i} que lo son aplicando una visi\'on global. Por ejemplo, habr\'{\i}a que
evaluar el da\~no que le supone a la imagen de nuestra organizaci\'on el que
un atacante env\'{\i}e mensajes insultantes o pornogr\'aficos utilizando nuestro
nombre o nuestros recursos. Tambi\'en es un problema la mala educaci\'on de los
usuarios en materias de seguridad inform\'atica: tienden a creer todo lo que
leen en ciertos grupos de noticias, por lo que un atacante podr\'{\i}a utilizar
ingenier\'{\i}a social para perjudicar a nuestra organizaci\'on. Otra amenaza
com\'un es el uso de grupos de {\it news} privados (internos) para tratar 
informaci\'on confidencial en la organizaci\'on: esto es un error, ya que si
la privacidad del servidor se ve comprometida un atacante puede obtener datos
que {\it a priori} no estar\'{\i}a autorizado a saber.\\
\\Realmente, es muy poco probable que necesitemos ofrecer este servicio, por lo
que lo m\'as razonable para nuestra seguridad es {\bf deshabilitarlo}. 
Generalmente s\'olo existen servidores de noticias en grandes organizaciones 
-- como las universidades --, y adem\'as lo normal es que s\'olo haya uno por
entidad. Si debemos administrar ese equipo la mejor forma de proteger el 
servicio {\sc nntp} es utilizando un buen cortafuegos (\cite{kn:spa96}).
\subsection{NTP}
{\sc ntp} ({\it Network Time Protocol}, puerto 123 {\sc udp} y {\sc tcp}) es
un protocolo utilizado para sincronizar relojes de m\'aquinas de una forma muy
precisa; a pesar de su sofisticaci\'on no fu\'e dise\~nado con una idea de 
robustez ante ataques, por lo que puede convertirse en una gran fuente de 
problemas (\cite{kn:bis90}) si no est\'a correctamente configurado o si no 
utilizamos versiones
actualizadas de {\tt nntpd}, el demonio que ofrece este servicio.\\
\\Son muchos los problemas de seguridad relacionados con un tiempo correcto; el
m\'as simple y obvio es la poca fiabilidad que ofrecer\'a nuestro sistema de
{\it log} a la hora de determinar cu\'ando sucedi\'o determinado evento: 
aunque se registrara que alguien hizo un {\it telnet} a las tres de la tarde,
no podr\'{\i}amos ni siquiera asegurar que la hora es correcta. Otro problema
t\'{\i}pico radica en las facilidades que ofrece Unix para la planificaci\'on
de tareas: si el reloj tiene problemas, es posible que ciertas tareas no se
lleguen a ejecutar, que se ejecuten varias veces, o que se ejecuten cuando no
han de hacerlo; esto es especialmente peligroso para tareas de las que depende
nuestra seguridad, como la rotaci\'on de {\it logs}. Si hablamos de problemas
m\'as sofisticados, podemos pensar en sistemas distribuidos, en los que una
correcta sincronizaci\'on entre nodos es b\'asica para garantizar el correcto 
funcionamiento del sistema global (\cite{kn:tan95}, \cite{kn:cou94}\ldots); la
sincronizaci\'on es muy importantes en modelos de autenticaci\'on como {\it
Kerberos}, que utiliza marcas de tiempo como pruebas de frescura para evitar
ataques por reenv\'{\i}o.\\
\\Como hemos visto, una correcta sincronizaci\'on del reloj de nuestro equipo 
es vital para la seguridad; no obstante, muy pocos sistemas necesitan la 
precisi\'on de {\sc ntp}, por lo que es habitual tener este servicio 
deshabilitado. En la mayor\'{\i}a de ocasiones el propio reloj de la m\'aquina,
o un protocolo mucho m\'as simple, como {\it time}, es m\'as que suficiente para
sincronizar equipos.
\subsection{UUCP}
{\sc uucp} ({\it Unix to Unix CoPy}, puerto 540 {\sc tcp}) es un servicio que,
como su nombre indica, se utiliza para copiar ficheros entre m\'aquinas Unix,
generalmente a trav\'es de l\'{\i}neas telef\'onicas o redes de baja velocidad; 
aunque hoy en d\'{\i}a apenas se utiliza, durante a\~nos ha sido la base de los 
sistemas de correo electr\'onico y de {\it news} (incluso hoy en d\'{\i}a 
algunos sistemas {\sc uucp} son capaces de transmitir noticias de Usenet m\'as 
eficientemente que la m\'as moderna implementaci\'on de {\sc nntp}).\\
\\Dos riesgos fundamentales amenazan a {\sc uucp}: al tratarse de una 
transmisi\'on en texto claro, un potencial atacante puede tener acceso a 
informaci\'on privada de los usuarios, vulnerando su privacidad. Evidentemente,
en el caso de transmisi\'on de {\it news} esto no es muy relevante, ya que
todos los mensajes son en principio de acceso p\'ublico, pero la cosa cambia
si estamos transmitiendo correo electr\'onico. El segundo riesgo es incluso
m\'as preocupante que la p\'erdida de privacidad: las contrase\~nas de los 
usuarios tambi\'en se transmiten en texto claro, con el consiguiente peligro que
supone la interceptaci\'on por parte de un pirata de dichas claves. Aunque
si utilizamos l\'{\i}neas telef\'onicas la probabilidad de que un {\it sniffer}
capture los datos enviados es menor que si utilizamos una red {\sc tcp}, en
ambos casos el riesgo est\'a presente.\\
\\Como siempre, y dado que como hemos dicho {\sc uucp} no se suele utilizar hoy
en d\'{\i}a, lo m\'as recomendable es {\bf deshabilitar} este servicio; es
m\'as, dado que suele existir un usuario {\tt uucp} en todo sistema Unix (por
motivos simplemente de compatibilidad), hemos de estar atentos a los posibles
problemas que dicho usuario pueda generar. Es necesario asegurarse que no se
permiten conexiones bajo este nombre de usuario, que en su directorio {\it
\$HOME} no existen un fichero {\tt .rhosts}\ldots las precauciones habituales
con cualquier nombre de usuario de este tipo que tengamos en nuestro sistema;
incluso nos puede interesar sustituir su {\it shell} original (si lo tiene) por 
uno como {\tt /bin/false}, para que un posible atacante que se haga pasar por 
{\tt uucp} no tenga posibilidad de ejecutar \'ordenes en la m\'aquina. Si 
estamos obligados a ofrecer conexiones v\'{\i}a {\sc uucp} en nuestro sistema,
una buena referencia para conocer m\'as detalles de este mecanismo y su 
seguridad es \cite{kn:rei88} (s\'olo su fecha nos da una idea del grado de
desuso en que ha ca\'{\i}do {\sc uucp}); otra excelente fuente de informaci\'on
sobre la seguridad -- e inseguridad -- de {\sc uucp} es el cap\'{\i}tulo 15 de
\cite{kn:spa96}. Una medida de protecci\'on b\'asica es asignar un {\it login}
y {\it password} diferente para cada sistema que conecte con el nuestro 
mediante este m\'etodo; aparte de incrementar la seguridad -- si un atacante
averigua una clave s\'olo podr\'a utilizar un acceso, no todos -- as\'{\i} 
conseguimos un mayor refinamiento a la hora de registrar los eventos que se
produzcan en nuestro sistema, lo que es muy \'util de cara a perseguir un
abuso del servicio por parte de usuarios no autorizados. Adem\'as, en 
situaciones extremas podemos configurar los m\'odems para realizar un {\it
callback} cuando reciben una petici\'on, lo que asegura que estamos llamando
al sistema deseado y no a otro -- siempre que un atacante no haya podido 
modificar esos n\'umeros --.
\section{El servicio FTP}
{\sc ftp} ({\it File Transfer Protocol}, puerto 21 {\sc tcp}) es, como su 
nombre indica, un protocolo
de transferencia de ficheros entre sistemas. Desde un equipo cliente conectamos
a un servidor para descargar ficheros desde \'el -- lo habitual -- o para 
enviarle nuestros propios archivos.\\
\\Un problema b\'asico y grave de {\sc ftp} es que est\'a pensado para ofrecer
la m\'axima velocidad en la conexi\'on, pero ni mucho menos para ofrecer la
m\'axima seguridad; todo el intercambio de informaci\'on, desde el {\it login}
y {\it password} del usuario en el servidor hasta la transferencia de 
cualquier fichero, se realiza en texto claro, con lo que un atacante lo tiene
muy f\'acil para capturar todo ese tr\'afico y conseguir as\'{\i} un acceso
v\'alido al servidor. Incluso puede ser una amenaza a la privacidad de nuestros
datos el hecho de que ese atacante tambi\'en pueda capturar y reproducir los
ficheros transferidos. Para solucionar este problema es conveniente concienciar
a nuestros usuarios de la utilidad de aplicaciones como {\tt scp} y {\tt sftp},
incluidas en el paquete {\sc ssh}, que permiten transferir ficheros pero 
cifrando todo el tr\'afico; de esta forma, son el mejor sustituto de {\sc 
ftp}.\\
\\Parece evidente que la conexi\'on {\sc ftp} a nuestro sistema ha de estar 
restringida a los usuarios que realmente lo necesiten: por ejemplo, un usuario
como {\tt root} en principio no va a necesitar utilizar este servicio, ya que
por lo general va a trabajar en consola; otros usuarios considerados `del
sistema' (donde se incluye por ejemplo a {\tt postmaster}, {\tt bin}, {\tt 
uucp}, {\tt shutdown}, {\tt daemon}\ldots) tampoco necesitar\'an hacer uso de 
{\sc ftp}. Podemos indicar este tipo de usuarios a los que {\bf no} les est\'a 
permitida una conexi\'on v\'{\i}a {\sc ftp} a nuestra m\'aquina en {\tt 
/etc/ftpusers}, con un nombre por l\'{\i}nea; un ejemplo de este fichero es el
siguiente:
\begin{quote}
\begin{verbatim}
luisa:~# cat /etc/ftpusers
halt
operator
root
shutdown
sync
bin
daemon
adm
lp
mail
postmaster
news
uucp
man
games
guest
postgres # 'postgres' NO hace ftp
nobody
inferno
luisa:~#
\end{verbatim}
\end{quote}
\subsection{FTP an\'onimo}
Los problemas relacionados con la seguridad del servicio {\sc ftp} son 
especialmente preocupantes cuando se trata de configurar un servidor de
{\sc ftp} an\'onimo; muchos de estas m\'aquinas situadas en universidades
espa\~nolas se convierten en servidores de im\'agenes pornogr\'aficas o de 
{\it warez} (copias ilegales de programas comerciales). Conseguir un servidor 
de {\sc
ftp} an\'onimo seguro puede llegar a ser una tarea complicada: incluso en las
p\'aginas de ayuda de algunas variantes de Unix (como Solaris) se trata de
facilitar el proceso para el administrador mediante un {\it shellscript} que
-- por defecto -- presenta graves problemas de seguridad, ya que deja una
copia del fichero de claves del sistema como un archivo de acceso p\'ublico y
an\'onimo.\\
\\Para configurar correctamente un servidor de este tipo necesitamos
en primer lugar crear al usuario {\tt ftp} en {\tt /etc/passwd} y {\tt 
/etc/shadow}, as\'{\i} como su directorio de conexi\'on (algunos Unices, como 
Linux, ya incorporan esto al instalar el sistema). Este directorio ha de 
pertenecer a {\tt root} ({\bf ning\'un} fichero o subdirectorio ha de 
pertenecer {\bf nunca} a {\tt ftp}) y al grupo al que pertenece 
{\tt ftp}: con esto conseguimos que los permisos de propietario sean para el
administrador y los de grupo para los usuarios an\'onimos; estos permisos 
ser\'an 555.\\
\\Dentro del {\it \$HOME} de {\tt ftp} hemos de crear el \'arbol de directorios
m\'{\i}nimo para poder trabajar correctamente; esto es debido a la llamada a
{\tt chroot()} que se utiliza en los accesos an\'onimos, que permite a esos
usuarios ver el directorio ra\'{\i}z de su conexi\'on en el directorio real
{\tt \~{}ftp/}. Al menos dos directorios son necesarios: {\tt etc/} y {\tt 
bin/}, ambos propiedad de {\tt root} y con modo 111. En el primero de ellos 
hemos de crear un fichero {\tt passwd} y otro {\tt group}, utilizados {\bf no} 
con prop\'ositos de autenticaci\'on sino para visualizar el propietario y grupo 
de cada fichero en el entorno sobre el que se ha aplicado {\tt chroot()} al 
ejecutar {\tt ls}: por
tanto, no hace falta {\bf ninguna} contrase\~na en ese fichero {\tt passwd},
y s\'olo ha de contener entradas para los usuarios que posean ficheros bajo
la jerarqu\'{\i}a de {\tt ftp}, como {\tt root}; de la misma forma, el fichero 
{\tt group} s\'olo ha de contener las entradas correspondientes a grupos que
posean ficheros en dicha jerarqu\'{\i}a:
\begin{quote}
\begin{verbatim}
anita:~# cat /export/home/ftp/etc/passwd
root:*:0:1:El Spiritu Santo:/:/sbin/sh
anita:~# cat /export/home/ftp/etc/group
root::0:
other::1:
daemon::2:
ftp::30000:
anita:~#
\end{verbatim}
\end{quote}
Como vemos, el usuario {\tt ftp} tiene un {\it shell} denominado {\tt 
/bin/false}; aunque aqu\'{\i} no tiene ning\'un efecto, en el archivo de
contrase\~nas real de la m\'aquina esto es \'util para prevenir que dicho 
usuario pueda conectar mediante {\sc telnet} o similar.\\
\\Por su parte, en el otro directorio que hemos creado ({\tt bin/}) hemos de
almacenar una copia del programa {\tt ls}, de forma que los usuarios puedan
listar los contenidos de los directorios cuyos permisos lo permitan; si 
utilizamos una versi\'on est\'atica del programa, como hace por ejemplo Linux,
no hemos de configurar nada para que la aplicaci\'on funcione, pero si en 
cambio utilizamos un {\tt ls} din\'amico (como SunOS o Solaris) hemos de
crear el directorio {\tt lib/} dentro de {\tt \~{}ftp/} y copiar en \'el las
librer\'{\i}as necesarias para que el programa funcione (podemos ver de cu\'ales
se trata con {\tt ldd}).\\
\\Con estos pasos ya tenemos configurada la base de nuestro servidor de {\sc
ftp} an\'onimo; no obstante, es habitual crear dos directorios m\'as, uno 
denominado {\tt pub/} y otro {\tt incoming/}, dentro de la misma jerarqu\'{\i}a 
que los anteriores (esto es, en el {\it \$HOME} del usuario {\tt ftp}). El
primero suele contener directorios con todos los ficheros que deseemos ofrecer 
a los usuarios an\'onimos; su modo ha de ser 555, o 2555 en los sistemas
que utilicen el bit {\it setgid} en un directorio para que sus subdirectorios y
ficheros hereden el grupo del propietario. El directorio {\tt incoming} es 
justo lo contrario: sirve para que esos usuarios an\'onimos puedan enviar 
archivos a nuestra m\'aquina. Y es aqu\'{\i} donde suelen comenzar muchos 
problemas: al permitir el {\it upload} de {\it software}, es posible que algunos
piratas utilicen nuestra m\'aquina para crear servidores {\it warez}, subiendo
programas comerciales a este directorio y luego indicando su localizaci\'on
exacta a otras personas, para que los puedan descargar. Por tanto, los permisos
de {\tt incoming} son vitales para nuestra seguridad (incluso si no deseamos
que los usuarios an\'onimos nos env\'{\i}en ficheros podemos borrar este
directorio): esos permisos han de ser 1733, y el propietario del directorio es
el {\tt root}. >Para qu\'e ponemos el bit de permanencia? Muy sencillo: para 
que los usuarios no puedan sobreescribir o borrar ficheros existentes; aunque
la mayor\'{\i}a de servidores {\sc ftp} no permiten a los usuarios an\'onimos
sobreescribir ficheros, si no pusi\'eramos este modo un usuario normal del
sistema s\'{\i} que podr\'{\i}a hacerlo.\\
\\El siguiente {\it shellscript} puede utilizarse para configurar c\'omodamente
un entorno restringido destinado a los usuarios de {\sc ftp} an\'onimo 
siguiendo las directrices que acabamos de comentar; funciona correctamente
(en teor\'{\i}a) sobre Solaris, Linux y AIX\footnote{En este \'ultimo es 
necesario instalar la utilidad {\tt ldd}, que por defecto no se distribuye con
el operativo.}. Al igual que sucede con muchas tareas automatizadas, conviene 
repasar manualmente la estructura de directorios y ficheros creados para 
comprobar que todo es como esper\'abamos:
\begin{quote}
\begin{verbatim}
anita:~# cat /usr/local/sbin/creaentorno
#!/bin/sh
# Script para crear un entorno chroot()eado.
# Funciona OK en Linux, Solaris y AIX.
# 

# Esta variable es una lista con los programas que necesitamos en el 
# entorno restringido. 
PROGS="/bin/ls"
# Imprime modo de uso
if (test $# -lt 1); then
        echo "Usage: $0 /path/to/chroot-environment"
        exit
fi
# Detectamos clon de Unix
OS=`uname -s`
# Creamos estructura de directorios
echo "Creando estructura de directorios para $OS"
if [ ! -d $1 ]; then
        mkdir -p $1
fi
chown root $1
for i in bin etc; do
        if [ ! -d $1/$i ] ; then
               mkdir -p $1/$i
        fi
        chown root $1/$i
done
# En funcion del Unix, la estructura sera una u otra...
if [ $OS = "Linux" ]; then
        if [ ! -d $1/lib ]; then
               mkdir -p $1/lib
        fi
        chown root $1/lib
fi
if ( test $OS = "SunOS" || test $OS = "AIX" ); then
        if [ ! -d $1/usr/lib ]; then
                mkdir -p $1/usr/lib
        fi
        chown root $1/usr/lib
        cd $1
        ln -s ./usr/lib $1/lib
fi
# Instalamos programas y las librerias que necesitan
echo "Instalando programas y librerias..."
for i in $PROGS; do
        if [ ! -f $1/$i ]; then
                cp $i $1/bin
        fi
        chmod 111 $1/bin
        chown root $1/bin
        if [ $OS = "AIX" ]; then 
        for j in `ldd $i|awk -F"(" '{if(NR!=1) print $1}'`; do
                if [ ! -f $1/$j ]; then
                        cp $j $1/lib
                fi
                chown root $1/$j
        done
        else 
                for j in `ldd $i|awk '{print $3}'`; do
                        if [ ! -f $1/$j ]; then
                                cp $j $1/lib
                        fi
                        chown root $1/$j
                done
        fi
done
# Estos ficheros quizas sea necesario retocarlos a mano, en funcion del tipo
# de entorno restringido que fabriquemos.
# Generamos PASSWD
echo "Generando /etc/passwd..."
awk -F: '$1=="root" {print $1":*:"$3":"$4":"$5":"$6":"$7}' /etc/passwd >\
$1/etc/passwd
awk -F: '$1=="bin" {print $1":*:"$3":"$4":"$5":"$6":"$7}' /etc/passwd>>\
$1/etc/passwd
awk -F: '$1=="daemon" {print $1":*:"$3":"$4":"$5":"$6":"$7}' /etc/passwd>>\
$1/etc/passwd
chmod 444 $1/etc/passwd
chown root $1/etc/passwd
# Quizas hay que anyadir otros grupos que nos interesen
# Generamos GROUP con algunas entradas
echo "Generando /etc/group..."
awk -F: '$1=="root" {print $1":*:"$3":"$4}' /etc/group>$1/etc/group
awk -F: '$1=="bin" {print $1":*:"$3":"}' /etc/group>>$1/etc/group
awk -F: '$1=="daemon" {print $1":*:"$3":"}' /etc/group>>$1/etc/group
chmod 444 $1/etc/group
chown root $1/etc/group
# Generamos pub/ e incoming/
echo "Generando pub/ e incoming/..."
if [ ! -d $1/pub ]; then
        mkdir -p $1/pub
fi
chmod 2555 $1/pub
chown root $1/pub
if [ ! -d $1/incoming ]; then
        mkdir -p $1/incoming
fi
chmod 1733 $1/incoming
chown root $1/incoming
# Si estamos en Solaris, aun no hemos acabado
if [ $OS = "SunOS" ]; then
        # Mas librerias
        echo "$OS: Instalando librerias..."
        for i in ld.so.1 libc.so.1 libdl.so.1 libmp.so.2 libnsl.so.1 \
        libsocket.so.1 nss_compat.so.1 nss_dns.so.1 nss_files.so.1 \
        nss_nis.so.1 nss_nisplus.so.1 nss_xfn.so.1 straddr.so \
        straddr.so.2; do
               cp /usr/lib/$i $1/usr/lib
        done
        if [ ! -d $1/dev ]; then
                mkdir -p $1/dev
        fi
        chown root $1/dev
        # Generamos dispositivos
        echo "$OS: Generando dispositivos..."
        for i in /dev/zero /dev/tcp /dev/udp /dev/ticotsord; do
                MAJOR=`ls -lL $i|awk '{print $5}'|sed s/","//g`
                MINOR=`ls -lL $i|awk '{print $6}'`
                TYPE=`ls -lL $i|cut -c1-1`
                /usr/sbin/mknod $1/$i $TYPE $MAJOR $MINOR
        done
        chmod 666 $1/dev/*
fi
echo "FIN"
# FIN de Solaris
anita:~#
\end{verbatim}
\end{quote}
Algunos problemas relacionados con {\tt incoming/} provienen de los permisos 
con que se 
crean sus ficheros y subdirectorios: aunque los usuarios an\'onimos no puedan
leer el directorio, con algunos servidores {\tt ftpd} s\'{\i} que es posible
que puedan leer los ficheros contenidos en \'el (y sus subdirectorios), con lo
que sigue siendo posible acceder a los archivos conociendo su nombre exacto;
para evitar este problema, muchos administradores planifican un sencillo
{\it shellscript} para que cada cierto tiempo mueva los contenidos de {\tt 
incoming} a otro lugar, fuera del alcance de los usuarios an\'onimos (por 
ejemplo, un subdirectorio con modo 000 de {\tt /tmp/}). Ciertos servidores,
como {\it WU--ftpd}, tienen un fichero de configuraci\'on ({\tt 
/etc/ftpaccess}) donde indicar -- entre otras cosas -- los modos con que se van 
a crear entradas en {\tt incoming/}.\\
\\Un ataque t\'{\i}pico a los servidores de {\sc ftp} es una negaci\'on de
servicio llenando todo el espacio disponible para el {\it upload} de ficheros;
para minimizar las consecuencias de este ataque, es conveniente situar el
directorio {\tt \~{}ftp/} en una partici\'on separada del resto del sistema
de ficheros, donde s\'olo se encuentre dicho directorio; algunos demonios
permiten directamente limitar la cantidad de ficheros subidos al servidor en
cada sesi\'on.\\
\\Otra negaci\'on de servicio muy habitual contra los servidores de {\sc ftp}
an\'onimo es obligar a las m\'aquinas a consumir una excesiva cantidad de
CPU, ralentizando el sistema hasta que la calidad de servicio es muy baja; esto
se produce en servidores que permiten descargar directorios completos como un
\'unico archivo, empaquetados con {\tt tar} y/o comprimidos con {\tt gzip}. 
Veamos un ejemplo extra\'{\i}do de una sesi\'on de {\sc ftp} an\'onimo:
\begin{quote}
\begin{verbatim}
ftp> pwd
257 "/pub/utils" is current directory.
ftp> ls
200 PORT command successful.
150 Opening ASCII mode data connection for /bin/ls.
total 377
drwxr-xr-x   2 root     root         1024 Sep 18 22:28 .
drwxrwxr-x   3 root     wheel        1024 Sep 18 22:28 ..
-rw-r--r--   1 root     root       163519 Sep 18 22:28 transfig-3.1.2.tar.gz
-rw-r--r--   1 root     root       217850 Sep 18 22:27 tth_C.tgz
226 Transfer complete.
ftp>  cd ..
250 CWD command successful.
ftp> pwd
257 "/pub" is current directory.
ftp> ls
200 PORT command successful.
150 Opening ASCII mode data connection for /bin/ls.
total 3
drwxrwxr-x   3 root     wheel        1024 Sep 18 22:28 .
drwxrwxr-x   8 root     wheel        1024 Aug  1  1994 ..
drwxr-xr-x   2 root     root         1024 Sep 18 22:28 utils
226 Transfer complete.
ftp> get utils.tar.gz
local: utils.tar.gz remote: utils.tar.gz
200 PORT command successful.
150 Opening BINARY mode data connection for /bin/tar.
226 Transfer complete.
381369 bytes received in 1.07 secs (3.5e+02 Kbytes/sec)
ftp>
\end{verbatim}
\end{quote}
Como podemos ver, acabamos de descargar un directorio completo empaquetado y
comprimido sin m\'as que a\~nadir al nombre de dicho directorio la extensi\'on
{\tt .tar.gz} (o simplemente {\tt .tar} si no hubi\'eramos querido comprimirlo).
Evidentemente esto resulta muy \'util en determinadas situaciones pero, >nos
hemos parado a pensar que suceder\'{\i}a si alguien intentara descargar el
directorio {\tt /pub.tar.gz} en un servidor con unos cuantos {\it Gigabytes}
de ficheros? La carga del sistema crecer\'{\i}a much\'{\i}simo, ya que estamos
empaquetando y comprimiendo un archivo de grandes dimensiones; si ahora 
extendemos esto a varios usuarios que ejecutan la misma orden 
simult\'aneamente podemos hacernos una idea de la sobrecarga introducida en la
m\'aquina: una raz\'on m\'as que suficiente para tener cuidado con los 
directorios que permitimos descargar de esta forma\ldots\\
\\Por \'ultimo, es una buena idea mostrar un mensaje cuando los usuarios 
an\'onimos conectan a nuestra m\'aquina donde se indiquen claramente los fines 
del sistema y la atenci\'on a su uso indebido; este mensaje puede sernos \'util
tanto con fines jur\'{\i}dicos (as\'{\i} el atacante no podr\'a argumentar que
desconoc\'{\i}a la finalidad del sistema) como con fines disuasorios: si el
pirata se da cuenta de que nos preocupamos por la seguridad de nuestro servidor,
es posible que lo abandone y busque otro menos protegido. Por ejemplo, si
utilizamos {\it WU--ftpd}, en {\tt \~{}ftp/welcome.msg} podemos escribir el
mensaje mostrado al conectar al sistema, y en diferentes ficheros {\tt 
.message} el mensaje que se vuelca a acceder a un directorio (estos nombres son
configurables en {\tt /etc/ftpaccess}). Un ejemplo del mensaje de entrada puede
ser el siguiente:
\begin{quote}
\begin{verbatim}
anita:~# cat /export/home/ftp/welcome.msg

                  * * *         ANITA        * * *
           -----------  Bienvenid@s a ANITA  ------------
  Esta maquina es propiedad de la Universidad Politecnica de Valencia y
  sus fines son exclusivamente academicos y de investigacion. Cualquier
  otro uso sera perseguido y castigado con el maximo rigor.
  Cualquier actividad realizada en, desde o hacia este sistema esta
  sujeta a monitorizacion sin previo aviso.

anita:~#
\end{verbatim}
\end{quote}
\subsection{FTP invitado}
Hasta ahora hemos visto dos formas de transferir ficheros desde una m\'aquina
Unix mediante {\sc ftp}: o bien el usuario conecta utilizando su {\it login} y
su clave al sistema y descarga de \'el cualquier archivo sobre el que tenga
permiso de lectura, de {\it cualquier} parte del sistema de ficheros, o bien
accede a un entorno restringido mediante {\tt chroot()} como usuario an\'onimo, 
con un {\it login} gen\'erico y usando como contrase\~na una direcci\'on de 
correo -- seguramente falsa --. En muchas ocasiones, estos modelos pueden 
resultar insuficientes o al menos poco adecuados a nuestras necesidades.\\
\\Imaginemos esta situaci\'on: un proveedor de acceso a Internet decide ofrecer
a sus clientes la posibilidad de actualizar sus p\'aginas {\it web} personales
mediante
{\sc ftp}, de forma que cada uno de ellos no tiene m\'as que conectar con su
nombre de usuario y su contrase\~na al servidor y subir sus ficheros {\sc 
html}; dichos {\it login} y {\it password} ser\'an por supuesto diferentes para
cada usuario, por lo que parece claro que un entorno de {\sc ftp} an\'onimo no 
es aplicable -- al menos de forma inmediata -- en esta situaci\'on. El {\sc
ftp} `normal' funcionar\'{\i}a correctamente, pero su utilizaci\'on tampoco es
\'optima: si un usuario no necesita acceder m\'as que a su {\it \$HOME} para
actualizar sus p\'aginas, >por qu\'e permitirle que vea todo nuestro sistema de 
ficheros, aunque sea v\'{\i}a {\sc ftp}, y que pueda descargar archivos tan
comprometedores como {\tt /etc/passwd}?\\
\\Los potenciales problemas de seguridad que la situaci\'on anterior implica
han dado lugar a un tercer tipo de acceso {\sc ftp} denominado {\bf invitado}
({\it guest}), que se puede contemplar como una mezcla de los dos vistos hasta
el momento. La idea de este mecanismo es muy sencilla: se trata de permitir que
cada usuario conecte a la m\'aquina mediante su {\it login} y su contrase\~na, 
pero evitando que tenga acceso a partes del sistema de ficheros que no necesita 
para realizar su trabajo; conectar\'a a un entorno restringido mediante {\tt
chroot()}, algo muy similar a lo que sucede en los accesos an\'onimos.\\
\\Para poder crear f\'acilmente entornos {\sc ftp} restringidos a cada usuario
es conveniente instalar {\it WU-ftpd} en la m\'aquina; este servidor est\'a
disponible libremente a trav\'es de Internet, en la direcci\'on {\tt 
ftp://ftp.wu-ftpd.org/pub/wu-ftpd/}. Otros servidores, como el distribuido con
Solaris, permiten crear usuarios {\sc ftp} invitados pero de una forma m\'as
compleja; en los ejemplos que veamos en este punto vamos a asumir que utilizamos
{\it WU-ftpd}.\\
\\Lo primero que necesitamos para configurar el entorno al que van a conectar
este tipo de usuarios es una estructura de directorios y archivos muy similar
a la que hemos estudiado para los accesos a trav\'es de {\sc ftp} an\'onimo, 
pero esta vez colgando del directorio de conexi\'on del usuario invitado; con 
unas peque\~nas variaciones, podemos utilizar para crear este entorno el 
{\it shellscript} que hemos presentado en el punto anterior. As\'{\i}, si
queremos que nuestro usuario {\tt toni} acceda como invitado v\'{\i}a {\sc
ftp} podemos crear esta estructura en su {\it \$HOME}:
\begin{quote}
\begin{verbatim}
anita:~# /usr/local/sbin/creaentorno /export/home/toni
Creando estructura de directorios para SunOS
Instalando programas y librerias...
Generando /etc/passwd...
Generando /etc/group...
Generando pub/ e incoming...
SunOS: Instalando librerias...
SunOS: Generando dispositivos...
FIN
anita:~# 
\end{verbatim}
\end{quote}
Realmente, son necesarias peque\~nas modificaciones sobre el esquema anterior
para que todo funcione correctamente; por un lado, los directorios {\tt pub/}
e {\tt incoming/} no son necesarios en los accesos como invitado, ya que {\it a
priori} los usuarios que accedan de esta forma necesitar\'an escribir en
varios directorios del entorno. Adem\'as, quiz\'as nos interese repasar los
permisos de toda la jerarqu\'{\i}a de directorios creada, para afinar m\'as
los lugares en los que se les permita escribir a los usuarios; por ejemplo,
si s\'olo van a subir archivos a un directorio {\tt \$HOME/public$\_$html/},
donde se ubicar\'an sus p\'aginas {\it web}, no tienen por qu\'e escribir en el
resto del entorno. De la misma forma, si el directorio {\it \$HOME} es 
propiedad de cada usuario quiz\'as pueda borrar archivos como {\tt lib}, que es
un enlace a {\tt usr/lib/}, lo que puede llegar a comprometer nuestra 
seguridad.\\
\\Otro punto a tener en cuenta es qui\'en va a poseer ficheros dentro del 
entorno restringido, ya que esos usuarios y sus grupos deber\'an tener una 
entrada en los archivos {\tt etc/passwd} y {\tt etc/group}; como suced\'{\i}a 
con los usuarios an\'onimos, estos ficheros no se van a usar aqu\'{\i} para
realizar autenticaci\'on, sino simplemente para ver los nombres del usuario y
grupo propietarios de cada fichero al realizar un listado, por lo que en
ninguno de ellos es necesaria una contrase\~na real: basta con un asterisco en
el campo correspondiente.\\
\\Una vez que hemos creado correctamente el entorno es necesario configurar
el acceso del usuario en cuesti\'on. Generalmente no nos interesar\'a que 
acceda por {\tt telnet} o similar, por lo que su {\it shell} en {\tt 
/etc/passwd} (el original de la m\'aquina, no el del entorno restringido) ha de 
ser algo como {\tt /bin/false}. Es necesario que exista una entrada para este
{\it shell} en {\tt /etc/shells}, ya que de lo contrario el usuario no podr\'a
autenticarse; si este \'ultimo archivo no existe, es necesario crearlo. Su
directorio {\it \$HOME}, indicado en {\tt /etc/passwd}, tambi\'en ha de ser 
modificado de la siguiente forma:
\begin{quote}
\begin{verbatim}
toni:x:1002:10:Toni at ANITA:/export/home/toni/./:/bin/sh
\end{verbatim}
\end{quote}
Como vemos, se a\~nade {\tt `/./'} al directorio {\it \$HOME} del usuario. Esta 
cadena 
indica d\'onde se va a efectuar el {\tt chroot()} (por ejemplo, si quisi\'eramos
que el {\tt chroot()} se hiciera sobre {\tt /export/home/} y tras esta llamada
el usuario entrara a su directorio {\tt toni}, lo indicar\'{\i}amos como\\
{\tt /export/home/./toni/}).\\
\\Tras modificar {\tt /etc/passwd} hemos de modificar {\tt /etc/group} para
incluir al usuario {\tt `toni'} en un grupo que luego definiremos como 
invitado, por ejemplo {\tt `rftp'}:
\begin{quote}
\begin{verbatim}
anita:~# grep toni /etc/group
rftp::400:toni
anita:~# 
\end{verbatim}
\end{quote}
Ahora falta por configurar el archivo {\tt /etc/ftpaccess}; hemos de indicarle
al demonio que utilice este fichero (por ejemplo, mediante la opci\'on {\tt 
`-a'}). En \'el definimos el grupo {\tt `guest'} en las clases apropiadas:
\begin{quote}
\begin{verbatim}
class   local   real,guest,anonymous *.domain 0.0.0.0
class   remote  real,guest,anonymous *
\end{verbatim}
\end{quote}
Tambi\'en le damos a los usuarios {\tt `guest'} los permisos que consideremos
oportunos; habitualmente, interesar\'a que puedan borrar, sobreescribir y 
renombrar sus archivos. Pero no es normal que necesiten ejecutar cambios en los
modos de los ficheros o en su m\'ascara de permisos:
\begin{quote}
\begin{verbatim}
delete     no   anonymous                     # delete permission?
overwrite  no   anonymous                     # overwrite permission?
rename     no   anonymous                     # rename permission?
chmod      no   anonymous,guest               # chmod permission?
umask      no   anonymous,guest               # umask permission?
\end{verbatim}
\end{quote}
Y por \'ultimo, tambi\'en en {\tt /etc/ftpaccess}, definimos al grupo {\tt 
`rftp'} como invitado:
\begin{quote}
\begin{verbatim}
guestgroup rftp
\end{verbatim}
\end{quote}
Una vez llegados a este punto el usuario ya est\'a en disposici\'on de 
conectar como invitado v\'{\i}a {\sc ftp}; aunque realmente acceder\'a a su 
{\it \$HOME}, para \'el ser\'a el directorio ra\'{\i}z, y no ver\'a ning\'un
archivo del sistema que no se encuentre en este directorio.\\
\\Antes de finalizar, un \'ultimo apunte: el entorno restringido que acabamos
de ver s\'olo se aplica para accesos por {\sc ftp}; as\'{\i}, si el usuario
tiene definido un {\it shell} est\'andar en {\tt /etc/passwd}, cuando conecte
mediante {\it telnet} o similar seguir\'a teniendo acceso a todo el sistema de 
ficheros, por lo que todo el trabajo que hemos realizado perder\'{\i}a su
sentido. Aunque en el siguiente punto daremos alguna idea para crear entornos
restringidos en los accesos por terminal remota, esta situaci\'on es mucho 
m\'as extra\~na que la de los accesos invitados, por lo que normalmente (y esto
es {\bf muy importante}) los {\it shells} de los usuarios invitados han de ser 
del tipo {\tt /bin/false}, es decir, no les permitiremos una sesi\'on 
interactiva en el sistema por terminal remota. Con un {\it shell} de este
estilo, si intentan acceder a la m\'aquina (por ejemplo mediante {\it telnet}),
nada m\'as introducir correctamente su {\it login} y su {\it password} ser\'an
desconectados:
\begin{quote}
\begin{verbatim}
luisa:~# telnet anita
Trying 192.168.0.3...
Connected to anita.
Escape character is '^]'.

SunOS 5.7

login: toni
Password: 
Connection closed by foreign host.
luisa:~# 
\end{verbatim}
\end{quote}
\section{El servicio TELNET}
El protocolo {\sc telnet} ({\sc tcp}, puerto 23) permite utilizar una m\'aquina 
como terminal virtual de otra a trav\'es de la red, de forma que se crea un 
canal virtual de comunicaciones similar -- pero mucho m\'as inseguro -- a 
utilizar una terminal f\'{\i}sicamente conectada a un servidor; la idea es 
sencilla: estamos accediendo remotamente en modo texto a un equipo -- en 
principio potente -- igual que si
estuvi\'eramos utilizando su consola o una de sus terminales f\'{\i}sicas, lo
que nos permite aprovechar toda su potencia de c\'alculo si necesidad de 
desplazarnos hasta la ubicaci\'on de ese servidor, sino trabajando c\'omodamente
desde nuestro propio equipo.\\
\\{\sc telnet} es el cl\'asico servicio que hasta hace unos a\~nos no se 
sol\'{\i}a deshabilitar nunca: no es habitual adquirir una
potente m\'aquina corriendo Unix y permitir que s\'olo se trabaje en ella desde
su consola; lo m\'as normal es que este servicio est\'e disponible para que
los usuarios puedan trabajar remotamente, al menos desde un conjunto de 
m\'aquinas determinado. Evidentemente, reducir al m\'{\i}nimo imprescindible 
el conjunto de sistemas desde donde es posible la conexi\'on es una primera
medida de seguridad; no obstante, no suele ser suficiente: recordemos que {\sc
telnet} no utiliza ning\'un tipo de cifrado, por lo que todo el tr\'afico entre
equipos se realiza en texto claro. Cualquier atacante con un analizador de red
(o un vulgar {\it sniffer}) puede capturar el {\it login} y el {\it password}
utilizados en una conexi\'on; el {\it sniffing} siempre es peligroso, pero 
m\'as a\'un en sesiones {\sc telnet} en las que transmitimos nombres de 
usuarios y contrase\~nas: estamos otorgando a cualquiera que lea esos datos un
acceso total a la m\'aquina destino, bajo nuestra identidad. Por tanto, es {\bf 
muy recomendable} no utilizar 
{\sc telnet} para conexiones remotas, sino sustituirlo por aplicaciones 
equivalentes pero que utilicen cifrado para la transmisi\'on de datos: {\sc 
ssh} o {\it SSL-Telnet} son las m\'as comunes. En estos casos necesitamos 
adem\'as de la parte cliente en nuestro equipo, la parte servidora en la 
m\'aquina remota escuchando en un puerto determinado.\\
\\Aparte del problema de los atacantes esnifando claves, los demonios {\tt
telnetd} han sido tambi\'en una fuente cl\'asica de problemas de programaci\'on
(se puede encontrar un excelente repaso a algunos de ellos en el cap\'{\i}tulo
29 de \cite{kn:ano97});
b\'asicamente, cualquier versi\'on de este demonio que no est\'e actualizada 
es una potencial fuente de problemas, por lo que conviene conseguir la \'ultima
versi\'on de {\tt telnetd} para nuestro Unix particular, especialmente si a\'un 
tenemos una versi\'on anterior a 1997. Otros problemas, como la posibilidad de
que un atacante consiga recuperar una sesi\'on que no ha sido cerrada 
correctamente, el uso de {\tt telnet} para determinar qu\'e puertos de un {\it
host} est\'an abiertos, o la utilizaci\'on del servicio {\it telnet} (junto a 
otros, como {\sc ftp}) para averiguar el clon de Unix concreto (versi\'on de 
{\it kernel} incluida) que un servidor utiliza, tambi\'en han hecho famosa la
inseguridad de este servicio.\\
\\Antes hemos hablado de la configuraci\'on de un entorno restringido para
usuarios {\sc ftp} invitados, que acced\'{\i}an mediante su {\it login} y su 
contrase\~na pero que no ve\'{\i}an la totalidad del sistema de ficheros de
nuestra m\'aquina. Es posible -- aunque ni de lejos tan habitual -- hacer algo
parecido con ciertos usuarios interactivos, usuarios que conectar\'an al sistema
mediante {\it telnet} utilizando tambi\'en su {\it login} y su {\it password},
pero que no ver\'an el sistema de ficheros completo: s\'olo la parte que a 
nosotros nos interese (en principio).\\
\\Para que un usuario acceda mediante {\it telnet} a un entorno restringido con
{\tt chroot()} necesitamos en primer lugar un entorno parecido al que hemos
visto antes: a partir de su directorio {\it \$HOME}, una serie de 
subdirectorios {\tt bin/}, {\tt lib/}, {\tt etc/}\ldots Dentro de este \'ultimo
existir\'a al menos un fichero {\tt group} y otro {\tt passwd} (igual que 
suced\'{\i}a antes, no se usan con prop\'ositos de autenticaci\'on, por lo que
no es necesario -- ni recomendable -- que existan claves reales en ninguno de 
ellos). En el directorio {\tt bin/} incluiremos los ejecutables que queremos
que nuestro usuario pueda ejecutar, y en {\tt lib/} (o {\tt usr/lib/}) las
librer\'{\i}as que necesiten; si usamos el {\it shellscript} anterior -- de
nuevo, con alguna peque\~na modificaci\'on -- para 
crear este entorno, en la variable {\it \$PROGS} podemos definir tales 
ejecutables para que autom\'aticamente se copien junto a las librer\'{\i}as 
necesarias en el directorio correspondiente:
\begin{quote}
\begin{verbatim}
PROGS="/bin/ls /bin/sh"
\end{verbatim}
\end{quote}
Finalmente, en el archivo {\tt /etc/passwd} real hemos de definir un {\it shell}
para el usuario como el siguiente:
\begin{quote}
\begin{verbatim}
luisa:~# cat /home/toni/prog/shell.c
#include <stdio.h>
#include <stdlib.h>
#include <unistd.h>
#include <sys/types.h>
#include <pwd.h>

#define SHELL "/bin/sh"

int main(){
struct passwd *entry=(struct passwd *)malloc(sizeof(struct passwd));
char *const ARGS[2]={SHELL,NULL};
while((entry=getpwent())->pw_uid!=getuid());
endpwent();
if(chdir(entry->pw_dir)<0) perror("chdir()");
if(chroot(entry->pw_dir)<0) perror("chroot()");
if(setuid(getuid())<0) perror("setuid()");
if(execvp(SHELL,ARGS)<0) perror("execvp()");
// No alcanzado
return(0);
}
luisa:~# 
\end{verbatim}
\end{quote}
Este c\'odigo, convenientemente compilado, ser\'a el {\it shell} real del
usuario restringido; como vemos, obtiene el directorio {\it \$HOME} del mismo,
hace un {\tt chroot()} a \'el, y ejecuta en este entorno el {\it shell} 
secundario ({\tt bin/sh}, que realmente ser\'a {\tt \$HOME/bin/sh}). Para 
que el {\tt chroot()} sea correcto el programa ha de estar {\it setuidado} bajo 
la identidad de {\tt root} (s\'olo el superusuario puede realizar esta llamada),
con los riesgos que esto implica; al contrario de lo que dir\'{\i}a Knuth, yo 
s\'olo defiendo que el c\'odigo anterior funciona, no que sea correcto\ldots o 
seguro :)\\
\\Si tenemos que crear un entorno como este para usuarios interactivos hemos
de tener en cuenta ciertas medidas de seguridad relativas a los ejecutables que
situemos -- o que permitamos situar -- en dicho entorno. Para empezar, hemos de
evitar a toda costa los ejecutables {\it setuidados}, as\'{\i} como las llamadas
{\tt mknod()}, {\tt chmod()} o la propia {\tt chroot()}; adem\'as, no debe ser
posible obtener privilegios de administrador dentro del entorno restringido, ya
que para el {\tt root} estas restricciones pierden su sentido: no tenemos m\'as
que pensar que si un usuario con privilegios de {\tt root} dentro del entorno 
es capaz de generar un dispositivo que represente un disco duro, con algo
tan sencillo como la utilidad {\tt mknod}, autom\'aticamente acceder\'a a la
totalidad de ese disco, olvidando ya el {\tt chroot()} y la potencial 
protecci\'on que pueda ofrecernos. Algo similar ocurre con la memoria del
sistema, ciertos dispositivos f\'{\i}sicos, o estructuras de datos del n\'ucleo:
si esto es accesible desde el entorno restringido, es muy probable que nuestra 
seguridad se vea rota tarde o temprano (m\'as bien temprano). Tampoco es 
aconsejable permitir la ejecuci\'on de compiladores de C o de int\'erpretes de
Perl.\\
\\Como hemos dicho, este tipo de entornos es mucho menos habitual
que los de {\sc ftp}, aparte de bastante m\'as peligrosos. Una tarea tan
habitual como cambiar la contrase\~na no es posible -- al menos de forma 
trivial -- en este entorno (aunque podr\'{\i}amos modificar el c\'odigo anterior
para que se ofrezca al usuario esta posibilidad antes de situarlo en el entorno
restringido). >Y que sucede si necesitamos que el usuario acceda no a un s\'olo
directorio, sino a dos? Las soluciones -- al menos las seguras -- no son 
inmediatas.
\section{El servicio SMTP}
El servicio {\sc smtp} ({\it Simple Mail Transfer Protocol}, puerto 25 {\sc 
tcp}) se utiliza para
transferir correo electr\'onico entre equipos remotos; estas m\'aquinas pueden
ubicarse f\'{\i}sicamente en la misma sala, en la misma universidad, o en la 
otra parte del mundo, a miles de kil\'ometros de distancia. Este servicio suele
ser atendido por un demonio denominado {\tt sendmail}, que ha sido uno de los 
que m\'as problemas de seguridad ha tenido a lo largo de la historia de Unix;
y no es para menos: se trata de un {\it software} muy complejo y potente -- 
incluso demasiado para las necesidades de la mayor\'{\i}a de servidores --, por
lo es inevitable que en su c\'odigo existan {\it bugs}; para hacernos una idea
del grado de complejidad de {\tt sendmail} simplemente tenemos que echarle un
vistazo a su fichero de configuracion principal, {\tt /etc/sendmail.cf}. Existen
incluso libros casi dedicados exclusivamente a este archivo (\cite{kn:cos97a},
\cite{kn:cos97b}\ldots).\\
\\Una medida de protecci\'on b\'asica para nuestro servicio {\sc smtp}, y que
muchos administradores desconocen, es la posibilidad de servir {\tt sendmail}
desde {\tt inetd} en lugar de hacerlo como un demonio independiente, y por tanto
poder restringir el acceso al mismo mediante {\it TCP Wrappers}. En la 
mayor\'{\i}a de organizaciones 
existe un servidor de correo principal que es el encargado de recoger el {\it
mail} para todas las direcciones {\tt `*@*.upv.es'}; el resto de equipos s\'olo
recibir\'an correo desde este equipo -- o desde otro que sirve s\'olo a un 
subdominio, y que a su vez recibe s\'olo desde el principal --. Entonces, 
parece claro que si nuestro {\tt sendmail} s\'olo recibe correo v\'alido desde
una m\'aquina, lo l\'ogico es configurarlo para que s\'olo acepte peticiones
desde ella: en lugar de lanzar el demonio al arrancar el sistema, en uno de
los {\it scripts} de {\tt /etc/rc.d/} o similar, lo serviremos desde {\tt 
inetd}. Para esto necesitamos en primer lugar modificar el {\it script} 
correspondiente para que {\tt sendmail} no se lance como demonio en el 
arranque: en lugar de
invocarlo como {\tt `sendmail -bd -q15m'} lo haremos como {\tt `sendmail 
-q15m'}. Ademas, es necesario identificar el servicio en {\tt /etc/services}, 
con una l\'{\i}nea como la siguiente:
\begin{quote}
\begin{verbatim}
luisa:~# grep smtp /etc/services
smtp            25/tcp          mail
luisa:~#
\end{verbatim}
\end{quote}
Tras reconocer el servicio, hemos de a\~nadir una l\'{\i}nea en {\tt 
/etc/inetd.conf} indicando c\'omo se ha de ejecutar {\tt sendmail} cuando {\tt
inetd} reciba una petici\'on en el puerto 25; dicha l\'{\i}nea es similar a la
siguiente:
\begin{quote}
\begin{verbatim}
luisa:~# grep smtp /etc/inetd.conf
smtp  stream  tcp     nowait  root    /usr/sbin/tcpd  sendmail -bs
luisa:~#
\end{verbatim}
\end{quote}          
Una vez realizados estos cambios podemos controlar el acceso a nuestro servicio
{\sc smtp} mediante {\it TCP Wrappers}; por ejemplo, en el caso de la 
Universidad Polit\'ecnica, el servidor de correo principal se denomina {\tt
vega.cc.upv.es}. Para que s\'olo esta m\'aquina nos pueda enviar correo, 
incluiremos una l\'{\i}nea como la siguiente en {\tt /etc/hosts.allow}:
\begin{quote}
\begin{verbatim}
luisa:~# grep sendmail /etc/hosts.allow
sendmail: vega.cc.upv.es
luisa:~# 
\end{verbatim}
\end{quote}        
El resto de sistemas no han de estar autorizados a conectar al puerto; esto
incluye tambi\'en a la m\'aquina local: para un correcto funcionamiento de
nuestro sistema de correo, ni siquiera hace falta que {\it localhost} tenga
permiso para acceder a su puerto 25. En \cite{kn:gon97} se explica c\'omo 
combinar estas restricciones ofrecidas por {\it TCP Wrappers} con un 
cortafuegos como {\it TIS Firewall Toolkit}; en esta obra tambi\'en se habla
con m\'as detalle de los problemas que puede implicar el correo electr\'onico,
y por supuesto de c\'omo solucionarlos.\\
\\Evidentemente, esto es aplicable a sistemas que reciban correo de un \'unico
{\it mailer}; si debemos configurar el propio {\it mailer} de la organizaci\'on,
que por lo general recibir\'a correo de un n\'umero indeterminado de m\'aquinas,
no podemos bloquear el acceso a su {\tt sendmail} de esta forma. No obstante,
en este caso podemos aplicar unas medidas de seguridad simples, como realizar
una consulta inversa a DNS para asegurarnos de que s\'olo m\'aquinas registradas
env\'{\i}an correo o no permitir que nuestro sistema reenv\'{\i}e correo que
no provenga de direcciones registradas bajo su dominio. Estas medidas, b\'asicas
para evitar problemas de {\it spam} y {\it mail bombing}, son necesarias en
la configuraci\'on de los sistemas de cualquier entidad.
\section{Servidores WWW}
Hoy en d\'{\i}a las conexiones a servidores {\it web} son sin duda las m\'as
extendidas entre usuarios de Internet, hasta el punto de que muchas personas
piensan que este servicio ({\sc http}, puerto 80 {\sc tcp}) es el \'unico que 
existe en
la red -- junto al {\sc irc} --. Lo que en un principio se dise\~n\'o para
que unos cuantos f\'{\i}sicos intercambiaran y consultaran art\'{\i}culos 
f\'acilmente, en la actualidad mueve a diario millones de d\'olares y es uno de
los pilares fundamentales de cualquier empresa: es por tanto un objetivo muy
atractivo para cualquier pirata.\\
\\Los problemas de seguridad relacionados con el protocolo {\sc http} se 
dividen en tres grandes grupos en funci\'on de los datos a los que pueden 
afectar (\cite{kn:gar97}):
\begin{itemize}
\item Seguridad en el servidor.\\
Es necesario garantizar que la informaci\'on almacenada en la m\'aquina 
servidora no pueda ser modificada sin autorizaci\'on, que permanezca disponible
y que s\'olo pueda ser accedida por los usuarios a los que les est\'e 
leg\'{\i}timamente permitido.
\item Seguridad en la red.\\
Cuando un usuario conecta a un servidor {\it web} se produce un intercambio de 
informaci\'on entre ambos; es vital garantizar que los datos que recibe el
cliente desde el servidor sean los mismos que se est\'an enviando (esto es, que
no sufran modificaciones de terceros), y tambi\'en garantizar que
la informaci\'on que el usuario env\'{\i}a hacia el servidor no sea capturada,
destruida o modificada por un atacante. Esto es especialmente importante si la
informaci\'on en tr\'ansito es secreta, como en el caso de los {\it passwords} 
que el usuario teclea para autenticarse en el servidor, o en el comercio 
electr\'onico y el intercambio de n\'umeros de tarjetas de cr\'edito.
\item Seguridad en el cliente.\\
Por \'ultimo es necesario garantizar al usuario que lo que descarga de un
servidor no va a perjudicar a la seguridad de su equipo; sin llegar a extremos
de {\it applets} maliciosos o programas con virus, si simplemente el navegador
del usuario `se cuelga' al acceder al visitar las p\'aginas de una 
organizaci\'on, seguramente esa persona dejar\'a de visitarlas, con la 
consecuente p\'erdida de imagen -- y posiblemente de un futuro cliente -- para
esa entidad.
\end{itemize}
Asegurar el servidor implica -- aparte de las medidas habituales para cualquier
m\'aquina Unix -- medidas excepcionales dedicadas al demonio servidor de {\it
web} y su entorno de trabajo; estas medidas son propias para cada programa
servidor, por lo que aqu\'{\i} no entraremos en detalles concretos sobre 
cada uno de ellos. No obstante, y sea cual sea el servidor utilizado (Apache,
NCSA, Netscape\ldots), es necesario seguir un consejo b\'asico: minimizar el
n\'umero de usuarios en la m\'aquina y minimizar el n\'umero de servicios 
ofrecidos en ella; aunque lo normal es que una m\'aquina dedicada a cualquier
tarea con decenas -- o con miles -- de usuarios sea tambi\'en el servidor {\it
web}, es recomendable que dicho servidor sea un equipo dedicado a esa tarea.\\
\\Los problemas relacionados con servidores {\it web} suelen proceder de errores
de programaci\'on en los CGIs ubicados en el servidor. Un CGI ({\it Common
Gateway Interface}) es un c\'odigo capaz de comunicarse con aplicaciones del
servidor, de forma que desde una p\'agina se invoque a dichas aplicaciones 
pas\'andoles argumentos y el resultado se muestre en el navegador de un cliente;
cuando rellenamos un formulario, vemos una imagen sensible, o simplemente
incrementamos el contador de cierta p\'agina, estamos utilizando CGIs. Esta 
capacidad del CGI para comunicarse con el resto del sistema que alberga las 
p\'aginas es lo que le otorga su potencia, pero tambi\'en lo que causa mayores
problemas de seguridad: un fallo en estos programas suele permitir a cualquier
visitante de las p\'aginas ejecutar \'ordenes en el sistema. Los errores m\'as
habituales en un CGI provienen de los datos recibidos desde el navegador del
cliente: un simple formulario, en el que el visitante rellena ciertos campos,
puede ser una puerta de acceso a nuestro sistema; es necesario comprobar la
validez de todos y cada uno de los datos le\'{\i}dos antes de que sean 
procesados. Por ejemplo, imaginemos un CGI que pida un nombre de usuario por
teclado y a continuaci\'on ejecute un {\tt finger} contra ese nombre de usuario
y muestre el resultado en el navegador; >que suceder\'{\i}a si el visitante
introduce como nombre de usuario {\tt `toni;cat /etc/passwd'}? Es posible que
se ejecute el {\tt finger} a {\tt toni}, pero a continuaci\'on se vuelque el 
fichero de contrase\~nas 
simplemente porque no se ha tenido la precauci\'on de ignorar los caracteres
especiales para el {\it shell} (recordemos que un {\tt `;'} en Unix separa
varias \'ordenes en una misma l\'{\i}nea); este ejemplo, que hoy en d\'{\i}a
parece absurdo, ha estado presente en algunos servidores durante mucho tiempo.
Cualquier CGI es susceptible de presentar problemas de seguridad sin importar
el lenguaje en que se haya escrito (\cite{kn:gun96}); por tanto, es muy
importante preocuparse de mantener actualizado el \'arbol
de CGIs (no copiarlo completamente al actualizar la versi\'on de demonio),
e incluso revisar los programas m\'as importantes en busca de posibles {\it 
bugs}. Otra medida de seguridad b\'asica es ejecutar el demonio servidor bajo
la identidad de un usuario con privilegios m\'{\i}nimos para que todo funcione
correctamente, pero nunca como {\tt root}; generalmente, el usuario {\tt
nobody} suele ser m\'as que suficiente: recordemos que los CGIs se ejecutan 
bajo la identidad del 
usuario propietario del demonio, por lo que si ese propietario es el 
administrador un potencial atacante podr\'{\i}a ejecutar cualquier aplicaci\'on
como {\tt root} del sistema.\\
\\Para garantizar la seguridad de los datos que circulan entre un cliente y
el servidor es casi obligatorio cifrar dichos datos (otras medidas, como
asegurar f\'{\i}sicamente la red, suelen ser impracticables) mediante SSL
({\it Secure Socket Layer}), un protocolo desarrollado por Netscape 
Communications para cifrar informaci\'on al enviarla por la red y descifrarla
antes de ser utilizada en el cliente; en la actualidad, se est\'a viendo 
relegado a un segundo plano a causa de los certificados digitales, aunque sigue
siendo una excelente opci\'on para administraci\'on remota y para transmitir
informaci\'on confidencial en redes de prop\'osito general.\\
\\En \'ultimo lugar es necesario hablar de la seguridad desde el punto de
vista del cliente que visita p\'aginas {\it web}; para el usuario, un servidor
es seguro si protege la informaci\'on que recibe y env\'{\i}a hacia \'el,
manteniendo su privacidad, y si no conduce al usuario a descargar programas 
maliciosos -- generalmente virus -- en su equipo; si sucede lo contrario, la
compa\~n\'{\i}a responsable de las p\'aginas se enfrenta a una importante
p\'erdida de imagen -- aparte de posibles problemas judiciales -- de cara a
sus usuarios: simplemente imaginemos que salta a los medios un fallo de 
seguridad en la versi\'on electr\'onica de cierto banco; ser\'a dif\'{\i}cil
que todos sus usuarios sigan manteniendo la suficiente confianza en \'el como
para guardar all\'{\i} su dinero. Tambi\'en es necesario hablar de los {\it
applets} hostiles -- o simplemente de los mal dise\~nados -- que en muchas
ocasiones llegan a detener todas las copias del navegador en memoria; aunque
sus implicaciones de seguridad no suelen ser muy graves, la p\'erdida de imagen
de la compa\~n\'{\i}a es tambi\'en considerable en estos casos.\\
\\En muy pocas m\'aquinas se pueden permitir el lujo de deshabilitar este 
servicio, ya que como hemos dicho es de los m\'as utilizados actualmente; no
obstante, por alguna extra\~na raz\'on -- personalmente no la llego a 
comprender -- en algunos clones de Unix (por ejemplo, ciertas variantes de 
Linux) el servicio {\sc http} est\'a activado por defecto a\'un a sabiendas de
que muchos de los usuarios de este sistema van a utilizarlo en su casa o como
estaci\'on de trabajo independiente, donde evidentemente no es habitual -- ni
necesario en la mayor\'{\i}a de ocasiones -- ofrecerlo. Por supuesto, en estos
casos es importante detener el demonio {\tt httpd} y evitar que se vuelva a 
iniciar con el arranque de la m\'aquina, modificando el {\it script} 
correspondiente. Siempre hemos de recordar que hemos de ofrecer s\'olo los
servicios {\bf imprescindibles} en cada sistema.
\section{Los servicios r-$\ast$}
Los servicios {\tt r-*} de Unix BSD (aparecieron inicialmente en la versi\'on
4.2 de esta variante de Unix) son herramientas con una parte cliente y 
una servidora que permiten la conexi\'on remota entre m\'aquinas, principalmente
para servicios de terminal remota y transferencia de ficheros. Las herramientas
clientes son {\tt rsh}, {\tt rlogin} y {\tt rcp}, mientras que las servidoras
son demonios como {\tt rexecd}, {\tt rshd} o {\tt rlogind} (en algunas versiones
de Unix, con {\tt in.} delante del nombre del demonio); {\tt rdist} y {\tt 
rdistd}, otro par de estas herramientas {\tt r-*}, no los vamos a tratar 
aqu\'{\i}.\\
\\{\tt rlogin} (puerto 513, {\sc tcp}) se utiliza como terminal virtual de un 
sistema Unix , de una forma muy parecida a {\sc telnet}. {\tt rsh} (puerto 514,
{\sc tcp}) es utilizado para ejecutar comandos en una m\'aquina remota sin 
necesidad de acceder a ella, y {\tt rcp} (v\'{\i}a {\tt rsh}) para copiar
ficheros entre diferentes m\'aquinas:
\begin{quote}
\begin{verbatim}
luisa:~# rlogin -l toni rosita

Overflow on /dev/null, please empty the bit bucket.

rosita:~$ exit
logout
rlogin: connection closed.
luisa:~# rsh -l toni rosita id
uid=1000(toni) gid=100(users) groups=100(users)
luisa:~# rcp prueba.tex toni@rosita:/tmp/
luisa:~#
\end{verbatim}
\end{quote}
Como vemos, la \'ultima orden no ha solicitado ninguna contrase\~na; ha
copiado el fichero local {\tt `prueba.tex'} en el directorio {\tt /tmp/} del
sistema remoto, bajo la identidad del usuario {\tt toni}. A continuaci\'on
veremos por qu\'e no se ha pedido clave para realizar esta acci\'on.\\
\\Estos servicios pretenden evitar el tr\'ansito de contrase\~nas por la red,
ya que este movimiento de claves implica molestias a los usuarios y tambi\'en
problemas de seguridad; para conseguirlo, entran en juego lo que los 
dise\~nadores del sistema de red de Unix BSD denominaron `m\'aquinas fiables'
y `usuarios fiables': cualquier usuario, puede hacer uso 
de recursos de una m\'aquina remota sin necesidad de una clave si su conexi\'on 
proviene de una m\'aquina {\it fiable} o su nombre de usuario es {\it 
fiable}.\\ 
\\Una m\'aquina se puede considerar {\it fiable} de dos formas: o bien su
nombre se encuentra en {\tt \\/etc/hosts.equiv}, o bien se encuentra en un
fichero denominado {\tt .rhosts} y situado en el {\it \$HOME} de alg\'un
usuario. Si estamos en el primer caso, cualquier usuario (excepto el {\it root})
del sistema remoto -- y {\it fiable} -- puede hacer acceder a nuestro equipo 
bajo el mismo {\it login} que tiene en el primero, sin necesidad de claves. En
el segundo caso, utilizando los ficheros {\tt .rhosts}, cualquier usuario del
sistema remoto podr\'a conectar al nuestro pero s\'olo bajo el nombre de 
usuario en cuyo {\it \$HOME} se encuentra el archivo. Por ejemplo, imaginemos
la siguiente configuraci\'on:
\begin{quote}
\begin{verbatim}
rosita:~# cat /etc/hosts.equiv
luisa
rosita:~# cat ~toni/.rhosts
anita
rosita:~#
\end{verbatim}
\end{quote}
En esta situaci\'on, cualquier usuario de {\tt luisa} puede acceder a {\tt 
rosita} si su nombre de usuario es el mismo; adem\'as, el usuario {\tt 
toni} de {\tt anita} puede tambi\'en conectar a {\tt rosita} sin necesidad de 
ninguna contrase\~na:
\begin{quote}
\begin{verbatim}
anita:~$ rlogin rosita

In the long run, every program becomes rococo, and then rubble.
                -- Alan Perlis

rosita:~$ id
uid=1000(toni) gid=100(users) groups=100(users)
rosita:~$
\end{verbatim}
\end{quote}
Aparte de m\'aquinas fiables hab\'{\i}amos hablado de usuarios fiables; la
idea es la misma que antes, pero aplic\'andola ahora a nombres de usuario junto
a (o en lugar de) nombres de m\'aquina. Podemos indicar estos nombres tanto en 
{\tt /etc/hosts.equiv} como en los archivos {\tt .rhosts}; no obstante, la 
primera
opci\'on {\bf no} es recomendable, ya que estar\'{\i}amos permitiendo al usuario
fiable del sistema remoto acceder sin contrase\~na {\bf a cualquier cuenta de
nuestra m\'aquina}. De esta forma, si deseamos crear usuarios fiables de 
sistemas remotos, es necesario hacerlo en los archivos {\tt .rhosts}. Por 
ejemplo, imaginemos que el usuario {\tt toni} de nuestra m\'aquina tiene un 
nombre de usuario distinto ({\tt antonio}) en un sistema remoto, y desea 
establecer una relaci\'on de confianza; para ello crear\'a en su {\it \$HOME}
el siguiente archivo {\tt .rhosts}:
\begin{quote}
\begin{verbatim}
rosita:~# cat ~toni/.rhosts
amparo antonio
rosita:~#
\end{verbatim}
\end{quote}
Entonces, desde la m\'aquina {\tt amparo} el usuario {\tt antonio} podr\'a 
acceder a la cuenta de {\tt toni} en nuestro sistema sin utilizar contrase\~nas:\begin{quote}
\begin{verbatim}
amparo:~$ id
uid=102(antonio) gid=10(staff)
amparo:~$ rlogin -l toni rosita

It is practically impossible to teach good programming style to
students that have had prior exposure to BASIC: as potential
programmers they are mentally mutilated beyond hope of
regeneration.
                -- Dijkstra

rosita:~$ id
uid=1000(toni) gid=100(users) groups=100(users)
rosita:~$
\end{verbatim}
\end{quote}
Como podemos ver, las relaciones de confianza entre equipos Unix pueden ser
muy \'utiles y c\'omodas, pero al mismo tiempo muy peligrosas: estamos confiando
plenamente en sistemas remotos, por lo que si su seguridad se ve comprometida
tambi\'en se ve la nuestra. Las m\'aquinas fiables se han de reducir a equipos
de la misma organizaci\'on, y administrados por la misma persona; adem\'as, 
es necesario tener siempre presente que si tenemos habilitados los servicios
{\tt r-*} cualquier usuario puede establecer relaciones de confianza, lo que
puede suponer una violaci\'on de nuestra pol\'{\i}tica de seguridad. Es 
conveniente chequear los directorios {\it \$HOME} en busca de ficheros {\tt
.rhosts} (en la secci\'on \ref{scriptrhosts} se presentaba un {\it shellscript}
que convenientemente planificado puede ayudarnos en esta tarea); muchos 
administradores prefieren no complicarse buscando estos ficheros, y configuran
sus sistemas para que en cada {\it \$HOME} exista un fichero con este nombre,
propiedad de {\tt root} y con modo 000: as\'{\i} los usuarios no tienen 
ocasi\'on de otorgar confianza a sistemas remotos. Esto se puede conseguir 
con el siguiente {\it shellscript}:
\begin{quote}
\begin{verbatim}
#!/bin/sh
for i in `cat /etc/passwd |awk -F: '{print $6}'`; do
    cd $i
    > .rhosts
    chmod 0 .rhosts
done
\end{verbatim}
\end{quote}
Las relaciones de confianza son transitivas: si una m\'aquina conf\'{\i}a en
otra, lo hace tambi\'en en todas en las que conf\'{\i}a ella. De esta forma
se crean anillos de confianza entre m\'aquinas, y como las relaciones suelen
estar basadas en el nombre del equipo se trata de objetivos ideales para un
atacante mediante {\it IP Spoofing}: si un pirata consigue hacer pasar su equipo
por uno de los confiables, autom\'aticamente ha conseguido acceso -- casi 
ilimitado -- al resto de las m\'aquinas.
\section{X�Window}
El entorno {\it X Window} proporciona herramientas incre\'{\i}blemente potentes,
pero que si no son correctamente configuradas pueden convertirse en peligrosas.
Este sistema est\'a formado por una serie de piezas que trabajan 
conjuntamente para ofrecer al usuario final un interfaz gr\'afico:
\begin{itemize}
\item La m\'as importante de ellas, sobre todo desde el punto de vista de la 
seguridad es el {\bf servidor X}. Este programa generalmente se ejecuta en la 
terminal de usuario, y tiene como funci\'on principal ofrecer unas primitivas 
b\'asicas de dibujo (trazado de rectas, relleno de \'areas\ldots) sobre la 
pantalla; adem\'as gestiona eventos de teclado y rat\'on.
\item Las {\bf aplicaciones X} son programas de usuario que lanzan llamadas 
contra un servidor {\it X}. Mientras que el servidor se ejecuta habitualmente 
en la
terminal desde donde conecta el usuario las aplicaciones se pueden lanzar desde
el mismo equipo o tambi\'en desde una m\'aquina m\'as potente, de forma que
aprovechamos la capacidad de procesamiento de ese equipo pero visualizamos el
resultado en la terminal gr\'afica; en este caso se ha de indicar a los clientes
la ubicaci\'on del servidor, mediante la variable de entorno {\it \$DISPLAY} o
mediante la opci\'on de l\'{\i}nea de comandos {\tt `-display'}.
\item El {\bf gestor de ventanas} es un caso particular de aplicaci\'on, ya que
se encarga de ofrecer un entorno de trabajo m\'as amigable al usuario que est\'a
trabajando en la terminal: dibujo de marcos, men\'us, cerrado de ventanas\ldots
\end{itemize}
Es el servidor {\it X Window} quien establece su pol\'{\i}tica de seguridad para
permitir a determinados clientes utilizar sus servicios. Para ello existen dos
mecanismos b\'asicos: la autenticaci\'on por testigo y la autenticaci\'on por
m\'aquina (\cite{kn:fis95}; otros esquemas, como {\sc sun-des-1}, no los vamos
a contemplar aqu\'{\i}.
\subsection{Autenticaci\'on por m\'aquina}
La autenticaci\'on por m\'aquina cliente ({\it host authentication}) es el
mecanismo m\'as simple, pero la seguridad que proporciona es muy limitada; es
\'util en entornos donde los clientes {\it X} se ejecutan o bien en estaciones 
monousuarios o bien en equipos donde todos los usuarios son confiables 
(\cite{kn:vic94}). Adem\'as, en sistemas antiguos es el \'unico modelo de 
seguridad disponible, por lo que en ocasiones no queda m\'as remedio que 
limitarse a \'el. Funciona configurando el servidor para permitir conexiones
a \'el provenientes de una lista de m\'aquinas, por ejemplo con la orden {\tt
xhosts}:
\begin{quote}
\begin{verbatim}
anita:~# xhost +luisa
luisa being added to access control list
anita:~#
\end{verbatim}
\end{quote}
Si ejecutamos la sentencia anterior en la m\'aquina donde se ejecuta el 
servidor, {\bf cualquier usuario} del sistema remoto estar\'a autorizado a 
lanzar aplicaciones contra \'el\footnote{En determinados casos, por ejemplo
utilizando autenticaci\'on {\sc sun-des-1} o utilizando {\it Kerberos}, es 
posible indicar nombres de usuario autorizados de cada sistema; no lo veremos
aqu\'{\i} por no ser el caso m\'as habitual.}:
\begin{quote}
\begin{verbatim}
luisa:~# xterm -display anita:0.0 &
[1] 11974
luisa:~#
\end{verbatim}
\end{quote}
La orden {\tt xhost} sin opciones nos dar\'a una lista de los clientes que
pueden lanzar aplicaciones contra el servidor, mientras que la opci\'on 
especial {\tt `+'} deshabilitar\'a este control de acceso, algo que 
evidentemente no es recomendable: cualquier usuario de cualquier sistema 
podr\'a utilizar nuestro servidor:
\begin{quote}
\begin{verbatim}
anita:~# xhost
access control enabled, only authorized clients can connect
LOCAL:
INET:anita
INET:localhost
INET:luisa
anita:~# xhost +
access control disabled, clients can connect from any host
anita:~# xhost 
access control disabled, clients can connect from any host
LOCAL:
INET:anita
INET:localhost
INET:luisa
anita:~#
\end{verbatim}
\end{quote}
Una medida de seguridad b\'asica utilizando este modelo es habilitar la 
m\'aquina en nuestra lista de {\it hosts} s\'olo el tiempo necesario para que
el cliente arranque, y deshabilitarla despu\'es; as\'{\i} la ejecuci\'on de
la aplicaci\'on cliente funcionar\'a normalmente, pero no se podr\'an lanzar
nuevas peticiones al servidor. Tambi\'en para eliminar una direcci\'on de la
lista utilizamos la orden {\tt xhost}:
\begin{quote}
\begin{verbatim}
anita:~# xhost
access control enabled, only authorized clients can connect
LOCAL:
INET:anita
INET:localhost
INET:luisa
anita:~# xhost -luisa
luisa being removed from access control list
anita:~# xhost
access control enabled, only authorized clients can connect
LOCAL:
INET:anita
INET:localhost
anita:~#
\end{verbatim}
\end{quote}
De esta forma, cuando alguien intente lanzar una aplicaci\'on contra nuestro
servidor desde un sistema no autorizado ver\'a un mensaje de error similar al
siguiente:
\begin{quote}
\begin{verbatim}
luisa:~# xterm -display anita:0.0 
Xlib: connection to "anita:0.0" refused by server
Xlib: Client is not authorized to connect to Server
Error: Can't open display: anita:0.0
luisa:~#
\end{verbatim}
\end{quote}
Como hemos dicho, este modelo de seguridad es demasiado vulnerable; por un lado,
estamos autenticando clientes en base a una direcci\'on o a un nombre de 
m\'aquina, algo f\'acilmente falsificable por un atacante. Por otro, aunque 
los usuarios de los sistemas a los que permitimos utilizar nuestro servidor sean
conocidos, fiables, y amantes de la naturaleza, nada nos demuestra que sus 
sistemas sean seguros, por lo que si sus equipos se ven comprometidos, nuestro
servidor tambi\'en.
\subsection{Autenticaci\'on por testigo}
Este mecanismo de {\it X Window} es el m\'as seguro, y por tanto el m\'as
recomendado; en \'el, el servidor controla el acceso de los clientes mediante
una {\it `cookie'} {\sc mit-magic-cookie-1}, que no es m\'as que un c\'odigo 
de acceso aleatorio de 128 bits en
un formato legible por la m\'aquina: esta {\it cookie} actua como un {\it 
password} temporal, de forma que s\'olo los clientes que conozcan ese {\it
password} podr\'an acceder al servidor. La {\it cookie} es generada por {\tt
xdm} o por el propio usuario al principio de cada sesi\'on, con {\tt xauth}, y
guardada en el fichero {\tt \$HOME/.Xauthority}; a partir de ese momento, los
programas clientes leer\'an su valor y lo enviar\'an al servidor cada vez que
deseen conectar a \'el. Podemos comprobar que poseemos -- al menos -- la {\it 
cookie} correspondiente a nuestro {\it display} con una orden como la
siguiente:
\begin{quote}
\begin{verbatim}
luisa:~# xauth list
luisa:0  MIT-MAGIC-COOKIE-1  8c1d09aab44573a524467c4e8faaaeb5
luisa/unix:0  MIT-MAGIC-COOKIE-1  8c1d09aab44573a524467c4e8faaaeb5
luisa:~#
\end{verbatim}
\end{quote}
El comando anterior, {\tt xauth}, se utiliza para manejar la informaci\'on de
las {\it cookies} de cada usuario; por ejemplo, un uso muy habitual es la 
transferencia de {\it cookies} a m\'aquinas remotas, para que puedan as\'{\i}
conectar al servidor {\it X} de un determinado equipo. Para ello debemos 
extraer la {\it cookie} de nuestro {\it \$DISPLAY} y enviarla al fichero 
{\tt \$HOME/.Xauthority} del sistema remoto, con una orden como esta:
\begin{quote}
\begin{verbatim}
luisa:~# xauth extract - $DISPLAY | ssh anita -l toni xauth merge -
luisa:~#
\end{verbatim}
\end{quote}
Este mecanismo tiene principalmente dos problemas de seguridad: por un
lado, las {\it cookies} se transmiten en texto claro por la red, por lo que son
susceptibles de ser interceptadas; por otro, al estar guardadas en el fichero
{\tt \$HOME/.Xauthority}, cualquiera que lo pueda leer tendr\'a acceso a ellas:
es muy importante que este archivo tenga permiso de lectura y escritura s\'olo
para su propietario, y que tambi\'en tomemos precauciones si los directorios
{\it \$HOME} de los usuarios son exportados v\'{\i}a NFS.
