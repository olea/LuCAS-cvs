%%$Id: Presentacion.1.tex,v 1.1.1.1 2003/03/10 23:28:22 juanfernandez Exp $
\documentclass[a4paper,11pt]{article}
\usepackage{palatino}
\usepackage[T1]{fontenc}
%\usepackage{marvosym}
\usepackage[latin1]{inputenc}
%\usepackage{babel}	% Hay problemas al hacer TOC con spanish
%\usepackage{geometry}
%\geometry{a4paper,tmargin=2.54cm,bmargin=2.54cm,lmargin=3cm,rmargin=3cm}
\usepackage{indentfirst}
\usepackage{soul}
\usepackage{xspace,colortbl}
\usepackage[screen,panelright,code,paneltoc,sectionbreak,spanish]{pdfscreen}
%\usepackage[display,slifonts]{texpower}
%\usepackage{hyperref}	% No hace falta
\usepackage[display,verbose]{texpower}


%%%%%%%%%%%%%%%%%%%%%%%%%%%%%%%%%%%%%%%%%%%%%%%%%%%%%%%%%%%%%%%%%%%%%%%%
%%% The panel
\margins{.65in}{.65in}{.65in}{.65in}
\screensize{6.25in}{8in}
\emblema{Pensador.pdf}
\urlid{es.tldp.org}
\changeoverlay		% El fondo ir� rotando
\paneloverlay{but.pdf}	% background of the panel
%\overlay{logo.pdf}
\overlay{overlay4.pdf}  % background of the screen
\def\pfill{\vskip6pt}
\makeatletter
    \providecommand{\LyX}{L\kern-.1667em\lower.25em\hbox{Y}\kern-.125emX\@}
    \newcommand{\noun}[1]{\textsc{#1}}
    %\addto\extrasspanish{\bbl@deactivate{~}}
\makeatother

\renewcommand\floatpagefraction{1}
\renewcommand\textfraction{0}
%\def\pdfscreen{\texttt{\small\color{section1}pdfscreen}\xspace}

\setcounter{secnumdepth}{-1}

%%%%%%%%%%%%%%%%%%%%%%%%%%%%%%%%%%%%%%%%%%%%%%%%%%%%%%%%%%%%%%%%%%%%%%%%
\begin{document}

\title{\color{section0}{{\Huge GNU Linux para usuarios}. \\
    Introducci�n al uso y la cultura \\
    del software libre}}

\author{\color{section1}\Large Juan Rafael Fern�ndez Garc�a \\
        {\small\href{mailto:juanrafael.fernandez[@]hispalinux.es}
        {\color{section1}\texttt{juanrafael.fernandez[@]hispalinux.es}}}}
\date{Curso 2002-2003}
\maketitle

\vfill

\begin{abstract}
El software libre se impone. Razones econ�micas, pol�ticas y t�cnicas
hacen que cada vez sea m�s evidente la necesidad de que la administraci�n 
adopte medidas encaminadas a la introducci�n de soluciones
relacionadas con el software de fuente abierta; es necesario por
tanto formar a los docentes para que puedan utilizar los recursos libres.
Ha quedado demostrado que una soluci�n triunfa si logra que los
usuarios se familiaricen con ella, si es percibida como la forma natural de
trabajar. Esta naturalidad es la que pretendemos conseguir: mostrando
la calidad de los productos realizados con software libre (documentos,
gr�ficos, presentaciones\ldots{}) y su superioridad t�cnica como sistema
operativo multiusuario y seguro.
\end{abstract}

\vfill

\pageTransitionReplace{}
%%%%%%%%%%%%%%%%%%%%%%%%%%%%%%%%%%%%%%%%%%%%%%%%%%%%%%%%%%%%%%%%%%%%%%%%
\section[Nombres]{El nombre y un poco de historia}

\subsection{El proyecto GNU}

\href{http://es.tldp.org/Articulos/0000otras/doc-traduccion-libre/doc-traduccion-libre/ch-traductor.html}
    {Historia del proyecto y del concepto de software libre}

\subsection[Y Linus cre� Linux]{1991: Y Linus cre� Linux}

\href{http://es.tldp.org/Manuales-LuCAS/LIPP2/lipp-2.0-beta-html/node16.html}
    {Breve historia de Linux}

\subsection{Woody, Sarge y Sid se apellidan Debian}

\href{http://www.es.debian.org/}{P�ginas en espa�ol del proyecto Debian}


%%%%%%%%%%%%%%%%%%%%%%%%%%%%%%%%%%%%%%%%%%%%%%%%%%%%%%%%%%%%%%%%%%%%%%%%
\section[Software libre]{�Por qu� software libre?}
\pageTransitionDissolve{}

\subsection*{Razones econ�micas}

\liststepwise%
{%
    \begin{itemize}
    \item Coste de licencias, sistemas operativos y utilidades

    \step{\item Coste de mantenimiento y auditor�a de seguridad

	\href{http://www.hispalinux.es/modules.php?op=modload&name=phpWiki&file=index&pagename=SLAdministracionInformes}%
	{Informes de la campa�a pro software libre en la administraci�n, 
	de Hispa\-Linux}%
    }

    \step{\item \href{http://es.tldp.org/Otros/catedral-bazar/}
	{El modelo \emph{bazar} de desarrollo abierto}}

    \step{\item Oportunidades para la peque�a empresa andaluza
    
        \href{https://listas.hispalinux.es/pipermail/sl-administracion/2003-January/001898.html}
	{Proposici�n no de ley sobre software libre en Andaluc�a}%
    }

    \end{itemize}
}

\pause

\subsection*{Razones �ticas}

El software es conocimiento: 
\href{http://www.gnu.org/philosophy/philosophy.es.html}
    {Filosof�a del proyecto GNU}

\pause

\subsection*{Razones t�cnicas}

\liststepwise%
{%
    \begin{itemize}
    \item c�digo abierto

    \href{http://www.hevanet.com/peace/microsoft-es.htm}
	{Windows XP muestra la direcci�n que Microsoft est� tomando}


    %\step{\item libre de virus}
    \step{\item Es un Unix en r�pido desarrollo}

    \end{itemize}
}



%%%%%%%%%%%%%%%%%%%%%%%%%%%%%%%%%%%%%%%%%%%%%%%%%%%%%%%%%%%%%%%%%%
\section[Problemas]{Los problemas del software libre}
\pageTransitionSplitHO{}
\liststepwise%
{%
    \begin{itemize}
    \item no funciona todo el hardware

    \step{\href{http://192.168.1.4/doc/HOWTO/en-html/Hardware-HOWTO/index.html}
	{ Linux Hardware Compatibility HOWTO}

    \href{http://lhd.zdnet.com/}{Linux Hardware Database}}

    \step{\item no tiene soporte}

    \step{--- Es el c�digo propietario el que no tiene soporte
	
	--- Comunidad entusiasta
	
	\href{http://www.linux-malaga.org/}{Linux M�laga}
	
	\href{http://lists.debian.org/debian-user-spanish/}
	{Lista de usuarios de Debian en espa�ol}
	
	\href{http://www.adala.org}
	{Asociaci�n para la Difusi�n y el Avance del Software Libre de Andaluc�a}
	
	\href{http://www.hispalinux.org}
	{Asociaci�n de Usuarios Espa�oles de GNU/LiNUX}}

    \step{\item es m�s dif�cil}
        
    \step{--- Es m�s dif�cil conducir un trailer que llevar un seiscientos

        --- La curva de aprendizaje lleva hacia alg�n sitio

        --- Con los nuevos desarrollos esta tesis ha dejado de ser cierta}
    \end{itemize}
}

\end{document}
