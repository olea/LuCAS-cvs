% Comando para incluir figuras con su comentario
% Primer argumento: Comentario
% Segundo argumento: Nombre del archivo eps (sin la extensi�n), debe estar
%		     en el directorio ./imagenes/eps/
%
% Sintaxis: \figura{arg1}{arg2}
%
% Para hacer referencia a una im�gen se utiliza el comando:
% \ref{fig:<arg2>}
% donde <arg2> es el segundo argumento que se le pas� a la imagen (es decir,
% el nombre de la im�gen sin su extensi�n)
%
\newcommand{\figura}[2]{
   \begin{figure}
   \begin{center}
   \includegraphics[scale=0.5]{imagenes/eps/#2.eps}
   \caption{#1}\label{fig:#2}
   \end{center}
   \end{figure}
}


%% Comandos generales de formateo de diferentes expresiones

% Comando para referirse a nombres de usuario
\newcommand{\usuario}[1]{{\tt #1}}

% Comando para referirse a comandos (viva la recursi�n!)
\newcommand{\comando}[1]{{\tt #1}}

% Comando para referirse a sitios web
\newcommand{\sitio}[1]{{\tt #1}}

% Comando para referirse a botones
\newcommand{\boton}[1]{\fbox{\tt #1}}

% Comando para referirse a men�es y sus opciones (gracias Sara!)
\newcommand{\menu}[1]{{[}{\tt #1}{]}}

% Comando para referirse a...�rboles? (por favor chequear con Nicol�s)
\newcommand{\rama}[1]{\emph{#1}}
