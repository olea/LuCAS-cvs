%%%%%%%%%%%%%%%%%%%%%%%%%%%
% Secci�n: Requerimientos %
%%%%%%%%%%%%%%%%%%%%%%%%%%%
\section{Requerimientos}

Como es de esperarse las \emph{consolas} y \emph{terminales} son mucho
m�s antiguas que las pantallas con entornos gr�ficos, por lo tanto los
requerimientos son menores.

Los requerimientos m�nimos para usar GNU/Linux son, tan solo, una 386
con 4 MB de RAM. Esto no quiere decir que va a andar r�pido, pero va a
andar. En algunos trabajos no es necesario un equipo m�s grande. Es
com�n pensar que si no se tiene la �ltima m�quina que est� en el
mercado junto los �ltimos perif�ricos no se puede usar nada. Este es
un concepto err�neo.

Tener en cuenta el uso que se dar� a la computadora. Por ejemplo para
compartir Internet entre varios amigos o dentro de una peque�a
empresa, una 386 con 4 MB de Ram ser�a suficiente. En cambio para la
creaci�n de planos en un sistema CAD esto no ser�a un equipo �ptimo.

Si bien la computadora funciona con esa configuraci�n, el aspecto es
similar al de terminales hace varios a�os atr�s. Para usuarios reci�n
iniciados y poco familiarizados con terminales, suele parecer un entorno
poco amigable. Pero si se necesita rapidez con poco hardware no queda 
otra opci�n.

En nuestro curso vamos a usar una \emph{terminal gr�fica}\footnote{En
realidad es un servidor X/Window}, con KDE y eso requiere
aproximadamente un Pentium (166MHz o 200MHz) con 32MB de RAM para
trabajar. Y para trabajar c�modamente con StarOffice es necesario 64MB
de RAM.

\subsection{M�s informaci�n}

Para mayor detalle sobre \emph{terminales gr�ficas} ejecutar
\comando{man X} que detalla como funciona el servidor que atiende 
a KDE, llamado \textbf{X/Window} (es sin `s' final)

Para mayor detalle sobre \emph{KDE} ir a \rama{K-Ayuda KDE}
