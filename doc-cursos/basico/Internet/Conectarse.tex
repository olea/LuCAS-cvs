%%%%%%%%%%%%%%%%%%%%%%%
% Secci�n: Conectarse %
%%%%%%%%%%%%%%%%%%%%%%%
\section{Conectarse}

Lo primero que hay que hacer es configurar una cuenta de Internet.
Los pasos son relativamente simples usando el \comando{kppp}.

El \comando{kppp} se puede ejecutar desde terminal o clickeando en 
\rama{K-Internet-Conexi�n a Internet}.

\figura{Vista inicial del \comando{kppp}}{kppp-Inicial}
\figura{Configuracion del \comando{kppp}}{kppp-Configuracion}
\figura{Nueva conexi�n}{kppp-NuevaConexion}

Luego hay que ir a \boton{Configuraci�n}
(fig. \ref{fig:kppp-Configuracion}) para crear una conexi�n a Internet
con un click en \boton{Nueva} (fig. \ref{fig:kppp-NuevaConexion}).

Pudiendo as� llenar los datos del provedor de Internet. Los datos
importantes son: el \emph{nombre} y el \emph{n�mero a marcar}.  Luego
hay que ir a la leng�eta \emph{Servidor Nombres} y poner la direcci�n
IP del \emph{Servidor de Nombres} o \emph{Servidor DNS} que es parte
de la informaci�n que nos brinda el proveedor de Internet.

\figura{Conexi�n a internet con todos los datos}{kppp-completo}

Una vez configurado ya se puede elegir como parte de las posibles
configuraciones en \emph{Conectar con}. S�lo falta el nombre de
usuario y la clave como muestra la figura \ref{fig:kppp-completo}.

Con s�lo apretar \boton{Conectar} se deber�a conectar sin problemas
(fig \ref{fig:kppp-conectando}). 

\figura{Conect�ndose a Internet}{kppp-conectando}

De todas formas, muchas veces la realidad es muy distinta a la
teor�a. Esta es una lista de posibles problemas:

\begin{itemize}
\item Problemas con el modem
\item Problemas con pppd
\item Problemas relativos al proveedor
\end{itemize}

No pretende ser una lista exhaustiva, tan s�lo son los problemas 
m�s comunes.

\subsection{Problemas con el modem}

Antes de empezar a ilusionarse con conectarse a Internet, hay que
saber si el modem es o no un \emph{Winmodem}. Esta l�nea especial de
modems (o no tan modems) utilizan \emph{drivers} propietarios que la
mayor�a s�lo funcionan en Windows. En caso de una futura compra de
modem, lo primero que hay que fijarse es que no sea uno de estos.

Sabiendo que no es un \emph{Winmodem} el modem instalado, hay muchos
problemas que pueden tener los modems. Desde los m�s simples, como ser
que la l�nea de tel�fono est� desconectada, hasta los m�s complicados
y misteriosos que cuesta bastante encontrar sus causas. En el archivo
\comando{/usr/doc/HOWTO/Modem-HOWTO} hay mucha referencia sobre modems
y en especial un cap�tulo dedicado a problemas (Troubleshooting).

Para quienes no leen ingl�s, existe una traducci�n de ese archivo,
junto con varios HOWTOs (traducidos al castellano como COMOs),
llamado \comando{Modem-COMO} y puede estar en el directorio
\comando{/usr/doc/HOWTO/translations/es}.

\subsection{Problemas con el pppd}

Luego de pulsar \boton{Conectar}, el programa de \emph{Conexi�n a Internet},
llamado \comando{kppp}, transfiere los datos ingresados al
\comando{pppd} para que este realice la conexi�n en si.

El \comando{pppd} es llamado tambi�n \emph{demonio pppd} y en el caso 
de que no se pueda conectar, puede surgir con varios cuadros de di�logo.

Uno muy com�n y con poca referencia es el que dice: ``El demonio pppd
muri� inesperadamente''. Esta frase suena absurda en especial a
usuarios reci�n iniciados en Linux.

Normalmente cuando el proveedor de Internet corta la conexi�n sin
motivo, aparece el demonio muerto. Y esto puede ser por diversos
motivos:

\begin{itemize} 
 \item El provedor da ocupado y el modem no detecta que es se�al de
ocupado. (muchas compan�as tienden a poner grabaciones).

 \item Al ingresar una clave incorrecta, el provedor no da explicaci�n
y corta la comunicaci�n. 

 \item La cuenta puede estar siendo usada y el provedor s�lo permite 
una conexi�n por cuenta.
\end{itemize}

\subsection{Problemas relativos al provedor}

Una vez conectado, es probable que no ``ande'' Internet, o al ingresar
un nombre de m�quina. Ejemplo: en el navegador tipear
\sitio{http://www.google.com} y que no se conecte a ese servidor.

Esto se debe a que el \emph{Servidor de nombres} o \emph{Servidor DNS}
no est� bien configurado. Hay que revisar las configuraciones
descriptas anteriormente.



