%%%%%%%%%%%%%%%%%%%%%%%
% Secci�n: Navegaci�n %
%%%%%%%%%%%%%%%%%%%%%%%
\section{Navegaci�n}

Una de las principales actividades en Internet es la navegaci�n.  El
programa m�s usado es Netscape. Este no es software libre, por lo
tanto algunas distribuciones no lo traen instalado, como por ejemplo
Debian. De todas formas esta disponible para bajar desde
\sitio{http://www.netscape.com}. 

En la distribuci�n Red Hat, en KDE la ubicaci�n del Netscape es
 \rama{K-Red Hat-Internet-Netscape Communicator}, si bien en cada
 distribuci�n cambian los men�es, una forma segura de encontrarlo es
 tipeando en una terminal \comando{netscape}.

La primera vez que se ejecuta \comando{netscape}
(fig. \ref{fig:netscape-www.lunix.com.ar} ) aparecer� una ventana con la
licencia y los botones \boton{Accept} y \boton{Do Not
Accept}. Aclaramos que no es sotfware libre, no tiene la licencia GNU
que viene con la mayor�a del software en Linux.

\figura{P�gina de LUNIX en Netscape}{netscape-www.lunix.com.ar}

%No es la intenci�n de este curso dar una clase de c�mo navegar, de
%todas formas vamos a presentar algunos tips: 

Otra aplicaci�n que sirve para navegar en p�ginas no muy complejas es
el \comando{kfm}, que no se lo conoce por ese nombre sino m�s bien por
\emph{Navegador}(fig. \ref{fig:kfm-www.lunix.com.ar}). 
Cada vez que uno navega entre las carpetas locales utiliza ese
programa. Como por ejemplo cuando se clickea en la opci�n de menu
\rama{Directorio Personal}. Y aparecer� una ubicaci�n similar a {\tt
file:/home/usuario}. Si se reemplaza por {\tt http://} se podr�
navegar por la web.

Esta en desarrollo y pretende llegar a reemplazar al software
comercial.

\figura{P�gina de LUNIX en \comando{kfm}}{kfm-www.lunix.com.ar}

Otra alternativa es la navegaci�n en texto, si bien a muchos no les
parecer� atractiva, es una experiencia distinta. Un programa bastante
elaborado es el \comando{lynx}(figura
\ref{fig:lynx-www.lunix.com.ar}). Abriendo una terminal y tipeando
\comando{lynx http://www.lynx.org} podremos acceder a la p�gina de los
creadores. Al no mostrar gr�ficos es mucho m�s r�pido que otros
navegadores.  Y sin lugar a dudas ocupa mucho menos espacio.

\figura{P�gina de LUNIX en Lynx}{lynx-www.lunix.com.ar}








