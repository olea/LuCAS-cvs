%%%%%%%%%%%%%%%%%
% Secci�n: Chat %
%%%%%%%%%%%%%%%%%
\section{Chat}

Una de las activiadades m�s adictivas en Internet
es el chat. Cada d�a se hace m�s popular el chat a travez de 
\emph{programas de chat} o \emph{irc}\footnote{No es lo mismo \emph{irc}
que chatear a travez de web.}.

Para chatear existen varios programas, en KDE el m�s usado es
\comando{ksirc} Se encuentra en \rama{K-Internet-Cliente de irc}. Si bien 
existen muchos clientes m�s, este es relativamente simple como para empezar.

En la pantalla principal (fig. \ref{fig:ksirc-Inicial}) hay un menu
llamado \emph{Conexiones} con una entrada que dice \emph{Nuevo
servidor..}

\figura{Pantalla principal de \comando{ksirc}}{ksirc-Inicial}

En la figura \ref{fig:ksirc-Conexion} muestra un cuadro de di�logo
para completar el nombre de servidor (como por ejemplo {\tt us.undernet.org})

\figura{Conectarse a un servidor en \comando{ksirc}}{ksirc-Conexion}

Una vez conectado se puede unir a un un canal tipeando \comando{/join
<nombre de canal>} como por ejemplo \comando{/join \#santafe}. Tambi�n 
hay otros comandos:

\begin{itemize}

 \item\comando{/whois} \emph{nick}: muestra informaci�n de un determinado nick

 \item\comando{/ping} \emph{nick}: intenta definir cuanto tiempo tarda
 un mensaje en llegar a nick.

 \item\comando{/help}: brinda ayuda de otros comandos.
\end{itemize}

\figura{Conectado a un servidor en \comando{ksirc}}{ksirc-Conectado}


