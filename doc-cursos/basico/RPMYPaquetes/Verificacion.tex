%%%%%%%%%%%%%%%%%%%%%%%%%
% Secci�n: Verificaci�n %
%%%%%%%%%%%%%%%%%%%%%%%%%
\section{Verificaci�n}

Muchas veces el uso incorrecto del sistema borra o modifica archivos
que pertenecen a paquetes y que son necesarios para su funcionamiento.
Pasado un tiempo, al intentar ejecutar programas, �stos no funcionan
correctamente.

Una alternativa antes de instalar de nuevo el paquete defectuoso es
la verificaci�n del mismo para que no se reinstale equivocadamente.

C�mo verificar un paquete:

\begin{itemize}

\item Seleccionamos el paquete (ej. RPM-Applicaciones-Multimedia-xmms)


\item Ir a la leng�eta Lista de Archivos

\figura{Listado de archivos correpondientes a un paquete}{kpackage-ListaDeArchivos}
\end{itemize}

Se indicar� con una cruz roja cualquier archivo que no se encuentre, en cambio, se
marcar�n los archivos existentes con una tilde verde.

%Un archivo \�faltante? va a ser indicado con una cruz roja, en cambio un
%archivo existente va a ser indicado con un tilde verde.

