\section{Archivos de configuraci�n del int�rprete de comandos {\tt bash}}

\subsection{Introduccion}
La configuraci�n de un int�rprete de comandos consiste mayormente en
establecer las variables de entorno, los \emph{aliases} de comandos y
el formato del \emph{prompt} que se necesiten. Dependiendo del tipo de
uso que se le dar� al int�rprete de comandos, habr� diferentes
necesidades de configuraci�n de estos valores.

Existen dos modalidades de uso del \emph{int�rprete de comandos} bash:

\begin{description}
\item[interactiva] es la com�n
\item[no interactiva] es propicia para scripts
\end{description}

Por esto, el int�rprete de comandos tiene unos cuantos archivos que se 
ejecutan al inicio.

Los archivos pueden categorizarse en:

\begin{description}
  
\item[no�login] cada vez que se ejecuta el bash (incluso desde el
  interprete de comandos), se leen estos archivos. Es el caso m�s
  com�n, cada vez que se abre una \comando{konsole}, \comando{xterm},
  \comando{gnome-terminal} o equivalente, se ejecutan.
  
\item[login] s�lo cuando el usuario comienza la sesi�n se ejecuta.  En
  tiempos de terminales y consolas era f�cil identificar cuando el
  usuario se \emph{logueaba}. Hoy en d�a es muy popular el login
  gr�fico, el cual no carga inmediatamente un \comando{bash} como
  antes.
\end{description}

\subsection{.bashrc}

Este archivo es del tipo no-login. Primero se carga el archivo global
al sistema \archivo{/etc/bashrc} y luego sea pasa al archivo
\archivo{.bashrc} en el directorio personal del usuario.

Es un archivo que probablemente llame a otros como por ejemplo
\archivo{.profile} y establezca las variable se entorno.

\subsection{.bash\_profile}

Este archivo es del tipo login, por lo que se ejecuta una sola vez en
una sesi�n. 

Todos los archivos son scripts por lo que en la secci�n
\ref{seccion:progbash} se ven ejemplos de programaci�n en
\comando{bash}.  Permiten personalizar el \comando{bash}
enormemente u obligando a los usuario a ejecutar ciertas tareas
administrativas en cuanto ingresen al sistema o bien ejecutan el
interprete de comandos.

